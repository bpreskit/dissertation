\documentclass[12pt]{ucsd-thesis}

% The OGS manual cited here was located at the following URL as of 2007-07-31:
% http://ogs.ucsd.edu/academicpolicy/Dissertations_Theses_Formatting_Manual.pdf

% The ucsd-thesis bibliography-style file requires the natbib package.
\usepackage[numbers,longnamesfirst]{natbib}
%\bibliographystyle{ucsd-thesis}
\bibliographystyle{abbrvnat}

% OGS manual (2007): "A bibliography lists works that students consulted or to
% which the reader may be referred, while works cited or reference list cites
% works in the doctoral dissertation or master's thesis."
%
% Use this command to specify the appropriate title for yours.
\renewcommand{\bibname}{References}

% These packages are recommended by the author, but not required.
%\usepackage{times}
\usepackage{amsmath,amsfonts,amssymb,amsthm,amscd,mathtools}
\usepackage{algorithm,algorithmic,caption,subcaption}
\usepackage{xstring}
\usepackage[dvipsnames]{xcolor}
\usepackage{import}
\usepackage{verbatim}
\usepackage{url}
\urlstyle{rm} % OGS said they require URLs to be typeset like normal text
\usepackage[pdftex, bookmarks=true, colorlinks=true, linkcolor=black, citecolor=black, urlcolor=black]{hyperref} % OGS said colored text is unacceptable
\usepackage{cleveref}
\usepackage{graphicx,subcaption}
\usepackage{helvet}
\usepackage{bbm,enumerate,array}
\usepackage{graphics, parskip}
\usepackage{listings}
%\usepackage{afterpage}

\usepackage{tikz,xparse}
\usetikzlibrary{matrix,backgrounds}
\pgfdeclarelayer{myback}
\pgfsetlayers{myback,background,main}
\tikzset{mycolor/.style = {line width=1bp,color=#1}}%
\tikzset{myfillcolor/.style = {fill=#1}}%
\NewDocumentCommand{\highlight}{O{blue!40} m m}{%
  \draw[mycolor=#1] (#2.north west)rectangle (#3.south east);
}

\NewDocumentCommand{\fhighlight}{O{blue!40} m m}{%
  \fill[myfillcolor=#1] (#2.north west)rectangle (#3.south east);
}


\graphicspath{{figs/}{more-figs/}} % directories where figures are kept

\newtheorem{theorem}{Theorem}
\newtheorem*{thm*}{Theorem}
\newtheorem{corollary}{Corollary}
\newtheorem{lemma}{Lemma}
\newtheorem*{lemma*}{Lemma}
\newtheorem{proposition}{Proposition}
\newtheorem*{proposition*}{Proposition}
\newtheorem{exercise}{Exercise}
\newtheorem*{remark*}{Remark}
\newtheorem*{surfacecor}{Corollary 1}
\newtheorem{conjecture}{Conjecture} 
\newtheorem{question}{Question} 
\theoremstyle{definition}
\newtheorem{definition}{Definition}

\let\mod\undefined
\let\Re\undefined
\let\Im\undefined
\let\vec\undefined

\DeclareMathOperator{\diag}{diag}
%\newcommand{\diag}{\mathrm{diag}}
\DeclareMathOperator{\Col}{Col}
\DeclareMathOperator{\Row}{Row}
\DeclareMathOperator{\Span}{span}
\DeclareMathOperator{\Nul}{Nul}
\DeclareMathOperator{\rank}{rank}
\DeclareMathOperator{\supp}{supp}
\DeclareMathOperator{\Tr}{Tr}
\DeclareMathOperator{\conv}{conv}
\DeclareMathOperator{\cone}{cone}
\DeclareMathOperator{\aff}{aff}
\DeclareMathOperator{\sgn}{sgn}
\DeclareMathOperator*{\argmin}{argmin}
\DeclareMathOperator*{\argmax}{argmax}
\DeclareMathOperator{\circop}{circ}
\DeclareMathOperator{\mod}{\mathbin{mod}}
\DeclareMathOperator{\vol}{vol}
\DeclareMathOperator{\Re}{Re}
\DeclareMathOperator{\Im}{Im}
\DeclareMathOperator{\Proj}{Proj}
\DeclareMathOperator{\vec}{vec}
\DeclareMathOperator{\mat}{mat}
\DeclareMathOperator{\Skew}{Skew}

\DeclarePairedDelimiter{\floor}{\lfloor}{\rfloor}
\DeclarePairedDelimiter{\ceil}{\lceil}{\rceil}
\DeclarePairedDelimiter{\norm}{\lVert}{\rVert}
\DeclarePairedDelimiter{\abs}{\lvert}{\rvert}
\DeclarePairedDelimiter{\inner}{\langle}{\rangle}

\newcommand{\R}{\ensuremath{\mathbb{R}}}  % The real numbers.
\newcommand{\C}{\ensuremath{\mathbb{C}}}
\newcommand{\N}{\ensuremath{\mathbb{N}}}
\newcommand{\Z}{\ensuremath{\mathbb{Z}}}
\renewcommand{\H}{\ensuremath{\mathcal{H}}}
\newcommand{\sym}{\mathcal{S}}
\newcommand{\Sbb}{\ensuremath{\mathbb{S}}}
\newcommand{\B}{\ensuremath{\mathcal{B}}}
\newcommand{\Rn}{\ensuremath{\R^n}}
\newcommand{\Rm}{\ensuremath{\R^m}}
\newcommand{\Rd}{\ensuremath{\R^d}}
\newcommand{\Cn}{\ensuremath{\C^n}}
\newcommand{\Cm}{\ensuremath{\C^m}}
\newcommand{\Cd}{\ensuremath{\C^d}}
\newcommand{\mfr}{\mathfrak{R}}
\newcommand{\ulmfr}{\underline{\mfr}}
\newcommand{\Rmxn}{\ensuremath{\R^{m \times n}}}
\newcommand{\Cmxn}{\ensuremath{\C^{m \times n}}}
\newcommand{\Rnxn}{\ensuremath{\R^{n \times n}}}
\newcommand{\Cnxn}{\ensuremath{\C^{n \times n}}}
\newcommand{\Cdxd}{\ensuremath{\C^{d \times d}}}
\newcommand{\Rdxd}{\ensuremath{\R^{d \times d}}}
\newcommand{\bigfrac}[2]{\ensuremath{\frac{\displaystyle #1}{\displaystyle #2}}}
\newcommand{\kron}{\otimes}
\newcommand{\iid}{\overset{\text{i.i.d.}}{\sim}}
\newcommand{\ee}{\mathrm{e}}
\newcommand{\ii}{\mathrm{i}}
\renewcommand{\th}{\ensuremath{^{\text{th}}}}
\newcommand{\st}{\ensuremath{^{\text{st}}}}
\newcommand{\ow}{\text{otherwise}}
\renewcommand{\c}{\ensuremath{\mathbf{c}}}
\newcommand{\x}{\ensuremath{\mathbf{x}}}
\newcommand{\m}{\ensuremath{\mathbf{m}}}
\newcommand{\y}{\ensuremath{\mathbf{y}}}
\newcommand{\tf}{\ensuremath{\tilde{f}}}
\newcommand{\ux}{\underline{x}}
\newcommand{\uz}{\underline{z}}
\newcommand{\uX}{\underline{X}}
\newcommand{\uR}{\underline{R}}
\newcommand{\utX}{\underline{\tX}}
\newcommand{\uL}{\underline{L}}
\newcommand{\uW}{\underline{W}}
%% \newcommand{\Fb}{\overline{F}}
\newcommand{\tF}{\widetilde{F}}
\newcommand{\Lc}{\mathcal{L}}
\newcommand{\Rc}{\mathcal{R}}
\newcommand{\Ac}{\mathcal{A}}
\newcommand{\Nc}{\mathcal{N}}
\newcommand{\CN}{\mathcal{C N}}
\newcommand{\Dc}{\mathcal{D}}
\newcommand{\hz}{\hat{z}}
\newcommand{\hZ}{\hat{Z}}
\newcommand{\hL}{\hat{L}}
\renewcommand{\v}{\ensuremath{\mathbf{v}}}
\renewcommand{\u}{\ensuremath{\mathbf{u}}}
\newcommand{\F}{\ensuremath{\mathcal{F}}}
\newcommand{\mg}{martingale}
\newcommand{\E}{\ensuremath{\mathbb{E}}}
\newcommand{\rv}{random variable}
\newcommand{\psd}{\ensuremath{\mathcal{S}_+}}
\newcommand{\psdn}{\ensuremath{\mathcal{S}_+^n}}
\newcommand{\bigO}{\mathcal{O}}
\newcommand{\one}{\ensuremath{\mathbbm{1}}}
\renewcommand{\a}{\ensuremath{\overline{a}}}
\newcommand{\dbar}{\overline{d}}
\newcommand{\deltabar}{\overline{\delta}}
\newcommand{\mintheta}{\min_{\theta \in [0, 2\pi]}}
\newcommand{\eit}{\ee^{\ii \theta}}
\newcommand{\eiti}{\ee^{\ii \theta_i}}
\newcommand{\conj}[1]{\overline{#1}}
\renewcommand{\subset}{\subseteq}

\newcommand{\rowmat}[3]{\ensuremath{\begin{bmatrix} #1_{#2} & \cdots & #1_{#3} \end{bmatrix}}}
\newcommand{\rowmatfun}[3]{
  \ensuremath{\noexpandarg
    \begin{bmatrix}
      \StrSubstitute{#1}{@}{{#2}} &
      \cdots &
      \StrSubstitute{#1}{@}{{#3}}
    \end{bmatrix}
  }
}
\newcommand{\colmat}[3]{\ensuremath{\begin{bmatrix} #1_{#2} \\ \vdots \\ #1_{#3} \end{bmatrix}}}
\newcommand{\colmatfun}[3]{
  \ensuremath{\noexpandarg
    \begin{bmatrix}
      \StrSubstitute{#1}{@}{{#2}} \\
      \vdots \\
      \StrSubstitute{#1}{@}{{#3}}
    \end{bmatrix}
  }
}
\newcommand{\diagcornmat}[2]{
  \ensuremath{
    \begin{bmatrix}
      #1 & \cdots & 0 \\
      \vdots & \ddots & \vdots \\
      0 & \cdots & #2
    \end{bmatrix}
  }
}
\newcommand{\bigdiagcornmat}[2]{
  \ensuremath{
    \begin{bmatrix}
      #1 & 0 & \cdots & 0 \\
      0 & \ddots & & \vdots \\
      \vdots & & \ddots & 0 \\
      0 & \cdots & 0 & #2
    \end{bmatrix}
    }
  }
\newcommand{\diagmat}[3]{
  \diagcornmat{#1_{#2}}{#1_{#3}}
}
\newcommand{\diagmatfun}[3]{\noexpandarg
  \diagcornmat{\StrSubstitute{#1}{@}{#2}}{\StrSubstitute{#1}{@}{#3}}
}
\newcommand{\cornmat}[5]{
  \ensuremath{
  \begin{bmatrix} #1_{#2#4} & \cdots & #1_{#2#5} \\
    \vdots & \ddots & \vdots \\
    #1_{#3#4} & \cdots & #1_{#3#5}
  \end{bmatrix}
  }
}
\newcommand{\cornmatfun}[5]{
  \ensuremath{\noexpandarg
    \StrSubstitute{#1}{@r}{{#2}}[\cmftop]
    \StrSubstitute{#1}{@r}{{#3}}[\cmfbot]
    \expandarg
    \begin{bmatrix}
      \StrSubstitute{\cmftop}{@c}{{#4}}  & \cdots & \StrSubstitute{\cmftop}{@c}{{#5}} \\
    \vdots & \ddots & \vdots \\
    \StrSubstitute{\cmfbot}{@c}{{#4}} & \cdots & \StrSubstitute{\cmfbot}{@c}{{#5}}
  \end{bmatrix}
  }
}

\newcommand{\cornmatfunexp}[5]{
  \ensuremath{%\noexpandarg
    \StrSubstitute{#1}{@r}{{#2}}[\cmftop]
    \StrSubstitute{#1}{@r}{{#3}}[\cmfbot]
    \expandarg
    \begin{bmatrix}
      \StrSubstitute{\cmftop}{@c}{{#4}}  & \cdots & \StrSubstitute{\cmftop}{@c}{{#5}} \\
    \vdots & \ddots & \vdots \\
    \StrSubstitute{\cmfbot}{@c}{{#4}} & \cdots & \StrSubstitute{\cmfbot}{@c}{{#5}}
  \end{bmatrix}
  }
}

\newcommand{\BPnote}[1]{\textcolor{ForestGreen}{[{\em {\bf **BP Note:} #1}]}}

\newcommand{\crefrangeconjunction}{--}

\newenvironment{remark}
               {\textbf{Remark.}}
               {}
               
\newenvironment{piecewise}
               {\left\{\begin{array}{r@{,\quad}l}}
               {\end{array}\right.}


% OGS manual (2007): "The title should be specific, unambiguous, and
% descriptive of the research, with easily identifiable key words that will
% ensure electronic retrieval."
\title{General Phase Retrieval with Locally Supported Measurements}

% OGS manual (2007): "Students should list their full legal name (name on
% record with UCSD's Office of the Registrar) and follow this format
% throughout"
\author{Brian P. Preskitt}

% OGS manual (2007): "Doctoral students should refer to their document as a
% dissertation. Master's students should refer to their document as a thesis."
\doctype{dissertation}

% OGS manual (2007): "The degree title listed should be the degree title that
% UCSD will actually confer; if unsure, contact your Graduate Coordinator."
\degree{Doctorate of Philosophy}
\field{Mathematics with Specialization in Computational Science}

% OGS manual (2007): "Degree year: Students must use only the year of the
% quarter of degree conferral.
\degreeyear{2018}

% OGS manual (2007): "All committee members must be listed, chair first, using
% the generic title Professor. If professor is not appropriate for all
% committee members, list all names without any titles. Alphabetize all members
% listed after chair."
\chair{Professor Rayan Saab}
\othermembers{Professor Kamalika Chaudhuri \\
Professor Todd Kemp \\
Professor Jiawang Nie \\
Professor Yoav Freund }
\numberofmembers{5} % committee members, including the chair, for signatures

\begin{document}

\begin{frontmatter}

\maketitle

% OGS manual (2007): "Students who do not wish to include a copyright notice
% within their doctoral dissertation or master.s thesis must include a second
% page, regardless. In this instance, the second page will be blank. This page
% does not have a page number."
\copyrightpage

% OGS manual (2007): "The signature page is always numbered as page 'iii'."
\approvalpage

% OGS manual (2007): "This section may be used to dedicate the doctoral
% dissertation or master's thesis to someone or to acknowledge particular
% persons. Within the usual margin restrictions, any format is acceptable for
% this page."
\begin{dedication}
This dissertation is lovingly dedicated to my brother, Charles Preskitt.
\end{dedication}

% OGS manual (2007): "An epigraph is a quotation that is pertinent but not
% integral to the text. Within the usual margin restrictions, any format is
% acceptable for this page."
\begin{epigraph}
For in much wisdom is much vexation, and he who increases knowledge increases sorrow. -- Ecclesiastes 1:18
\end{epigraph}

% OGS manual (2007): "All doctoral dissertations or master's theses are
% required to use a table of contents."
\tableofcontents

% OGS manual (2007): If plates or illustrations such as figures, tables,
% graphs, slides, maps, diagrams, charts, photos, etc., are scattered
% throughout the doctoral dissertation or master's thesis, make a separate
% 'List of Figures', 'List of Illustrations', 'List of Tables', etc. to follow
% the table of contents."
\listoffigures

\listoftables

% Acknowledgements are sometimes required; see the OGS manual for more detail.
\begin{acknowledgements}
  First and foremost, I would like to thank my advisor Rayan Saab for all of his involvement in my academic career.  In my first year of graduate school, it was his class on compressed sensing that convinced me to switch into the computational math track, which in turn is where I found the material and the problems that motivated me to continue in academic research.  Since then, his support -- in our many hours of teamwork at the whiteboard; his connections to problems, authors, conferences, and collaborators; and the simple things such as critiques of my writing and talks -- was uniquely indispensable in shaping the environment in which I found my place as a researcher.  I am immensely thankful for the inspiration of his talent and energetic leadership.

  I would also like to thank Mark Iwen and Aditya Viswanathan, our two co-authors and collaborators on the subjects covered in this thesis.  The ground that they broke on this new strain of phase retrieval proved to be incredibly fruitful, and I am thankful for being welcomed as a contributor to their ongoing research efforts.  This partnership was foundational to my beginnings in research and much of my subsequent progress, and I am deeply indebted to their generous spirit of academic camaraderie.  Chapter \ref{ch:our_model}, \crefrange{sec:intro}{sec:AltPerturbBounds}, in full, is a reprint of material published with Iwen, Saab, and Viswanathan as published in Applied and Computational Harmonic Analysis 2018.  \Cref{ch:2d_base_model} \crefrange{sec:2d_intro}{sec:2d_num}, in part, is a reprint of material published with these authors in the Proceedings of SPIE vol.~10394, 2017.

\end{acknowledgements}

% OGS manual (2007): "A biographical notice, or vita, is required only of
% doctoral students. Master's students do not include a vita." See the OGS
% manual for more detail.
\begin{vitapage}
% A list environment. Each item is a separate entry in the vita. Text between
% the brackets goes in the left column. Everything else goes in the right.
% Text wraps automatically, but you can force a line break with \\.
\begin{vita}
\item[2013] Bachelor of Science, Texas Christian University
\item[2013-2018] Teaching Assistant, University of California, San Diego
\item[2015] Master of Science, University of California, San Diego
\item[2016-2017] Associate Instructor, University of California, San Diego
\item[2018] Doctor of Philosophy, University of California, San Diego
\end{vita}

% Also a list. Each entry is a separate item. Unlike the vita, there is only
% one column, so no labels are needed. The list just makes formatting easy.
\begin{publications}
\item  M. Iwen, B. Preskitt, R. Saab, and A. Viswanathan.  \emph{An Eigenvector-Based Angular Synchronization Method for Phase Retrieval from Local Correlation Measurements}.  \href{https://arxiv.org/abs/1612.01182}{arXiv:1612.01182}.  Accepted for publication June 2018.
  \item M. Iwen, B. Preskitt, R. Saab, and A. Viswanathan.  \emph{Phase retrieval from local measurements in two dimensions}.  Proceedings of SPIE 10394, Wavelets and Sparsity XVII, San Diego, CA, 2017.
\end{publications}

% Fields of Study is trickier. Each item is one field, and there are several
% entries in each item. Again, you can force a line break with \\, but note the
% \ \\ on the first line. This forces the label on a line by itself.
%% \begin{studyfields}
%% \item [Major Field: Mathematics with Specialization in Computational Sciences] \ \\
%%       Studies in Signal Processing \\
%%       Professor Rayan Saab
%% \end{studyfields}
\end{vitapage}

% OGS manual (2007): "It is important to write an abstract which gives a clear
% impression of the content and major divisions of the doctoral dissertation or
% master's thesis. Use whole sentences, not elliptic phrases. Abstracts of
% masters' theses must not exceed 250 words."
\begin{abstract}
In this dissertation, we study a new approach to the problem of phase retrieval, which is the task of reconstructing a complex-valued signal from magnitude-only measurements.  This problem occurs naturally in several specialized imaging applications such as electron microscopy and X-ray crystallography.  Although solutions were first proposed for this problem as early as the 1970s, these algorithms have lacked theoretical guarantees of success, and phase retrieval has suffered from a considerable gap between practice and theory for almost the entire history of its study.

A common technique in fields that use phase retrieval is that of \emph{ptychography}, where measurements are collected by only illuminating small sections of the sample at any time.  We refer to measurements designed in this way as \emph{local measurements}, and in this dissertation, we develop and expand the theory for solving phase retrieval in measurement regimes of this kind.  Our first contribution is a basic model for this setup in the case of a one-dimensional signal, along with an algorithm that robustly solves phase retrieval under this model.  This work is unique in many ways that represent substantial improvements over previously existing solutions: perhaps most significantly, many of the recovery guarantees in recent work rely on the measurements being generated by a random process, while we devise a class of measurements for which the conditioning of the system is known and quickly checkable (see \cref{sec:con_number}).  These advantages constitute major progress towards producing theoretical results for phase retrieval that are directly usable in laboratory settings.

Chapter 1 conducts a survey of the history of phase retrieval and its applications.  Chapter 2 reviews the mathematical literature on the subject, including the first solutions and the theoretical work of the last decade.  Chapter 3 presents co-authored results defining and establishing the setting and solution of the base model explored in this dissertation.  Chapter 4 expands the theory on what measurement schemes are admissible in our model, including an analysis of conditioning and runtime.  Chapter 5 explores results that bring our model nearer to the actual practice of ptychography.  Chapter 6 includes a few relevant results that may be used for future expansion on this topic.



%Phase retrieval is the problem of solving a system of equations of the form $\y = |A \x|^2 + \eta$, where $\x \in \C^d, A \in \C^{D \times d}, \y, \eta \in \R^D$, and $|\cdot|^2$ represents taking the elementwise magnitude-squared of a complex vector.  Here, $\x$ represents the objective vector to be found, $A$ is referred to as the ``measurement matrix,'' $\eta$ represents a perturbation or noise vector, and $\y$ represents the measurement data.  Given $\y$ and $A$, we attempt to solve for an approximation of $\x$.  We can imagine this as a set of noisy linear equations in complex variables where the \emph{phases} of the matrix-vector product $A\x$ have been erased; in solving for $\x$, it could therefore be said that we are reconstructing or ``retrieving'' its phase.

%This thesis develops and expands the theory for this problem with a particular family of measuremental setups: in particular, we study the special case where the rows of $A$ represent a collection of vectors with small support, which are each 

\end{abstract}

\end{frontmatter}

\chapter{History of Phase Retrieval}
\import{sections/history/}{history_sec}

\chapter{Applications}
\import{sections/applications/}{applications_sec}

\chapter{Phase retrieval from local correlation measurements}
\label{ch:our_model}
\label{ch:base_model}
\import{sections/base_model/}{base_model_sec}

\chapter{Spanning Masks}
\import{sections/meas/}{meas_sec}

\chapter{Ptychographic Model}
\import{sections/ptychography/}{ptychography_sec}

\chapter{Angular Synchronization}
\import{sections/ang_sync/}{ang_sync_sec}

%% \begin{figure}[t]
%% \begin{center}
%% \leavevmode % abstruse LaTeX invocation required to center within a figure
%% \includegraphics{sample-figure}
%% \end{center}
%% \caption[A lovely figure.]{Isn't this just a lovely figure?}
%% \label{fig:lovely} % for later \ref
%% \end{figure}

%% This chapter has a lovely figure, Figure~\ref{fig:lovely}, and a footnote\footnote{Here is the footnote.}. For no reason apart from demonstration, it cites \citet{goguen:introduction} and two other papers \citep{goguen:information,grudin:groupware}.

\appendix

\chapter{Sample Appendix}

If you seek a pleasant appendix, look no further.

\bibliography{bibs/dissertation}

\end{document}
