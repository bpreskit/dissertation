In this chapter, we consider when a local measurement system $\{m_j\}_{j = 1}^K$ composes a \emph{spanning family} of masks.

\begin{definition}
  We say that $\{m_j\}_{j = 1}^D \subset \C^d$ is a \emph{local measurement system} or \emph{family of masks} of support $\delta$ if $1 \in \supp(m_j)$ and $\supp(m_j) \subset [\delta]$ for each $j$.
\end{definition}

\begin{definition}
  Given a local measurement system $\{m_j\}_{j = 1}^D$ in $\C^d$, the associated \emph{lifted measurement system} is the set $\mathcal{L}_{\{m_j\}} = \{S^{\ell} m_j m_j^* S^{- \ell}\}_{(\ell, j) \in [d]_0 \times [D]} \subset \C^{d \times d}$.
\end{definition}

\begin{definition}
  We say that a family of masks $\{m_j\}_{j = 1}^K \subset \C^d$ of support $\delta$ is a \emph{spanning family} if $\Span \{S^\ell m_j m_j^* S^{-\ell}\}_{(\ell, j) \in [d]_0 \times [K]} = T_\delta(\H^d)$.
\end{definition}



%% \section*{Preliminaries}

%% \begin{itemize}
%% \item Indices of matrices in $\C^{d \times d}$ and vectors in $\C^d$ are always taken modulo $d$.
%% \item For $k \in \N, n \in \Z, [k]_n = \{n, n + 1, \ldots, n + k - 1\}$ and $[k] = [k]_1$.
%% \item $S_d \in \R^{d \times d}$ is the $d \times d$ shift operator, such that $(S_d \x)_i = \x_{i - 1}$.  Typically we imply the subscript by context, writing $S$.
%% \item $R \in \R^{d \times d}$ is the operator that reverses a vector's entries, leaving the first entry fixed.  Namely, $(R \x)_i = \x_{2 - i}$.
%% \item Given $\x \in \C^d$ and $k \in [d], \circop_k(\x) \in \C^{d \times k}$ denotes the first $k$ columns of the circulant matrix whose first column is $\x$.  In particular, $\circop_k(\x) e_i = S^{i - 1} \x$ for $i \in [k]$.  When the subscript is omitted, $\circop(\x) := \circop_d(\x)$.
%% \item $\omega_d := \ee^{\frac{2 \pi \ii}{d}}$ is the $d^{\text{th}}$ root of unity.  When context permits, $d$ is implied and we use just $\omega$.
%%   \item For $i, n \in \N, e_i^n \in \R^n$ is the $i^{\text{th}}$ column of the $n \times n$ identity matrix.  When context permits, $n$ is implied and we write $e_i$.  In particular, whenever $e_i$ is used in a matrix multiplication, $n$ is taken to be appropriate so that the multiplication is legal.
%% \item For $k \in \Z, F_k \in \C^{k \times k}$ is the $k \times k$ Fourier matrix with $(F_k)_{ij} = \omega_k^{(i-1)(j-1)}$.
%%   \item For $m, n \in \N, f_m^n = F_m e_n$ is the $n^{\text{th}}$ column of the $m \times m$ unitary Fourier matrix, where $e_n \in \R^m$ has its index taken modulo $m$.
%% \item Given $\x, \y \in \C^d, \x \circ \y$ denotes the Hadamard/elementwise product of $\x$ and $\y$; specifically $(\x \circ \y)_i = \x_i \y_i$.
%% \item Given $A \in \C^{d \times d}, \diag(A, m) \in \C^d$ denotes the $m^{\text{th}}$ circulant off-diagonal of $A$.  That is, $\diag(A, m)_i = A_{i, i + m}$.
%%   \item Given $\x \in \C^d, \diag(\x) \in \C^{d \times d}$ is the diagonal matrix whose diagonal entries are the entries of $\x$.  Namely, $\diag(\x) e_i = \x_i e_i$.  When the intention is clear from context, we may write $D_\x := \diag(\x)$.
%%   \item $\H^d$ is the set of Hermitian matrices in $\C^{d \times d}$, to be viewed as a $d^2$-dimensional vector space over $\R$.
%%     \item $\mathcal{R}_d : \bigcup_{k = 1}^\infty \C^k \to \C^d$ is a universal resize mapping, where for $v \in \C^k,$ \[\mathcal{R}_d(v)_i = \left\{\begin{array}{r@{,\quad}l} v_i & i \le k \\ 0 & \text{otherwise} \end{array}\right. \text{for}\ i \in [d]\]
%% \end{itemize}

\section{Conditions for a spanning family}

\begin{proposition}
  Suppose that $\gamma \in \R^d$ has $1 \in \supp(\gamma) = [\delta].$  Set $D = \min\{2 \delta - 1, d\},$ take $K \ge 2 \delta - 1$ and let \[\begin{array}{r@{\;=\;}r}v_j & \sqrt{K} \mathcal{R}_d(F_K e_j) \\ v_j^D & \sqrt{K} \mathcal{R}_D(F_K e_j) \end{array},\quad j \in [D],\  2 \delta - 1 \le K.\]  Define a local measurement system  $\{m_j\}_{j \in [D]}$ by setting $m_j = \gamma \circ v_j$.  Then $\{m_j\}_{j \in [D]}$ is a spanning family if and only if all the sets $J_k := \{m \in [\delta]_0 : (F_d (\gamma \circ S^{-m} \gamma))_k \neq 0\}$, for all $k \in [d]$ satisfy \[\left\{\begin{array}{r@{,\quad}l} 2 |J_k| - 1 \ge D & 0 \in J_k \\ 2 |J_k| \ge D & \text{otherwise}\end{array}\right..\] \label{prop:spanning_family}
\end{proposition}

%% \begin{remark}
%%   We remark that if $D = 2 \delta - 1$, this condition is equivalent to requiring that $J_k = [\delta]$ for each $k$, which means that $F \circop(g_m)$ has no entries equal to zero for any $m \in [\delta]_0$.  Perhaps more intuitively, this stronger condition means that $F (\gamma \circ S^{-m} \gamma)$ has no zeros for any $m \in [\delta]_0$.
%% \end{remark}

The proof will make use of the following lemmas.

\begin{lemma}
  Define $w_j = \mathcal{R}_{N_1}(f_j^{N_2}), j \in [N_2]$ and set \[\rho_j = \mathrm{Re}(w_j) \quad\text{and}\quad \mu_j = \mathrm{Im}(w_j)\] to be vectors containing the real and imaginary components of $w_j$.  Then for $1 \le \ell_1 < \cdots < \ell_k \le \frac{N_2 + 1}{2}$ with $k \le N_1$, we have \begin{align*} \dim\,\Span\{w_{\ell_i}, w_{2 - \ell_i}\}_{i = 1}^k &= \dim\,\Span\{\rho_{\ell_i}, \mu_{\ell_i}\}_{i = 1}^k \\ &= \left\{\begin{array}{r@{,\quad}l} 2k - 1 & \ell_1 = 1 \\ 2k & \text{otherwise}\end{array}\right.,\end{align*} where the indices are taken modulo $N_2$. \label{lem:conjugate_span_dim}
\end{lemma}

\begin{proof}[Proof of lemma \ref{lem:conjugate_span_dim}]

The first equality is clear by considering that $w_{2 - i} = \overline{w_i}$, so $\rho_k = \frac{1}{2}(w_i + w_{2-i})$ and $\mu_i = -\frac{i}{2}(w_i - w_{2 - i})$.  We set $M = \dim\,\Span\{w_{\ell_i}, w_{2 - \ell_i}\}_{i = 1}^k$ to be the common dimension of the two spaces under consideration.

We now divide into two cases: if $N_1 < N_2$, then $\{w_j\}_{j \in [N_2]}$ is full spark, as any $N_1 \times N_1$ submatrix of $\begin{bmatrix} w_1 & \cdots & w_{N_2} \end{bmatrix}$ will be a Vandermonde matrix of the form \[V = \frac{1}{\sqrt{N_2}}\begin{bmatrix} w_{\ell_1} & \cdots & w_{\ell_{N_1}} \end{bmatrix}\] with determinant \[N_2^{-N_1 / 2}\prod_{1 \le i < j \le N_1} (\omega_{N_2}^{\ell_i - 1} - \omega_{N_2}^{\ell_j - 1}),\] which is immediately non-zero since $\omega_{N_2}^{\ell_i - 1} - \omega_{N_2}^{\ell_j - 1} = 0$ only when $\ell_i - \ell_j = 0 \mod N_2$, which cannot happen when $N_1 < N_2$.

When $N_1 \ge N_2, \{w_j\}_{j \in [N_2]}$ is linearly independent, since its members form the matrix $\begin{bmatrix} F_{N_2} \\ 0_{N_1 - N_2 \times N_2} \end{bmatrix}$.

In either case, $M$ is equal to the cardinality of $\{\ell_i, 2 - \ell_i\}_{i = 1}^k,$ which has $2k - 1$ elements if and only if $\ell_1 = 1$; otherwise it has $2k$.  We remark that a collision where $\ell_i = (2 - \ell_i \mod N_2) = N_2 / 2 + 1$ is precluded since we have asserted $\ell_i \le \frac{N_2 + 1}{2}$.

\end{proof}

\begin{lemma}
  For $v \in \R^d,$ we have \begin{align} \circop(v) \rho_k^d &= \frac{1}{2}\mathrm{Re}((Fv)_k f_k^d) \label{eq:real_part} \\ \circop(v) \mu_k^d &= \frac{1}{2}\mathrm{Im}((Fv)_k f_k^d) \label{eq:imag_part}. \end{align} In particular, if $(Fv)_k \neq 0$ and $k \notin \{1, \frac{d}{2} + 1\}$, then $\rho_k^d, \mu_k^d \notin \Nul(\circop(v))$; if $k \in \{1, \frac{d}{2} + 1\}$, then $\rho_k^d \notin \Nul(\circop(v))$ and $\mu_k^d = 0$.  On the other hand, if $(Fv)_k = 0$, then $\rho_k^d, \mu_k^d \in \Nul(\circop(v))$.  \label{lem:eigenbits}
\end{lemma}

\begin{proof}[Proof of lemma \ref{lem:eigenbits}]
  We set $\lambda_k^d = (Fv)_k$, and recalling that $\circop(v) = F \diag(F v) F^*$, we observe that \begin{align*}\circop(v) \mu_k^d &= \circop(v) \frac{1}{2} (f_k^d + f_{2 - k}^d) = \frac{1}{2}(\circop(v) f_k^d + \circop(v) f_{2 - k}^d) \\ &= \frac{1}{2}(\lambda_k^d f_k^d + \lambda_{2 - k}^d f_{2 - k}^d).\end{align*}  \eqref{eq:real_part} follows immediately since $\lambda_k^d = \overline{\lambda_{2 - k}^d}$ when $v \in \R^D$.  \eqref{eq:imag_part} follows from an analogous calculation.

  If $\lambda_k^d \neq 0$ and $k \notin \{1, \frac{d}{2} + 1\}$, then $\omega_d^{k - 1}$ is a non-real root of unity and there exists some $j$ such that $\mathrm{Re}(\omega_d^{(j - 1)(k - 1)} \lambda_k^d) \neq 0,$ and similarly for $\mathrm{Im}(\omega_d^{(j - 1)(k - 1)} \lambda_k^d) \neq 0.$  When $k \in \{1, \frac{d}{2} + 1\}, \omega_d^{(k - 1)} \in \R$ so $\mu_k^d = 0$, but $\lambda_k^d \in \R$ in this case (because $v \in \R^d$), so $\circop(v) \rho_k^d = \lambda_k^d \rho_k^d \neq 0$.  The claim concerning the case of $\lambda_k^d = 0$ is immediate from \eqref{eq:real_part} and \eqref{eq:imag_part}.
\end{proof}

\begin{proof}[Proof of proposition \ref{prop:spanning_family}]
  For this proof, we set \begin{align*} (\rho_k^d, \mu_k^d) &= (\mathrm{Re}(f_k^d), \mathrm{Im}(f_k^d)) \\ (\rho_k, \mu_k) &= (\mathrm{Re}(v_k), \mathrm{Im}(v_k)) \\ (\rho_k^D, \mu_k^D) &= (\mathrm{Re}(v_k^D), \mathrm{Im}(v_k^D)) \end{align*}
  
  In this case, we identify $\mathcal{L}_\gamma := \mathcal{L}_{\{m_j\}}$.  By a basic dimension count, $\{m_j\}_{j \in [D]}$ is a spanning family if and only if $\mathcal{L}_\gamma$ is linearly independent, so we consider the conditions under which a linear combination of this lifted measurement system can be equal to zero.  To this end, we define the operator $\mathcal{A}^* : \R^{d \times D} \to \C^{d \times d}$ by \begin{equation} \mathcal{A}^*(C) = \sum_{\ell \in [d], j \in [D]} C_{\ell, j} S^\ell m_j m_j^* S^{-\ell} \label{eq:synth_op} \end{equation} and begin with the observation that, for any $A \in \C^{d \times d}$ we have \[\diag(S^\ell A S^{-\ell}, m) = S^\ell \diag(A, m).\]  We then have
  \[\arraycolsep=1.4pt\def\arraystretch{2}
  \begin{array}{c>{\displaystyle}rcl@{\quad}r}
    & \sum_{j \in [D], \ell \in [d]} C_{\ell, j} S^\ell m_j m_j^* S^{-\ell} & = & 0 & \\
    \iff & \diag\left(\sum_{j \in [D], \ell \in [d]} C_{\ell, j} S^\ell m_j m_j^* S^{-\ell}, m\right) & = & 0 & \text{for all} \ m \in [\delta]_0 \\
    \iff & \sum_{j \in [D], \ell \in [d]} C_{\ell, j} \diag(S^\ell m_j m_j^* S^{-\ell}, m) & = & 0 & \text{for all} \ m \in [\delta]_0 \\
    \iff & \sum_{j \in [D], \ell \in [d]} C_{\ell, j} S^{\ell} \diag(m_j m_j^*, m) & = & 0 & \text{for all} \ m \in [\delta]_0 \\
  \end{array}\]
  
  At this point, we consider that \begin{align} \diag(m_j m_j^*, m) &= \diag((\gamma \circ v_j) (\gamma \circ v_j)^*, m) = \diag(D_{v_j} \gamma \gamma^* D_{\overline{v_j}}, m) \\ &= \omega_K^{m(j - 1)} \diag(\gamma \gamma^*, m). \label{eq:gam_diag}\end{align}  We now set $g_m := \diag(\gamma \gamma^*, m) = \gamma \circ S^{-m} \gamma$ and proceed with the previous chain of implications:
  \[\arraycolsep=1.4pt%\def\arraystretch{1.5}
  \begin{array}{c>{\displaystyle}rcl@{\quad}r}
    & \sum_{j \in [D], \ell \in [d]} C_{\ell, j} S^{\ell} \diag(m_j m_j^*, m) & = & 0 & \text{for all} \ m \in [\delta]_0 \\
    \iff & \sum_{j \in [D], \ell \in [d]} C_{\ell, j} S^{\ell} (\omega_K^{m(j - 1)} g_m) & = & 0 & \text{for all} \ m \in [\delta]_0 \\
    \iff & \sum_{j \in [D], \ell \in [d]} C_{\ell, j} \omega_K^{m(j - 1)} S^{\ell} g_m & = & 0 & \text{for all} \ m \in [\delta]_0 \\
    \iff & \circop(g_m) C \, v^D_{m+1} & = & 0 & \text{for all} \ m \in [\delta]_0 \\
  \end{array}\]

  We now recall that any circulant matrix $\circop(v)$ is diagonalized by the Discrete Fourier Matrix, such that, for $v \in \C^d,$ \[\circop(v) = F_d \diag(\sqrt{d} F_d v) F_d^* = \sqrt{d} \sum_{j = 1}^d (F_d v)_j f_j^d (f_j^d)^*.\]  By writing $\lambda_k^m = \sqrt{d}(F g_m)_k$, we get a natural decoupling of the previous equations: for a fixed $m$, we have that $\circop(g_m) C f_{m + 1} = 0$ if and only if \[\sum_{k = 1}^d \lambda_k^m f_k^d (f_k^d)^* C \, f_{m + 1} = \sum_{k = 1}^d (\lambda_k^m (f_k^d)^* C \, f_{m + 1}) f_k^d = 0.\]  Since this last expression is a linear combination of an orthonormal basis, it occurs only when $\lambda_k^m (f_k^d)^* C \, f_{m + 1} = 0$ for all $k \in [d]$.  We collect these equations over $m \in [\delta]_0$, considering the definition of $J_k$ and that $g_m \in \R^d$ implies $\lambda_k^m = 0 \iff \lambda_{2 - k}^m = 0$ to restate this condition as $\begin{bmatrix} f_k^d & f_{2 - k}^d \end{bmatrix}^* C v^D_{m + 1} = 0$ for all $k \in [d], m \in J_k$.  Since $\Span\{f_k^d, f_{2 - k}^d\} = \Span\{\rho_k^d, \mu_k^d\}$, we further restate this as $\begin{bmatrix} \rho_k^d & \mu_k^d \end{bmatrix}^* C v_{m + 1}^D = 0$ for all $k \in [d], m \in J_k$; setting $W_k = C^* \begin{bmatrix} \rho_k^d & \mu_k^d \end{bmatrix} \in \R^{D \times 2},$ we now get that $\mathcal{A}^*(C) = 0 \iff \Col(W_k) \subset \{v_{m + 1}^D\}_{m \in J_k}^\perp \cap \R^D$ for all $k \in [d]$.

  We now claim that $\mathcal{A}^*$ is invertible if and only if the subspaces $\{v_{m + 1}^D\}_{m \in J_k}^\perp \cap \R^D$ are all trivial.  Indeed, if we fix a $k$ and have some non-zero $u \in \{v_{m + 1}^D\}_{m \in J_k}^\perp \cap \R^D,$ then we may set $C = \rho_k^d u^*$, such that \[\circop(g_m) C v_{m + 1}^D = (\circop(g_m) \rho_k^d) (u^* v_{m + 1}^D).\]  For $m \in J_k, u^* v_{m + 1}^D = 0$ by hypothesis on $u$, and for $m \notin J_k, \circop(g_m) \rho_k^d = 0$ by definition of $J_k$ and lemma \ref{lem:eigenbits}.

  For the other direction, assume $\{v_{m + 1}^D\}_{m \in J_k}^\perp \cap \R^D = 0$ for each $k \in [d]$.  Then $\mathcal{A}^*(C) = 0 \iff \Col(W_k) = \{0\} \iff W_k = 0$ for all $k$.  However, $\{\rho_k^d\}_{k \in [d]} \cup \{\mu_k^d\}_{k \in [d] \setminus \{1, \frac{d}{2} + 1\}}$ is an orthogonal basis for $\R^d$, so \[\begin{array}{cr@{\,=\,}lr} & W_k & 0 & \text{for all} \ k \in [d] \\ \iff & C^* \rho_k^d = C^* \mu_k^d & 0 & \text{for all} \ k \in [d] \\ \iff & C & 0 & \end{array}\]

%  For every $m \in [\delta]_0$ such that $\circop(g_m)$ is non-singular, we must have that $f_{m + 1} \in \Nul(C)$.    Of course, the eigenvalues of $\circop(g_m)$ are given by $F_d g_m = F_d (\gamma \circ S^{-m} \gamma)$, so $\circop(g_m)$ is non-singular exactly when $F_d (\gamma \circ S^{-m})$ has no entries equal to zero (i.e. when $m \in J$).  These conditions then give us that $\Row(C) \subset \{f_{m+1}\}_{m \in J}^\perp \cap \R^{D}$.  Hence $B_\gamma$ is linearly independent if $\{f_{m+1}\}_{m \in J}^\perp \cap \R^{D} = \{0\}$.

  We complete the proof by considering that, for $u \in \R^{D}, \langle v_j^D, u \rangle = 0$ if and only if $\langle \rho_j^D, v \rangle = \langle \mu_j, v \rangle = 0$, so \[\{v_{m + 1}\}_{m \in J_k}^\perp \cap \R^{D} = \{\rho_{m + 1}, \mu_{m + 1}\}_{m \in J_k}^\perp\] which has dimension $\max\{D - (2|J_k| - \one_{0 \in J_k}), 0\}$ by lemma \ref{lem:conjugate_span_dim}.  Therefore, $\mathcal{A}^*$ is invertible if and only if $2|J_k| - \one_{0 \in J_k} \le D$ for all $k \in [d]$, as claimed.
\end{proof}

\begin{remark}
  It turns out that this condition is generic, in the sense that it fails to hold only on a subset of $\R^d$ with Lebesgue measure zero.  We consider that the set of $\gamma \in \R^d$ giving at least one zero in $F(\gamma \circ S^{-m} \gamma)$ is a finite union of zero sets of non-trivial quadratic polynomials (except when $2 \mid d, \delta \ge d / 2,$ and $m = d / 2$, discussed below) and hence a set of zero measure; therefore, $J_k = [\delta]_0$ for all $\gamma$ outside a set of measure zero and $B_\gamma$ is linearly independent under generic conditions.

To address the case of $m = d/2$, we first remark that this is the only possible exception: indeed, when $m \neq d / 2$, we have that \[F((e_1 + e_{m + 1}) \circ S^m(e_1 + e_{m + 1}))_k = f_k^* e_{m + 1} = \omega^{m(k-1)},\] so $\gamma \to F(\gamma \circ S^m \gamma)_k$ is a non-zero, homogeneous quadratic polynomial and therefore has a zero locus of measure zero.

However, when $d = 2m$, then $\gamma \circ S^m \gamma$ is periodic with period $m$ and $F(\gamma \circ S^m \gamma)_{2i} = 0$ for $i \in [m]_0$.  In particular, if $\delta \ge m$, then $D = d$ and $m \notin J_{2i}$ for all $i \in [m]_0$ for any $\gamma$.  In particular, $|J_2| \le \delta - 1$ and $2 |J_2| - \one_{0 \in J_2} \le 2 \delta - 3 $, so if $\delta \in \{d / 2, d/2 + 1\}$,  all choices of $\gamma$ automatically fail to produce a spanning family.

This exception is quite pathological, though: since our intention is to have $\delta \ll d$, this will rarely be an impediment.  Nonetheless, in the case that you \emph{do} want to have $\Span B_\gamma = \H^d$, then taking $\delta > d / 2 + 1$ gives some space for the condition $2|J_k| - \one_{0 \in J_k}$, and we again have that generic $\gamma$ will produce spanning families.
\end{remark}

\section{Condition number}
Now that we have characterized this collection of spanning families, we are interested in the condition number for solving the linear system $y = \mathcal{A}(T_{\delta}(xx^*)) + \eta$ to estimate $T_\delta(xx^*)$.  We begin by introducing the main result of this section:

\begin{proposition}
  Let $\{m_j\}_{j = 1}^D \subset \C^d$ be a local Fourier measurement system with support $\delta$, mask $\gamma$, and modulation index $K$, where $D = \min(d, 2 \delta - 1)$.  Let $\Ac$ be the associated measurement operator as in \eqref{eq:meas_op}, with canonical matrix representation $A$ as in \eqref{eq:meas_mat}.

If we additionally assume that $2 \delta - 1 \le d$ and $K = 2 \delta - 1,$ then the condition number of $\mathcal{A}$ is \begin{equation}\kappa(\mathcal{A}) = \dfrac{\max\limits_{m \in [\delta]_0, j \in [d]} \lvert F_d (\gamma \circ S^{-m} \gamma)_j \rvert}{\min\limits_{m \in [\delta]_0, j \in [d]} \lvert F_d (\gamma \circ S^{-m} \gamma)_j \rvert}.\label{eq:clean_cond}\end{equation}  Otherwise, if $2 \delta - 1 > d$ or $K > 2 \delta - 1$, we may bound the condition number by \begin{equation}\kappa(\mathcal{A}) \le \dfrac{\max\limits_{m \in [\delta]_0, j \in [d]} \lvert F_d (\gamma \circ S^{-m} \gamma)_j \rvert}{\min\limits_{m \in [\delta]_0, j \in [d]} \lvert F_d (\gamma \circ S^{-m} \gamma)_j \rvert} \kappa(\tF_K), \label{eq:messy_cond}\end{equation} where $\tF_K \in \C^{D \times D}$ is the $D \times D$ principal submatrix of $F_K$.
\label{prop:span_fam_cond}
\end{proposition}

\subsection{Interleaving operators and circulant structure}
\label{sec:interlemma}

To set the stage for the proof, we introduce a certain collection of permutation operators and study their interactions with circulant and block-circulant matrices.  The structure we identify here will be of much use to us in unraveling the linear systems we encounter in our model for phase retrieval with local correlation measurements.

To this end, we introduce the \emph{interleaving operators} $P^{(d, N)} : \C^{dN} \to \C^{dN}$ for any $d, N \in \N$, each of which is a permutation defined by \begin{equation} (P^{(d, N)} v)_{(i - 1)N + j} = v_{(j - 1)d + i}.\label{eq:interleave_def}\end{equation}  We can view this is beginning with $v \in C^{dN}$ written as $N$ blocks of $d$ entries, and interleaving them into $d$ blocks each of $N$ entries.  Additionally, for $\ell, N_1, N_2 \in \N, v \in \C^{\ell N_1}, k \in [\ell],$ and $H \in \C^{\ell N_1 \times N_2}$, we define the block circulant operator $\circop^{N_1}$ by
\begin{align*}
  \circop_k^{N_1}(v) &= \begin{bmatrix} v & S^{N_1} v & \cdots & S^{(k - 1)N_1} v \end{bmatrix} \\
  \circop_k^{N_1}(H) &= \begin{bmatrix} H & S^{N_1} H & \cdots & S^{(k - 1) N_1}H \end{bmatrix},
\end{align*}
where, as with $\circop(\cdot)$, when we omit the subscript we define $\circop^{N_1}(H) = \circop_\ell^{N_1}(H)$ and $\circop^{N_1}(v) = \circop_\ell^{N_1}(v)$.  We now proceed with the following lemmas; the first establishes the inverse of $P^{(d, N)}$.

\begin{lemma}
  For $d, N \in \N,$ we have \[(P^{(d, N)})^{-1} = P^{(d, N) *} = P^{(N, d)}.\]
  \label{lem:interleave_inverse}
\end{lemma}

\begin{proof}[Proof of \cref{lem:interleave_inverse}]
  To prove the statement, we simply take $v \in \C^{d N}$ and calculate, for $i \in [d], j \in [N]$,
  \begin{align*}
    (P^{(d, N)} P^{(N, d)} v)_{(i - 1) N + j} &= (P^{(d, N)} (P^{(N, d)} v))_{(i - 1) N + j} \\
    &= (P^{(N, d)} v)_{(j - 1) d + i} \\
    &= v_{(i - 1) N + j},
  \end{align*}
  with these equalities coming from the definition in \eqref{eq:interleave_def}.  
\end{proof}

We now observe some useful ways in which the interleaving operators commute with the construction of circulant matrices.

\begin{lemma}
  Suppose $v_i, v_{ij} \in \C^k, w_j \in \C^{k N_1}$ for $i \in [N_1], j \in [N_2]$ and
  \begin{gather*}
    M_1 = \begin{bmatrix} \circop(v_1) \\ \vdots \\ \circop(v_{N_1}) \end{bmatrix},\quad
    M_2 = \begin{bmatrix} \circop^{N_1}(w_1) & \cdots & \circop^{N_1}(w_{N_2}) \end{bmatrix},\ \text{and} \\
    M_3 = \begin{bmatrix} \circop(v_{11}) & \cdots & \circop(v_{1 N_2}) \\ \vdots & \ddots & \vdots \\ \circop(v_{N_1 1}) & \cdots & \circop(v_{N_1 N_2}) \end{bmatrix}.\end{gather*}
  Then
  \begin{align}
    P^{(k, N_1)} M_1 &= \circop^{N_1}\left(P^{(k, N_1)} \begin{bmatrix} v_1 \\ \vdots \\ v_{N_1} \end{bmatrix}\right) \label{eq:M_1} \\
    M_2 P^{(k, N_2)*} &= \circop^{N_1}\left(\begin{bmatrix} w_1 & \cdots & w_{N_2} \end{bmatrix}\right) \label{eq:M_2} \\
    P^{(k, N_1)}M_3P^{(k, N_2)*} &= \circop^{N_1}\left(P^{(k, N_1)} \begin{bmatrix} v_{11} & \cdots & v_{1 N_2} \\ \vdots & \ddots & \vdots \\ v_{N_1 1} & \cdots & v_{N_1 N_2} \end{bmatrix}\right). \label{eq:M_3}
  \end{align}
  \label{lem:interleave}
\end{lemma}

\begin{proof}[Proof of lemma \ref{lem:interleave}]
  We index the matrices to check the equalities.  For \eqref{eq:M_1}, we have \begin{align*}
    (P^{(k, N_1)} M_1)_{(a-1)N_1 + b, j} &= (M_1)_{(b - 1) k + a, j} \\ &= \begin{bmatrix} S^{j - 1}v_1 \\ \vdots \\ S^{j - 1} v_{N_1} \end{bmatrix}_{(b - 1)k + a} \\ &= (S^{j - 1}v_b)_a = (v_b)_{a + j - 1}
  \end{align*}
  and
  \begin{align*}
    \circop^{N_1}\left(P^{(k, N_1)} \begin{bmatrix} v_1 \\ \vdots \\ v_{N_1} \end{bmatrix}\right)_{(a-1)N_1 + b, j} &= \left(P^{(k, N_1)} \begin{bmatrix} v_1 \\ \vdots \\ v_{N_1} \end{bmatrix}\right)_{(a - 1)N_1 + b + (j-1)N_1} \\
    &= %\begin{bmatrix} v_1 \\ \vdots \\ v_{N_1} \end{bmatrix}_{(b-1)k + a + j - 1} = 
    (v_b)_{a + j - 1}
  \end{align*}
  For \eqref{eq:M_2}, we have
%  \begin{align*}
 \[(P^{(k, N_2)} M_2^*)_{(a - 1)N_2 + b, j} = (M_2)_{j, (b - 1) k + a} = (w_b)_{j + (a - 1)N_1}\]
%  \end{align*}
  and
  \[\left(\circop^{N_1}\left(\begin{bmatrix} w_1 & \cdots & w_{N_2} \end{bmatrix}\right)\right)_{j, (a - 1)N_2 + b} = (S^{N_1(a - 1)} w_b)_j = (w_b)_{j + N_1(a - 1)}\]

  \eqref{eq:M_3} follows immediately by combining \eqref{eq:M_1} and \eqref{eq:M_2}.
\end{proof}

\Cref{lem:interkron} introduces a few useful identities relating the interleaving operators to kronecker products.

\begin{lemma}
  For $v \in \C^N, V = \rowmat{V}{1}{\ell} \in \C^{N \times \ell}, A = \rowmat{A}{1}{m} \in \C^{d \times m}$, and $B_i \in \C^{m \times k}, i \in [\ell]$, we have
  \begin{align}
    P^{(d, N)} (v \kron A) &= A \kron v
    \label{eq:interkron_vec} \\
    P^{(d, N)} (V \kron A) &= \rowmat{A \kron V}{1}{\ell}
    \label{eq:interkron_mat} \\
    P^{(d, N)} (V \kron A) P^{(\ell, m)} &= A \kron V
    \label{eq:interkron_swap} \\
    (V \kron A) \diagmat{B}{1}{\ell} &= \rowmatfun{V_@ \kron A B_@}{1}{\ell}
    \label{eq:kron_diag}
  \end{align}
  \label{lem:interkron}
\end{lemma}

\begin{proof}[Proof of \cref{lem:interkron}]
  For \eqref{eq:interkron_vec}, we see that, for $i, j, k \in [d] \times [N] \times [m]$, we have
  \begin{align*}
    (P^{(d, N)} v \otimes A)_{(i - 1) N + j, k} &= (v \otimes A)_{(j - 1) d + i, k} \\
    &= v_j A_{i k}, \ \text{while} \\
    (A \kron v)_{(i - 1) N + j, k} &= A_{i k} v_j,
  \end{align*}
  and \eqref{eq:interkron_mat} follows by considering that $V \otimes A = \rowmatfun{V_@ \kron A}{1}{\ell}.$  To get \eqref{eq:interkron_swap}, we trace the positions of columns, considering that $(V \kron A) e_{(i - 1) m + j} = V_j \kron A_i$.  From \eqref{eq:interkron_mat}, we observe that $P^{(d, N)} (V \kron A) e_{(i - 1) m + j} = A_j \kron V_i,$ so \begin{align*}
    P^{(d, N)} (V \kron A) P^{(m, \ell)} e_{(j - 1) \ell + i} &= P^{(d, N)} (V \kron A) e_{(i - 1) m + j} \\
    &= A_j \kron V_i = (A \kron V) e_{(j - 1) \ell + i}.
  \end{align*}
 As for \eqref{eq:kron_diag}, we remark that \begin{gather*} (V \kron A) \diagmat{B}{1}{\ell}%% \begin{bmatrix} B_1 & 0 & \cdots & 0 \\ 0 & \ddots &  & \vdots \\ \vdots & & \ddots & 0 \\ 0 & \cdots & 0 & B_\ell \end{bmatrix}
    = (V \kron A) \rowmatfun{e^\ell_@ \kron B_@}{1}{\ell}  \\ = \rowmatfun{(V \kron A) (e_@^\ell \kron B_@)}{1}{\ell} = \rowmatfun{V_@ \kron A B_@}{1}{\ell},\end{gather*} as desired.
\end{proof}

The following lemma is a standard result (e.g., Theorem 13.26 in \cite{laub2004matrix}) regarding the kronecker product.

\begin{lemma}
  We have $\vec(A B C) = (C^T \kron A) \vec(B)$ for any $A \in \C^{m \times n}, B \in \C^{n \times p}, C \in \C^{p \times k}.$
  \label{lem:kronvec}
\end{lemma}
The next lemma generalizes the diagonalization of circulant matrices, stated in \eqref{eq:circ_dft_diag}, for block circulant matrices.

\begin{lemma}
  Suppose $V \in C^{k N \times m}$, then $\circop^N(V)$ is block diagonalizable by \[\circop^N(V) = \left(F_k \otimes I_N\right) \left(\diag(M_1, \ldots, M_k)\right) \left(F_k \otimes I_m\right)^*,\] where \[\sqrt{k}\left(F_k \otimes I_N\right)^* V = \begin{bmatrix} M_1 \\ \vdots \\ M_k \end{bmatrix}, \quad \text{or} \quad M_j = \sqrt{k} (f_j^k \otimes I_N)^* V\] \label{lem:circ_diag}
\end{lemma}

\begin{proof}[Proof of lemma \ref{lem:circ_diag}]
  We set $V_i$ to be the $k \times m$ blocks of $V$ such that $V^* = \rowmat{V^*}{1}{k}$ and begin by observing that, for $u \in \C^k$ and $W \in \C^{m \times p}$, the $\ell\th$ $k \times p$ block of $\circop^N(V)(u \otimes W)$ is given by \[\left(\circop^N(V)(u \otimes W)\right)_\ell = \sum_{i = 1}^k u_i (S^{N (i - 1)}V)_\ell W = \sum_{i = 1}^k u_i V_{\ell - i + 1} W.\]  Taking $u = f_j^k$ and $W = I_m$, this gives \begin{align*} \left(\circop^N(V)(f_j^k \otimes I_m)\right)_\ell &= \frac{1}{\sqrt{k}}\sum_{i = 1}^k \omega_k^{(j - 1) (i - 1)} V_{\ell - i + 1} I_m \\ &= \frac{1}{\sqrt{k}} \omega_k^{(j - 1) (\ell - 1)} \sum_{i = 1}^k \omega_k^{-(j - 1)(i - 1)} V_i \\ &= (f_j^k)_\ell \left(\sqrt{k} (f_j^k \otimes I_N)^* V \right) = (f_j^k)_\ell M_j. \end{align*}  This relation is equivalent to having \[\circop^N(V) (f_j^k \otimes I_m) = (f_j^k \otimes M_j) = (f_j^k \otimes I_N) M_j,\] which is the statement of the lemma.
\end{proof}

Lemma \ref{lem:circ_diag} immediately gives the following corollary regarding the conditioning of $\circop^N(V)$, with which we return to considering spanning families of masks.

\begin{corollary}
  With notation as in lemma \ref{lem:circ_diag}, the condition number of $\circop^N(V)$ is \[\dfrac{\max\limits_{i \in [k]} \sigma_{\max} (M_i)}{\min\limits_{i \in [k]} \sigma_{\min} (M_i)}.\] \label{cor:circ_diag_condition}
\end{corollary}

\subsection{Proof of \cref{prop:span_fam_cond}}

To begin the proof of \cref{prop:span_fam_cond}, we apply the results of \cref{sec:interlemma} to the case of the measurement operator for a family of masks, as defined in \cref{sec:span_fam}.

\begin{proposition}
  Given a family of masks $\{m_j\}_{j \in [D]}$ of support $\delta \le \frac{d + 1}{2}$, we define $g_m^j = \diag(m_j m_j^*, m),$ \[H = P^{(d, D)} \cornmatfun{R g^@r_@c}{1}{D}{1 - \delta}{\delta - 1}%% \begin{bmatrix} R g_{1 - \delta}^1 & \cdots & R g_{\delta - 1}^1 \\ \vdots & \ddots & \vdots \\ R g_{1 - \delta}^D & \cdots & R g_{\delta - 1}^D \end{bmatrix}
  ,\] and $M_j = \sqrt{d}\left(f_j^d \otimes I_D\right)^* H$.  Then the condition number of $\mathcal{A}$ as defined in \eqref{eq:meas_op} is \[\kappa(\mathcal{A}) = \dfrac{\max\limits_{i \in [d]} \sigma_{\max} (M_i)}{\min\limits_{i \in [d]} \sigma_{\min} (M_i)}.\] \label{prop:meas_cond}
\end{proposition}

\begin{proof}
We consider the rows of the measurement operator $\mathcal{A}$ defined in \eqref{eq:meas_op}.  We vectorize $X \in T_\delta(\C^{d \times d})$ by its diagonals, taking $\chi_m = \diag(X, m), m \in [2 \delta - 1]_{1 - \delta}$.  Each measurement then looks like \begin{align*} \mathcal{A}(X)_{(\ell, j)} &= \langle S^{\ell} m_j m_j^* S^{-\ell}, X \rangle \\ &= \sum_{m = 1 - \delta}^{\delta - 1} \langle S^{\ell} g_m^j, \chi_m \rangle,\end{align*} so if we define the matrix $A \in \C^{dD \times (2 \delta - 1)d}$ such that \begin{equation}\left(A \begin{bmatrix} \chi_{1 - \delta} \\ \vdots \\ \chi_{\delta - 1} \end{bmatrix}\right)_{(j-1) d + \ell} = \mathcal{A}(X)_{(\ell, j)}, \label{eq:vectorized_meas}\end{equation} the $(j - 1) d + \ell\th$ row of $A$ is given by \[\begin{bmatrix} S^{\ell - 1} g_{1 - \delta}^j \\ \vdots \\ S^{\ell - 1} g_{\delta - 1}^j \end{bmatrix}^*.\] By obvserving that $\circop(v)^* = \circop(Rv)$, we see that $A$ is the block matrix given by \[A = \begin{bmatrix} \circop(g_{1 - \delta}^1) & \cdots & \circop(g_{1 - \delta}^D) \\ \vdots & \ddots & \vdots \\ \circop(g_{\delta - 1}^1) & \cdots & \circop(g_{\delta - 1}^D) \end{bmatrix}^* = \begin{bmatrix} \circop(R g_{1 - \delta}^1) & \cdots & \circop(R g_{\delta - 1}^1) \\ \vdots & \ddots & \vdots \\ \circop(R g_{1 - \delta}^D) & \cdots & \circop(R g_{\delta - 1}^D) \end{bmatrix}\] which may be transformed, by Lemma \ref{lem:interleave}, to \begin{equation}P^{(d, D)} A P^{(d, 2 \delta - 1)*} = \circop^D\left(P^{(d, D)} \begin{bmatrix} R g_{1 - \delta}^1 & \cdots & R g_{\delta - 1}^1 \\ \vdots & \ddots & \vdots \\ R g_{1 - \delta}^D & \cdots & R g_{\delta - 1}^D \end{bmatrix}\right) = \circop^D(H). \label{eq:interleaved_meas}\end{equation}

%We label the $D \times 2\delta - 1$ blocks of $H$ by $H^* = \begin{bmatrix} H_1^* & \cdots & H_d^* \end{bmatrix}$, so that $(H_\ell)_{ij} = (R g_{j - \delta}^i)_\ell = (g_{j - \delta}^i)_{2 - \ell}$.
Quoting corollary \ref{cor:circ_diag_condition} establishes the proposition.
\end{proof}

We are now able to prove proposition \ref{prop:span_fam_cond}.

\begin{proof}[Proof of Proposition \ref{prop:span_fam_cond}]
  For the moment, we assert that $D = 2 \delta - 1 \le d$ and set $\tF_K \in \C^{2 \delta - 1 \times 2 \delta - 1}, (\tF_K)_{ij} = \frac{1}{\sqrt{K}}\omega_K^{(i-1)(j-\delta)}$ to be the principal submatrix of $\sqrt{K} \diag(f^K_{1 - \delta}) F_K$.  In this case, $g_m^j = \diag(m_j m_j^*, m) = \omega_K^{m(j - 1)} g_m$, as in \eqref{eq:gam_diag}.  Therefore, we label the $2 \delta - 1 \times 2 \delta - 1$ blocks of $H$ by $H^* = \begin{bmatrix} H_1^* & \cdots & H_d^* \end{bmatrix}$, so that \[(H_\ell)_{ij} = (R g_{j - \delta}^i)_\ell = \omega_K^{(i - 1)(j - \delta)}(R g_{j - \delta})_\ell\] and $M_\ell = \sum_{k = 1}^d \omega_d^{(\ell - 1)(k - 1)} H_k,$ giving \begin{align*} (M_\ell)_{ij} &= \sum_{k = 1}^d \omega_d^{(\ell - 1)(k - 1)} (H_k)_{ij} = \omega_K^{(i - 1)(j - \delta)} \sum_{k = 1}^d \omega_d^{(\ell - 1)(k - 1)} (Rg_{j - \delta})_k \\ &= \omega_K^{(i - 1)(j - \delta)} (F_d^* g_{j - \delta})_\ell. \end{align*}  In other words, \begin{equation} M_\ell = \sqrt{K} \diag(f_\ell^{d*} g_{1 - \delta},\, \ldots\, , f_\ell^{d*} g_{\delta - 1}) %% D_{R_{2 \delta - 1} (f^K_{1 - \delta})}
    \tF_K. \label{eq:block_diag_components} \end{equation}  If $K = 2 \delta - 1$, then $\tF_K$ is unitary, and the singular values of $M_\ell$ are $\{\sqrt{K} f_\ell^{d *} g_j\}_{j = 1 - \delta}^{\delta - 1}$.  Recognizing that $S^j g_j = g_{-j}$, then proposition \ref{prop:meas_cond} takes us to \eqref{eq:clean_cond}.

  If $D = 2 \delta - 1 < K$, then the argument remains unchanged, except that the singular values of $M_\ell$, instead of being known explicitly, are bounded above and below by $\max\limits_{|j| < \delta} |f_\ell^{d *} g_j| \sigma_{\max}(\tF_K)$ and $\min\limits_{|j| < \delta} |f_\ell^{d *} g_j| \sigma_{\min}(\tF_K)$ respectively, which gives the more general result of \eqref{eq:messy_cond}.

  If $2 \delta - 1 > d$, then instead of using diagonals $1 - \delta, \ldots, \delta - 1$, we use diagonals $0, 1, \ldots, d - 1$.  This change propagates from \eqref{eq:vectorized_meas} to \eqref{eq:interleaved_meas}, so that \[(H_\ell)_{ij} = \omega_K^{(i - 1)(j - 1)} (R g_{j - 1})_\ell \quad \text{and} \quad (M_\ell)_{ij} = \omega_K^{(i - 1)(j - 1)} (F_d^* g_{j - 1})_\ell,\] giving $M_\ell = \sqrt{K} \diag(f_\ell^{d*} g_0,\, \ldots\, , f_\ell^{d*} g_{d - 1})\mathcal{R}_{d \times d}(F_K)$, which immediately gives us \eqref{eq:messy_cond}.  We remark that indexing only over the diagonals $m \in [\delta]_0$ in \eqref{eq:messy_cond} suffices, again because $S^j g_j = g_{-j}$, so having $2 \delta - 1 > d$ makes $1 - \delta, \ldots, -1$ redundant.
  
\end{proof}

\label{sec:con_number}
%% \section{Generalization to ptychographic case}
%% In the case of ptychography, instead of using all shifts in our lifted measurement system, we instead fix a shift size $s \in \N$ where $d = \dbar s$ with $\dbar \in \N$ and use $S^{s \ell} m_j m_j^* S^{-s \ell}$ for $\ell \in [\dbar]$.  Therefore, we introduce the following generalization of the lifted measurement system.

\begin{definition}
  Given a family of masks in $\{m_j\}_{j \in [D]} \subset \C^d$ and $s, \dbar \in \N$ with $\dbar = d / s$, the associated \emph{lifted measurement system of shift $s$} is the set $\mathcal{L}_{\{m_j\}}^s := \{S^{s \ell} m_j m_j^* S^{-s \ell}\}_{(\ell, j) \in [\dbar] \times [D]} \subset \C^{d \times d}.$
\end{definition}

Of course, with a shift size $s > 1$, it is impossible for $\mathcal{L}^s$ to span $T_\delta(\C^{d \times d})$, so we consider the analagous subspace.  We will define $\mathcal{J}_{\delta, s} = \bigcup_{\ell \in [\dbar]_0}\supp(S^{s \ell} \one \one^* S^{-s \ell})$ to be the set of indices ``reached'' by this system, and \[T_\delta^s(X) = \left\{\begin{array}{r@{,\quad}l} X_{ij} & (i, j) \in \mathcal{J}_{\delta, s} \\ 0 & \text{otherwise}\end{array}\right.\] to be the projection onto the associated subspace of $\C^{d \times d}$.  Namely, we observe that \[\left(S^{s \ell} m_k m_k^* S^{-s \ell}\right)_{ij} = (S^{s \ell} m_k)_i (\overline{S^{s \ell} m_k})_j = (m_k)_{i - s \ell} (\overline{m_k})_{j - s\ell},\] so $\left(S^{s \ell} m_k m_k^* S^{-s \ell}\right)_{ij} = 0$ when $(i - s \ell, j - s \ell) \notin [\delta]^2$, i.e.~when $(i, j) \notin [\delta]^2_{s \ell + 1}$.  Hence the indices onto which we are projecting are those in $\bigcup_{\ell \in [\dbar]_0} [\delta]_{s \ell + 1}^2$.  This set may be revisualized by calculating which $j$'s are admissible for each $i$; for a fixed $i$, we look at all shifts $\ell$ such that $i \in [\delta]_{s\ell + 1}$, and $j$ is allowed to be in their union.  

In the (pathological) case where $s \ge \delta$, obviously any given index can only appear in one of the $[\delta]_{s \ell + 1}$, namely $i \in [\delta]_{s \ell + 1}$ iff $i \mod s \le \delta$ and $\floor{i / s} = \ell$, so in this case we would have \[\mathcal{J}_{\delta, s} = \{(i, j) : \floor{i / s} = \floor{j / s} \ \text{and} \ i \mathbin{\mathrm{mod}} s, j \mod s \le \delta\}.\]  However, this case is not typical, since $T_{\delta, s}(\one \one^*)$ will be the adjacency matrix of an unconnected graph, and the phase synchronization of section \ref{sec:phase_synch} will fail, as the graph Laplacian \eqref{eq:graph_laplace} will be singular.  In the ordinary case, where $s < \delta$, it is clear that we need only consider the first and last shifts that cover $i$, namely $i \in [\delta]_{1 + s \ell}$ iff $\ceil{\frac{i - \delta}{s}} \le \ell \le \ceil{\frac{i - s}{s}}$, and therefore
\begin{gather*}
  \mathcal{J}_{\delta, s} = \left\{(i, j) : j \in \{\ceil*{\frac{i - \delta}{s}}s + 1, \cdots, \ceil*{\frac{i - \delta}{s}} + \delta\}\right. \\
  \left. \cup\, \{\ceil*{\frac{i - s}{s}}s + 1, \cdots, \ceil*{\frac{i - s}{s}}s + \delta\} \right\} \\
  = \left\{(i, j) : j = \ceil*{\frac{i - \delta}{s}}s + 1, \ldots, \ceil*{\frac{i - s}{s}}s + \delta \right\}
\end{gather*}
Unfortunately, this formulation is not particularly transparent, but we mention an important special case.  When $s$ is also a divisor of $\delta$, say $\delta = \deltabar s$, then this condition becomes
\begin{gather*}
  (i, j) \in \mathcal{J}_{\delta, s} \iff \left(\ceil*{\frac{i}{s}} - \deltabar\right) s + 1 \le j \le \left(\ceil*{\frac{i}{s}} - 1\right) s + \delta \\
  \iff \frac{1}{s} - \deltabar \le  \frac{j}{s} - \ceil*{\frac{i}{s}} \le \deltabar - 1 \\
%  \iff 1 - \deltabar \le \ceil*{\frac{j}{s}} - \ceil*{\frac{i}{s}} \le \deltabar - 1 \\
  \iff \left| \ceil*{\frac{j}{s}} - \ceil*{\frac{i}{s}} \right| < \deltabar.
\end{gather*}

Before addressing invertibility and conditioning of lifted measurement systems with shifts, for $N \in \N$, we introduce $\mathcal{T}_N : \bigcup_{\ell \in \N} \C^{\ell N \times m} \to \bigcup_{\ell \in \N} \C^{\ell m \times N},$ the blockwise transpose operator, defined by \[\mathcal{T}_N\left(\begin{bmatrix} V_1 \\ \vdots \\ V_\ell \end{bmatrix}\right) = \begin{bmatrix} V_1^* \\ \vdots \\ V_{\ell}^* \end{bmatrix}\] for $V_1, \ldots, V_\ell \in \C^{N \times m}$.  We also define, for $\{k_j\}_{j = 1}^n$ and $V \in \C^{m \times n}$, \[\mathcal{I}(V, (k_j)_{j = 1}^n) = \begin{bmatrix} v_1 \otimes I_{k_1} & \cdots & v_n \otimes I_{k_n} \end{bmatrix},\]  where $v_j = V e_j$ are the columns of $V$.  This prepares us to prove the following lemmas.

\begin{lemma}
  Given $k, N, m \in \N$ and $V \in \C^{kN \times M}$, we have \[\circop^N(V)^* = \circop^m\left( (R_k \otimes I_m) \mathcal{T}_N(V) \right).\] \label{lem:circ_transpose}
\end{lemma}

\begin{proof}
  Suppose $V_i$ are the $N \times m$ blocks of $V$, such that $V = \left[V_1^T \cdots V_k^T\right]^T$.  Indexing blockwise, we have $\circop^N(V)_{[ij]} = V_{i - j + 1}$, so that $\circop^N(V)^*_{[ij]} = V_{j - i + 1}^*$.  In other words,
  \[
  \circop^N(V)^* = \begin{bmatrix} V_1^* & V_2^* & \cdots & V_N^* \\ V_N^* & V_1^* & \cdots & V_{N - 1}^* \\ \vdots & & \ddots & \vdots \\ V_2^* & V_3^* & \cdots & V_1^* \end{bmatrix} = \circop^m((R_k \otimes I_m) \mathcal{T}_N(V) )
  %\begin{bmatrix} V_1 & V_N & \cdots & V_2 \\ V_2 & V_1 & \cdots & V_3 \\ \vdots & & \ddots & \vdots \\ V_N & V_{N - 1} & \cdots & V_1 \end{bmatrix}^* \\
  \]

  as claimed.
  
\end{proof}

\begin{lemma}
  Given $N_1, N_2, k, m \in \N$ and $V_i \in \C^{k N_1 \times m}$ for $i \in [N_2]$, we have \[\begin{bmatrix} \circop^{N_1}(V_1) & \cdots & \circop^{N_1}(V_{N_2}) \end{bmatrix} (P^{(k, N_2)} \otimes I_m)^* = \circop^{N_1}(\begin{bmatrix} V_1 & \cdots & V_{N_2}\end{bmatrix}).\] \label{lem:block_circ_right}
\end{lemma}

\begin{proof}
  We quote \eqref{eq:M_2} from lemma \ref{lem:interleave} and consider that $P^{(k, N_2)} \otimes I_m$ is a permutation that changes the blockwise indices of $m \times p$ blocks (or, acting from the right, $p \times m$ blocks) exactly the way that $P^{(k, N_2)}$ changes the indices of a vector.
\end{proof}

\begin{lemma}
  Given $k, n \in \N$ and $V_j \in \C^{m_j \times n_j}$, we have \[\diag(I_k \otimes V_j)_{j = 1}^n =  P_1 (I_k \otimes \diag(V_j)_{j = 1}^n) P_2^*\] where $P_1 = \mathcal{I}(P^{(n, k)}, (m_j)_{j = 1}^n)$ and $P_2 = \mathcal{I}(P^{(n, k)}, (n_j)_{j = 1}^n)$. \label{lem:diag_kron_perm}
\end{lemma}

\begin{proof}
  We immediately reduce to the case $m_j = n_j = 1$ (and we replace $V$ with $v \in \C^n$) for all $j$ by observing that $P_1$ and $P_2$ will act on blockwise indices precisely as $P^{(n, k)}$ acts on individual indices.  Hence, we need only remark that \[\left(\diag(I_k \otimes v_\ell)_{\ell = 1}^n\right)_{((i_1 - 1)k + i_2) ((j_1 - 1)k + j_2)} = \left\{\begin{array}{r@{,\quad}l} v_{i_1} & i_1 = j_1 \ \text{and} \ i_2 = j_2 \\ 0 & \text{otherwise} \end{array}\right.,\] while \begin{align*} (P^{(n, k)} (&I_k \otimes \diag(v)) P^{(n, k)*})_{((i_1 - 1)k + i_2) ((j_1 - 1)k + j_2)} \\ &= (I_k \otimes \diag(v))_{((i_2 - 1)n + i_1) ((j_2 - 1)n + j_1)} \\ &= \left\{\begin{array}{r@{,\quad}l} v_{i_1} & i_1 = j_1 \ \text{and} \ i_2 = j_2 \\ 0 & \text{otherwise} \end{array}\right..\end{align*}
\end{proof}

For the remainder of this section, we assume that $\delta > s$.  We now consider the question of when $\Span \mathcal{L}_{\{m_j\}}^s = T_{\delta, s}$ and what the condition number of $\mathcal{A}$ will be; naturally, this requires us to have redefined $\mathcal{A}$ by \[\mathcal{A}(X)_{(\ell, j)} = \langle S^{s \ell} m_j m_j^* S^{-s \ell}, X \rangle, \quad (\ell, j) \in [\dbar]_0 \times [D].\]  As in \eqref{eq:vectorized_meas}, we vectorize $X$ by its diagonals and write $A \in \C^{\dbar D \times (2 \delta - 1) d}$ such that \begin{equation*}\left(A \begin{bmatrix} \chi_{1 - \delta} \\ \vdots \\ \chi_{\delta - 1} \end{bmatrix}\right)_{(j-1) \dbar + \ell} = \mathcal{A}(X)_{(\ell, j)}, %\label{eq:vectorized_meas_ptych}
\end{equation*} which gives the $(j - 1) \dbar + \ell\th$ row of $A$ as \[\begin{bmatrix} S^{s (\ell - 1)} g_{1 - \delta}^j \\ \vdots \\ S^{s (\ell - 1)} g_{\delta - 1}^j \end{bmatrix}^*\] so that, by lemma \ref{lem:circ_transpose}, we have \begin{equation} A = \begin{bmatrix} \circop^s(g_{1 - \delta}^1) & \cdots & \circop^s(g_{1 - \delta}^D) \\ \vdots & \ddots & \vdots \\ \circop^s(g_{\delta - 1}^1) & \cdots & \circop^s(g_{\delta - 1}^D) \end{bmatrix}^* = \begin{bmatrix} \circop(R_{\dbar} \mathcal{T}_s g_{1 - \delta}^1) & \cdots & \circop(R_{\dbar} \mathcal{T}_s g_{\delta - 1}^1) \\ \vdots & \ddots & \vdots \\ \circop(R_{\dbar} \mathcal{T}_s g_{1 - \delta}^D) & \cdots & \circop(R_{\dbar} \mathcal{T}_s g_{\delta - 1}^D) \end{bmatrix}. \label{eq:A_block_ptych} \end{equation}  However, because $T_{\delta, s} \subsetneq T_\delta$, this operator can never be invertible.  Figure \ref{fig:T_delta_s} shows this visually.  Indeed, we consider that (restricting to $m \ge 0$), even if $\supp(m_k) = [\delta]$ for all $k$, $\supp(g^k_m) = [\delta - m],$ so when $\delta - m < s, \bigcup_{\ell = 1}^{\dbar} \supp(S^{s (\ell - 1)} g^k_m) \subsetneq [d]$.  In particular, $\circop^s(g^k_m)_{ij} = 0$ for all $j$ when $i \mod_1 s > \delta - m$.  By a similar argument, for $m < 0$ we have $\circop^s(g_m^k)_{ij} = 0$ when $i \mod_1 s < s - (\delta - |m|)$.  We remark that these inequalities can only be satisfied when $m > \delta - s$ or $m > \delta - s$, respectively.

\begin{figure}
  \centering
  \begin{subfigure}[b]{0.4\textwidth}    
    \begin{tikzpicture}[ampersand replacement=\&,baseline=-\the\dimexpr\fontdimen22\textfont2\relax]
    \matrix (m)[matrix of math nodes,left delimiter=(,right delimiter=)]
            {
              * \& * \& * \&   \&   \&   \& * \& * \\
              * \& * \& * \& * \&   \&   \&   \& * \\
              * \& * \& * \& * \& * \&   \&   \&   \\
                \& * \& * \& * \& * \& * \&   \&   \\
                \&   \& * \& * \& * \& * \& * \&   \\
                \&   \&   \& * \& * \& * \& * \& * \\
              * \&   \&   \&   \& * \& * \& * \& * \\
              * \& * \&   \&   \&   \& * \& * \& * \\
            };

            \begin{pgfonlayer}{myback}
              \fhighlight[blue!30]{m-1-1}{m-3-3}
              \fhighlight[blue!30]{m-2-2}{m-4-4}
              \fhighlight[blue!30]{m-3-3}{m-5-5}
              \fhighlight[blue!30]{m-4-4}{m-6-6}
              \fhighlight[blue!30]{m-5-5}{m-7-7}
              \fhighlight[blue!30]{m-6-6}{m-8-8}
              \fhighlight[blue!30]{m-7-1}{m-8-1}
              \fhighlight[blue!30]{m-1-7}{m-1-8}
              \fhighlight[blue!30]{m-8-8}{m-8-8}
              \fhighlight[blue!30]{m-8-1}{m-8-2}
              \fhighlight[blue!30]{m-1-8}{m-2-8}
              \fhighlight[blue!30]{m-1-1}{m-2-2}
            \end{pgfonlayer}
  \end{tikzpicture}
    \caption{$T_3(\C^{8 \times 8})$}
  \end{subfigure}
  \begin{subfigure}[b]{0.4\textwidth}
    \begin{tikzpicture}[ampersand replacement=\&,baseline=-\the\dimexpr\fontdimen22\textfont2\relax]
    \matrix (m)[matrix of math nodes,left delimiter=(,right delimiter=)]
            {
              * \& * \& * \&   \&   \&   \& * \& * \\
              * \& * \& * \& * \&   \&   \&   \& * \\
              * \& * \& * \& * \& * \&   \&   \&   \\
                \& * \& * \& * \& * \& * \&   \&   \\
                \&   \& * \& * \& * \& * \& * \&   \\
                \&   \&   \& * \& * \& * \& * \& * \\
              * \&   \&   \&   \& * \& * \& * \& * \\
              * \& * \&   \&   \&   \& * \& * \& * \\
            };

            \begin{pgfonlayer}{myback}
              \fhighlight[blue!30]{m-1-1}{m-3-3}
              %\fhighlight[blue!30]{m-2-2}{m-4-4}
              \fhighlight[blue!30]{m-3-3}{m-5-5}
              %\fhighlight[blue!30]{m-4-4}{m-6-6}
              \fhighlight[blue!30]{m-5-5}{m-7-7}
              %\fhighlight[blue!30]{m-6-6}{m-8-8}
              \fhighlight[blue!30]{m-7-1}{m-8-1}
              \fhighlight[blue!30]{m-1-7}{m-1-8}
              \fhighlight[blue!30]{m-7-7}{m-8-8}
              %\fhighlight[blue!30]{m-8-8}{m-8-8}
              %\fhighlight[blue!30]{m-8-1}{m-8-2}
              %\fhighlight[blue!30]{m-1-8}{m-2-8}
              %\fhighlight[blue!30]{m-1-1}{m-2-2}
            \end{pgfonlayer}
    \end{tikzpicture}
    \caption{$T_{3, 2}(\C^{8 \times 8})$}
  \end{subfigure}
  \caption{$T_\delta(\C^{d \times d})$ vs. $T_{\delta, s}(\C^{d \times d})$ for $d = 8, \delta = 3, s = 2$}
  \label{fig:T_delta_s}  
\end{figure}

By reference to \eqref{eq:A_block_ptych}, it is clear that each of these ``missing indices'' results in a column of all zeros in $A$; specifically, viewing $A e_{(m + \delta - 1)d + i}, (m, i) \in [2 \delta - 1]_{1 - \delta} \times [d]$ as the $i\th$ column of the $m + \delta\th$ block of $A$, we see \[A e_{(m + \delta - 1) d + i} = 0\quad \text{if} \quad \left\{\begin{array}{rcl@{\,,\quad}l} i \mod_1 s & > & \delta - m & m \ge 0 \\ i \mod_1 s & < & s - (\delta + m) & m \le 0 \end{array}\right..\]  Since $\delta > s$, we may reduce this condition to ``$i \mod s > \delta - m$ or $i \mod s < s - (\delta + m)$,'' or further to $i \mod s \notin [2 \delta - s + 1]_{s - \delta - m}$.  Therefore, the matrix representing $\mathcal{A}$ restricted to $T_{\delta, s}(\C^{d \times d})$ is found by right multiplying $A$ by \begin{equation}N = \diag(I_{\dbar} \otimes N_{j - \delta})_{j = 1}^{2 \delta - 1},\ \text{where} \ N_m = \left\{\begin{array}{c@{,\quad}l} \begin{bmatrix} 0_{\delta + m} \\ I_{s - (\delta + m)} \end{bmatrix} & m < s - \delta \vspace{2pt}\\ \begin{bmatrix} I_{s - (\delta - m)} \\ 0_{\delta - m} \end{bmatrix} & m > \delta - s \\ I_s & \text{otherwise} \end{array}\right.. \label{eq:N_and_N_m}\end{equation}

But does this restriction commute well with the permutations used in the condition number analysis of section \ref{sec:con_number}?  Thankfully it does; following the intuition of \eqref{eq:interleaved_meas} and making use of lemma \ref{lem:block_circ_right}, we can arrive at \begin{align*} A' := P^{(\dbar, D)} A \left(P^{(\dbar, 2 \delta - 1)} \otimes I_s\right)^* &= \circop^D\left(P^{(\dbar, D)} \begin{bmatrix} R_{\dbar} \mathcal{T}_s g_{1 - \delta}^1 & \cdots & R_{\dbar} \mathcal{T}_s g_{\delta - 1}^1 \\ \vdots & \ddots & \vdots \\ R_{\dbar} \mathcal{T}_s g_{1 - \delta}^D & \cdots & R_{\dbar} \mathcal{T}_s g_{\delta - 1}^D \end{bmatrix}\right). \\ &= \circop^D\left(P^{(\dbar, D)} (I_D \otimes R_{\dbar}) \begin{bmatrix}  \mathcal{T}_s g_{1 - \delta}^1 & \cdots &  \mathcal{T}_s g_{\delta - 1}^1 \\ \vdots & \ddots & \vdots \\  \mathcal{T}_s g_{1 - \delta}^D & \cdots &  \mathcal{T}_s g_{\delta - 1}^D \end{bmatrix}\right). \end{align*}  In the interest of finding the locations of the zero columns after this permutation, we remark that the inner matrix is of size $\dbar D \times s (2 \delta - 1)$, and that the $\circop^D$ operation will therefore repeat it $\dbar$ times.  It is then clear that \begin{gather*} A'e_{(\ell - 1)s (2 \delta - 1) + i} = 0 \iff \begin{bmatrix}  \mathcal{T}_s g_{1 - \delta}^1 & \cdots &  \mathcal{T}_s g_{\delta - 1}^1 \\ \vdots & \ddots & \vdots \\  \mathcal{T}_s g_{1 - \delta}^D & \cdots &  \mathcal{T}_s g_{\delta - 1}^D \end{bmatrix} e_i = 0,\ \text{and} \\ \begin{bmatrix}  \mathcal{T}_s g_{1 - \delta}^1 & \cdots &  \mathcal{T}_s g_{\delta - 1}^1 \\ \vdots & \ddots & \vdots \\  \mathcal{T}_s g_{1 - \delta}^D & \cdots &  \mathcal{T}_s g_{\delta - 1}^D \end{bmatrix} e_{(m + \delta - 1)s + i} = 0 \iff i \notin [2 \delta - s + 1]_{s - \delta - m},\end{gather*} so we may remove the zero columns from $A'$ by right multiplying the interior matrix by $N' = \diag(N_m)_{m = 1 - \delta}^{\delta - 1}$.  That is, \begin{align} A' (I_{\dbar} \otimes N') &= \circop^D\left(P^{(\dbar, D)} (I_D \otimes R_{\dbar}) \begin{bmatrix}  \mathcal{T}_s g_{1 - \delta}^1 & \cdots &  \mathcal{T}_s g_{\delta - 1}^1 \\ \vdots & \ddots & \vdots \\  \mathcal{T}_s g_{1 - \delta}^D & \cdots &  \mathcal{T}_s g_{\delta - 1}^D \end{bmatrix} \begin{bmatrix} N_{1 - \delta} & & \\ & \ddots & \\ & & N_{\delta - 1} \end{bmatrix}\right) \nonumber\\ & = P^{(\dbar, D)} A N P', \label{eq:ptych_permute_final}\end{align} where the second equality comes from lemma \ref{lem:diag_kron_perm} and \[P' = \mathcal{I}(P^{(\dbar, 2 \delta - 1)}, (\min\{s, \delta - |m|\})_{m = 1 - \delta}^{\delta - 1}).\]  This result, along with corollary \ref{cor:circ_diag_condition}, gives us the following proposition.

\begin{proposition}
  Taking $A$ as in \eqref{eq:A_block_ptych}, $N$ and $N_m$ as in \eqref{eq:N_and_N_m}, and setting \[H = P^{(\dbar, D)} (I_D \otimes R_{\dbar}) \begin{bmatrix}  \mathcal{T}_s g_{1 - \delta}^1 & \cdots &  \mathcal{T}_s g_{\delta - 1}^1 \\ \vdots & \ddots & \vdots \\  \mathcal{T}_s g_{1 - \delta}^D & \cdots &  \mathcal{T}_s g_{\delta - 1}^D \end{bmatrix} \diag(N_m)_{m = 1 - \delta}^{\delta - 1}\] and $M_j = \sqrt{\dbar}(f_j^{\dbar} \otimes I_D)^* H$ for $j \in [\dbar]$, the condition number of $AN$ is given by \[\dfrac{\max\limits_{i \in [\dbar]} \sigma_{\max} (M_i)}{\min\limits_{i \in [\dbar]} \sigma_{\min} (M_i)}.\]  In particular, $\left.\mathcal{A}\right|_{T_{\delta, s}(\C^{d \times d})}$ is invertible if and only if each of the $M_i$ are of full rank.
\end{proposition}

%% \label{sec:con_number_ptych}
