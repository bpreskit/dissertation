In this section, we consider the problem of angular synchronization, which appears as a subproblem in many approaches to phase retrieval, including ours.  In particular, we make a study of it in this section in order to improve the results in \cref{sec:Perturb,sec:RecovGuarantee} and to supply a crucial lemma for \cref{sec:ptych_recovery}.

Angular synchronization is the problem of recovering a vector of complex units $x_i \in \Sbb^1, i \in [n]$, or $x \in (\Sbb^1)^n$ from estimates of their relative phases \[\tX_{ij} = x_j^* x_i \eta_{ij}, (i, j) \in E\] where $\eta_{ij} \in \Sbb^1$ are the ``noise terms'' and $E \subset [n]^2$ is a list of the pairs of indices for which we have such estimates.  This immediately suggests associating a graph to this problem; namely, taking the vertex set to be $V = [n]$, we set $G = (V, E)$.  We also remark that this problem invites the same global phase ambiguity as phase retrieval: indeed $(\ee^{\ii \theta} x_i)^* (\ee^{\ii \theta} x_j) \eta_{ij} = x_i^* x_j \eta_{ij}$ for any $\theta \in [0, 2\pi)$.  Obviously, then, our goal is to get an estimate $\tilde{x}$ that is guaranteeably close to the ground truth $x$ in some chosen metric, say our usual $\min_{\theta \in \R} || x - \ee^{\ii \theta} y ||_2,$ and to acquire this estimate with the least computational cost.  Naturally, $x$ is not known, so we instead attempt to minimize some cost function corresponding to how well our estimate explains the measurement data, usually taking a form similar to the frustration function stated in \eqref{eq:frustration}.  Often, we more simply take only the numerator of this expression, \begin{equation} \sum_{(i, j) \in E} w_{ij} |x_i - X_{ij} x_j|^2 = x^* (D - W \circ X) x, \label{eq:ang_sync_cost} \end{equation} where $D$ and $W$ are the degree and weight matrices specified in \eqref{eq:degandweight}.  For convenience, we define \begin{equation} L = D - W \circ X, \ \uL = D - W \circ x x^*, \ \text{and} \ L_G = D - W. \label{eq:L_defs} \end{equation}  The approach to angular synchronization that we consider in this dissertation, and the approach most studied in the literature, then is to attempt the non-convex optimization problem \begin{equation} \begin{array}{cl} \min\limits_{z \in \C^n} & z^* L z \\ \text{s.t.} & |z_i| = 1 \end{array} \label{eq:ang_sync}\end{equation}

The modern study of eigenvector-based methods for angular synchronization appears to have begun with a 2011 paper by Amit Singer \cite{singer2011ang_sync}, in which he proposed two ways of solving this problem which remain as the basis of the state of the art.  The first is almost identical to the eigenvector-based approach that we have used in the algorithm proposed in chapter \ref{ch:our_model}, and the second is a semidefinite relaxation of the same.  Namely, using the unweighted adjacency matrix $A_{ij} = X_{ij} \chi_E(i, j)$, his eigenvector method solves \[ \max_{||z||_2^2 = n} z^* A z \] to find the largest eigenvector $\hz$ of $A$ and rounds to a vector of units by taking $\tx = \sgn(\hz)$.  The SDP method solves \begin{equation}\begin{array}{cl} \max\limits_{Z \in \H^d} & \Tr(A Z) \\ \text{s.t.} & Z \succeq 0 \\ & Z_{ii} = 1 \end{array} \label{eq:angsyncSDPsinger}\end{equation}  In this paper, he studies the problem under a noise model where the disturbances $\eta_{ij}$ are distributed according to \[\eta_{ij} = \left\{\begin{array}{r@{,\quad}l} 1 & \text{with probability}\ p \\ \mathrm{Unif}(\Sbb^1) & \text{with probability}\ 1 - p\end{array}\right.,\] such that the measurement $\tX_{ij}$ is exact with probability $p$ and is completely meaningless -- being drawn from the uniform distribution on $\Sbb^1$ -- with probability $1 - p$.  He proves the robustness of this method in the sense that there is a probability $p_c$, dependent on the spectral gap and size of the graph $G$, for which parameter values $p > p_c$ guarantee ``better than random'' approximations of $x$ with high probability.  Moreover, he shows that, experimentally, both of these recovery algorithms work acceptably, if not extremely, well, with little to be gained by transferring from the eigenvector problem to the computationally more expensive semidefinite program.

The literature on angular synchronization since this paper has largely consisted of analyzing generalizations and variations of these methods.  One major generalization has been to apply these methods to larger classes of group synchronization problems such as synchronization over the orthogonal groups $O(d)$ or the special Euclidean groups $SE(d)$ \cite{Cheeger,briales2017cartan_sync,bandeira2016se_sync}.  Naturally, much of the interest in this subject has been the treatment of synchronization over $SO(3)$ and $SE(3)$, as these correspond to pose estimation problems fundamental to computer vision, as in \cite{enqvist2011nonsequential, olsson2017rot_avg, fischler1981ransac, govindu2006motion_avg}.  Significant results giving guarantees of robustness, as well as proofs that these relaxations are solved \emph{exactly} in certain cases may be found in \cite{alexeev2014phase, bandeira2016tightness, olsson2017rot_avg, bandeira2016se_sync}.
