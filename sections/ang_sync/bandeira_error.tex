\documentclass[12pt]{article}

\usepackage{amsmath,amsfonts,amssymb,amsthm,amscd,mathtools}
\usepackage{algorithm,algorithmic,caption,subcaption}
\usepackage[dvipsnames]{xcolor}
\usepackage{import}
\usepackage{verbatim}
\usepackage{url}
\urlstyle{rm} % OGS said they require URLs to be typeset like normal text
\usepackage[pdftex, bookmarks=true, colorlinks=true, linkcolor=black, citecolor=black, urlcolor=black]{hyperref} % OGS said colored text is unacceptable
\usepackage{cleveref,titling}
\usepackage{graphicx,subcaption}
\usepackage{helvet}
\usepackage{bbm,enumerate,array}
\usepackage{graphics, parskip}
\usepackage{listings}

\bibliographystyle{abbrv}

\newtheorem{theorem}{Theorem}
\newtheorem*{thm*}{Theorem}
\newtheorem{corollary}{Corollary}
\newtheorem{lemma}{Lemma}
\newtheorem*{lemma*}{Lemma}
\newtheorem{proposition}{Proposition}
\newtheorem*{proposition*}{Proposition}
\newtheorem{exercise}{Exercise}
\newtheorem*{remark*}{Remark}
\newtheorem*{surfacecor}{Corollary 1}
\newtheorem{conjecture}{Conjecture} 
\newtheorem{question}{Question} 
\theoremstyle{definition}
\newtheorem{definition}{Definition}

\let\mod\undefined
\let\Re\undefined
\let\Im\undefined
\let\vec\undefined

\DeclareMathOperator{\diag}{diag}
%\newcommand{\diag}{\mathrm{diag}}
\DeclareMathOperator{\Col}{Col}
\DeclareMathOperator{\Row}{Row}
\DeclareMathOperator{\Span}{span}
\DeclareMathOperator{\Nul}{Nul}
\DeclareMathOperator{\rank}{rank}
\DeclareMathOperator{\supp}{supp}
\DeclareMathOperator{\Tr}{Tr}
\DeclareMathOperator{\conv}{conv}
\DeclareMathOperator{\cone}{cone}
\DeclareMathOperator{\aff}{aff}
\DeclareMathOperator{\sgn}{sgn}
\DeclareMathOperator*{\argmin}{argmin}
\DeclareMathOperator*{\argmax}{argmax}
\DeclareMathOperator{\circop}{circ}
\DeclareMathOperator{\mod}{\mathbin{mod}}
\DeclareMathOperator{\vol}{vol}
\DeclareMathOperator{\Re}{Re}
\DeclareMathOperator{\Im}{Im}
\DeclareMathOperator{\Proj}{Proj}
\DeclareMathOperator{\vec}{vec}
\DeclareMathOperator{\mat}{mat}

\DeclarePairedDelimiter{\floor}{\lfloor}{\rfloor}
\DeclarePairedDelimiter{\ceil}{\lceil}{\rceil}
\DeclarePairedDelimiter{\norm}{\lVert}{\rVert}
\DeclarePairedDelimiter{\abs}{\lvert}{\rvert}
\DeclarePairedDelimiter{\inner}{\langle}{\rangle}

\newcommand{\R}{\ensuremath{\mathbb{R}}}  % The real numbers.
\newcommand{\C}{\ensuremath{\mathbb{C}}}
\newcommand{\N}{\ensuremath{\mathbb{N}}}
\newcommand{\Z}{\ensuremath{\mathbb{Z}}}
\renewcommand{\H}{\ensuremath{\mathcal{H}}}
\newcommand{\sym}{\mathcal{S}}
\newcommand{\Sbb}{\ensuremath{\mathbb{S}}}
\newcommand{\B}{\ensuremath{\mathcal{B}}}
\newcommand{\Rn}{\ensuremath{\R^n}}
\newcommand{\Rm}{\ensuremath{\R^m}}
\newcommand{\Cn}{\ensuremath{\C^n}}
\newcommand{\mfr}{\mathfrak{R}}
\newcommand{\ulmfr}{\underline{\mfr}}
\newcommand{\Rmxn}{\ensuremath{\R^{m \times n}}}
\newcommand{\Cmxn}{\ensuremath{\C^{m \times n}}}
\newcommand{\Rnxn}{\ensuremath{\R^{n \times n}}}
\newcommand{\Cnxn}{\ensuremath{\C^{n \times n}}}
\newcommand{\bigfrac}[2]{\ensuremath{\frac{\displaystyle #1}{\displaystyle #2}}}
\newcommand{\kron}{\otimes}
\newcommand{\iid}{\overset{\text{i.i.d.}}{\sim}}
\newcommand{\ee}{\mathrm{e}}
\newcommand{\ii}{\mathrm{i}}
\renewcommand{\th}{\ensuremath{^{\text{th}}}}
\newcommand{\ow}{\text{otherwise}}
\renewcommand{\c}{\ensuremath{\mathbf{c}}}
\newcommand{\x}{\ensuremath{\mathbf{x}}}
\newcommand{\m}{\ensuremath{\mathbf{m}}}
\newcommand{\y}{\ensuremath{\mathbf{y}}}
\newcommand{\tf}{\ensuremath{\tilde{f}}}
\newcommand{\ux}{\underline{x}}
\newcommand{\uz}{\underline{z}}
\newcommand{\uX}{\underline{X}}
\newcommand{\uR}{\underline{R}}
\newcommand{\utX}{\underline{\tX}}
\newcommand{\uL}{\underline{L}}
\newcommand{\uW}{\underline{W}}
%% \newcommand{\Fb}{\overline{F}}
\newcommand{\tF}{\widetilde{F}}
\newcommand{\Lc}{\mathcal{L}}
\newcommand{\Rc}{\mathcal{R}}
\newcommand{\Ac}{\mathcal{A}}
\newcommand{\hz}{\hat{z}}
\newcommand{\hZ}{\hat{Z}}
\newcommand{\hL}{\hat{L}}
\renewcommand{\v}{\ensuremath{\mathbf{v}}}
\renewcommand{\u}{\ensuremath{\mathbf{u}}}
\newcommand{\F}{\ensuremath{\mathcal{F}}}
\newcommand{\mg}{martingale}
\newcommand{\E}{\ensuremath{\mathbb{E}}}
\newcommand{\rv}{random variable}
\newcommand{\psd}{\ensuremath{\mathcal{S}_+}}
\newcommand{\psdn}{\ensuremath{\mathcal{S}_+^n}}
\newcommand{\bigO}{\mathcal{O}}
\newcommand{\one}{\ensuremath{\mathbbm{1}}}
\renewcommand{\a}{\ensuremath{\overline{a}}}
\newcommand{\dbar}{\overline{d}}
\newcommand{\deltabar}{\overline{\delta}}
\newcommand{\mintheta}{\min_{\theta \in [0, 2\pi]}}
\newcommand{\eit}{\ee^{\ii \theta}}
\newcommand{\eiti}{\ee^{\ii \theta_i}}
\newcommand{\conj}[1]{\overline{#1}}
\renewcommand{\subset}{\subseteq}

\newcommand{\rowmat}[3]{\ensuremath{\begin{bmatrix} #1_{#2} & \cdots & #1_{#3} \end{bmatrix}}}
\newcommand{\rowmatfun}[3]{
  \ensuremath{
    \begin{bmatrix}
      \StrSubstitute{#1}{@}{{#2}} &
      \cdots &
      \StrSubstitute{#1}{@}{{#3}}
    \end{bmatrix}
  }
}
\newcommand{\colmat}[3]{\ensuremath{\begin{bmatrix} #1_{#2} \\ \vdots \\ #1_{#3} \end{bmatrix}}}
\newcommand{\colmatfun}[3]{
  \ensuremath{\noexpandarg
    \begin{bmatrix}
      \StrSubstitute{#1}{@}{{#2}} \\
      \vdots \\
      \StrSubstitute{#1}{@}{{#3}}
    \end{bmatrix}
  }
}
\newcommand{\diagcornmat}[2]{
  \ensuremath{
    \begin{bmatrix}
      #1 & \cdots & 0 \\
      \vdots & \ddots & \vdots \\
      0 & \cdots & #2
    \end{bmatrix}
  }
}
\newcommand{\bigdiagcornmat}[2]{
  \ensuremath{
    \begin{bmatrix}
      #1 & 0 & \cdots & 0 \\
      0 & \ddots & & \vdots \\
      \vdots & & \ddots & 0 \\
      0 & \cdots & 0 & #2
    \end{bmatrix}
    }
  }
\newcommand{\diagmat}[3]{
  \diagcornmat{#1_{#2}}{#1_{#3}}
}
\newcommand{\diagmatfun}[3]{
  \diagcornmat{\StrSubstitute{#1}{@}{#2}}{\StrSubstitute{#1}{@}{#3}}
}
\newcommand{\cornmat}[5]{
  \ensuremath{
  \begin{bmatrix} #1_{#2#4} & \cdots & #1_{#2#5} \\
    \vdots & \ddots & \vdots \\
    #1_{#3#4} & \cdots & #1_{#3#5}
  \end{bmatrix}
  }
}
\newcommand{\cornmatfun}[5]{
  \ensuremath{
    \StrSubstitute{#1}{@r}{{#2}}[\cmftop]
    \StrSubstitute{#1}{@r}{{#3}}[\cmfbot]
  \begin{bmatrix} \StrSubstitute{\cmftop}{@c}{{#4}} & \cdots & \StrSubstitute{\cmftop}{@c}{{#5}} \\
    \vdots & \ddots & \vdots \\
    \StrSubstitute{\cmfbot}{@c}{{#4}} & \cdots & \StrSubstitute{\cmfbot}{@c}{{#5}}
  \end{bmatrix}
  }
}

\newcommand{\BPnote}[1]{\textcolor{ForestGreen}{[{\em {\bf **BP Note:} #1}]}}

\newcommand{\crefrangeconjunction}{--}

\newenvironment{remark}
               {\textbf{Remark.}}
               {}
               
\newenvironment{piecewise}
               {\left\{\begin{array}{r@{,\quad}l}}
               {\end{array}\right.}


\begin{document}
\begin{center}
  \Large Error in proof of Theorem 12 of \cite{bandeira2016se_sync}
\end{center}

Theorem 12, stated on p.~43, is proven on pp.~41-43.  The final line of the proof says ``combining inequalities (134), (140), (151), and (152), we obtain the following [Theorem 12],'' but unfortunately (151) and (152) are not true in general.  Nonetheless, the statement of Theorem 12 is true as stated, and I propose an alternative argument that restores the proof.  In addition, we are able to prove a bound that is stronger as the measurement error goes to zero.

\section{Recalling the notation}
We recall that $\uR, R^*$ are elements of $O(d)^n$, meaning \[\uR = \begin{bmatrix} \uR_1 & \cdots & \uR_n \end{bmatrix} \ \text{and} \ R^* = \begin{bmatrix} R^*_1 & \cdots & R^*_n \end{bmatrix},\] where $\uR_i, R^*_i \in \R^{d \times d}$ are orthogonal matrices.  $P$ is the orthogonal projection of $R^*$ onto the orthogonal complement of $\uR$.  In $O(d)^n$, this means \[P = R^* - \frac{1}{n} R^* \uR^T \uR,\] and in (143) we discover that \[\norm{P}_F^2 = dn - \frac{1}{n} \norm*{\uR R^{*T}}_F^2.\]  We define the $O(d)$ orbital distance between $\uR$ and $R^*$ by \[d_{\mathcal{O}}(\uR, R^*) = \min_{G \in O(d)} \norm{\uR - G R^*}_F,\] and from Theorem 5 (p.~35) we have that \[d_{\mathcal{O}}(\uR, R^*)^2 = 2 d n - 2 \norm{\uR R^{*T}}_*.\]

\section{The argument for (150) is incorrect}
\label{sec:incorrect}
The goal is to prove \begin{equation} \norm{P}_F^2 \ge \frac{1}{2} d_{\mathcal{O}}(\uR, R^*)^2, \label{goal} \end{equation} which is essential in the proof of Theorem 12 (and Theorem 12 is quoted in the proof of the main result, Proposition 2, on p.~44!).  However, the ``optimizaton strategy'' of (147)-(150) doesn't work.  Setting $\delta = d_{\mathcal{O}}(\uR, R^*)$, we can rephrase the optimization problem of (147) as
\[\begin{array}{r>{\displaystyle}l}
\max & \sum_{i = 1}^d \sigma_i^2 \\
\text{s.t.} & \sum_{i = 1}^d \sigma_i = a \\
& \sigma_i \ge 0
\end{array},
\]
where $a = dn - \delta^2 / 2$.  This, in turn, is obviously equivalent to \[\max_{\norm{x}_1 = a} \norm{x}_2^2.\]  From standard norm inequalities, we have that $\frac{1}{\sqrt{d}} \norm{x}_1 \le \norm{x}_2 \le \norm{x}_1$, with equality on the left when $x_i = t \one$ and equality on the right when $x = e_i$ for some $i \in [d]$.  In this instance, the maximal value will be $\norm{x}_1^2$, or \[\epsilon^2 = \left(\sum_{i = 1}^d \sigma_i\right)^2 = \left(dn - \frac{\delta^2}{2}\right)^2 = d \left(d\left(n - \frac{\delta^2}{2d}\right)^2\right),\] which is greater than the value given in (150) by a factor of $d$.  This leads to (151) becoming \[\norm{P}_F^2 \ge dn - \frac{d^2}{n}\left(n - \frac{\delta^2}{2d}\right)^2 = d \delta^2 + dn(1 - d) - \frac{\delta^4}{4n}.\]  Following the argument of (152), we get \[\norm{P}_F^2 \ge \frac{d}{2} \delta^2 - dn(d - 1),\] which, if combined with (134) and (140), gives \[\delta^2 \le \frac{4n \norm{\Delta Q}_2}{\lambda_{d + 1}(Q)} + 2n(d - 1).\]  

In this state, the bound proves nothing; in particular, we don't have that $\lim_{\norm{\Delta Q} \to 0} \norm{\Delta R} = 0$, not to mention that Theorem 5 (p.~35) trivially bounds $\delta^2 \le 2dn$.

\section{An alternative argument}
\label{sec:alternative}
Consider that, from (143) and (144), $\norm{P}_F^2 \ge \frac{1}{2} d_{\mathcal{O}}(\uR, R^*)^2$ holds for $\uR, R^* \in O(d)^n$ iff \[dn - \frac{1}{n} \norm*{\uR R^{*T}}_F^2 \ge dn - \norm*{\uR R^{*T}}_* \iff \norm*{\uR R^{*T}}_* \ge \frac{1}{n} \norm*{\uR R^{*T}}_F^2,\] so it suffices to prove this last inequality for all $\uR, R^* \in O(d)^n$.  Fix $\uR, R^* \in O(d)^n$; taking $\sigma_1 \ge \ldots \ge \sigma_d$ to be the singular values of $\uR R^{*T}$ and setting $(\sigma_1, \ldots, \sigma_n)^T =: \sigma \in \R^n$, this happens iff \begin{equation} \norm{\sigma}_1 \ge \frac{1}{n} \norm{\sigma}_2^2. \label{necsuff} \end{equation}  By H\"{o}lder's inequality, we have \begin{equation} \norm{\sigma}_2^2 \le \norm{\sigma}_1 \norm{\sigma}_\infty. \label{holder}\end{equation}  Now \begin{equation} \norm{\sigma}_\infty = \norm*{\uR R^{*T}} = \norm*{\sum_{i = 1}^n \uR_i R^{*T}_i} \le \sum_{i = 1}^n \norm*{\uR_i R^{*T}_i} = n, \label{infty} \end{equation} since $\uR_i R^{*T}_i$ is orthogonal.  Combining \eqref{holder} and \eqref{infty}, we have \[\frac{1}{n} \norm{\sigma}_2^2 \le \frac{1}{n} \norm{\sigma}_1 \norm{\sigma}_\infty \le \frac{1}{n} \norm{\sigma}_1 n = \norm{\sigma}_1,\] which yields \eqref{necsuff} as needed.

\section{Improvement to the bound of Theorem 12}
\label{sec:improve}
In sections \ref{sec:incorrect} and \ref{sec:alternative}, we ignored the origin of $\uR$ and $R^*$, but in the present section \ref{sec:improve}, we note that $\uR$ and $R^*$ arise as optima of certain constrained optimization problems, though for now we ignore the specifics and assume only the relevant identities.  Given symmetric matrices $\tilde{Q}, \underline{Q} \succeq 0$ in $\R^{dn \times dn}$, we assume $\Tr(\underline{Q} \uR^T \uR) = 0$ and that $\Tr(\tilde{Q} R^{*T} R^*) \le \Tr(\tilde{Q} \uR^T \uR)$.  Setting $\Delta Q = \tilde{Q} - \underline{Q}$, these may be combined into \begin{align*} \Tr(\tilde{Q} \uR^T \uR) &= \Tr(\Delta Q \uR^T \uR) + \Tr(\underline{Q} \uR^T \uR) \\&\ge \Tr(\Delta Q R^{*T} R^*) + \Tr(\underline{Q} R^{*T} R^*) = \Tr(\tilde{Q} R^{*T} R^*), \end{align*} which we rearrange to get \begin{equation} \Tr(\underline{Q} R^{*T} R^*) \le \Tr(\Delta Q \uR^T \uR) - \Tr(\Delta Q R^{*T} R^*). \label{trinq}\end{equation}

In the style of the proof of Lemma 4.1 in \cite{bandeira2016tightness}, we are able to achieve a notable improvement to the bound of Theorem 12 in \cite{bandeira2016se_sync}.  From \eqref{trinq}, and using $\Tr(\Delta Q \uR^T \uR) = \vec(\uR)^T (\Delta Q \otimes I_n) \vec(\uR)$, we get \begin{align*} \Tr(\underline{Q} R^{*T} R^*) &\le \vec(\uR - R^*)^T (\Delta Q \otimes I_n) \vec(\uR + R^*) \\ &\le \norm{\vec(\uR - R^*)}_2 \norm{\Delta Q \otimes I_n}_2 \norm{\vec(\uR + R^*)}_2 \\ &= \norm{\uR - R^*}_F \norm{\Delta Q}_2 \norm{\uR + R^*}_F \\ &\le 2\sqrt{dn} \norm{\uR - R^*}_F \norm{\Delta Q}_2\end{align*}

Assuming, without loss of generality, that $\uR$ and $R^*$ are representatives of their orbits such that $\norm{\uR - R^*}_F = d_{\mathcal{O}}(\uR, R^*)$ and combining this with \eqref{goal} and (140) of \cite{bandeira2016se_sync}, which gives \[\Tr(\underline{Q} R^{*T} R^*) \ge \lambda_{d + 1}(Q) \norm{P}_F^2,\] we have \begin{equation} d_{\mathcal{O}}(\uR, R^*) \le \dfrac{4 \sqrt{dn} \norm{\Delta Q}_2}{\lambda_{d + 1}(Q)}. \label{eq:new_bound}\end{equation}

We compare this to the original result, which gives \begin{equation} d_{\mathcal{O}}(\uR, R^*) \le \sqrt{\dfrac{4 dn \norm{\tilde{Q} - \underline{Q}}_2}{\lambda_{d + 1}(\underline{Q})}}. \label{eq:sq_bound} \end{equation}  We remark that, as $\norm{\tilde{Q} - \underline{Q}}_2$ goes to zero, the higher exponent in the new bound guarantees a faster rate of convergence of $R^* \to \uR$.  Since both bounds are valid, we may always use whichever is stronger: indeed, the square root bound of \eqref{eq:sq_bound} is stronger exactly when $\norm{\tilde{Q} - \underline{Q}}_2 / \lambda_{d + 1}(\underline{Q}) > \frac{1}{4}$.  We remark that \eqref{eq:new_bound} is trivial when $\norm{\tilde{Q} - \underline{Q}}_2 / \lambda_{d + 1}(\underline{Q}) \ge \frac{1}{2 \sqrt{2}}$, while \eqref{eq:sq_bound} is trivial when $\norm{\tilde{Q} - \underline{Q}}_2 / \lambda_{d + 1}(\underline{Q}) \ge \frac{1}{2}$, as $d_{\mathcal{O}}(\uR, R^*) \le \sqrt{2 d n}$.

\bibliography{bandeira_error}

%% \section{(150) holds when $d = 2$}
%% It turns out the argument works in the $SO(2)$ case, because if we identify complex-unit vectors $\hz, \uz \in (\Sbb^1)^n$ such that $\uR = \mfr(\uz^T)$ and $R^* = \mfr(\hz^T)$, then we have that \[\norm{\uR R^{*T}}^2_F = \norm{\mfr(\uz^T \overline{\hz)}}_F^2 = \norm{\mfr(\inner{\hz, \uz})}_F = 2 \abs{\inner{\uz, \hz}}.\]  We now consider that, for any $z \in \C$, the singular values of $\mfr(z)$ are both equal to $\abs{z}$; indeed, \[\mfr(z) = \begin{bmatrix} \Re(\sgn(z)) & -\Im(\sgn{z}) \\ \Im(\sgn(z)) & \Re(\sgn(z)) \end{bmatrix} \begin{bmatrix} \abs{z} & 0 \\ 0 & \abs{z} \end{bmatrix} \begin{bmatrix} 1 & 0 \\ 0 & 1 \end{bmatrix},\] as $\mfr(z) e_1 = (\Re(z), \Im(z))^T$ and $\mfr(z) e_2 = (-\Im(z), \Re(z))^T$ are orthogonal.  In this instance, then, following (147) of \cite{bandeira2016se_sync} we have $\sigma_i = \sigma = \abs{\inner{\hz, \uz}}$ and $d = 2$, giving \[\sum_{i = 1}^d \sigma_i = d \sigma = dn - \frac{\delta^2}{2}\] so that $\sigma = n - \frac{\delta^2}{2d} = n - \frac{\delta^2}{4}$, while \[\epsilon^2 = \sum_{i = 1}^d \sigma_i^2 = d \sigma^2 = d(n - \frac{\delta^2}{2d})^2 = 2(n - \frac{\delta^2}{4})^2,\] as Bandeira, et.~al have in (150).
\end{document}
