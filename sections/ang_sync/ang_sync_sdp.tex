\subsection{Introduction and Main Result}
We have already presented one angular synchronization result in \cref{sec:Perturb}, which drew largely on \cite{alexeev2014phase}.  We remark that this theorem %% \BPnote{rephrase: don't want to downplay your own result} leaves something to be desired 
lacks some generality in that the graph $G$ is not permitted to be weighted, which restricts us from applying some knowledge that we may have about the problem.  For example, suppose our relative phase measurements $\{X_{ij}\}_{(i,j) \in E}$ are disturbed by noise drawn from a fixed, phase-invariant probability distribution, say \[X_{ij} = \sgn(x_i^*x_j + \epsilon_{ij}), \epsilon_{ij} = a_{ij} + \ii\, b_{ij} \ \ \text{with} \ \ a_{ij}, b_{ij} \iid N(0, \sigma^2),\] then we will have more confidence in the relative phases represented by the larger magnitude entries of $X = \mathcal{A}^{-1}(\y)$.  It would be intuitive to use this knowledge to privilege some edges of the graph over others in the frustration function \eqref{eq:frustration} that we are trying to minimize, say by using $w_{ij} = |X_{ij}|$.  Unfortunately, theorem \ref{thm:SpecGraphPertBound} assumes an unweighted graph and its proof technique does not readily admit a satisfactory adjustment towards weighted edges, though we shall impose one later.  Therefore, we will take a distinct approach, drawing upon recent results in the literature that consider certain convex relaxations of \eqref{eq:ang_sync} \cite{bandeira2016tightness, calafiore2016complex_pgo, bandeira2016se_sync}.

To begin this discussion, we gather our notation: $G = (V = [n], E)$ is a connected graph with a weighted adjacency matrix $W = [w_{ij}] \in \sym^n$ satisfying $w_{ij} \ge 0$ and $w_{ij} \neq 0$ only if $(i, j) \in E$.  We take $D = \diag(W \one)$ to be the degree matrix.  $\ux \in (\Sbb^1)^n$ is the ground truth vector, which we attempt to recover, and $X, \uX \in \H^n$ are our noisy and ground truth edge data matrices, satisfying \[X_{ij} = \begin{piecewise} \eta_{ij} x_i x_j^* & (i, j) \in E \\ 0 & \ow \end{piecewise} \quad \text{and} \quad \uX_{ij} = \begin{piecewise} x_i x_j^* & (i, j) \in E \\ 0 & \ow \end{piecewise},\] where $\eta_{ij} = \eta_{ji}^* \in \Sbb^1$ for each $(i, j) \in E$.  We then define $L, \uL$, and $L_G$ as in \eqref{eq:L_defs}.  Finally, given a graph $G = (V = [n], E)$ with unweighted adjacency matrix $A_G$, for any set $0 \in S \subset \C$, by $S^E$ we mean the set of matrices $X$ satisfying $X \in \H^n \cap S^{n \times n}$ and $X = X \circ A_G$ (so that $X_{ij} = 0$ for all non-adjacent pairs $(i, j) \notin E$).  Specifically, $\R_{++}^E$ will denote the set of matrices representing valid positive weightings of $G$ and $(\Sbb^1)^E$ will denote matrices containing valid assignments of relative phases to edges of $G$.

Towards formulating the appropriate SDP relaxations, we recall the transformation $\mfr : \C \to \R^{2 \times 2}$ %and $\ulmfr : \C \to \R^2$ defined by \[\mfr(a + bi) = \begin{bmatrix} a & -b \\ b & a \end{bmatrix}, \ \ulmfr(a + bi) = \begin{bmatrix} a \\ b \end{bmatrix}.\]
defined by \[\mfr(a + \ii\, b) = \begin{bmatrix} a & -b \\ b & a \end{bmatrix}.\]  It is well-known that $\mfr$ is the canonical isomorphism from $\C$ into $\R^{2 \times 2}$, and indeed if we extend it to matrices by taking, for $A \in \C^{m \times n}$, $\mfr(A) \in \R^{2m \times 2n}$ to be a block matrix with $\mfr(A)_{ij} = \mfr(A_{ij})$, then it remains an isomorphism on $\C^{m \times n}$, and indeed it preserves the eigenvalues and eigenvectors of Hermitian matrices (see, e.g.~\cite[p.~101]{wedderburn1934matrices}).  In particular, a Hermitian matrix $A \in \H^n$ is semi-definite if and only if $\mfr(A)$ is semi-definite; we notice that the multiplicities of its eigenvalues are all doubled, as $A v = \lambda v$ implies $\mfr(A) \mfr(v) = \lambda \mfr(v)$, giving that the two columns of $\mfr(v)$ are each eigenvectors of $\mfr(A)$ with eigenvalue $\lambda$.  For convenience, we extend the inverse $\mfr^{-1} : \bigcup_{m, n \in \N} \R^{2m \times 2n} \to \bigcup_{m, n \in \N} \C^{m \times n}$ to $\bigcup_{n \in \N} \R^{2n}$ in the obvious way, by letting \[\mfr^{-1}\begin{bmatrix} a \\ b \end{bmatrix} = a + b i.\]

With this, we consider SDP relaxations of \eqref{eq:ang_sync}.  Specifically, we observe that $z^* L z = \Tr(L z z^*)$, so an equivalent optimization problem will be
\begin{equation}
  \begin{array}{cl}
    \min\limits_{Z \in \H^n} & \Tr(L Z) \\
    \text{s.t.} & Z_{ii} = 1 \\
    & \rank(Z) = 1 \\
    & Z \succeq 0
  \end{array}, \label{eq:ang_sync_rankone}
\end{equation}
where the optimizer $\hat{z}$ of \eqref{eq:ang_sync} is recovered from the optimal matrix $\hZ$ by merely factoring $\hZ = \hat{z} \hat{z}^*$.  To get a convex relaxation, we simply omit the non-convex rank one constraint, yielding \begin{equation} \begin{array}{cl} \min\limits_{Z \in \H^n} & \Tr(L Z) \\ \text{s.t.} & Z_{ii} = 1 \\ & Z \succeq 0 \end{array}. \label{eq:ang_sync_sdp} \end{equation}  We remark that if an optimizer $\hZ$ of \eqref{eq:ang_sync_sdp} is rank one, then it is also an optimizer of \eqref{eq:ang_sync_rankone} since the feasible set of \eqref{eq:ang_sync_sdp} is strictly larger than that of \eqref{eq:ang_sync_rankone}; in this case, then, factoring $\hZ$ gives the global minimizer of \eqref{eq:ang_sync}.  Considering that many results in the optimization literature (and many of the software libraries) are written for real-valued SDPs, we take advantage of these by further relaxing the feasible set, casting into the real domain with
\begin{equation}
  \begin{array}{cl}
    \min\limits_{Z \in \sym^{2n \times 2n}} & \Tr(\mfr(L) Z) \\
    \text{s.t.} & Z_{ii} = I_2 \\
    & Z \succeq 0
  \end{array}, \label{eq:ang_sync_real}
\end{equation}
where $Z_{ii} = [e_{2i - 1} \ e_{2i}]^* Z [e_{2i - 1} \ e_{2i}]$ in this case refers to the $i\th \ 2 \times 2$ diagonal block of $Z$.  With this in mind, we state two of the algorithms for angular synchronization that will be considered in this dissertation.\footnote{For \cref{alg:ang_sync_sdp}, we will actually implement the optimization with \eqref{eq:ang_sync_real}, but since all the SDPs involved are strictly feasible -- we may take $Z = I_n$ for \eqref{eq:ang_sync_sdp} and $\Lambda = -(\norm{L}_2 + \epsilon) I_n$ for \eqref{eq:ang_sync_dual} -- optimizers exist for each \cite[Thm. 3.1]{boyd1996semidef_prog}, and these coincide when the opimization ``works,'' as we shall see in \cref{lem:necsuff_ang_sync_dual}.}

\begin{algorithm}[htbp]
\renewcommand{\algorithmicrequire}{\textbf{Input:}}
\renewcommand{\algorithmicensure}{\textbf{Output:}}
\caption{Angular Synchronization by Eigenvectors}
\label{alg:ang_sync_eig}
\begin{algorithmic}[1]
  \REQUIRE Connected graph $G = (V = [n], E)$, weight matrix $W \in \R_{++}^E$, and relative phase data $X \in (\Sbb^1)^E$.
  \ENSURE A vector $x \in (\Sbb^1)^n$ of phases.
  \STATE Let $L = \diag(W one) - W \circ X$ be the Laplacian of $G$.
  \STATE Let $u \in \argmin_{\norm{z}_2 = 1} z^* L z$ be an eigenvector corresponding to the smallest eigenvalue of $L$.
  \STATE Return $x = \sgn(u)$.
\end{algorithmic}
\end{algorithm}

\begin{algorithm}[htbp]
\renewcommand{\algorithmicrequire}{\textbf{Input:}}
\renewcommand{\algorithmicensure}{\textbf{Output:}}
\caption{Angular Synchronization by SDP Relaxation}
\label{alg:ang_sync_sdp}
\begin{algorithmic}[1]
  \REQUIRE Connected graph $G = (V = [n], E)$, weight matrix $W \in \R_{++}^E$, and relative phase data $X \in (\Sbb^1)^E$.
  \ENSURE A vector $x \in (\Sbb^1)^n$ of phases.
  \STATE Let $L = \diag(W \one) - W \circ X$ be the connection Laplacian of $G$.
  \STATE Let $\hZ \Cnxn$ be a minimizer of \eqref{eq:ang_sync_sdp} with this data.
  \STATE Let $u \in \Cn$ be an eigenvector of $\hZ$ corresponding to its largest eigenvalue.
  \STATE Return $x = \sgn(u)$.
\end{algorithmic}
\end{algorithm}

%% However, neither of these results may be directly applied, since \cite{bandeira2016tightness} is restricted to  The result stated in \cite{bandeira2016se_sync}, on the other hand, is far more general, applying to synchronization over $SE(n)$, but narrowing to angular synchronization (which is a special case of this problem) allows a considerable improvement over the error bound stated in the paper.  This sharpened result, as we shall see, leads to better guarantees than any we have yet proven.

%% This allows us to restate the SDP of \eqref{eq:angsyncSDPsinger} as a real-valued SDP in 

At this point, we recognize the previous work existing on this problem.  Namely, in \cite{bandeira2016tightness}, \citeauthor*{bandeira2016tightness} prove that the optimizer $\hZ$ of \eqref{eq:ang_sync_sdp} is rank one (and therefore yields a minimizer of \eqref{eq:ang_sync}) when $L$ is sufficiently close to $\uL$.  Unfortunately for our purposes, this paper only considers the case when $G = K_n$ is the complete graph and the weights $W = \one \one^* - I_n$ are constant.  A more general result appears in \cite{bandeira2016se_sync}, where \citeauthor*{bandeira2016se_sync} prove a similar result for synchronization over $SE(d)$, of which angular synchronization is a special case.  Moreover, these results allow for a weighted graph, and include a bound on $\min_{\theta \in [0, 2\pi)} \lVert \hat{z} - \ee^{\ii\theta} \ux\rVert_2$ in terms of $\lVert L - \uL \rVert_2$ and the spectral gap of the graph.  Nonetheless, we find that narrowing to the case of $SO(2)$ (equivalent to angular synchronization) allows for a tighter error bound compared to their original result.  In \cite{calafiore2016complex_pgo}, \citeauthor*{calafiore2016complex_pgo} use methods similar to those in \cite{bandeira2016se_sync} to analyze $SE(2)$ synchronization.  Furthermore, and pertinent to the present work, the authors exchange the rotational components in $SO(2)$ for complex units, but they do not admit weighted graphs, nor do they supply explicit bounds on the error of their estimate or on what level of noise may be tolerated and still guarantee that their convex relaxation solves \eqref{eq:ang_sync} exactly.  Significantly, all three of these works supply an \emph{a posteriori}-certifiable condition that can verify whether the solution obtained is indeed optimal for \cref{eq:ang_sync,eq:ang_sync_sdp,eq:ang_sync_rankone,eq:ang_sync_real}.

The results in this chapter, particularly \cref{thm:ang_sync_dual,thm:improved_spec_pert} are heavily based on this previous work, although they are original, having not appeared in this form in the literature.  In particular, our result is more general than those in \cite{bandeira2016tightness}, since we consider all possible graphs rather than $K_n$ alone.  It is slightly more general than \cite{calafiore2016complex_pgo}, since it admits weighted graphs.  Although \cite{bandeira2016se_sync} applies to a far more general problem -- that of $SE(d)$ synchronization -- the result of \cref{thm:ang_sync_dual} is slightly sharper than what is given by applying their result to angular synchronization as a special case.  The present result is achieved by synthesizing proof techniques from all three papers. 

Additionally, \cref{app:bandeira_thing} demonstrates a mistake that appeared in the original argument of a proposition necessary to the main results of \cite{bandeira2016se_sync}.  Fortunately, we were able to produce an alternative proof that establishes the same statement.  The same appendix also brought in ideas from \cite{bandeira2016tightness} that extend our ``special case'' bound to the general setting of $SE(d)$.  With this improvement, for which the author of this dissertation is acknowledged in an erratum to be issued for \cite{bandeira2016se_sync}, \cref{thm:ang_sync_dual} is almost a corollary of their Theorem 12, except that we explicitly prove a basin of attraction, inside which the SDP \eqref{eq:ang_sync_sdp} produces a solution $\hZ = \hz \hz^*$ that may be factorized to obtain a minimizer of \eqref{eq:ang_sync}, and our bound is better by a constant factor of 2 than the bound produced by the improved result of \cite{bandeira2016se_sync}.

As they appear in the original paper, Proposition 2 and Theorem 12 of \cite{bandeira2016se_sync} give us the following.

\begin{proposition}[Proposition 2 and Theorem 12 in \cite{bandeira2016se_sync}]
  There exists a constant $\beta > 0$, depending on $\uL$, such that, if $\lVert L - \uL \rVert_2 < \beta$, then \eqref{eq:ang_sync_real} has a unique solution $\hZ$ which may be factored as $\hZ = R R^*$, with $R = \mfr(\hat{z})$ where $\hat{z} \in (\Sbb^1)^n$ is a global optimizer of \eqref{eq:ang_sync}.  Furthermore, \[\min_{\theta \in [0, 2\pi)} \lVert \hz - \ee^{\ii \theta} \ux \rVert_2 \le 2 \sqrt{\dfrac{n \lVert \uL - L \rVert_2}{\lambda_2(L_G)}},\] where $\lambda_2(L_G)$ is the second smallest eigenvalue of $L_G$.
\end{proposition}

We strengthen this result in \cref{thm:ang_sync_dual} by giving $\beta$ explicitly and by increasing the exponent of $\lVert \uL - L \rVert_2$ in the error bound, which improves the convergence rate as $L \to \uL$.
\medskip
\begin{theorem} \label{thm:ang_sync_dual}
  Given a connected, weighted graph $G = (V = [n], E)$, $W \in \R_{++}^E$ with spectral gap $\tau = \lambda_2(D - W)$ and rotational data $X \in (\Sbb^1)^E$, suppose that $\hz$ is a minimizer of \eqref{eq:ang_sync}, where $L = D - W \circ X$.  By $\ux$ we denote the ground truth, and we take $\uL = D - W \circ \ux \ux^*$ and $\hL = D - W \circ \hz \hz^*$.  Then if $\lVert L - \hL \rVert_2 < \frac{\tau}{1 + \sqrt{n}}$, $\hZ = \hz \hz^*$ and $\mfr(\hZ)$ are the unique minimizers of \eqref{eq:ang_sync_sdp} and \eqref{eq:ang_sync_real}.  In any case, we have \begin{equation} \min_{\theta \in [0, 2\pi)} \lVert \hz - \ee^{\ii \theta} \ux \rVert_2 \le \dfrac{2 \sqrt{2 n} \lVert\uL - L\rVert_2}{\tau}.\label{eq:ang_sync_bound}\end{equation}
  %Then $\hz$ is also the unique minimizer of \cref{eq:ang_sync_sdp,eq:ang_sync_rankone,eq:ang_sync_real} if it satisfies \begin{equation} L - \diag(\Re(L \hz \hz^*)) \succeq 0 \ \text{and} \ \Nul(L - \diag(\Re(L \hz \hz^*))) = \Span(\hz). \label{eq:ang_sync_unique_cond} \end{equation}  In particular, $\hz$ may be found by solving an SDP under these conditions.  Also, these conditions hold if $||W \circ (\hz \hz^* - X)\rVert_2 < \tau$, and we have that \[\min_{\theta \in [0, 2\pi)} || \hz - \ee^{\ii \theta} \ux \rVert_2 \le \dfrac{2 \sqrt{n} ||L - \uL\rVert_2}{\tau}.\]
\end{theorem}

\subsection{Dual Problems}
To prove theorem \ref{thm:ang_sync_dual}, we introduce dual problems for \eqref{eq:ang_sync} and \eqref{eq:ang_sync_real}.  Specifically, we give the Lagrangian function $\Lc : \Cn \times \Rn \to \R$ of \eqref{eq:ang_sync}, \[\Lc(z, \lambda) = z^* L z + \sum_{i = 1}^n \lambda_i (1 - z_i^* z_i) = z^*(L - \diag(\lambda)) z + \sum_{i = 1}^n \lambda_i,\] and the dual function $q : \Rn \to \R$ \[q(\lambda) = \inf_{z \in \Cn} \Lc(z, \lambda),\] which have the properties that for any $\lambda \in \Rn, z \in (\Sbb^1)^n,$ we have \[q(\lambda) \le \Lc(z, \lambda) = z^* L z.\]  In particular, $\sup\limits_{\lambda \in \Rn} q(\lambda) \le \min\limits_{z \in (\Sbb^1)^n} z^* L z$.  Additionally, for any $\lambda \in \Rn$ such that $L - \diag(\lambda) \nsucceq 0,$ we have $q(\lambda) = - \infty$.  Indeed, if $v^*(L - \diag(\lambda))v < 0$, then we may take \[q(\lambda) \le \lim_{t \to \infty} (tv)^* (L - \diag(\lambda)) (tv) = \lim_{t \to \infty} t^2 v^*  (L - \diag(\lambda)) v = -\infty.\]  Otherwise, in the case that $L - \diag(\lambda) \succeq 0,$ the quadratic form $z^* (L - \diag(\lambda)) z$ is minimized when $z = 0$, so \[q(\lambda) = \Lc(0, \lambda) = \sum_{i = 1}^n \lambda_i.\]  Together, this gives $q(\lambda)$ as \[q(\lambda) = \begin{piecewise} \sum_{i = 1}^n \lambda_i & L - \diag(\lambda) \succeq 0 \\ -\infty & L - \diag(\lambda) \nsucceq 0 \end{piecewise}\]  Writing $\Lambda = \diag(\lambda)$, it is clear that the supremum of $\sup q = \sup_{L - \Lambda \succeq 0} \Tr(\Lambda)$, which is an SDP.  Recalling $\sup q \le \min\limits_{z \in (\Sbb^1)^n} z^* L z$, we declare the dual problem of \eqref{eq:ang_sync} to be
\begin{equation}
  \begin{array}{cl} \max\limits_{\Lambda \in \Rnxn} & \Tr(\Lambda) \\
    \text{s.t.} & L - \Lambda \succeq 0 \\
    & \Lambda = \diag(\lambda_1, \ldots, \lambda_n)
  \end{array}, \label{eq:ang_sync_dual}
\end{equation}
where it is ``dual'' in the sense that, if $\Lambda^*$ optimizes \eqref{eq:ang_sync_dual} and $\hz$ optimizes \eqref{eq:ang_sync}, then $\Tr(\Lambda^*) \le \hz^* L \hz$.  

To find the dual of \eqref{eq:ang_sync_sdp}, we define $\Lc_{\SDP} : \H_+^n \times \Rn \to \R$ and $d_{\SDP} : \Rn \to \R$ by
\begin{align*}
  \Lc_{\SDP}(Z, \lambda) &= \Tr(LZ) + \sum_{i = 1}^n \lambda_i(1 - Z_{ii}) = \Tr((L - \diag(\lambda)) Z) + \Tr(\diag(\lambda)), \ \text{and} \\
  d_{\SDP}(\lambda) &= \inf_{Z \in \H^n_+} \Lc_{\SDP}(Z, \lambda) = \begin{piecewise} \Tr(\diag(\lambda)) & L - \diag(\lambda) \succeq 0 \\ -\infty & \text{otherwise} \end{piecewise}
\end{align*}
As before, we see that $\sup\limits_{\lambda \in \Rn} d_{\SDP}(\lambda) \le \Lc_{\SDP}(Z, \lambda) = \Tr(LZ)$ for any $Z$ which is feasible to \eqref{eq:ang_sync_sdp}; therefore, if $\Lambda^*$ and $\hZ$ optimize \eqref{eq:ang_sync_dual} and \eqref{eq:ang_sync_sdp}, we will have $\Tr(\Lambda^*) \le \Tr(L\hZ)$.

To prove the uniqueness of the solution to \eqref{eq:ang_sync_sdp}, we will need to quote a result from \cite{alizadeh1997nondegeneracy}.  They use the primal-dual format with the primal problem
\begin{equation}
  \begin{array}{rlc}
    \displaystyle \min_{X \in \sym^n} & \Tr(C X) \\
    \text{s.t.} & \Tr(A_k X) = b_k, & k \in [m] \\
    & X \succeq 0
  \end{array} \label{eq:alizadeh_primal}
\end{equation}
where $C, A_k \in \sym^n$ and $b \in \Rm$ are fixed.  The dual problem is
\begin{equation}
  \begin{array}{rl}
    \displaystyle\max_{y \in \Rm} & b^T y \\
    \text{s.t.} & C - \sum_{k = 1}^m y_k A_k \succeq 0
  \end{array} \label{eq:alizadeh_dual}
\end{equation}
where $b \in \Rm$ and $A_k$ are as in the primal.  Among other results, they prove the following:

\begin{proposition}[Theorems 9 and 10 in \cite{alizadeh1997nondegeneracy}]
  Suppose that $y \in \Rm$ is dual feasible and optimal, with $\rank(C - \sum_{k = 1}^m y_k A_k) = s$.  Let the columns of $Q \in \R^{n \times n - s}$ be an orthonormal basis for $\Nul(C - \sum_{k = 1}^m y_k A_k)$, such that  \[\Col(Q) = \Nul(C - \sum_{k = 1}^m y_k A_k) \quad \text{and} \quad Q^T Q = I.\] then if \begin{equation} \Span \{Q^T A_k Q\}_{k \in [m]} = \sym^{n - s} \label{eq:dual_nondegen},\end{equation} there is a unqiue optimal primal solution $X$. \label{prop:primal_unique}
\end{proposition}

In order to use this result, we will need the dual of \eqref{eq:ang_sync_real}.  Following \cref{eq:alizadeh_primal,eq:alizadeh_dual}, this gives
\begin{equation}
  \begin{array}{cl}
    \max\limits_{\Lambda \in \sym^{2n}} & \Tr(\Lambda) \\
    \text{s.t.} & \mfr(L) - \Lambda \succeq 0 \\
    & \Lambda = \diag(\Lambda_1, \ldots, \Lambda_n) \\
    & \Lambda_i \in \sym^2
  \end{array} \label{eq:ang_sync_real_dual}
\end{equation}

\subsection{Proof of \Cref{thm:ang_sync_dual}}
With this in mind, we state a few lemmas whose proofs will constitute a proof of \cref{thm:ang_sync_dual}.  Each of these assumes the notation and hypotheses of \cref{thm:ang_sync_dual}.

\begin{lemma}[Sufficient conditions for strong duality]
  If $\hz \in (\Sbb^1)^n$ satisfies \begin{gather} L - \diag\Re(\hz \hz^* L) \succeq 0, \ \text{and} \label{eq:ang_sync_certificate} \\ \Nul(L - \diag\Re(\hz \hz^* L)) = \Span(\hz), \label{eq:ang_sync_nullspace} \end{gather} then $\hZ = \hz \hz^*$ is the unique optimizer of \eqref{eq:ang_sync_sdp} and $\mfr(\hZ)$ is the unique optimizer of \eqref{eq:ang_sync_real}.
  \label{lem:necsuff_ang_sync_dual}
\end{lemma}

\begin{lemma}
  If $\lVert L - \hL\rVert_2 < \frac{\tau}{1 + \sqrt{n}}$, then $\hz$ meets the conditions of \cref{lem:necsuff_ang_sync_dual} and $\hZ = \hz \hz^*$ is the unique optimizer of \eqref{eq:ang_sync_sdp} and $\mfr(\hZ)$ is the unique optimizer of \eqref{eq:ang_sync_real}.
  \label{lem:unique_ang_sync}
\end{lemma}

\begin{lemma} \label{lem:error_ang_sync}
  Suppose that $\hz$ minimizes \eqref{eq:ang_sync}.  Then \[\min_{\theta \in [0, 2\pi)} \lVert \hz - \ee^{\ii \theta} \ux \rVert_2 \le \dfrac{2 \sqrt{2 n} \lVert\uL - L\rVert_2}{\tau}.\]
\end{lemma}

Before proving these, we begin with a further lemma that explains the recurrent $L - \diag \Re (\hz \hz^* L)$ term.

\begin{lemma}\label{lem:first_order_cond}
  Suppose $A \succeq 0$.  Then if $\hz$ is an optimizer of \[\min\limits_{z \in (\Sbb^1)^n} \ z^* A z,\] we have \[\hz \in \Nul (A - \diag \Re (\hz \hz^* A)).\]
\end{lemma}

\begin{proof}[Proof of \cref{lem:first_order_cond}]
  We define $f : (\Sbb^1)^n \to \R$ by $f(z) = z^* A z$.  We observe that $(\Sbb^1)^n$ is an $n$-dimensional manifold with charts given by \[\phi_U : U \to (\Sbb^1)^n,\quad \phi_U(\theta_1, \ldots, \theta_n)_i = \ee^{\ii \theta_i},\] where $U \subset \Rn$ is any open set satisfying $\lVert x - y\rVert_\infty < 2 \pi$ for all $x, y \in U$ (or, more generally, $x - y \notin 2 \pi (\Z^n) \setminus \{0\}$ for all $x, y \in U$).  Assume $\hz$ is a minimizer of \eqref{eq:ang_sync}.  Because $z^* A z$ is smooth on $(\Sbb^1)^n$, for any chart $\phi_U$ with $\hz \in \phi_U(U)$, we must have $\left.\nabla (f \circ \phi_U)\right|_{\phi_U^{-1}(\hz)} = 0$.  In particular, if $\theta \in \Rn$ is such that $\hz_i = \ee^{\ii \theta_i}$, then \[\left.\nabla (f \circ \phi_U)\right|_{\phi_U^{-1}(\hz)} = \nabla_\theta \hz^* A \hz.\]  We remark that \begin{align*} \hz^* A \hz &= \sum_{ij} \ee^{\ii (\theta_j - \theta_i)} A_{ij} = \sum_i \ee^{\ii \theta_i} \hz^* A_i \end{align*} where $A_i = A e_i$ and $A_{ij} = (A_i)_j$.  This gives \begin{align*} \frac{\partial}{\partial \theta_j} \hz^* A \hz &= \ii \ee^{\ii \theta_j} \hz^* A_j + \sum_{i = 1}^n \ee^{\ii \theta_i} \frac{\partial}{\partial \theta_j} \hz^* A_i \\ &= \ii \ee^{\ii \theta_j} \hz^* A_j - \ii \ee^{\ii \theta_j} \sum_{i = 1}^n \ee^{-\ii \theta_i} A_{ji} \\ &= \ii \ee^{\ii \theta_j} \hz^* A_j - \ii\ee^{-\ii \theta_j} A_j^* \hz \\ &= -2 \Im(\ee^{\ii \theta_j} \hz^* A_j) = -2 \Im (\hz \hz^* A)_{jj}\end{align*}  Therefore, $\left.\nabla (f \circ \phi_U)\right|_{\phi_U^{-1}(\hz)} = 0$ if and only if $\diag \Im(\hz \hz^* A) = 0$.  On the other hand, we observe that \begin{align*} ((A - \diag \Re(\hz \hz^* A)) \hz)_i &= A_i^* \hz - \Re(\hz_i \hz^* A_i) \hz_i \\ &= A_i^* \hz - \Re(\overline{\hz}_i A_i^* \hz) \hz_i,\end{align*} such that $\hz \in \Nul(A - \diag \Re(\hz \hz^* A))$ iff
  \[
  \begin{array}{crcll}
    & A_i^* \hz & = & \Re(\overline{\hz}_i A_i^* \hz) \hz_i, & \text{for all}\ i \\
    \iff & \overline{\hz}_i A_i^* \hz & = & \Re(\overline{\hz}_i A_i^* \hz), & \text{for all}\ i \\
    \iff & \overline{\hz}_i A_i^* \hz & \in & \R & \text{for all}\ i \\
    \iff & \diag \Im(\hz \hz^* A) & = & 0
  \end{array}
  \]
  This may be summarized as \begin{equation} z \in \Nul(A - \diag \Re(\hz \hz^* A)) \iff \overline{\hz}_i A_i^* \hz \in \R \ \text{for all}\ i \label{eq:null_iff_real} \end{equation} so that $\hz$ is a minimizer of \eqref{eq:ang_sync} only if $\hz \in \Nul(A - \diag\Re(\hz\hz^* A))$.
\end{proof}

Beyond being a useful result, this suggests that the matrix $L - \diag\Re(\hz\hz^* L)$ can play a role in certifying minima of these optimization problems, as we shall see in the proofs of \cref{lem:necsuff_ang_sync_dual,lem:unique_ang_sync,lem:error_ang_sync}.  We remark, however, that this condition is far from being \emph{sufficient} to show that $\hz$ is an optimizer of $\min\limits_{z \in (\Sbb^1)^n} z^* A z$.  Indeed, even the stronger condition $\Nul(A - \diag\Re(\hz\hz^* A)) = \Span(\hz)$ is insufficient.  For example, if we take $A \in \sym^2$ to be $A = \one \one^*$, then with $z = \one$ we have \[A - \diag \Re (\one \one^* A) = \begin{bmatrix} -1 & 1 \\ 1 & -1 \end{bmatrix},\] so that $\Nul(A - \diag \Re (\one \one^* A)) = \Span(\one)$, but \[\begin{bmatrix} 1 \\ \ee^{\ii \theta} \end{bmatrix}^* A \begin{bmatrix} 1 \\ \ee^{\ii \theta} \end{bmatrix} = 2 + 2 \cos \theta,\] which is \emph{maximized} at $z = \one$ and minimized at $\hz = (1, -1)^T$.  We now prove \cref{lem:necsuff_ang_sync_dual,lem:unique_ang_sync,lem:error_ang_sync}.

\begin{proof}[Proof of \cref{lem:necsuff_ang_sync_dual}]
  We begin by showing that \[\hz \in \Nul(L -  \diag \Re(\hz \hz^* L)) \ \text{and} \ L - \diag \Re(\hz \hz^* L) \succeq 0 \] suffices to show that $\hZ = \hz \hz^*$ and $\mfr(\hZ)$ are optimizers of \eqref{eq:ang_sync_sdp} and \eqref{eq:ang_sync_real}, respectively.  We then show that, if $\Nul(L -  \diag \Re(\hz \hz^* L)) = \Span(\hz)$ holds in addition to these, then they are the \emph{unique} optimizers of their problems.

  Suppose that $\hz \in (\Sbb^1)^n$ satisfies $L - \diag \Re(\hz \hz^* L) \succeq 0$ and $(L - \diag \Re(\hz \hz^* L))\hz = 0$, and set $\hZ = \hz \hz^*$.  This means $\diag \Re (\hz \hz^* L)$ is feasible for \eqref{eq:ang_sync_dual}, so we set $\hat{\Lambda} = \diag \Re(\hz \hz^* L)$ to obtain
  \[
  \begin{array}{rclcll}
    \Tr(\hat{\Lambda}) & = & \Tr(\diag\Re(\hz \hz^* L))& = & \Tr(\Re(\hz \hz^* L)) \\
    & = & \sum_{i = 1}^n \Re(\hz_i \hz^* L_i) & = & \sum_{i = 1}^n \hz_i \hz^* L_i & \text{(from \eqref{eq:null_iff_real})} \\
    & = & \hz^* L \hz
  \end{array}
  \]
  Therefore $(\hz, \hat{\Lambda})$ is a primal-dual pair for \eqref{eq:ang_sync} and \eqref{eq:ang_sync_dual} with equal objective function values, showing that they are both optimizers for their respective problems.  Since \eqref{eq:ang_sync_dual} is also the dual problem for \eqref{eq:ang_sync_sdp}, and since $\Tr(L \hZ) = \Tr(L \hz \hz^*) = \hz^* L \hz$, $\hat{\Lambda}$ also certifies optimality of $\hZ$ for \emph{its} problem, and similarly also for $\mfr(\hZ)$ by observing that
  % $\mfr(\hat{\Lambda})$ and $\mfr(\hZ) = \mfr(\hz) \mfr(\hz)^*$ are feasible for their problems, and
 $\Tr(\mfr(A)) = 2 \Tr(A)$ for general $A \in \H^n$.

  To show uniqueness, we will show that, under the additional assumption that $\Nul(L -  \diag \Re(\hz \hz^* L)) = \Span(\hz)$, then $\mfr(\hat{\Lambda})$ satisfies the hypotheses of \cref{prop:primal_unique}.  To this end, observe that $\Nul(L - \hat{\Lambda}) = \Span(\hz)$ implies \[\Nul(\mfr(L) - \mfr(\hat{\Lambda})) = \Col \mfr(\hz).\]  The two columns of $\mfr(\hz)$ are trivially orthogonal, so we set $Z_i = \mfr(\hz_i)$ and \[Q = \frac{1}{n} \mfr(\hz) = \frac{1}{n} \begin{bmatrix} Z_1 \\ \vdots \\ Z_n \end{bmatrix}.\] Towards proving that $Q$ satisfies \eqref{eq:dual_nondegen}, we note that $Z_i^T Z_i = Z_i Z_i^T = I_2$ and claim that, for $A \in \sym^2$, one block-diagonal pre-image of $A$ is $n^2 \diag(Z_1 A Z_1^T, 0, \ldots, 0)$.  Indeed, \[Q^T n^2 \diag(Z_1 A Z_1^T, 0, \ldots, 0) Q = Z_1^T Z_1 A Z_1^T Z_1 = A.\]  This establishes that $\Span\{Q^T \diag(\Lambda_i) Q\}_{\Lambda_i \in \sym^2} = \sym^2,$ which, by \cref{prop:primal_unique}, gives us that \eqref{eq:ang_sync_real} has a unique solution.  Since we have already established optimality of $\mfr(\hZ)$, this unique solution is $\mfr(\hZ)$.

  Label the feasible sets of \eqref{eq:ang_sync_sdp} and \eqref{eq:ang_sync_real} $F_1$ and $F_2$, respectively.  Then $\mfr(F_1) \subset F_2$, such that if $\mfr(\hZ)$ is the unique minimizer of $\Tr(\mfr(L) Z)$ over $F_2$, then $\hZ$ is the unique minimizer of $\Tr(L Z)$ over $F_1$.  This completes the proof.
\end{proof}

\begin{proof}[Proof of \cref{lem:unique_ang_sync}]
  Accepting the notation of \cref{thm:ang_sync_dual}, suppose that $\hz$ is an optimizer of \eqref{eq:ang_sync}.  For convenience, we set $Y = L - \diag \Re (\hz \hz^* L)$.  Then, by \cref{lem:first_order_cond} we have $Y \hz = 0$.  Therefore, by \cref{lem:necsuff_ang_sync_dual}, to show that the solution to \eqref{eq:ang_sync_sdp} is unique, it suffices to have $P^* Y P \succ 0$, where the columns of $P \in \C^{n \times n - 1}$ form an orthonormal basis for $\hz^\perp$.

  To this end, we introduce $\hL = D - W \circ \hz \hz^*$, which has that $\Nul(\hL) = \Span(\hz)$ and $P^* \hL P \succ 0$ when $G$ is connected (see, e.g., lemma 1.7 of \cite{chungspectral}), so that \[\hL - \diag \Re \hz \hz^* \hL = \hL.\]  By Weyl's inequalities (Theorem 4.3.1 in \cite{horn2012matrix}), the smallest eigenvalue $\lambda_1(P^* Y P)$ of $P^* Y P$ satisfies \[\lambda_1(P^* Y P) \ge \lambda_1(P^* \hL P) - \lVert P^* (Y - \hL) P\rVert_2 = \lambda_2(\hL) - \lVert Y - \hL\rVert_2,\] so it suffices to have $\lambda_2(\hL) = \tau > \lVert Y - \hL\rVert_2$.  The lemma follows by observing that
  \begin{align*}
    \lVert Y - \hL\rVert_2 &\le \lVert L - \hL\rVert_2 + \lVert \diag\Re(\hz \hz^* L)\rVert_2 \\
    &\le \lVert L - \hL\rVert_2 + \lVert L \hz\rVert_\infty \\
    &\le \lVert L - \hL\rVert_2 + \lVert L \hz\rVert_2 \\
    &=   \lVert L - \hL\rVert_2 + \lVert (L - \hL) \hz\rVert_2 \\
    &\le \lVert L - \hL\rVert_2 + \sqrt{n} \lVert L - \hL\rVert_2
  \end{align*}
\end{proof}

\begin{proof}[Proof of \cref{lem:error_ang_sync}]
  We begin by assuming that $\ux^* \hz = \hz^* \ux = |\ux^* \hz|$, which is accomplished by taking appropriate representatives of $\ux$ and $\hz$ in $(\Sbb^1)^n / \Sbb^1$.  This gives that \[\min_{\theta \in [0, 2 \pi)} \lVert \ux - \ee^{\ii \theta} \hz\rVert_2 = \lVert \ux - \hz\rVert_2.\]

  Writing $\Delta L = L - \uL$, we have, by optimality of $\hz$, that $\hz^* L \hz \le \ux^* L \ux$.  Since $\ux \in \Nul(\uL)$, this gives \[\hz^* \uL \hz + \hz^* \Delta L \hz = \hz^* L \hz \le \ux^* L \ux = \ux^* \uL \ux + \ux^* \Delta L \ux = \ux^* \Delta L \ux,\] which yields \begin{equation} \begin{aligned} \hz^* \uL \hz &\le \ux^* \Delta L \ux - \hz^* \Delta L \hz \\ &= (\ux - \hz)^* \Delta L (\ux + \hz) \\ &\le \lVert \ux - \hz\rVert_2 \lVert \Delta L\rVert_2 \sqrt{2 n} \end{aligned} \label{eq:ang_sync_error_bound}\end{equation}  We then lower-bound the left-hand side of \eqref{eq:ang_sync_error_bound} by setting \[y = \mathrm{Proj}_{\ux^\perp} \hz = \hz - \frac{1}{n} \ux \ux^* \hz,\] so that \[\hz^* \uL \hz = y^* \uL y.\]  At this point, we remark that $\lVert \ux \ux^* - \hz \hz^* \rVert_F^2 = 2n^2 - 2 |\ux^* \hz|^2$ and $\lVert \ux - \hz\rVert_2^2 = 2 n - 2|\ux^* \hz|$, such that \begin{align*} n \lVert \ux - \hz\rVert_2^2 &\le \lVert \ux \ux^* - \hz \hz^* \rVert_F^2 = 2 (n - |\ux^* \hz|) (n + |\ux^* \hz|) \\ &= \lVert \ux - \hz\rVert_2^2(n + |\ux^* \hz|) \le 2 n \lVert \ux - \hz\rVert_2^2. \end{align*}  Therefore, we may observe that \[\lVert y \rVert_2^2 = \lVert \hz \rVert_2^2 - \frac{1}{n^2} |\ux^* \hz|^2 \lVert \ux \rVert_2^2 = n - \frac{1}{n} |\ux^* \hz|^2 = \dfrac{\lVert \ux \ux^* - \hz \hz^2\rVert_F^2}{2 n},\] giving $\frac{1}{2} \lVert \ux - \hz\rVert_2^2 \le \lVert y \rVert_2^2 \le \lVert \ux - \hz \rVert_2^2$.  In this way, since $y$ is orthogonal to the null space of $\uL$, we have \begin{equation} \hz^* \uL \hz = y^* \uL y \ge \lambda_2(\uL) \lVert y \rVert_2^2 \ge \frac{\lambda_2(\uL)}{2} \lVert \ux - \hz \rVert_2^2. \label{eq:dist_quad}\end{equation}  Combining this with \eqref{eq:ang_sync_error_bound} completes the proof.
\end{proof}

\subsection{Spanning Tree Strategies}
\label{sec:ang_sync_tree}

By way of practical interest, in this section we briefly consider a strategy for solving the angular synchronization that is cheap to compute and theoretically tight.  Namely, given an angular synchronization problem \eqref{eq:ang_sync} with data $W \circ X \in \H^n, W = \abs{W \circ X}, X = \sgn{W \circ X}$ such that $W$ is the weighted adjacency matrix of a connected graph $G$, we will solve \eqref{eq:ang_sync} by taking $G' \subset G$ to be a spanning tree of $G$ and deciding the phases of $\hz_i$ by arbitrarily fixing the phase of one vertex and directly propagating the relative phases along the edges.  This works by leveraging the property that, for any two vertices $v, w$ in a tree $G' = (V', E')$, there exists a unique path (see, e.g., Theorem 1.5.1 of \cite{diestel2017graphtheory}) $p(v, w) = v v_1 \cdots v_{k-1} w$ from $v$ to $w$.  Therefore, to build an appropriate vector of phases $z$, we may simply fix a root vertex $r$, and set $z_r = 1$.  For each vertex $v$, if $e_1, \ldots, e_k$ are the edges in the path $p(r, v)$ from $r$ to $v$, we set $z_v = \prod_{i = 1}^k X_{e_i}$.\footnote{Here, note that we have used the ordered pair $e_i = (v_i, v'_i)$, representing the edge between $v_i, v'_i \in [n]$, as a subscript denoting for $X$; specifically, $X_{e_i}$ in this instance means $X_{v_i v'_i}$.}  \Cref{alg:span_tree} makes this procedure precise, and \cref{fig:tree_example} illustrates a simple example.

This method is quite popular in the literature, precisely because of its quick implentation and amenability to simple analysis.  In fact, in a paper that preceded the joint work presented in \cref{ch:base_model}, this was the strategy analyzed for phase retrieval with local measurements \cite{IVW2015_FastPhase}; \cref{prop:spinach} somewhat generalizes the bound achieved there.  Of more broad interest to the community is the canonical generalization of \cref{alg:span_tree} to arbitrary group synchronization over a graph, and several authors working on various group synchronization problems -- mostly pose and position estimation -- have used and studied spanning tree-based techniques, including \cite{calafiore2016complex_pgo, calafiore2016planar_pgo, enqvist2011nonsequential,govindu2006motion_avg}.  Notably, because the solution of group synchronization over a spanning tree is so cheap, it admits stochastic consensus methods such as that proposed in \cite{fischler1981ransac}, and this is the strategy followed in \cite{govindu2006motion_avg}.

Before stating \cref{alg:span_tree}, we say that, given a tree $G' = (V' = [n], E')$, $(k_1, \ldots, k_n)$ is a \emph{rooted ordering} of $V'$ if $(k_i)_{i = 1}^n$ is a permutation of $[n]$ such that, for each $i \ge 2$, $k_i$ has exactly one neighbor in $\{k_1, \ldots, k_{i - 1}\}$.  Having fixed a rooted ordering, we call $k_1$ the \emph{root}, and the preceding neighbor (the unique $k_j$ such that $j < i$ and $(k_i, k_j) \in E'$) of $k_i$ is called the \emph{parent} of $k_i$, and neighbors that suceed it are called its \emph{children} or \emph{descendants}.  Indeed, by a breadth-first search \cite{skiena2012algorithmgraph}, we may find, from an adjacency list,\footnote{This is equivalent to an adjacency matrix stored sparsely.  Using the Yale sparse format, which is still standard for everyday sparse computation \cite{bank1993smmp}, vertex $n$ is adjacent to $JA(k)$ for $k \in \clopen{IA(n), IA(n+1)}$.  $IA$ and $JA$ are defined in \cite{bank1993smmp} and stored such that this block is continuous in memory.} a rooted ordering with any vertex as the root with time complexity $\bigO(\abs{V} + \abs{E})$.  With this, we state \cref{alg:span_tree}, which also has a runtime of $\bigO(\abs{V} + \abs{E})$.

\begin{algorithm}[htbp]
\renewcommand{\algorithmicrequire}{\textbf{Input:}}
\renewcommand{\algorithmicensure}{\textbf{Output:}}
\caption{Angular Synchronization on a Spanning Tree}
\label{alg:span_tree}
\begin{algorithmic}[1]
    \REQUIRE A spanning tree $G' = (V = [n], E') \subset G = (V, E)$ of a connected graph.  Edge phase data $X_{ij} \in \Sbb^1$ for $(i, j) \in E$ with $X \in \H^n$.
    \ENSURE A vector of phases $\hz \in (\Sbb^1)^n$.
    \STATE Choose $(k_1, \ldots, k_n)$ to be any rooted ordering of $V$.
    \STATE Set $\hz_1 = 1$.
    \STATE For $i = 2, \ldots, n$, set $\hz_{k_i} = \hz_{k_j} X_{k_i k_j}$ where $k_j$ is the parent of $k_i$.
    \end{algorithmic}
\end{algorithm}

\begin{figure}
  \centering
  \includegraphics[width=\textwidth]{figs/tree}
  \caption{Example of angular synchronization on a spanning tree}
  \label{fig:tree_example}
\end{figure}

We remark that \cref{alg:span_tree} makes no mention of a weight matrix $W$.  This is because the output of \cref{alg:span_tree} uniquely (up to scaling by $\eit$) satisfies $\hz^* L_{G'} \hz = 0$, regardless of $W$, such that the solution solves angular synchronization exactly over the tree.  Additionally, since $L_{G'} \succeq 0$, $\hz$ is an optimum for \eqref{eq:ang_sync} and therefore coincides with the results produced by factorizing the solution to \eqref{eq:ang_sync_sdp} or by taking $z = \sgn(u)$ where $u$ is the eigenvector for $L_{G'}$'s smallest eigenvalue.  We prove this in \cref{prop:spinach}.

\begin{proposition} \label{prop:spinach}
  Let $G' = ([n], E') \subset G = ([n], E)$ be a spanning tree for some connected graph $G$.  Suppose further that $X, \uX \in \H^n$ satisfy $X_{ij}, \uX_{ij} \in \Sbb^1$ if $(i, j) \in E'$ and $X_{ij}, \uX_{ij} = 0$ otherwise.  Let $W \in \R_+^{n \times n} \cap \sym^n$ have $\supp(W) = \supp(X)$.  Then setting $L_{G'} = D - W \circ X$ and $\uL_{G'} = D - W \circ \uX$, where $D = \diag(W \one_n)$, the output $\hz$ of \cref{alg:span_tree} for $X$ satisfies $\Span(\hz) = \Nul(L_{G'})$.  Furthermore, if $\uz$ is the output for $\uX$, we have \begin{equation} \mintheta \norm{\hz - \eit \uz}_2 \le \sqrt{\dfrac{2}{\tau_{G'}}} \norm{X - \uX}_F, \label{eq:spinach_bnd} \end{equation} where $\tau_{G'} = \lambda_2(\diag(\abs{X} \one) - X)$ is the unweighted spectral gap of $G'$.
\end{proposition}

\begin{proof}[Proof of \cref{prop:spinach}]
  Recalling \eqref{eq:ang_sync_cost}, we have \[\hz^* L_{G'} \hz = \sum_{(i, j) \in E'} w_{ij} \abs{\hz_i - X_{ij} \hz_j}^2,\] which immediately equals zero, since if $(i, j) \in E'$, we have set $\hz_j^* \hz_i = X_{ij}$ in line 3 of \cref{alg:span_tree}.  Since $G'$ is connected, $\dim \Nul(L_{G'}) = 1$, so $\Nul(L_{G'}) = \Span(\hz)$.

  To get the error bound, we remark that \cref{alg:span_tree} does not depend on $W$, so, having found $\hz$ from $X$ alone, $\Span(\hz) = \Nul(L_{G'})$ for any weight matrix $W = \abs{W}, \supp(W) = \supp(X)$.  In particular, taking $W = \abs{X}$, so that $L_{G'} = \diag(\abs{X} \one) - X$, we have \[\hz^* \uL_{G'} \hz = \sum_{(i, j) \in E'} \abs{\hz_i - \uX_{ij} \hz_j}^2 = \sum_{(i, j) \in E'} \abs{X_{ij} - \uX_{ij}}^2 = \norm{X - \uX}_F^2.\]  The proof is completed by finding, as in \eqref{eq:dist_quad}, that $\hz^* \uL_{G'} \hz \ge \min_\theta \frac{\tau_{G'}}{2} \norm{\hz - \eit \uz}_2^2$.
\end{proof}

This result demonstrates the real appeal of this method: the solution is exact, and marvelously cheap to compute.  When $G$ is a tree, we can get an exact solution to the SDP of \eqref{eq:ang_sync_sdp} in $\bigO(n)$!  The drawback, however, is that trees suffer from dismal spectral gaps, and this weakens the bound proposed in \cref{prop:spinach}.  From Theorem 4.1 in \cite{deabreu2006algebraicconnectivity} and Theorem 4.2 in \cite{mohar1991diameter} we have that, for a tree $G'$ on $n$ vertices, \begin{equation}\dfrac{4}{n \diam(G')} \le \tau_{G'} \le 2 \left(1 - \cos\left(\dfrac{\pi}{\diam(G') + 1}\right)\right) \le \dfrac{\pi^2}{(\diam(G') + 1)^2},\label{eq:diam_tree}\end{equation} where $\diam(G')$ is the diameter of $G'$.  To make $\tau_{G'}$ large, then, we will want $\diam(G')$ as small as possible, but in the case of the graph $G_{d, \delta}$ associated with\footnote{Specifically, $G_{d, \delta} = ([d], E)$ where $(i, j) \in E$ if $i \neq j$ and $\abs{i - j} \mod d < \delta$.} $T_\delta(\Cdxd)$, the path from $1$ to $\ceil{d/2}$ will have length at least \[\diam(G') \ge \ceil*{\frac{\ceil{d/2} - 1}{\delta - 1}} \ge \dfrac{d}{3 \delta},\] as long as $d \ge 6$.  However, a spanning tree $G' \subset G_{d, \delta}$ with diameter of this asymptotic order is always achievable by setting $G' = ([d], E')$ by setting \[p = p(d, \delta - \delta + 1) = \delta, 2 \delta - 1, \ldots, (d - \delta + 1)\] to be the increasing path of length $\ell = \ceil{\frac{d - 2 \delta + 1}{\delta- 1}}$ from $\delta$ to $d - \delta + 1$ and connecting each vertex $i \in [d] \setminus p$ to any choice $j \in p$ such that $\abs{i - j} \mod d < \delta$.  Then \[\dist(i, j) \le 2 + \ell \le \dfrac{d - 1}{\delta - 1} + 1 \le \frac{3d}{\delta},\] so $\diam(G') \le \frac{3d}{\delta}$.  These considerations give us that we may always construct a spanning tree of $G_{d, \delta}$ with \[\dfrac{4 \delta}{3 d^2} \le \tau_{G'} \le \dfrac{100 \delta^2}{d^2}.\]  Of course, the lower bound is the only one that may be used in the result of \cref{prop:spinach}, and in any event, we know from Corollary 3.2 of \cite{fiedler1973algebraic_connectivity} that $\tau_{G'} \le \tau_G$ for any spanning tree $G' \subset G$.  Nonetheless, we formalize this lower bound in \cref{cor:spinach}.

\begin{corollary}\label{cor:spinach}
  If $G = G_{d, \delta}$, then there exists a spanning tree $G' \subset G$ with $\tau_{G'} \ge \frac{4 \delta}{3 d^2}$.  Using the notation of \cref{prop:spinach}, the output of \cref{alg:span_tree} on $G'$ achieves \begin{equation*} \mintheta \norm{\hz - \eit \uz}_2 \le \sqrt{3/2} \left(\frac{d}{\delta^{1/2}}\right) \norm{X - \uX}_F. \end{equation*}
\end{corollary}

