\subsection{Introduction and Main Result}
We have already presented one angular synchronization result in \cref{sec:Perturb}, which drew largely on \cite{alexeev2014phase}.  We remark that this theorem \BPnote{rephrase: don't want to downplay your own result} leaves something to be desired in that the graph $G$ is not permitted to be weighted, which restricts us from applying some knowledge that we may have about the problem.  For example, suppose our relative phase measurements $\{X_{ij}\}_{(i,j) \in E}$ are disturbed by noise drawn from a fixed, phase-invariant probability distribution, say \[X_{ij} = \sgn(x_i^*x_j + \epsilon_{ij}), \epsilon_{ij} = a_{ij} + \ii\, b_{ij} \ \ \text{with} \ \ a_{ij}, b_{ij} \iid N(0, \sigma^2),\] then we will have more confidence in the relative phases represented by the larger magnitude entries of $X = \mathcal{A}^{-1}(\y)$.  It would be intuitive to use this knowledge to privilege some edges of the graph over others in the frustration function \eqref{eq:frustration} that we are trying to minimize, say by using $w_{ij} = |X_{ij}|$.  Unfortunately, theorem \ref{thm:SpecGraphPertBound} assumes an unweighted graph and its proof technique does not readily admit a satisfactory adjustment towards weighted edges, though we shall impose one later.  Therefore, we will take a distinct approach, drawing upon recent results in the literature that consider certain convex relaxations of \eqref{eq:ang_sync} \cite{bandeira2016tightness, calafiore2016complex_pgo, bandeira2016se_sync}.

To begin this discussion, we gather our notation: $G = (V = [n], E)$ is a connected graph with a weighted adjacency matrix $W = [w_{ij}] \in \sym^n$ satisfying $w_{ij} \ge 0$ and $w_{ij} \neq 0$ only if $(i, j) \in E$.  We take $D = \diag(W \one)$ to be the degree matrix.  $\ux \in (\Sbb^1)^n$ is the ground truth vector, which we attempt to recover, and $X, \uX \in \H^n$ are our noisy and ground truth edge data matrices, satisfying \[X_{ij} = \begin{piecewise} \eta_{ij} x_i x_j^* & (i, j) \in E \\ 0 & \ow \end{piecewise} \quad \text{and} \quad \uX_{ij} = \begin{piecewise} x_i x_j^* & (i, j) \in E \\ 0 & \ow \end{piecewise},\] where $\eta_{ij} = \eta_{ji}^* \in \Sbb^1$ for each $(i, j) \in E$.  We then define $L, \uL$, and $L_G$ as in \eqref{eq:L_defs}.

Towards formulating the appropriate SDP relaxations, we recall the transformation $\mfr : \C \to \R^{2 \times 2}$ %and $\ulmfr : \C \to \R^2$ defined by \[\mfr(a + bi) = \begin{bmatrix} a & -b \\ b & a \end{bmatrix}, \ \ulmfr(a + bi) = \begin{bmatrix} a \\ b \end{bmatrix}.\]
defined by \[\mfr(a + \ii\, b) = \begin{bmatrix} a & -b \\ b & a \end{bmatrix}.\]  It is well-known that $\mfr$ is the canonical isomorphism from $\C$ into $\R^{2 \times 2}$, and indeed if we extend it to matrices by taking, for $A \in \C^{m \times n}$, $\mfr(A) \in \R^{2m \times 2n}$ to be a block matrix with $\mfr(A)_{ij} = \mfr(A_{ij})$, then it remains an isomorphism on $\C^{m \times n}$, and indeed it preserves the eigenvalues and eigenvectors of Hermitian matrices (see, e.g.~\cite[p.~101]{wedderburn1934matrices}).  In particular, a Hermitian matrix $A \in \H^n$ is semi-definite if and only if $\mfr(A)$ is semi-definite; we notice that the multiplicities of its eigenvalues are all doubled, as $A v = \lambda v$ implies $\mfr(A) \mfr(v) = \lambda \mfr(v)$, giving that the two columns of $\mfr(v)$ are each eigenvectors of $\mfr(A)$ with eigenvalue $\lambda$.

With this, we consider SDP relaxations of \eqref{eq:ang_sync}.  Specifically, we observe that $z^* L z = \Tr(L z z^*)$, so an equivalent optimization problem will be
\begin{equation}
  \begin{array}{cl}
    \min\limits_{Z \in \H^n} & \Tr(L Z) \\
    \text{s.t.} & Z_{ii} = 1 \\
    & \rank(Z) = 1 \\
    & Z \succeq 0
  \end{array}, \label{eq:ang_sync_rankone}
\end{equation}
where the optimizer $\hat{z}$ of \eqref{eq:ang_sync} is recovered from the optimal matrix $\hZ$ by merely factoring $\hZ = \hat{z} \hat{z}^*$.  To get a convex relaxation, we simply omit the non-convex rank one constraint, yielding \begin{equation} \begin{array}{cl} \min\limits_{Z \in \H^n} & \Tr(L Z) \\ \text{s.t.} & Z_{ii} = 1 \\ & Z \succeq 0 \end{array}. \label{eq:ang_sync_sdp} \end{equation}  We remark that if an optimizer $\hZ$ of \eqref{eq:ang_sync_sdp} is rank one, then it is also an optimizer of \eqref{eq:ang_sync_rankone} since the feasible set of \eqref{eq:ang_sync_sdp} is strictly larger than that of \eqref{eq:ang_sync_rankone}; in this case, then, factoring $\hZ$ gives the global minimizer of \eqref{eq:ang_sync}.  Considering that many results in the optimization literature are written for real-valued SDPs, we further relax the feasible set by casting into the real domain with
\begin{equation}
  \begin{array}{cl}
    \min\limits_{Z \in \sym^{2n \times 2n}} & \Tr(\mfr(L) Z) \\
    \text{s.t.} & Z_{ii} = I_2 \\
    & Z \succeq 0
  \end{array}, \label{eq:ang_sync_real}
\end{equation}
where $Z_{ii} = [e_{2i - 1} \ e_{2i}]^* Z [e_{2i - 1} \ e_{2i}]$ in this case refers to the $i\th \ 2 \times 2$ diagonal block of $Z$.

%% However, neither of these results may be directly applied, since \cite{bandeira2016tightness} is restricted to  The result stated in \cite{bandeira2016se_sync}, on the other hand, is far more general, applying to synchronization over $SE(n)$, but narrowing to angular synchronization (which is a special case of this problem) allows a considerable improvement over the error bound stated in the paper.  This sharpened result, as we shall see, leads to better guarantees than any we have yet proven.

%% This allows us to restate the SDP of \eqref{eq:angsyncSDPsinger} as a real-valued SDP in 

At this point, we recognize the previous work existing on this problem.  Namely, in \cite{bandeira2016tightness}, \citeauthor*{bandeira2016tightness} prove that the optimizer $\hZ$ of \eqref{eq:ang_sync_sdp} is rank one (and therefore yields a minimizer of \eqref{eq:ang_sync}) when $L$ is sufficiently close to $\uL$.  Unfortunately for our purposes, this paper only considers the case when $G = K_n$ is the complete graph and the weights $W = \one \one^* - I_n$ are constant.  A more general result appears in \cite{bandeira2016se_sync}, where \citeauthor*{bandeira2016se_sync} prove a similar result for synchronization over $SE(d)$, of which angular synchronization is a special case.  Moreover, these results allow for a weighted graph, and include a bound on $\min_{\theta \in [0, 2\pi)} \lVert \hat{z} - \ee^{\ii\theta} \ux\rVert_2$ in terms of $\lVert L - \uL \rVert_2$ and the spectral gap of the graph.  Nonetheless, we find that narrowing to the case of $SO(2)$ (equivalent to angular synchronization) allows for a tighter error bound.  In \cite{calafiore2016complex_pgo}, \citeauthor*{calafiore2016complex_pgo} use methods similar to those in \cite{bandeira2016se_sync} to analyze $SE(2)$ synchronization.  Furthermore, and pertinent to the present work, the authors exchange the rotational components in $SO(2)$ for complex units, but they do not admit weighted graphs, nor do they supply explicit bounds on the error of their estimate or on what level of noise may be tolerated and still guarantee that their convex relaxation solves \eqref{eq:ang_sync} exactly.  Significantly, all three of these works supply an \emph{a posteriori}-certifiable condition that can verify whether the solution obtained is indeed optimal for \cref{eq:ang_sync,eq:ang_sync_sdp,eq:ang_sync_rankone,eq:ang_sync_real}.

Among the results just mentioned, of greatest interest to us are proposition 2 and Theorem 12 in \cite{bandeira2016se_sync}, which may be restated as

\begin{proposition}[Proposition 2 and Theorem 12 in \cite{bandeira2016se_sync}]
  There exists a constant $\beta > 0$, depending on $\uL$, such that, if $\lVert L - \uL \rVert_2 < \beta$, then \eqref{eq:ang_sync_real} has a unique solution $\hZ$ which may be factored as $\hZ = R R^*$, with $R = \mfr(\hat{z})$ where $\hat{z} \in (\Sbb^1)^n$ is a global optimizer of \eqref{eq:ang_sync}.  Furthermore, \[\min_{\theta \in [0, 2\pi)} \lVert \hz - \ee^{\ii \theta} \ux \rVert_2 \le 2 \sqrt{\dfrac{n \lVert \uL - L \rVert_2}{\lambda_2(L_G)}},\] where $\lambda_2(L_G)$ is the second smallest eigenvalue of $L_G$.
\end{proposition}

We strengthen this result in \cref{thm:ang_sync_dual} by giving $\beta$ explicitly and by increasing the exponent of $\lVert \uL - L \rVert_2$ in the error bound, which improves the convergence rate as $L \to \uL$.
\medskip
\begin{theorem}
  Given a connected, weighted graph $G = (V = [n], E)$ with spectral gap $\tau = \lambda_2(D - W)$ and rotational data $X_{ij} \in \Sbb^1$ for $(i, j) \in E$, suppose that $\hz$ is a minimizer of \eqref{eq:ang_sync}, where $L = D - W \circ X$.  By $\ux$ we denote the ground truth, and we take $\uL = D - W \circ \ux \ux^*$ and $\hL = D - W \circ \hz \hz^*$.  Then if $\lVert L - \hL \rVert_2 < \frac{\tau}{1 + \sqrt{n}}$, $\hZ = \hz \hz^*$ and $\mfr(\hZ)$ are the unique minimizers of \eqref{eq:ang_sync_sdp} and \eqref{eq:ang_sync_real}.  In any case, we have \begin{equation} \min_{\theta \in [0, 2\pi)} \lVert \hz - \ee^{\ii \theta} \ux \rVert_2 \le \dfrac{2 \sqrt{2 n} \lVert\uL - L\rVert_2}{\tau}.\label{eq:ang_sync_bound}\end{equation}
  %Then $\hz$ is also the unique minimizer of \cref{eq:ang_sync_sdp,eq:ang_sync_rankone,eq:ang_sync_real} if it satisfies \begin{equation} L - \diag(\Re(L \hz \hz^*)) \succeq 0 \ \text{and} \ \Nul(L - \diag(\Re(L \hz \hz^*))) = \Span(\hz). \label{eq:ang_sync_unique_cond} \end{equation}  In particular, $\hz$ may be found by solving an SDP under these conditions.  Also, these conditions hold if $||W \circ (\hz \hz^* - X)\rVert_2 < \tau$, and we have that \[\min_{\theta \in [0, 2\pi)} || \hz - \ee^{\ii \theta} \ux \rVert_2 \le \dfrac{2 \sqrt{n} ||L - \uL\rVert_2}{\tau}.\]
  \label{thm:ang_sync_dual}
\end{theorem}

\subsection{Dual Problems}
To prove theorem \ref{thm:ang_sync_dual}, we introduce dual problems for \eqref{eq:ang_sync} and \eqref{eq:ang_sync_real}.  Specifically, we give the Lagrangian function $\Lc : \Cn \times \Rn \to \R$ of \eqref{eq:ang_sync}, \[\Lc(z, \lambda) = z^* L z + \sum_{i = 1}^n \lambda_i (1 - z_i^* z_i) = z^*(L - \diag(\lambda)) z + \sum_{i = 1}^n \lambda_i,\] and the dual function $d : \Rn \to \R$ \[d(\lambda) = \inf_{z \in \Cn} \Lc(z, \lambda),\] which have the properties that for any $\lambda \in \Rn, z \in (\Sbb^1)^n,$ we have \[d(\lambda) \le \Lc(z, \lambda) = z^* L z.\]  In particular, $\sup\limits_{\lambda \in \Rn} d(\lambda) \le \min\limits_{z \in (\Sbb^1)^n} z^* L z$.  Additionally, for any $\lambda \in \Rn$ such that $L - \diag(\lambda) \nsucceq 0,$ we have $d(\lambda) = - \infty$.  Indeed, if $v^*(L - \diag(\lambda))v < 0$, then we may take \[d(\lambda) \le \lim_{t \to \infty} (tv)^* (L - \diag(\lambda)) (tv) = \lim_{t \to \infty} t^2 v^* L v = -\infty.\]  Otherwise, in the case that $L - \diag(\lambda) \succeq 0,$ the quadratic form $z^* (L - \diag(\lambda)) z$ is minimized when $z = 0$, so \[d(\lambda) = \Lc(0, \lambda) = \sum_{i = 1}^n \lambda_i.\]  Together, this gives $d(\lambda)$ as \[d(\lambda) = \begin{piecewise} \sum_{i = 1}^n \lambda_i & L - \diag(\lambda) \succeq 0 \\ -\infty & L - \diag(\lambda) \nsucceq 0 \end{piecewise}\]  \BPnote{transition from $d$ to opt.~more gracefully} Therefore, we may maximize $d$ by considering the following optimization problem over the set of diagonal matrices where state the dual problem of \eqref{eq:ang_sync} as
\begin{equation}
  \begin{array}{cl} \max\limits_{\Lambda \in \Rnxn} & \Tr(\Lambda) \\
    \text{s.t.} & L - \Lambda \succeq 0 \\
    & \Lambda = \diag(\lambda_1, \ldots, \lambda_n)
  \end{array}, \label{eq:ang_sync_dual}
\end{equation}
in the sense that, if $\Lambda^*$ optimizes \eqref{eq:ang_sync_dual} and $\hz$ optimizes \eqref{eq:ang_sync}, then $\Tr(\Lambda^*) \le \hz^* L \hz$.  

To find the dual of \eqref{eq:ang_sync_sdp}, we define $\Lc_{SDP} : \H_+^n \times \Rn \to \R$ and $d_{SDP} : \Rn \to \R$ by
\begin{align*}
  \Lc_{SDP}(Z, \lambda) &= \Tr(LZ) + \sum_{i = 1}^n \lambda_i(1 - Z_{ii}) = \Tr((L - \diag(\lambda)) Z) + \Tr(\diag(\lambda)), \ \text{and} \\
  d_{SDP}(\lambda) &= \inf_{Z \in \H^n_+} \Lc_{SDP}(Z, \lambda) = \begin{piecewise} \Tr(\diag(\lambda)) & L - \diag(\lambda) \succeq 0 \\ -\infty & \text{otherwise} \end{piecewise}
\end{align*}
As before, we see that $\sup\limits_{\lambda \in \Rn} d_{SDP}(\lambda) \le \Lc_{SDP}(Z, \lambda) = \Tr(LZ)$ for any $Z$ which is feasible to \eqref{eq:ang_sync_sdp}; therefore, if $\Lambda^*$ and $\hZ$ optimize \eqref{eq:ang_sync_dual} and \eqref{eq:ang_sync_sdp}, we will have $\Tr(\Lambda^*) \le \Tr(L\hZ)$.

To prove the uniqueness of the solution to \eqref{eq:ang_sync_sdp}, we will need to quote a result from \cite{alizadeh1997nondegeneracy}.  They use the primal-dual format with the primal problem
\begin{equation}
  \begin{array}{rlc}
    \displaystyle \min_{X \in \sym^n} & \Tr(C X) \\
    \text{s.t.} & \Tr(A_k X) = b_k, & k \in [m] \\
    & X \succeq 0
  \end{array} \label{eq:alizadeh_primal}
\end{equation}
where $C, A_k \in \sym^n$ and $b \in \Rm$ are fixed.  The dual problem is
\begin{equation}
  \begin{array}{rl}
    \displaystyle\max_{y \in \Rm} & b^T y \\
    \text{s.t.} & C - \sum_{k = 1}^m y_k A_k \succeq 0
  \end{array} \label{eq:alizadeh_dual}
\end{equation}
where $b \in \Rm$ and $A_k$ are as in the primal.  Among other results, they prove the following:

\begin{proposition}[Theorems 9 and 10 in \cite{alizadeh1997nondegeneracy}]
  Suppose that $y \in \Rm$ is dual feasible and optimal, with $\rank(C - \sum_{k = 1}^m y_k A_k) = s$.  Let the columns of $Q \in \R^{n \times n - s}$ be an orthonormal basis for $\Nul(C - \sum_{k = 1}^m y_k A_k)$, such that  \[\Col(Q) = \Nul(C - \sum_{k = 1}^m y_k A_k) \quad \text{and} \quad Q^T Q = I.\] then if \begin{equation} \Span \{Q^T A_k Q\}_{k \in [m]} = \sym^{n - s} \label{eq:dual_nondegen},\end{equation} there is a unqiue optimal primal solution $X$. \label{prop:primal_unique}
\end{proposition}

In order to use this result, we will need the dual of \eqref{eq:ang_sync_real}.  Following \cref{eq:alizadeh_primal,eq:alizadeh_dual}, this gives
\begin{equation}
  \begin{array}{cl}
    \max\limits_{\Lambda \in \sym^{2n}} & \Tr(\Lambda) \\
    \text{s.t.} & \mfr(L) - \Lambda \succeq 0 \\
    & \Lambda = \diag(\Lambda_1, \ldots, \Lambda_n) \\
    & \Lambda_i \in \sym^2
  \end{array} \label{eq:ang_sync_real_dual}
\end{equation}

\subsection{Proof of \Cref{thm:ang_sync_dual}}
With this in mind, we state a few lemmas whose proofs will constitute a proof of \cref{thm:ang_sync_dual}.  Each of these assumes the notation and hypotheses of \cref{thm:ang_sync_dual}.

\begin{lemma}[Sufficient conditions for strong duality]
  If $\hz \in (\Sbb^1)^n$ satisfies \begin{gather} L - \diag\Re(\hz \hz^* L) \succeq 0, \ \text{and} \label{eq:ang_sync_certificate} \\ \Nul(L - \diag\Re(\hz \hz^* L)) = \Span(\hz), \label{eq:ang_sync_nullspace} \end{gather} then $\hZ = \hz \hz^*$ is the unique optimizer of \eqref{eq:ang_sync_sdp} and $\mfr(\hZ)$ is the unique optimizer of \eqref{eq:ang_sync_real}.
  \label{lem:necsuff_ang_sync_dual}
\end{lemma}

\begin{lemma}
  If $\lVert L - \hL\rVert_2 < \frac{\tau}{1 + \sqrt{n}}$, then $\hz$ meets the conditions of \cref{lem:necsuff_ang_sync_dual} and $\hZ = \hz \hz^*$ is the unique optimizer of \eqref{eq:ang_sync_sdp} and $\mfr(\hZ)$ is the unique optimizer of \eqref{eq:ang_sync_real}.
  \label{lem:unique_ang_sync}
\end{lemma}

\begin{lemma}
  Suppose that $\hz$ minimizes \eqref{eq:ang_sync}.  Then \[\min_{\theta \in [0, 2\pi)} \lVert \hz - \ee^{\ii \theta} \ux \rVert_2 \le \dfrac{2 \sqrt{2 n} \lVert\uL - L\rVert_2}{\tau}.\]
    \label{lem:error_ang_sync}
\end{lemma}

Before proving these, we begin with a further lemma that explains the recurrent $L - \diag \Re (\hz \hz^* L)$ term.

\begin{lemma}
  Suppose $A \succeq 0$.  Then if $\hz$ is an optimizer of \[\min\limits_{z \in (\Sbb^1)^n} \ z^* A z,\] we have \[\hz \in \Nul (A - \diag \Re (\hz \hz^* A)).\] \label{lem:first_order_cond}  
\end{lemma}

\begin{proof}[Proof of \cref{lem:first_order_cond}]
  We define $f : (\Sbb^1)^n \to \R$ by $f(z) = z^* A z$.  We observe that $(\Sbb^1)^n$ is an $n$-dimensional manifold with charts given by \[\phi_U : U \to (\Sbb^1)^n,\quad \phi_U(\theta_1, \ldots, \theta_n)_i = \ee^{\ii \theta_i},\] where $U \subset \Rn$ is any open set satisfying $\lVert x - y\rVert_\infty < 2 \pi$ for all $x, y \in U$ (or, more generally, $x - y \notin 2 \pi (\Z^n) \setminus \{0\}$ for all $x, y \in U$).  Assume $\hz$ is a minimizer of \eqref{eq:ang_sync}.  Because $z^* A z$ is smooth on $(\Sbb^1)^n$, for any chart $\phi_U$ with $\hz \in \phi_U(U)$, we must have $\left.\nabla (f \circ \phi_U)\right|_{\phi_U^{-1}(\hz)} = 0$.  In particular, if $\theta \in \Rn$ is such that $\hz_i = \ee^{\ii \theta_i}$, then \[\left.\nabla (f \circ \phi_U)\right|_{\phi_U^{-1}(\hz)} = \nabla_\theta \hz^* A \hz.\]  We remark that \begin{align*} \hz^* A \hz &= \sum_{ij} \ee^{\ii (\theta_j - \theta_i)} A_{ij} = \sum_i \ee^{\ii \theta_i} \hz^* A_i \end{align*} where $A_i = A e_i$ and $A_{ij} = (A_i)_j$.  This gives \begin{align*} \frac{\partial}{\partial \theta_j} \hz^* A \hz &= \ii \ee^{\ii \theta_j} \hz^* A_j + \sum_{i = 1}^n \ee^{\ii \theta_i} \frac{\partial}{\partial \theta_j} \hz^* A_i \\ &= \ii \ee^{\ii \theta_j} \hz^* A_j - \ii \ee^{\ii \theta_j} \sum_{i = 1}^n \ee^{-\ii \theta_i} A_{ji} \\ &= \ii \ee^{\ii \theta_j} \hz^* A_j - \ii\ee^{-\ii \theta_j} A_j^* \hz \\ &= -2 \Im(\ee^{\ii \theta_j} \hz^* A_j) = -2 \Im (\hz \hz^* A)_{jj}\end{align*}  Therefore, $\left.\nabla (f \circ \phi_U)\right|_{\phi_U^{-1}(\hz)} = 0$ if and only if $\diag \Im(\hz \hz^* A) = 0$.  On the other hand, we observe that \begin{align*} ((A - \diag \Re(\hz \hz^* A)) \hz)_i &= A_i^* \hz - \Re(\hz_i \hz^* A_i) \hz_i \\ &= A_i^* \hz - \Re(\overline{\hz}_i A_i^* \hz) \hz_i,\end{align*} such that $\hz \in \Nul(A - \diag \Re(\hz \hz^* A))$ iff
  \[
  \begin{array}{crcll}
    & A_i^* \hz & = & \Re(\overline{\hz}_i A_i^* \hz) \hz_i, & \text{for all}\ i \\
    \iff & \overline{\hz}_i A_i^* \hz & = & \Re(\overline{\hz}_i A_i^* \hz), & \text{for all}\ i \\
    \iff & \overline{\hz}_i A_i^* \hz & \in & \R & \text{for all}\ i \\
    \iff & \diag \Im(\hz \hz^* A) & = & 0
  \end{array}
  \]
  This may be summarized as \begin{equation} z \in \Nul(A - \diag \Re(\hz \hz^* A)) \iff \overline{\hz}_i A_i^* \hz \in \R \ \text{for all}\ i \label{eq:null_iff_real} \end{equation} so that $\hz$ is a minimizer of \eqref{eq:ang_sync} only if $\hz \in \Nul(A - \diag\Re(\hz\hz^* A))$.
\end{proof}

Beyond being a useful result, this suggests that the matrix $L - \diag\Re(\hz\hz^* L)$ can play a role in certifying minima of these optimization problems, as we shall see in the proofs of \cref{lem:necsuff_ang_sync_dual,lem:unique_ang_sync,lem:error_ang_sync}.  We remark, however, that this condition is far from being \emph{sufficient} to show that $\hz$ is an optimizer of $\min\limits_{z \in (\Sbb^1)^n} z^* A z$.  Indeed, even the stronger condition $\Nul(A - \diag\Re(\hz\hz^* A)) = \Span(\hz)$ is insufficient.  For example, if we take $A \in \sym^2$ to be $A = \one \one^*$, then with $z = \one$ we have \[A - \diag \Re (\one \one^* A) = \begin{bmatrix} -1 & 1 \\ 1 & -1 \end{bmatrix},\] so that $\Nul(A - \diag \Re (\one \one^* A)) = \Span(\one)$, but \[\begin{bmatrix} 1 \\ \ee^{\ii \theta} \end{bmatrix}^* A \begin{bmatrix} 1 \\ \ee^{\ii \theta} \end{bmatrix} = 2 + 2 \cos \theta,\] which is \emph{maximized} at $z = \one$ and minimized at $\hz = (1, -1)^T$.  We now prove \cref{lem:necsuff_ang_sync_dual,lem:unique_ang_sync,lem:error_ang_sync}.

\begin{proof}[Proof of \cref{lem:necsuff_ang_sync_dual}]
  We begin by showing that \[\hz \in \Nul(L -  \diag \Re(\hz \hz^* L)) \ \text{and} \ L - \diag \Re(\hz \hz^* L) \succeq 0 \] suffices to show that $\hZ = \hz \hz^*$ and $\mfr(\hZ)$ are optimizers of \eqref{eq:ang_sync_sdp} and \eqref{eq:ang_sync_real}, respectively.  We then show that, if $\Nul(L -  \diag \Re(\hz \hz^* L)) = \Span(\hz)$ holds in addition to these, then they are the \emph{unique} optimizers of their problems.

  Suppose that $\hz \in (\Sbb^1)^n$ satisfies $L - \diag \Re(\hz \hz^* L) \succeq 0$ and $(L - \diag \Re(\hz \hz^* L))\hz = 0$, and set $\hZ = \hz \hz^*$.  This means $\diag \Re (\hz \hz^* L)$ is feasible for \eqref{eq:ang_sync_dual}, so we set $\hat{\Lambda} = \diag \Re(\hz \hz^* L)$ to obtain
  \[
  \begin{array}{rclcll}
    \Tr(\hat{\Lambda}) & = & \Tr(\diag\Re(\hz \hz^* L))& = & \Tr(\Re(\hz \hz^* L)) \\
    & = & \sum_{i = 1}^n \Re(\hz_i \hz^* L_i) & = & \sum_{i = 1}^n \hz_i \hz^* L_i & \text{(from \eqref{eq:null_iff_real})} \\
    & = & \hz^* L \hz
  \end{array}
  \]
  Therefore $(\hz, \hat{\Lambda})$ is a primal-dual pair for \eqref{eq:ang_sync} and \eqref{eq:ang_sync_dual} with equal objective function values, showing that they are both optimizers for their respective problems.  Since \eqref{eq:ang_sync_dual} is also the dual problem for \eqref{eq:ang_sync_sdp}, and since $\Tr(L \hZ) = \Tr(L \hz \hz^*) = \hz^* L \hz$, $\hat{\Lambda}$ also certifies optimality of $\hZ$ for \emph{its} problem, and similarly also for $\mfr(\hZ)$ by observing that
  % $\mfr(\hat{\Lambda})$ and $\mfr(\hZ) = \mfr(\hz) \mfr(\hz)^*$ are feasible for their problems, and
 $\Tr(\mfr(A)) = 2 \Tr(A)$ for general $A \in \H^n$.

  To show uniqueness, we will show that, under the additional assumption that $\Nul(L -  \diag \Re(\hz \hz^* L)) = \Span(\hz)$, then $\mfr(\hat{\Lambda})$ satisfies the hypotheses of \cref{prop:primal_unique}.  To this end, observe that $\Nul(L - \hat{\Lambda}) = \Span(\hz)$ implies \[\Nul(\mfr(L) - \mfr(\hat{\Lambda})) = \Col \mfr(\hz).\]  The two columns of $\mfr(\hz)$ are trivially orthogonal, so we set $Z_i = \mfr(\hz_i)$ and \[Q = \frac{1}{n} \mfr(\hz) = \frac{1}{n} \begin{bmatrix} Z_1 \\ \vdots \\ Z_n \end{bmatrix}.\] Towards proving that $Q$ satisfies \eqref{eq:dual_nondegen}, we note that $Z_i^T Z_i = Z_i Z_i^T = I_2$ and claim that, for $A \in \sym^2$, one block-diagonal pre-image of $A$ is $n^2 \diag(Z_1 A Z_1^T, 0, \ldots, 0)$.  Indeed, \[Q^T n^2 \diag(Z_1 A Z_1^T, 0, \ldots, 0) Q = Z_1^T Z_1 A Z_1^T Z_1 = A.\]  This establishes that $\Span\{Q^T \diag(\Lambda_i) Q\}_{\Lambda_i \in \sym^2} = \sym^2,$ which, by \cref{prop:primal_unique}, gives us that \eqref{eq:ang_sync_real} has a unique solution.  Since we have already established optimality of $\mfr(\hZ)$, this unique solution is $\mfr(\hZ)$.

  Label the feasible sets of \eqref{eq:ang_sync_sdp} and \eqref{eq:ang_sync_real} $F_1$ and $F_2$, respectively.  Then $\mfr(F_1) \subset F_2$, such that if $\mfr(\hZ)$ is the unique minimizer of $\Tr(\mfr(L) Z)$ over $F_2$, then $\hZ$ is the unique minimizer of $\Tr(L Z)$ over $F_1$.  This completes the proof.
\end{proof}

\begin{proof}[Proof of \cref{lem:unique_ang_sync}]
  Accepting the notation of \cref{thm:ang_sync_dual}, suppose that $\hz$ is an optimizer of \eqref{eq:ang_sync}.  For convenience, we set $Y = L - \diag \Re (\hz \hz^* L)$.  Then, by \cref{lem:first_order_cond} we have $Y \hz = 0$.  Therefore, by \cref{lem:necsuff_ang_sync_dual}, to show that the solution to \eqref{eq:ang_sync_sdp} is unique, it suffices to have $P^* Y P \succ 0$, where the columns of $P \in \C^{n \times n - 1}$ form an orthonormal basis for $\hz^\perp$.

  To this end, we introduce $\hL = D - W \circ \hz \hz^*$, which has that $\Nul(\hL) = \Span(\hz)$ and $P^* \hL P \succ 0$ when $G$ is connected (see, e.g., lemma 1.7 of \cite{chungspectral}), so that \[\hL - \diag \Re \hz \hz^* \hL = \hL.\]  By Weyl's inequalities (Theorem 4.3.1 in \cite{horn2012matrix}), the smallest eigenvalue $\lambda_1(P^* Y P)$ of $P^* Y P$ satisfies \[\lambda_1(P^* Y P) \ge \lambda_1(P^* \hL P) - \lVert P^* (Y - \hL) P\rVert_2 = \lambda_2(\hL) - \lVert Y - \hL\rVert_2,\] so it suffices to have $\lambda_2(\hL) = \tau > \lVert Y - \hL\rVert_2$.  The lemma follows by observing that
  \begin{align*}
    \lVert Y - \hL\rVert_2 &\le \lVert L - \hL\rVert_2 + \lVert \diag\Re(\hz \hz^* L)\rVert_2 \\
    &\le \lVert L - \hL\rVert_2 + \lVert L \hz\rVert_\infty \\
    &\le \lVert L - \hL\rVert_2 + \lVert L \hz\rVert_2 \\
    &=   \lVert L - \hL\rVert_2 + \lVert (L - \hL) \hz\rVert_2 \\
    &\le \lVert L - \hL\rVert_2 + \sqrt{n} \lVert L - \hL\rVert_2, \\
  \end{align*}
\end{proof}

\begin{proof}[Proof of \cref{lem:error_ang_sync}]
  We begin by assuming that $\ux^* \hz = \hz^* \ux = |\ux^* \hz|$, which is accomplished by taking appropriate representatives of $\ux$ and $\hz$ in $(\Sbb^1)^n / \Sbb^1$.  This gives that \[\min_{\theta \in [0, 2 \pi)} \lVert \ux - \ee^{\ii \theta} \hz\rVert_2 = \lVert \ux - \hz\rVert_2.\]

  Writing $\Delta L = L - \uL$, we have, by optimality of $\hz$, that $\hz^* L \hz \le \ux^* L \ux$.  Since $\ux \in \Nul(\uL)$, this gives \[\hz^* \uL \hz + \hz^* \Delta L \hz = \hz^* L \hz \le \ux^* L \ux = \ux^* \uL \ux + \ux^* \Delta L \ux = \ux^* \Delta L \ux,\] which yields \begin{equation} \begin{aligned} \hz^* \uL \hz &\le \ux^* \Delta L \ux - \hz^* \Delta L \hz \\ &= (\ux - \hz)^* \Delta L (\ux + \hz) \\ &\le \lVert \ux - \hz\rVert_2 \lVert \Delta L\rVert_2 \sqrt{2 n} \end{aligned} \label{eq:ang_sync_error_bound}\end{equation}  We then lower-bound the left-hand side of \eqref{eq:ang_sync_error_bound} by setting \[y = \mathrm{Proj}_{\ux^\perp} \hz = \hz - \frac{1}{n} \ux \ux^* \hz,\] so that \[\hz^* \uL \hz = y^* \uL y.\]  At this point, we remark that $\lVert \ux \ux^* - \hz \hz^* \rVert_F^2 = 2n^2 - 2 |\ux^* \hz|^2$ and $\lVert \ux - \hz\rVert_2^2 = 2 n - 2|\ux^* \hz|$, such that \begin{align*} n \lVert \ux - \hz\rVert_2^2 &\le \lVert \ux \ux^* - \hz \hz^* \rVert_F^2 = 2 (n - |\ux^* \hz|) (n + |\ux^* \hz|) \\ &= \lVert \ux - \hz\rVert_2^2(n + |\ux^* \hz|) \le 2 n \lVert \ux - \hz\rVert_2^2. \end{align*}  Therefore, we may observe that \[\lVert y \rVert_2^2 = \lVert \hz \rVert_2^2 - \frac{1}{n^2} |\ux^* \hz|^2 \lVert \ux \rVert_2^2 = n - \frac{1}{n} |\ux^* \hz|^2 = \dfrac{\lVert \ux \ux^* - \hz \hz^2\rVert_F^2}{2 n},\] giving $\frac{1}{2} \lVert \ux - \hz\rVert_2^2 \le \lVert y \rVert_2^2 \le \lVert \ux - \hz \rVert_2^2$.  In this way, since $y$ is orthogonal to the null space of $\uL$, we have \begin{equation} \hz^* \uL \hz = y^* \uL y \ge \lambda_2(\uL) \lVert y \rVert_2^2 \ge \frac{\lambda_2(\uL)}{2} \lVert \ux - \hz \rVert_2^2. \label{eq:dist_quad}\end{equation}  Combining this with \eqref{eq:ang_sync_error_bound} completes the proof.
\end{proof}
