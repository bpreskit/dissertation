In this chapter, we consider when a local measurement system $\{m_j\}_{j = 1}^K$ composes a \emph{spanning family} of masks.

\begin{definition}
  We say that $\{m_j\}_{j = 1}^K \subset \C^d$ is a \emph{local measurement system} or \emph{family of masks} of support $\delta$ if $1 \in \supp(m_j)$ and $\supp(m_j) \subset [\delta]$ for each $j$.
\end{definition}

\begin{definition}
  We say that a family of masks $\{m_j\}_{j = 1}^K \subset \C^d$ of support $\delta$ is a \emph{spanning family} if $\Span \{S^\ell m_j m_j^* S^{-\ell}\}_{(\ell, j) \in [d]_0 \times [K]} = T_\delta(\H^d)$.
\end{definition}



%% \section*{Preliminaries}

%% \begin{itemize}
%% \item Indices of matrices in $\C^{d \times d}$ and vectors in $\C^d$ are always taken modulo $d$.
%% \item For $k \in \N, n \in \Z, [k]_n = \{n, n + 1, \ldots, n + k - 1\}$ and $[k] = [k]_1$.
%% \item $S_d \in \R^{d \times d}$ is the $d \times d$ shift operator, such that $(S_d \x)_i = \x_{i - 1}$.  Typically we imply the subscript by context, writing $S$.
%% \item $R \in \R^{d \times d}$ is the operator that reverses a vector's entries, leaving the first entry fixed.  Namely, $(R \x)_i = \x_{2 - i}$.
%% \item Given $\x \in \C^d$ and $k \in [d], \circop_k(\x) \in \C^{d \times k}$ denotes the first $k$ columns of the circulant matrix whose first column is $\x$.  In particular, $\circop_k(\x) e_i = S^{i - 1} \x$ for $i \in [k]$.  When the subscript is omitted, $\circop(\x) := \circop_d(\x)$.
%% \item $\omega_d := \ee^{\frac{2 \pi \ii}{d}}$ is the $d^{\text{th}}$ root of unity.  When context permits, $d$ is implied and we use just $\omega$.
%%   \item For $i, n \in \N, e_i^n \in \R^n$ is the $i^{\text{th}}$ column of the $n \times n$ identity matrix.  When context permits, $n$ is implied and we write $e_i$.  In particular, whenever $e_i$ is used in a matrix multiplication, $n$ is taken to be appropriate so that the multiplication is legal.
%% \item For $k \in \Z, F_k \in \C^{k \times k}$ is the $k \times k$ Fourier matrix with $(F_k)_{ij} = \omega_k^{(i-1)(j-1)}$.
%%   \item For $m, n \in \N, f_m^n = F_m e_n$ is the $n^{\text{th}}$ column of the $m \times m$ unitary Fourier matrix, where $e_n \in \R^m$ has its index taken modulo $m$.
%% \item Given $\x, \y \in \C^d, \x \circ \y$ denotes the Hadamard/elementwise product of $\x$ and $\y$; specifically $(\x \circ \y)_i = \x_i \y_i$.
%% \item Given $A \in \C^{d \times d}, \diag(A, m) \in \C^d$ denotes the $m^{\text{th}}$ circulant off-diagonal of $A$.  That is, $\diag(A, m)_i = A_{i, i + m}$.
%%   \item Given $\x \in \C^d, \diag(\x) \in \C^{d \times d}$ is the diagonal matrix whose diagonal entries are the entries of $\x$.  Namely, $\diag(\x) e_i = \x_i e_i$.  When the intention is clear from context, we may write $D_\x := \diag(\x)$.
%%   \item $\H^d$ is the set of Hermitian matrices in $\C^{d \times d}$, to be viewed as a $d^2$-dimensional vector space over $\R$.
%%     \item $\mathcal{R}_d : \bigcup_{k = 1}^\infty \C^k \to \C^d$ is a universal resize mapping, where for $v \in \C^k,$ \[\mathcal{R}_d(v)_i = \left\{\begin{array}{r@{,\quad}l} v_i & i \le k \\ 0 & \text{otherwise} \end{array}\right. \text{for}\ i \in [d]\]
%% \end{itemize}

\section{Conditions for a spanning family}

\begin{proposition}
  Suppose that $\gamma \in \R^d$ has $1 \in \supp(\gamma) = [\delta].$  Set $D = \min\{2 \delta - 1, d\},$ take $K \ge 2 \delta - 1$ and let \[\begin{array}{r@{\;=\;}r}v_j & \sqrt{K} \mathcal{R}_d(F_K e_j) \\ v_j^D & \sqrt{K} \mathcal{R}_D(F_K e_j) \end{array},\quad j \in [D],\  2 \delta - 1 \le K.\]  Define a local measurement system  $\{m_j\}_{j \in [D]}$ by setting $m_j = \gamma \circ v_j$.  Then $\{m_j\}_{j \in [D]}$ is a spanning family if and only if all the sets $J_k := \{m \in [\delta]_0 : (F_d (\gamma \circ S^{-m} \gamma))_k \neq 0\}$, for all $k \in [d]$ satisfy \[\left\{\begin{array}{r@{,\quad}l} 2 |J_k| - 1 \ge D & 0 \in J_k \\ 2 |J_k| \ge D & \text{otherwise}\end{array}\right..\] \label{prop:spanning_family}
\end{proposition}

%% \begin{remark}
%%   We remark that if $D = 2 \delta - 1$, this condition is equivalent to requiring that $J_k = [\delta]$ for each $k$, which means that $F \circop(g_m)$ has no entries equal to zero for any $m \in [\delta]_0$.  Perhaps more intuitively, this stronger condition means that $F (\gamma \circ S^{-m} \gamma)$ has no zeros for any $m \in [\delta]_0$.
%% \end{remark}

The proof will make use of the following lemmas.

\begin{lemma}
  Define $w_j = \mathcal{R}_{N_1}(f_j^{N_2}), j \in [N_2]$ and set \[\rho_j = \mathfrak{Re}(w_j) \quad\text{and}\quad \mu_j = \mathfrak{Im}(w_j)\] to be vectors containing the real and imaginary components of $w_j$.  Then for $1 \le \ell_1 < \cdots < \ell_k \le \frac{N_2 + 1}{2}$ with $k \le N_1$, we have \begin{align*} \dim\,\Span\{w_{\ell_i}, w_{2 - \ell_i}\}_{i = 1}^k &= \dim\,\Span\{\rho_{\ell_i}, \mu_{\ell_i}\}_{i = 1}^k \\ &= \left\{\begin{array}{r@{,\quad}l} 2k - 1 & \ell_1 = 1 \\ 2k & \text{otherwise}\end{array}\right.,\end{align*} where the indices are taken modulo $N_2$. \label{lem:conjugate_span_dim}
\end{lemma}

\begin{proof}[Proof of lemma \ref{lem:conjugate_span_dim}]

The first equality is clear by considering that $w_{2 - i} = \overline{w_i}$, so $\rho_k = \frac{1}{2}(w_i + w_{2-i})$ and $\mu_i = -\frac{i}{2}(w_i - w_{2 - i})$.  We set $M = \dim\,\Span\{w_{\ell_i}, w_{2 - \ell_i}\}_{i = 1}^k$ to be the common dimension of the two spaces under consideration.

We now divide into two cases: if $N_1 < N_2$, then $\{w_j\}_{j \in [N_2]}$ is full spark, as any $N_1 \times N_1$ submatrix of $\begin{bmatrix} w_1 & \cdots & w_{N_2} \end{bmatrix}$ will be a Vandermonde matrix of the form \[V = \frac{1}{\sqrt{N_2}}\begin{bmatrix} w_{\ell_1} & \cdots & w_{\ell_{N_1}} \end{bmatrix}\] with determinant \[N_2^{-N_1 / 2}\prod_{1 \le i < j \le N_1} (\omega_{N_2}^{\ell_i - 1} - \omega_{N_2}^{\ell_j - 1}),\] which is immediately non-zero since $\omega_{N_2}^{\ell_i - 1} - \omega_{N_2}^{\ell_j - 1} = 0$ only when $\ell_i - \ell_j = 0 \mod N_2$, which cannot happen when $N_1 < N_2$.

When $N_1 \ge N_2, \{w_j\}_{j \in [N_2]}$ is linearly independent, since its members form the matrix $\begin{bmatrix} F_{N_2} \\ 0_{N_1 - N_2 \times N_2} \end{bmatrix}$.

In either case, $M$ is equal to the cardinality of $\{\ell_i, 2 - \ell_i\}_{i = 1}^k,$ which has $2k - 1$ elements if and only if $\ell_1 = 1$; otherwise it has $2k$.  We remark that a collision where $\ell_i = (2 - \ell_i \mod N_2) = N_2 / 2 + 1$ is precluded since we have asserted $\ell_i \le \frac{N_2 + 1}{2}$.

\end{proof}

\begin{lemma}
  For $v \in \R^d,$ we have \begin{align} \circop(v) \rho_k^d &= \frac{1}{2}\mathfrak{Re}((Fv)_k f_k^d) \label{eq:real_part} \\ \circop(v) \mu_k^d &= \frac{1}{2}\mathfrak{Im}((Fv)_k f_k^d) \label{eq:imag_part}. \end{align} In particular, if $(Fv)_k \neq 0$ and $k \notin \{1, \frac{d}{2} + 1\}$, then $\rho_k^d, \mu_k^d \notin \Nul(\circop(v))$; if $k \in \{1, \frac{d}{2} + 1\}$, then $\rho_k^d \notin \Nul(\circop(v))$ and $\mu_k^d = 0$.  On the other hand, if $(Fv)_k = 0$, then $\rho_k^d, \mu_k^d \in \Nul(\circop(v))$.  \label{lem:eigenbits}
\end{lemma}

\begin{proof}[Proof of lemma \ref{lem:eigenbits}]
  We set $\lambda_k^d = (Fv)_k$, and recalling that $\circop(v) = F \diag(F v) F^*$, we observe that \begin{align*}\circop(v) \mu_k^d &= \circop(v) \frac{1}{2} (f_k^d + f_{2 - k}^d) = \frac{1}{2}(\circop(v) f_k^d + \circop(v) f_{2 - k}^d) \\ &= \frac{1}{2}(\lambda_k^d f_k^d + \lambda_{2 - k}^d f_{2 - k}^d).\end{align*}  \eqref{eq:real_part} follows immediately since $\lambda_k^d = \overline{\lambda_{2 - k}^d}$ when $v \in \R^D$.  \eqref{eq:imag_part} follows from an analogous calculation.

  If $\lambda_k^d \neq 0$ and $k \notin \{1, \frac{d}{2} + 1\}$, then $\omega_d^{k - 1}$ is a non-real root of unity and there exists some $j$ such that $\mathfrak{Re}(\omega_d^{(j - 1)(k - 1)} \lambda_k^d) \neq 0,$ and similarly for $\mathfrak{Im}(\omega_d^{(j - 1)(k - 1)} \lambda_k^d) \neq 0.$  When $k \in \{1, \frac{d}{2} + 1\}, \omega_d^{(k - 1)} \in \R$ so $\mu_k^d = 0$, but $\lambda_k^d \in \R$ in this case (because $v \in \R^d$), so $\circop(v) \rho_k^d = \lambda_k^d \rho_k^d \neq 0$.  The claim concerning the case of $\lambda_k^d = 0$ is immediate from \eqref{eq:real_part} and \eqref{eq:imag_part}.
\end{proof}

\begin{proof}[Proof of proposition \ref{prop:spanning_family}]
  For this proof, we set \begin{align*} (\rho_k^d, \mu_k^d) &= (\mathfrak{Re}(f_k^d), \mathfrak{Im}(f_k^d)) \\ (\rho_k, \mu_k) &= (\mathfrak{Re}(v_k), \mathfrak{Im}(v_k)) \\ (\rho_k^D, \mu_k^D) &= (\mathfrak{Re}(v_k^D), \mathfrak{Im}(v_k^D)) \end{align*}
  
  We consider the conditions under which a linear combination of the matrices \[B_\gamma := \{S^\ell m_j m_j^* S^{-\ell}\}_{(\ell, j) \in [d] \times [D]}\] can be equal to zero; by a basic dimension count, $\{m_j\}_{j \in [D]}$ is a spanning family if and only if $B_\gamma$ is linearly independent.  To this end, we define the operator $\mathcal{A}^* : \R^{d \times D} \to \C^{d \times d}$ by \begin{equation} \mathcal{A}^*(C) = \sum_{\ell \in [d], j \in [D]} C_{\ell, j} S^\ell m_j m_j^* S^{-\ell} \label{eq:synth_op} \end{equation} and begin with the observation that, for any $A \in \C^{d \times d}$ we have \[\diag(S^\ell A S^{-\ell}, m) = S^\ell \diag(A, m).\]  We then have
  \[\arraycolsep=1.4pt\def\arraystretch{2}
  \begin{array}{c>{\displaystyle}rcl@{\quad}r}
    & \sum_{j \in [D], \ell \in [d]} C_{\ell, j} S^\ell m_j m_j^* S^{-\ell} & = & 0 & \\
    \iff & \diag\left(\sum_{j \in [D], \ell \in [d]} C_{\ell, j} S^\ell m_j m_j^* S^{-\ell}, m\right) & = & 0 & \text{for all} \ m \in [\delta]_0 \\
    \iff & \sum_{j \in [D], \ell \in [d]} C_{\ell, j} \diag(S^\ell m_j m_j^* S^{-\ell}, m) & = & 0 & \text{for all} \ m \in [\delta]_0 \\
    \iff & \sum_{j \in [D], \ell \in [d]} C_{\ell, j} S^{\ell} \diag(m_j m_j^*, m) & = & 0 & \text{for all} \ m \in [\delta]_0 \\
  \end{array}\]
  
  At this point, we consider that \begin{align} \diag(m_j m_j^*, m) &= \diag((\gamma \circ v_j) (\gamma \circ v_j)^*, m) = \diag(D_{v_j} \gamma \gamma^* D_{\overline{v_j}}, m) \\ &= \omega_K^{m(j - 1)} \diag(\gamma \gamma^*, m). \label{eq:gam_diag}\end{align}  We now set $g_m := \diag(\gamma \gamma^*, m) = \gamma \circ S^{-m} \gamma$ and proceed with the previous chain of implications:
  \[\arraycolsep=1.4pt%\def\arraystretch{1.5}
  \begin{array}{c>{\displaystyle}rcl@{\quad}r}
    & \sum_{j \in [D], \ell \in [d]} C_{\ell, j} S^{\ell} \diag(m_j m_j^*, m) & = & 0 & \text{for all} \ m \in [\delta]_0 \\
    \iff & \sum_{j \in [D], \ell \in [d]} C_{\ell, j} S^{\ell} (\omega_K^{m(j - 1)} g_m) & = & 0 & \text{for all} \ m \in [\delta]_0 \\
    \iff & \sum_{j \in [D], \ell \in [d]} C_{\ell, j} \omega_K^{m(j - 1)} S^{\ell} g_m & = & 0 & \text{for all} \ m \in [\delta]_0 \\
    \iff & \circop(g_m) C \, v^D_{m+1} & = & 0 & \text{for all} \ m \in [\delta]_0 \\
  \end{array}\]

  We now recall that any circulant matrix $\circop(v)$ is diagonalized by the Discrete Fourier Matrix, such that, for $v \in \C^d,$ \[\circop(v) = F_d \diag(\sqrt{d} F_d v) F_d^* = \sqrt{d} \sum_{j = 1}^d (F_d v)_j f_j^d (f_j^d)^*.\]  By writing $\lambda_k^m = \sqrt{d}(F g_m)_k$, we get a natural decoupling of the previous equations: for a fixed $m$, we have that $\circop(g_m) C f_{m + 1} = 0$ if and only if \[\sum_{k = 1}^d \lambda_k^m f_k^d (f_k^d)^* C \, f_{m + 1} = \sum_{k = 1}^d (\lambda_k^m (f_k^d)^* C \, f_{m + 1}) f_k^d = 0.\]  Since this last expression is a linear combination of an orthonormal basis, it occurs only when $\lambda_k^m (f_k^d)^* C \, f_{m + 1} = 0$ for all $k \in [d]$.  We collect these equations over $m \in [\delta]_0$, considering the definition of $J_k$ and that $g_m \in \R^d$ implies $\lambda_k^m = 0 \iff \lambda_{2 - k}^m = 0$ to restate this condition as $\begin{bmatrix} f_k^d & f_{2 - k}^d \end{bmatrix}^* C v^D_{m + 1} = 0$ for all $k \in [d], m \in J_k$.  Since $\Span\{f_k^d, f_{2 - k}^d\} = \Span\{\rho_k^d, \mu_k^d\}$, we further restate this as $\begin{bmatrix} \rho_k^d & \mu_k^d \end{bmatrix}^* C v_{m + 1}^D = 0$ for all $k \in [d], m \in J_k$; setting $W_k = C^* \begin{bmatrix} \rho_k^d & \mu_k^d \end{bmatrix} \in \R^{D \times 2},$ we now get that $\mathcal{A}^*(C) = 0 \iff \Col(W_k) \subset \{v_{m + 1}^D\}_{m \in J_k}^\perp \cap \R^D$ for all $k \in [d]$.

  We now claim that $\mathcal{A}^*$ is invertible if and only if the subspaces $\{v_{m + 1}^D\}_{m \in J_k}^\perp \cap \R^D$ are all trivial.  Indeed, if we fix a $k$ and have some non-zero $u \in \{v_{m + 1}^D\}_{m \in J_k}^\perp \cap \R^D,$ then we may set $C = \rho_k^d u^*$, such that \[\circop(g_m) C v_{m + 1}^D = (\circop(g_m) \rho_k^d) (u^* v_{m + 1}^D).\]  For $m \in J_k, u^* v_{m + 1}^D = 0$ by hypothesis on $u$, and for $m \notin J_k, \circop(g_m) \rho_k^d = 0$ by definition of $J_k$ and lemma \ref{lem:eigenbits}.

  For the other direction, assume $\{v_{m + 1}^D\}_{m \in J_k}^\perp \cap \R^D = 0$ for each $k \in [d]$.  Then $\mathcal{A}^*(C) = 0 \iff \Col(W_k) = \{0\} \iff W_k = 0$ for all $k$.  However, $\{\rho_k^d\}_{k \in [d]} \cup \{\mu_k^d\}_{k \in [d] \setminus \{1, \frac{d}{2} + 1\}}$ is an orthogonal basis for $\R^d$, so \[\begin{array}{cr@{\,=\,}lr} & W_k & 0 & \text{for all} \ k \in [d] \\ \iff & C^* \rho_k^d = C^* \mu_k^d & 0 & \text{for all} \ k \in [d] \\ \iff & C & 0 & \end{array}\]

%  For every $m \in [\delta]_0$ such that $\circop(g_m)$ is non-singular, we must have that $f_{m + 1} \in \Nul(C)$.    Of course, the eigenvalues of $\circop(g_m)$ are given by $F_d g_m = F_d (\gamma \circ S^{-m} \gamma)$, so $\circop(g_m)$ is non-singular exactly when $F_d (\gamma \circ S^{-m})$ has no entries equal to zero (i.e. when $m \in J$).  These conditions then give us that $\Row(C) \subset \{f_{m+1}\}_{m \in J}^\perp \cap \R^{D}$.  Hence $B_\gamma$ is linearly independent if $\{f_{m+1}\}_{m \in J}^\perp \cap \R^{D} = \{0\}$.

  We complete the proof by considering that, for $u \in \R^{D}, \langle v_j^D, u \rangle = 0$ if and only if $\langle \rho_j^D, v \rangle = \langle \mu_j, v \rangle = 0$, so \[\{v_{m + 1}\}_{m \in J_k}^\perp \cap \R^{D} = \{\rho_{m + 1}, \mu_{m + 1}\}_{m \in J_k}^\perp\] which has dimension $\max\{D - (2|J_k| - \one_{0 \in J_k}), 0\}$ by lemma \ref{lem:conjugate_span_dim}.  Therefore, $\mathcal{A}^*$ is invertible if and only if $2|J_k| - \one_{0 \in J_k} \le D$ for all $k \in [d]$, as claimed.
\end{proof}

\begin{remark}
  It turns out that this condition is generic, in the sense that it fails to hold only on a subset of $\R^d$ with Lebesgue measure zero.  We consider that the set of $\gamma \in \R^d$ giving at least one zero in $F(\gamma \circ S^{-m} \gamma)$ is a finite union of zero sets of non-trivial quadratic polynomials (except when $2 \mid d, \delta \ge d / 2,$ and $m = d / 2$, discussed below) and hence a set of zero measure; therefore, $J_k = [\delta]_0$ for all $\gamma$ outside a set of measure zero and $B_\gamma$ is linearly independent under generic conditions.

To address the case of $m = d/2$, we first remark that this is the only possible exception: indeed, when $m \neq d / 2$, we have that \[F((e_1 + e_{m + 1}) \circ S^m(e_1 + e_{m + 1}))_k = f_k^* e_{m + 1} = \omega^{m(k-1)},\] so $\gamma \to F(\gamma \circ S^m \gamma)_k$ is a non-zero, homogeneous quadratic polynomial and therefore has a zero locus of measure zero.

However, when $d = 2m$, then $\gamma \circ S^m \gamma$ is periodic with period $m$ and $F(\gamma \circ S^m \gamma)_{2i} = 0$ for $i \in [m]_0$.  In particular, if $\delta \ge m$, then $D = d$ and $m \notin J_{2i}$ for all $i \in [m]_0$ for any $\gamma$.  In particular, $|J_2| \le \delta - 1$ and $2 |J_2| - \one_{0 \in J_2} \le 2 \delta - 3 $, so if $\delta \in \{d / 2, d/2 + 1\}$,  all choices of $\gamma$ automatically fail to produce a spanning family.

This exception is quite pathological, though: since our intention is to have $\delta \ll d$, this will rarely be an impediment.  Nonetheless, in the case that you \emph{do} want to have $\Span B_\gamma = \H^d$, then taking $\delta > d / 2 + 1$ gives some space for the condition $2|J_k| - \one_{0 \in J_k}$, and we again have that generic $\gamma$ will produce spanning families.
\end{remark}

\section{Condition number}
\begin{gather} \mathcal{A} : \C^{d \times d} \to \R^{[d] \times [D]} \nonumber \\ \mathcal{A}(X)_{(\ell, j)} = \langle S^{\ell} m_j m_j^* S^{-\ell}, X \rangle \label{eq:meas_op} \end{gather}

Now that we have characterized this collection of spanning families, we are interested in the condition number for solving the linear system $y = \mathcal{A}(T_{\delta}(xx^*)) + \eta$ to estimate $T_\delta(xx^*)$.  We begin by introducing the main result of this section:

\begin{proposition}
  Accept the hypotheses of proposition \ref{prop:spanning_family} and define $\mathcal{A}$ as in \eqref{eq:meas_op}.  If we additionally assume that $2 \delta - 1 \le d$ and $K = 2 \delta - 1,$ then the condition number of $\mathcal{A}$ is \begin{equation}\kappa(\mathcal{A}) = \dfrac{\max\limits_{m \in [\delta]_0, j \in [d]} \lvert F_d (\gamma \circ S^{-m} \gamma)_j \rvert}{\min\limits_{m \in [\delta]_0, j \in [d]} \lvert F_d (\gamma \circ S^{-m} \gamma)_j \rvert}.\label{eq:clean_cond}\end{equation}  Otherwise, we may bound the condition number by \begin{equation}\kappa(\mathcal{A}) \le \dfrac{\max\limits_{m \in [\delta]_0, j \in [d]} \lvert F_d (\gamma \circ S^{-m} \gamma)_j \rvert}{\min\limits_{m \in [\delta]_0, j \in [d]} \lvert F_d (\gamma \circ S^{-m} \gamma)_j \rvert} \kappa(\overline{F}_K), \label{eq:messy_cond}\end{equation} where $\overline{F}_K \in \C^{D \times D}$ is the $D \times D$ principal submatrix of $F_K$.
\label{prop:span_fam_cond}
\end{proposition}

To accomplish this, we introduce the operators $P^{(d, N)} : \C^{dN} \to \C^{dN}$, each of which is a permutation defined by \[(P^{(d, N)} v)_{(i - 1)N + j} = v_{(j - 1)d + i}.\]  We can view this is beginning with $v \in C^{dN}$ written as $N$ blocks of $d$ entries, and interleaving them into $d$ blocks each of $N$ entries.  Additionally, for $k, N_1, N_2 \in \N, v \in \C^{kN_1},$ and $H \in \C^{kN_1 \times N_2}$, we define $\circop^{N_1}$ by \begin{align*} \circop^{N_1}(v) &= \begin{bmatrix} v & S_{kN_1}^{N_1} v & \cdots & S_{kN_1}^{(k-1)N_1} v \end{bmatrix} \\ \circop^{N_1}(H) &= \begin{bmatrix} H & S_{k N_1}^{N_1} H & \cdots & S_{k N_1}^{(k-1) N_1}H \end{bmatrix}. \end{align*}  We now proceed with the following lemmas.

\begin{lemma}
  Suppose $v_i, v_{ij} \in \C^k, w_j \in \C^{k N_1}$ for $i \in [N_1], j \in [N_2]$ and \begin{gather*} M_1 = \begin{bmatrix} \circop(v_1) \\ \vdots \\ \circop(v_{N_1}) \end{bmatrix},\quad M_2 = \begin{bmatrix} \circop^{N_1}(w_1) & \cdots & \circop^{N_1}(w_{N_2}) \end{bmatrix},\ \text{and} \\ M_3 = \begin{bmatrix} \circop(v_{11}) & \cdots & \circop(v_{1 N_2}) \\ \vdots & \ddots & \vdots \\ \circop(v_{N_1 1}) & \cdots & \circop(v_{N_1 N_2}) \end{bmatrix}.\end{gather*}  Then \begin{align} P^{(k, N_1)} M_1 &= \circop^{N_1}\left(P^{(k, N_1)} \begin{bmatrix} v_1 \\ \vdots \\ v_{N_1} \end{bmatrix}\right) \label{eq:M_1} \\ M_2 P^{(k, N_2)*} &= \circop^{N_1}\left(\begin{bmatrix} w_1 & \cdots & w_{N_2} \end{bmatrix}\right) \label{eq:M_2} \\ P^{(k, N_1)}M_3P^{(k, N_2)*} &= \circop^{N_1}\left(P^{(k, N_1)} \begin{bmatrix} v_{11} & \cdots & v_{1 N_2} \\ \vdots & \ddots & \vdots \\ v_{N_1 1} & \cdots & v_{N_1 N_2} \end{bmatrix}\right). \label{eq:M_3} \end{align} \label{lem:interleave}
\end{lemma}

\begin{proof}[Proof of lemma \ref{lem:interleave}]
  We index the matrices to check the equalities.  For \eqref{eq:M_1}, we have \begin{align*}
    (P^{(k, N_1)} M_1)_{(a-1)N_1 + b, j} &= (M_1)_{(b - 1) k + a, j} \\ &= \begin{bmatrix} S^{j - 1}v_1 \\ \vdots \\ S^{j - 1} v_{N_1} \end{bmatrix}_{(b - 1)k + a} \\ &= (S^{j - 1}v_b)_a = (v_b)_{a + j - 1}
  \end{align*}
  and
  \begin{align*}
    \circop^{N_1}\left(P^{(k, N_1)} \begin{bmatrix} v_1 \\ \vdots \\ v_{N_1} \end{bmatrix}\right)_{(a-1)N_1 + b, j} &= \left(P^{(k, N_1)} \begin{bmatrix} v_1 \\ \vdots \\ v_{N_1} \end{bmatrix}\right)_{(a - 1)N_1 + b + (j-1)N_1} \\
    &= %\begin{bmatrix} v_1 \\ \vdots \\ v_{N_1} \end{bmatrix}_{(b-1)k + a + j - 1} = 
    (v_b)_{a + j - 1}
  \end{align*}
  For \eqref{eq:M_2}, we have
%  \begin{align*}
 \[(P^{(k, N_2)} M_2^*)_{(a - 1)N_2 + b, j} = (M_2)_{j, (b - 1) k + a} = (w_b)_{j + (a - 1)N_1}\]
%  \end{align*}
  and
  \[\left(\circop^{N_1}\left(\begin{bmatrix} w_1 & \cdots & w_{N_2} \end{bmatrix}\right)\right)_{j, (a - 1)N_2 + b} = (S^{N_1(a - 1)} w_b)_j = (w_b)_{j + N_1(a - 1)}\]

  \eqref{eq:M_3} follows immediately by combining \eqref{eq:M_1} and \eqref{eq:M_2}.
\end{proof}

\begin{lemma}
  Suppose $V \in C^{k N \times m}$, then $\circop^N(V)$ is block diagonalizable by \[\circop^N(V) = \left(F_k \otimes I_N\right) \left(\diag(M_1, \ldots, M_k)\right) \left(F_k \otimes I_m\right)^*,\] where \[\sqrt{k}\left(F_k \otimes I_N\right)^* V = \begin{bmatrix} M_1 \\ \vdots \\ M_k \end{bmatrix}, \quad \text{or} \quad M_j = \sqrt{k} (f_j^k \otimes I_N)^* V\] \label{lem:circ_diag}
\end{lemma}

\begin{proof}[Proof of lemma \ref{lem:circ_diag}]
  We set $V_i$ to be the $k \times m$ blocks of $V$ such that $V^* = \begin{bmatrix} V_1^* & \cdots & V_k^* \end{bmatrix}$ and begin by observing that, for $u \in \C^k$ and $W \in \C^{m \times p}$, the $\ell\th$ $k \times p$ block of $\circop^N(V)(u \otimes W)$ is given by \[\left(\circop^N(V)(u \otimes W)\right)_\ell = \sum_{i = 1}^k u_i (S^{N (i - 1)}V)_\ell W = \sum_{i = 1}^k u_i V_{\ell - i + 1} W.\]  Taking $u = f_j^k$ and $W = I_m$, this gives \begin{align*} \left(\circop^N(V)(f_j^k \otimes I_m)\right)_\ell &= \frac{1}{\sqrt{k}}\sum_{i = 1}^k \omega_k^{(j - 1) (i - 1)} V_{\ell - i + 1} I_m \\ &= \frac{1}{\sqrt{k}} \omega_k^{(j - 1) (\ell - 1)} \sum_{i = 1}^k \omega_k^{-(j - 1)(i - 1)} V_i \\ &= (f_j^k)_\ell \left(\sqrt{k} (f_j^k \otimes I_N)^* V \right) = (f_j^k)_\ell M_j. \end{align*}  This relation is equivalent to having \[\circop^N(V) (f_j^k \otimes I_m) = (f_j^k \otimes M_j) = (f_j^k \otimes I_N) M_j,\] which is the statement of the lemma.
\end{proof}

Lemma \ref{lem:circ_diag} immediately gives the following corollary.

\begin{corollary}
  With notation as in lemma \ref{lem:circ_diag}, the condition number of $\circop^N(V)$ is \[\dfrac{\max\limits_{i \in [k]} \sigma_{\max} (M_i)}{\min\limits_{i \in [k]} \sigma_{\min} (M_i)}.\] \label{cor:circ_diag_condition}
\end{corollary}

We now consider the rows of the measurement operator $\mathcal{A}$ defined in \eqref{eq:meas_op}.  For now, we assume $2 \delta - 1 \le d$, and we vectorize $X \in T_\delta(\C^{d \times d})$ by its diagonals, taking $\chi_m = \diag(X, m), m \in [2 \delta - 1]_{1 - \delta}$, and for now we set $g_m^{j} = \diag(m_j m_j^*, m)$, for now omitting the assumption that $m_j = \gamma \circ v_j$ as in proposition \ref{prop:spanning_family}.  Therefore, each measurement looks like \begin{align*} \mathcal{A}(X)_{(\ell, j)} &= \langle S^{\ell} m_j m_j^* S^{-\ell}, X \rangle \\ &= \sum_{m = 1 - \delta}^{\delta - 1} \langle S^{\ell} g_m^j, \chi_m \rangle,\end{align*} so if we define the matrix $A \in \C^{dD \times (2 \delta - 1)d}$ such that \begin{equation}\left(A \begin{bmatrix} \chi_{1 - \delta} \\ \vdots \\ \chi_{\delta - 1} \end{bmatrix}\right)_{(j-1) d + \ell} = \mathcal{A}(X)_{(\ell, j)}, \label{eq:vectorized_meas}\end{equation} the $(j - 1) d + \ell^{\text{th}}$ row of $A$ is given by \[\begin{bmatrix} S^{\ell} g_{1 - \delta}^j \\ \vdots \\ S^{\ell} g_{\delta - 1}^j \end{bmatrix}^*,\] such that $A$ is the block matrix given by \[A = \begin{bmatrix} \circop(g_{1 - \delta}^1)^* & \cdots & \circop(g_{\delta - 1}^1)^* \\ \vdots & \ddots & \vdots \\ \circop(g_{1 - \delta}^D)^* & \cdots & \circop(g_{\delta - 1}^D)^* \end{bmatrix} = \begin{bmatrix} \circop(R g_{1 - \delta}^1) & \cdots & \circop(R g_{\delta - 1}^1) \\ \vdots & \ddots & \vdots \\ \circop(R g_{1 - \delta}^D) & \cdots & \circop(R g_{\delta - 1}^D) \end{bmatrix},\]  which may be transformed, by Lemma \ref{lem:interleave}, to \begin{equation}P^{(d, D)} A P^{(d, 2 \delta - 1)*} = \circop^D\left(P^{(d, D)} \begin{bmatrix} R g_{1 - \delta}^1 & \cdots & R g_{\delta - 1}^1 \\ \vdots & \ddots & \vdots \\ R g_{1 - \delta}^D & \cdots & R g_{\delta - 1}^D \end{bmatrix}\right) =: \circop^D(H). \label{eq:interleaved_meas}\end{equation}

%We label the $D \times 2\delta - 1$ blocks of $H$ by $H^* = \begin{bmatrix} H_1^* & \cdots & H_d^* \end{bmatrix}$, so that $(H_\ell)_{ij} = (R g_{j - \delta}^i)_\ell = (g_{j - \delta}^i)_{2 - \ell}$.
Quoting corollary \ref{cor:circ_diag_condition}, this establishes the next proposition.

\begin{proposition}
  Setting $M_j = \sqrt{d}\left(f_j^d \otimes I_D\right)^* H$, the condition number of $A$ is \[\kappa(A) = \dfrac{\max\limits_{i \in [d]} \sigma_{\max} (M_i)}{\min\limits_{i \in [d]} \sigma_{\min} (M_i)}.\] \label{prop:meas_cond}
\end{proposition}

We are now able to prove proposition \ref{prop:span_fam_cond}.

\begin{proof}
  For the moment, we assert that $D = 2 \delta - 1 \le d$ and set $\overline{F}_K \in \C^{2 \delta - 1 \times 2 \delta - 1}, (\overline{F}_K)_{ij} = \frac{1}{\sqrt{K}}\omega_K^{(i-1)(j-\delta)}$ to be the principal submatrix of $\sqrt{K} \diag(f^K_{1 - \delta}) F_K$.  In this case, $g_m^j = \diag(m_j m_j^*, m) = \omega_K^{m(j - 1)} g_m$, as in \eqref{eq:gam_diag}.  Therefore, we label the $2 \delta - 1 \times 2 \delta - 1$ blocks of $H$ by $H^* = \begin{bmatrix} H_1^* & \cdots & H_d^* \end{bmatrix}$, so that \[(H_\ell)_{ij} = (R g_{j - \delta}^i)_\ell = \omega_K^{(i - 1)(j - \delta)}(R g_{j - \delta})_\ell\] and $M_\ell = \sum_{k = 1}^d \omega_d^{(\ell - 1)(k - 1)} H_k,$ giving \begin{align*} (M_\ell)_{ij} &= \sum_{k = 1}^d \omega_d^{(\ell - 1)(k - 1)} (H_k)_{ij} = \omega_K^{(i - 1)(j - \delta)} \sum_{k = 1}^d \omega_d^{(\ell - 1)(k - 1)} (Rg_{j - \delta})_k \\ &= \omega_K^{(i - 1)(j - \delta)} (F_d^* g_{j - \delta})_\ell. \end{align*}  In other words, $M_\ell = \sqrt{K} \diag(f_\ell^{d*} g_{1 - \delta},\, \ldots\, , f_\ell^{d*} g_{\delta - 1}) \overline{F}_K$.  If $K = 2 \delta - 1$, then $\overline{F}_K$ is unitary, and the singular values of $M_\ell$ are $\{\sqrt{K} f_\ell^{d *} g_j\}_{j = 1 - \delta}^{\delta - 1}$.  Recognizing that $S^j g_j = g_{-j}$, then proposition \ref{prop:meas_cond} takes us to \eqref{eq:clean_cond}.

  If $D = 2 \delta - 1 < K$, then the argument remains unchanged, except that the singular values of $M_\ell$, instead of being known explicitly, are bounded above and below by $\max\limits_{|j| < \delta} |f_\ell^{d *} g_j| \sigma_{\max}(\overline{F}_K)$ and $\min\limits_{|j| < \delta} |f_\ell^{d *} g_j| \sigma_{\min}(\overline{F}_K)$ respectively, which gives the more general result of \eqref{eq:messy_cond}.

  If $2 \delta - 1 > d$, then instead of using diagonals $1 - \delta, \ldots, \delta - 1$, we use diagonals $0, 1, \ldots, d - 1$.  This change propagates from \eqref{eq:vectorized_meas} to \eqref{eq:interleaved_meas}, so that \[(H_\ell)_{ij} = \omega_K^{(i - 1)(j - 1)} (R g_{j - 1})_\ell \quad \text{and} \quad (M_\ell)_{ij} = \omega_K^{(i - 1)(j - 1)} (F_d^* g_{j - 1})_\ell,\] giving $M_\ell = \sqrt{K} \diag(f_\ell^{d*} g_0,\, \ldots\, , f_\ell^{d*} g_{d - 1})\mathcal{R}_{d \times d}(F_K)$, which immediately gives us \eqref{eq:messy_cond}.  We remark that indexing only over the diagonals $m \in [\delta]_0$ in \eqref{eq:messy_cond} suffices, again because $S^j g_j = g_{-j}$, so having $2 \delta - 1 > d$ makes $1 - \delta, \ldots, -1$ redundant.
  
\end{proof}

