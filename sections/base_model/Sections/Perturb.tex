In this section we present a simpler (and easier to derive), albeit weaker,  perturbation result in the spirit of Section \ref{sec:Perturb}, which is associated with the analysis of line 3 of Algorithm~\ref{alg:phaseRetrieval1}.

Specifically, we will derive an upper bound on $\min_{\theta\in[0,2\pi]} \Vert \tx_0 - \mathbbm{e}^{\mathbbm{i}\theta}\tx \Vert_2$ (provided by Theorem~\ref{cor2:GenBound}), which scales like $d^3$.  While this dependence is strictly worse than the one derived in Section \ref{sec:Perturb}, it is easier to obtain and the technique may be of independent interest. 

We will begin with a result concerning the top eigenvector of any Hermitian matrix.

\begin{lemma}\label{lem:GenBound}
Let $X_0 = \sum^d_{j=1} \nu_j \x_j\x_j^*$ be Hermitian with eigenvalues $\nu_1 \geq \nu_2 \geq \dots \geq \nu_d$ and orthonormal eigenvectors $\x_1, \dots, \x_d \in \mathbbm{C}^d$.  Suppose that $X = \sum^d_{j=1} \lambda_j \v_j\v_j^*$ is Hermitian with eigenvalues $\lambda_1 \geq \lambda_2 \geq \dots \geq \lambda_d$, orthonormal eigenvectors $\v_1, \dots, \v_d \in \mathbbm{C}^d$, and $\Vert X-X_0 \Vert_{\rm F}~\leq \eta \Vert X_0 \Vert_F$ for some $\eta \geq 0$.  Then,
\[ \left( 1 - | \langle \x_1, \v_1 \rangle |^2 \right)  \leq  \frac{4 \eta^2 \Vert X_0 \Vert^2_{\rm F} }{(\nu_1- \nu_2)^2} .\]
\end{lemma}

\begin{proof}
An application of the $\sin \theta$ theorem \cite{davis1970rotation,stewart1990matrix} (see, e.g., the proof of Corollary 1 in \cite{yu2015useful}) tells us that 
\begin{equation*}
\sin \left( \arccos \left( | \langle \x_1, \v_1 \rangle | \right) \right) \leq \frac{2 \eta \Vert X_0 \Vert_F}{|\nu_1 - \nu_2|}.
\end{equation*}
Squaring both sides we then learn that 
\begin{equation}
 \left( 1 - | \langle \x_1, \v_1 \rangle |^2 \right)  = \sin^2 \left( \arccos \left( | \langle \x_1, \v_1 \rangle | \right) \right) \leq \frac{4 \eta^2 \Vert X_0 \Vert^2_F}{(\nu_1 - \nu_2)^2},
\end{equation}
giving us the desired inequality.  
\end{proof}

The following variant of Lemma~\ref{lem:GenBound} concerning rank 1 matrices $X_0$ is of use in the analysis of many other phase retrieval methods, and can be used, e.g., to correct and simplify the proof of equation (1.8) in Theorem 1.3 of \cite{candes2014solving}.

\begin{lemma}\label{cor:rank1Bound}
  Let $\x_0 \in \mathbb C^d$, set $X_0 = \x_0\x_0^*$, and let $X \in \mathbb C^{d \times d}$ be Hermitian with $\Vert X~-~X_0 \Vert_F~\leq \eta \Vert X_0 \Vert_F = \eta \Vert \x_0\Vert_2^2$ for some $\eta \geq 0$.  Furthermore, let $\lambda_i$ be the $i$-th largest magnitude eigenvalue of $X$ and $\v_i \in \mathbb{C}^d$ an associated eigenvector, such that the $\v_i$ form an orthonormal eigenbasis.  Then \[ \min_{\theta \in [0, 2 \pi]} \Vert  \mathbbm{e}^{\mathbbm{i} \theta}  \x_0 - \sqrt{|\lambda_1 |} \v_1 \Vert_2 \leq (1+2 \sqrt{2}) \eta \Vert \x_0 \Vert_2.\footnote{\emph{It is interesting to note that similar bounds can also be obtained using simpler techniques (see, e.g., \cite{CandesFix}).}} \]
\end{lemma}

\begin{proof}
In this special case of Lemma~\ref{lem:GenBound} we have $\nu_1 = \Vert X_0 \Vert_F = \Vert \x_0\Vert_2^2$ and $\x_1 := \x_0 / \Vert \x_0 \Vert$.  Choose $\phi \in [0, 2 \pi]$ such that $\langle \mathbbm{e}^{\mathbbm{i} \phi} \x_0, \v_1 \rangle  = \vert \langle  \x_0, \v_1 \rangle \vert$.  Then, 
%
\begin{align}
 \Vert \mathbbm{e}^{\mathbbm{i} \phi} \x_0 - \sqrt{\nu_1} \v_1 \Vert_2^2 &= 2\nu_1 - 2 \nu_1  \cdot \vert \langle  \x_0 / \Vert \x_0 \Vert, \v_1 \rangle \vert = 2\nu_1 - 2 \nu_1  \cdot \vert \langle  \x_1, \v_1 \rangle \vert \nonumber \\ &\leq 2 \nu_1 \left( 1 - \vert \langle  \x_1, \v_1 \rangle \vert \right) \left( 1 + \vert \langle  \x_1, \v_1 \rangle \vert \right) \label{equ:CorIntermed} \\ &= 2 \nu_1 \left( 1 - | \langle \x_1, \v_1 \rangle |^2 \right) \leq 8 \eta^2 \Vert X_0 \Vert_{\rm F} \nonumber
\end{align}
where the last inequality follows from Lemma~\ref{lem:GenBound} with $\nu_1 = \Vert X_0 \Vert_F = \Vert \x_0\Vert_2^2$.  Finally, by the triangle inequality, Weyl's inequality (see, e.g., \cite{horn2012matrix}), and \eqref{equ:CorIntermed}, we have
\begin{align*}
  \Vert \mathbbm{e}^{\mathbbm{i} \phi} \x_0 - \sqrt{ |\lambda_1|} \v_1 \Vert_2 & \leq \Vert 
    \mathbbm{e}^{\mathbbm{i} \phi} \x_0 -\sqrt \nu_1 \v_1 \Vert_2 + \Vert \sqrt \nu_1 \v_1 - 
    \sqrt{|\lambda_1|} \v_1 \Vert_2 \\
    & \leq 2 \sqrt{2} \cdot \eta \sqrt \nu_1 + \left \vert \sqrt \nu_1 - \sqrt{|\lambda_1|} \right \vert \\ & \leq 2 \sqrt{2} \cdot \eta \sqrt 
        \nu_1 + \frac{| \nu_1 - \lambda_1 |}{\sqrt \nu_1 + \sqrt{|\lambda_1|}} \\
     & \leq 2 \sqrt{2} \cdot \eta \sqrt \nu_1 + \frac{\eta \nu_1}{\sqrt \nu_1 + \sqrt{|\lambda_1|}}\\ 
    & \leq (1+2 \sqrt{2}) \eta \sqrt \nu_1.
\end{align*}
The desired result now follows.
\end{proof}

We may now use Lemma~\ref{lem:GenBound} to produce a perturbation bound for our banded matrix of phase differences $\tX_0$ from \eqref{eq:X}.

\begin{theorem}\label{cor2:GenBound}
Let $\tX_0 = T_{\delta}(\tx_0 \tx_0^*)$ where $|(\tx_0)_i| = 1$ for each $i$.  Further suppose $\tX \in T_{\delta}(\H^d)$ has $\tx$ as its top eigenvector, where $||\tx||_2 = \sqrt{d}$.  Suppose that $\Vert \tX_0 - \tX \Vert_F \le \eta \Vert \tX_0 \Vert_F$ for some $\eta>0$.  Then, there exists an absolute constant $C \in \mathbb{R}^+$ such that
\[\min_{\theta\in[0,2\pi]} \Vert \tx_0 - \mathbbm{e}^{\mathbbm{i}\theta} \tx \Vert_2 \le C\bigfrac{\eta d^3}{\delta^{\frac{5}{2}}}.\]
\end{theorem}

\begin{proof}
Recall that the phase vectors $\tilde{\x}$ and $\tilde{\x}_0$ are normalized so that $\| \tilde{\x} \|_2 = \| \tilde{\x}_0 \|_2 = \sqrt{d}$.  Combining Lemmas~\ref{lem:EigGap} and \ref{lem:GenBound} after noting that $\| \tX_0 \|^2_{\rm F} = d (2 \delta - 1)$ we learn that 
\begin{equation}
\left( 1 - \frac{1}{d^2}| \langle \tilde{\x}_0, \tilde{\x} \rangle |^2 \right)  \leq  C' \eta^2 \left( \frac{d}{\delta} \right)^5
\label{equ:cor2GenBIP}
\end{equation}
for an absolute constant $C' \in \mathbb{R}^+$.  Let $\phi \in [0,2 \pi)$ be such that $\operatorname{Re}\left( \langle \tilde{\x}_0,\mathbbm{e}^{\mathbbm{i} \phi} \tilde{\x} \rangle \right) = \left| \langle \tilde{\x}_0, \tilde{\x} \rangle \right|$.  Then,
\begin{align*}
\| \tilde{\x}_0 - \mathbbm{e}^{\mathbbm{i} \phi} \tilde{\x} \|^2_2 &= 2d - 2 \operatorname{Re}\left( \langle \tilde{\x}_0,\mathbbm{e}^{\mathbbm{i} \phi} \tilde{\x} \rangle \right) \\
&= 2d \left( 1 - \frac{1}{d} \left| \langle \tilde{\x}_0, \tilde{\x} \rangle \right| \right) \leq 2d \left( 1 - \frac{1}{d^2} \left| \langle \tilde{\x}_0, \tilde{\x} \rangle \right|^2 \right).
\end{align*}
Combining this last inequality with \eqref{equ:cor2GenBIP} concludes the proof.
\end{proof}
