%Let $\delta \in \N$ be such that $\delta \le \lfloor \frac{d+1}{2} \rfloor$.  Theorem~\ref{thm:Noiseless_Spectrum} guarantees that $\widetilde{X}_0$ from \eqref{equ:MatrixofPhases} has $\widetilde{\x}_0$ as its (unique) top eigenvector with eigenvalue $\nu_1 = 2 \delta -1$.  This combined with the fact that $\| \widetilde{X}_0 \|_{\rm F} = \sqrt{d (2 \delta - 1)}$ and ${\rm Tr}(\widetilde{X}_0) = d$ suggests that line 5 of Algorithm~\ref{alg:phaseRetrieval} is likely to produce a reasonably accurate approximation to $\widetilde{\x}_0$ as long as $\| \n \|_2$ is sufficiently small and $\delta$ is sufficiently large.  Of course, more detailed knowledge regarding the spectral properties of $\widetilde{X}_0$ are required before such observations can be made precise.  In this section the spectral properties of $\widetilde{X}_0$ will be derived, thereby allowing a detailed study of Algorithm~\ref{alg:phaseRetrieval}'s robustness to arbitrary measurement noise $\n$.

Consider line 3 of Algorithm~\ref{alg:phaseRetrieval1}, which shows that we are trying to recover $\widetilde{\x}_0:= \frac{{\x_0}}{|\x_0|}$ via an eigenvector method. Here, we show that  $\widetilde{X}_0$ has $\widetilde{\x}_0$ as its top eigenvector and we investigate the spectral properties of $\widetilde{X}_0$ in this section, following the intuition that the eigenvalue gap $|\lambda_1 - \lambda_2|$ will affect the robustness of the spectral step in the algorithm.

For the remainder of the paper, we let $\mathbbm{1}$ refer to a constant vector of all ones; its size will always be determined by context.  To begin, consider $U = T_\delta(\mathbbm{1}\mathbbm{1}^*)$, i.e., 
\begin{equation}
U_{j,k} =  \left\{ \begin{array}{ll} 1 & \textrm{if}~| j - k |~{\rm mod}~d  < \delta \\ 0 & \textrm{otherwise} \end{array} \right..
\label{equ:NormedUmatSPG}
\end{equation}
Observe that $U$ is circulant for all $\delta$, so its eigenvectors are always discrete Fourier vectors.  Setting $\omega_j = \ee^{2\pi\ii\frac{j-1}{d}}$ for $j = 1, 2, \ldots, d$, one can also see that the eigenvalues of $U$ are given by 
\begin{equation}
\nu_j = \sum_{k = 1}^d (U)_{1,k}\omega_j^{k-1} = 1 + \sum_{k = 1}^{\delta - 1}\omega_j^k + \omega_j^{-k} ~=~ 1 + 2 \sum_{k = 1}^{\delta - 1}\cos\left(\bigfrac{2\pi (j-1) k}{d}\right),
\label{eqn:lambda}
\end{equation}
for all $j = 1, \dots, d$.  In particular, $\nu_1 = 2 \delta - 1$.  Set $\Lambda = \diag\{\nu_1, \ldots, \nu_{d}\}$ and let $F$ denote the unitary $d \times d$ discrete Fourier matrix with entries 
$$F_{j,k} := \frac{1}{\sqrt{d}} \ee^{2\pi\ii\frac{(j-1)(k-1)}{d}},$$ then $U = F \Lambda F^*$.

We consider that $\widetilde{X}_0$ and $U$ are similar; indeed $\widetilde{\X}_0 = \widetilde{D}_0 U \widetilde{D}_0^*$, where $\widetilde{D}_0 = \diag\{(\widetilde{x}_0)_1, \ldots, (\widetilde{x}_0)_d\}$.  Since $|(\widetilde{\x}_0)_j| = 1$ for each $j$, we have that $\widetilde{D}_0$ is unitary.  Thus the eigenvalues of $\widetilde{X}_0$ are given by \eqref{eqn:lambda}, and its eigenvectors are simply the discrete Fourier vectors modulated by the entries of $\widetilde{\x}_0$.  %
We now have the following lemma.

\begin{lemma}
Let $\widetilde{\X}_0$ be defined as in \eqref{eq:X_0}.  Then $$\widetilde{\X}_0 = \widetilde{D}_0 F \Lambda F^* \widetilde{D}_0^*$$ where $F$ is the unitary $d \times d$ discrete Fourier transform matrix, $\widetilde{D}_0$ is the $d \times d$ diagonal matrix $\diag\{(\widetilde{x}_0)_1, \ldots, (\widetilde{x}_0)_d\}$, and $\Lambda$ is the $d \times d$ diagonal matrix $\diag\{\nu_1, \ldots, \nu_{d}\}$ where
$$\nu_j := 1 + 2 \sum_{k = 1}^{\delta - 1}\cos\left(\bigfrac{2\pi (j-1) k}{d}\right)$$
for $j = 1, \dots, d$.
\label{lem:spectrum}
\end{lemma}

We next estimate the principal eigenvalue gap of $\widetilde{\X}_0$.  This information will be crucial to our understanding of the stability and robustness of Algorithm~\ref{alg:phaseRetrieval1}.  

\subsection{The Spectral Gap of $\widetilde{\X}_0$}

Set $\theta_j = \frac{2 \pi j}{d}$ and begin by observing that (by, for example, \cref{lem:dirichlet}), for any $\theta \in \R$,
\begin{eqnarray*}
  \sum_{k = 1}^{\delta - 1}\cos(\theta k) %& = & \bigfrac{\sin(\theta / 2)}{\sin(\theta / 2)}\sum_{k = 1}^{\delta - 1}\cos(\theta k) \\
  %& = & \bigfrac{1}{2\sin(\theta / 2)} \sum_{k=1}^{\delta - 1}\left(\sin(\theta(k + 1 / 2)) - \sin(\theta(k - 1 / 2))\right) \\
  %& = & \bigfrac{1}{2\sin(\theta / 2)}\left(\sin(\theta(\delta - 1/2)) - \sin(\theta / 2)\right) \\
  & = & \bigfrac{1}{2}\left(\bigfrac{\sin(\theta(\delta - 1/2))}{\sin(\theta / 2)} - 1\right).
\end{eqnarray*}
Accordingly, defining $l_\delta : \R \to \R$ by $l_\delta(\theta) := 1 + 2\sum_{k=1}^{\delta - 1} \cos(\theta k)$ we have that
\begin{eqnarray}
  \nu_{j+1} %& = & l_\delta(\theta_j) = 1 + 2 \sum_{k = 1}^{\delta - 1}\cos(\theta_j k) \notag \\
  & = & l_\delta(\theta_j) ~=~ \bigfrac{\sin(\theta_j(\delta - 1/2))}{\sin(\theta_j / 2)}. \label{equ:EigenFormula}
\end{eqnarray}
Thus, the eigenvalues of $\widetilde{\X}_0$ are sampled from the $(\delta-1)^{\rm st}$ Dirichlet kernel.  
Of course, $\nu_1 = 2\delta - 1$ is the largest of these in magnitude, so the eigenvalue gap $\min_j \nu_1 - |\nu_j|$ is at most equal to
\begin{eqnarray*}
  \nu_1 - \nu_2 & = & (2\delta -  1) - \bigfrac{\sin(\pi / d (2\delta - 1))}{\sin(\pi / d)} \\
  & \le & (2 \delta - 1) - \bigfrac{\pi / d(2 \delta - 1) - \frac{1}{6}(\pi / d (2 \delta - 1))^3}{\pi / d} \\
  & = & \frac{1}{6}\left(\frac{\pi}{d}\right)^2(2 \delta - 1)^3 \le \frac{4\pi^2}{3} \frac{\delta^3}{d^2} \label{equ:gap12}.
\end{eqnarray*}
Thus, $\nu_1 - |\nu_2 | \lesssim \frac{\delta^3}{d^2}$.  However, a lower bound on the spectral gap is more useful.  The following lemma establishes that the spectral gap is indeed $\sim \frac{\delta^3}{d^2}$ for most reasonable choices of $\delta < d$.

\begin{lemma}
Let $\nu_1 = 2 \delta -1, \nu_2, \dots, \nu_d$ be the eigenvalues of $\widetilde{\X}_0$.  Then
$$\min_{j \in \{2, 3, \dots, d \} } (\nu_1 - |\nu_j| ) \geq \dfrac{\pi^2}{3} \frac{\delta^3}{d^2}$$
whenever $d \geq 4\delta$ and $\delta \geq 3$.
% More generally, defining the function $l_{\delta, d} : \Z \to \R$ by 
%\[l_{\delta, d}(k) = (\delta - 1) - \sum_{i=1}^{\delta - 1} \cos\left(\frac{2 \pi i}{d} \right),\] we have that $\min_{k \in [d]} l_{\delta, d}(k) \simeq \delta^3 / d^2$.
\label{lem:EigGap}
\end{lemma}

\begin{proof}

Let $\theta_j = \frac{2 \pi j}{d}$.  We find the lower bound by considering that $\theta_j \in [\pi / d, 2\pi - \pi / d]$ for every $j > 0$, so 
\[\nu_1 - \max | \nu_j | \ge \nu_1 - \max_{\theta \in [\pi / d, 2\pi - \pi / d]} | l_\delta(\theta) | = (2 \delta - 1) - \max_{\theta \in [\pi / d, \pi]} | l_\delta(\theta) |,\]where we have used our eigenvalue formula from \eqref{equ:EigenFormula}, and the symmetry of $l_\delta$ about $\theta = \pi$.

We now show that $l_\delta$ is decreasing towards its first zero at $\theta = \frac{2 \pi}{2 \delta - 1}$ by considering the derivative \[l_\delta'(\theta) = \bigfrac{(\delta - 1/2)\cos((\delta - 1/2)\theta) \sin(\theta/2) - 1/2 \sin((\delta - 1/2)\theta) \cos(\theta / 2)}{\sin(\theta / 2) ^ 2},\] which is non-positive if and only if \[(2\delta - 1) \sin(\theta/2)\cos((\delta - 1/2)\theta)  \le  \sin((\delta - 1/2)\theta) \cos(\theta /2 ).\]  Since $\tan(\cdot)$ is convex on $[0,\, \pi / 2)$, this last inequality will hold for $\theta \in [0,\, \frac{\pi}{2\delta - 1})$.  For $\theta \in [\frac{\pi}{2\delta-1}, \frac{2\pi}{2\delta -1})$, $\cos((\delta - 1/2)\theta) \leq 0$ while the remainder of the terms are non-negative, so the inequality also holds.  Therefore,
\[\nu_1 - \max_{j > 1} |\nu_j| \ge (2 \delta - 1) - \max \left\{\nu_2, \max_{\theta \in [\frac{2\pi}{2\delta-1}, \pi]} |l_\delta(\theta)| \right\},\]
which permits us to bound $(2 \delta - 1) - \nu_2$ and $(2 \delta - 1) - \max\limits_{\theta \in [\frac{2\pi}{2\delta-1}, \pi]} |l_\delta(\theta)|$ separately.

%% For $\max\limits_{\theta \in [\frac{2\pi}{2\delta-1}, \pi]} |l_\delta(\theta)|$, we consider the Dirichlet kernel \eqref{equ:EigenFormula} %, \[l_\delta(\theta) = \bigfrac{\sin((\delta - 1/2)\theta)}{\sin(\theta / 2)},\] 
%% which allows us to devise an upper bound by considering it as a sine wave modulated by a decreasing envelope in the usual fashion; therefore, we take the first non-negative half-cycle of $\sin((\delta - 1/2)\theta)$ and bound $l_\delta$ there.  More specifically,
%% $$  \max_{\theta \in [\frac{2\pi}{2\delta-1}, \pi]} | l_\delta(\theta) |  \le \left| \sin \left(\frac{2\pi}{2\delta - 1} \right) \right|^{-1}  \le \bigfrac{2\delta - 1}{4},$$
%% where the last line uses that $\bigfrac{2 \pi}{2\delta - 1} \le \pi / 2$ (since $\delta \ge 3$).  If this is indeed the top eigenvalue, it gives an eigenvalue gap of $\bigfrac{3}{4}(2\delta - 1)$, which we shall see is much too large.

For $\max\limits_{\theta \in [\frac{2\pi}{2\delta-1}, \pi]} |l_\delta(\theta)|$, we simply observe that \[\max\limits_{\theta \in [\frac{2\pi}{2\delta-1}, \pi]} |l_\delta(\theta)| \le \max\limits_{\theta \in [\frac{2\pi}{2\delta-1}, \pi]} \dfrac{1}{\sin(\theta / 2)} = \left( \sin \left(\frac{\pi}{2\delta - 1} \right) \right)^{-1}  \le \bigfrac{2\delta - 1}{2},\] where the last line uses that $\bigfrac{\pi}{2\delta - 1} \le \pi / 2$ (since $\delta \ge 3$).  This yields $\nu_1 - \max\limits_{\theta \in [\frac{2\pi}{2\delta-1}, \pi]} |l_\delta(\theta)| \ge \frac{1}{2}(2 \delta - 1)$.

As for $\nu_2$, we have $\theta_1 \cdot (\delta - 1) \le \pi / 2$ (since $4(\delta - 1) \le d$). Thus, $\cos(\cdot)$ will be concave on $[0,\, \theta_1(\delta-1)]$.  Considering \eqref{eqn:lambda}, this will give $\sum_{k=1}^{\delta - 1}\cos(k\theta_1) \le (\delta - 1) \cos \left(\theta_1 \frac{\delta}{2} \right)$, so \[\begin{array}{rcl}
  \nu_1 - \nu_2 & \ge & 2(\delta - 1) \left(1 - \cos \left(\pi\frac{\delta}{d} \right) \right)\\
  & \ge & 2(\delta - 1) \left(\dfrac{(\pi \frac{\delta}{d})^2}{4}\right) \\
  & \ge & \dfrac{\pi^2}{3} \cdot \dfrac{\delta^3}{d^2}.
\end{array}\]
The stated result follows, since $d \ge 4 \delta$ gives \[\nu_1 - \max\limits_{\theta \in [\frac{2\pi}{2\delta-1}, \pi]} |l_\delta(\theta)| \ge \frac{1}{2}(2 \delta - 1) \ge \dfrac{1}{2} \delta \ge \dfrac{1}{3}\left(\dfrac{\pi \delta}{d}\right)^2 \delta = \dfrac{\pi^2}{3} \cdot \dfrac{\delta^3}{d^2}.\]
\end{proof}

We are now sufficiently well informed about $\widetilde{\X}_0$ to consider perturbation results for its leading eigenvector.
