Let $\delta \in \N$ be such that $\delta \le \lfloor \frac{d+1}{2} \rfloor$.  Theorem~\ref{thm:Noiseless_Spectrum} guarantees that $\tilde{X}_0$ from \eqref{equ:MatrixofPhases} has $\tilde{\x}_0$ as its (unique) top eigenvector with eigenvalue $\nu_1 = 2 \delta -1$.  This combined with the fact that $\| \tilde{X}_0 \|_{\rm F} = \sqrt{d (2 \delta - 1)}$ and ${\rm Tr}(\tilde{X}_0) = d$ suggests that line 5 of Algorithm~\ref{alg:phaseRetrieval} is likely to produce a reasonably accurate approximation to $\tilde{\x}_0$ as long as $\| \n \|_2$ is sufficiently small and $\delta$ is sufficiently large.  Of course, more detailed knowledge regarding the spectral properties of $\tilde{X}_0$ are required before such observations can be made precise.  In this section the spectral properties of $\tilde{X}_0$ will be derived, thereby allowing a detailed study of Algorithm~\ref{alg:phaseRetrieval}'s robustness to arbitrary measurement noise $\n$.

To begin, consider $\u = \mathbbm{1}$ with $U = \u\u^*$ and $\tilde{U}$ defined along the lines of \eqref{equ:MatrixofPhases} so that
\begin{equation}
\tilde{U}_{j,k} =  \left\{ \begin{array}{ll} 1 & \textrm{if}~| j - k ~{\rm mod}~d | < \delta \\ 0 & \textrm{otherwise} \end{array} \right..
\label{equ:NormedUmatSPG}
\end{equation}
Observe that $\tilde{U}$ is circulant for all $\delta$, so its eigenvectors are always discrete Fourier vectors.  Setting $\omega_j = \ee^{2\pi\ii\frac{j-1}{d}}$ for $j = 1, 2, \ldots, d$, one can also see that the eigenvalues of $\tilde{U}$ are given by 

\begin{eqnarray}
\nu_j & = & \sum_{k = 1}^d (\tilde{U})_{1,k}\omega_j^{k-1} ~=~  \sum_{k = 1}^\delta \omega_j^{k-1} + \sum_{k=d-\delta + 2}^d \omega_j^{k-1} \notag \\
& = & 1 + \sum_{k = 1}^{\delta - 1}\omega_j^k + \omega_j^{-k} ~=~ 1 + 2 \sum_{k = 1}^{\delta - 1}\cos\left(\bigfrac{2\pi (j-1) k}{d}\right),
\label{eqn:lambda}
\end{eqnarray}
for all $j = 1, \dots, d$.  Thus, $\nu_1 = 2 \delta - 1$ as promised by Theorem~\ref{thm:Noiseless_Spectrum}.  Set $\Lambda = \diag\{\nu_1, \ldots, \nu_{d}\}$ and let $F$ denote the unitary $d \times d$ discrete Fourier matrix with entries 
$$F_{j,k} := \frac{1}{\sqrt{d}} \ee^{2\pi\ii\frac{(j-1)(k-1)}{d}}.$$ 
Then, \begin{equation}\tilde{U} = F \Lambda F^*\label{eq:Utilde}.\end{equation}

Returning to arbitrary $\x_0$, we set $D_0 = \diag\{(x_0)_1, \ldots, (x_0)_d\}$ and observe that $\x_0 = D \u$, so that $\X_0 = D_0 \tilde{U} D_0^*$ and $\tilde{\X}_0 = \tilde{D}_0 \tilde{U} \tilde{D}_0^*$, where $\tilde{D}_0 = \diag\{(\tilde{x}_0)_1, \ldots, (\tilde{x}_0)_d\}$.  Since $|(\tilde{\x}_0)_j| = 1$ for each $j$, we have that $\tilde{D}_0$ is unitary.  Thus, $\tilde{\X}_0$ is diagonalized as
\begin{equation} \tilde{\X}_0 = \tilde{D}_0 F \Lambda F^* \tilde{D}_0^*. \label{eqn:diagonalization}\end{equation}
Therefore, the eigenvalues of $\tilde{\X}_0$ are given by \eqref{eqn:lambda}, and its eigenvectors are always simply the discrete Fourier vectors modulated by the entries of $\tilde{\x}_0$.  
%%Finally, we note that $\tilde{\X}$ from line 4 of Algorithm~\ref{alg:phaseRetrieval} will also have this exact same structure (except with $\tx$ replacing $\tx_0$ above).  In particular, its eigenvalues will always match those of $\tilde{\X}_0$, and its eigenvectors will always be modulated discrete Fourier vectors.  
We now have the following lemma.

\begin{lem}
Let $\tilde{\X}_0$ be defined as in \eqref{equ:MatrixofPhases}.  Then $$\tilde{\X}_0 = \tilde{D}_0 F \Lambda F^* \tilde{D}_0^*$$ where $F$ is the unitary $d \times d$ discrete Fourier transform matrix, $\tilde{D}_0$ is the $d \times d$ diagonal matrix $\diag\{(\tilde{x}_0)_1, \ldots, (\tilde{x}_0)_d\}$, and $\Lambda$ is the $d \times d$ diagonal matrix $\diag\{\nu_1, \ldots, \nu_{d}\}$ where
$$\nu_j := 1 + 2 \sum_{k = 1}^{\delta - 1}\cos\left(\bigfrac{2\pi (j-1) k}{d}\right)$$
for $j = 1, \dots, d$.
\label{lem:spectrum}
\end{lem}

We next investigate the minimum eigenvalue gap for $\tilde{\X}_0$.  This information will be crucial to our understanding of the stability and robustness of Algorithm~\ref{alg:phaseRetrieval}.  

\subsection{The Spectral Gap of $\tilde{\X}_0$}

Set $\theta_j = \frac{2 \pi j}{d}$ and begin by observing that, for any $\theta \in \R$,
\begin{eqnarray*}
  \sum_{k = 1}^{\delta - 1}\cos(\theta k) %& = & \bigfrac{\sin(\theta / 2)}{\sin(\theta / 2)}\sum_{k = 1}^{\delta - 1}\cos(\theta k) \\
  %& = & \bigfrac{1}{2\sin(\theta / 2)} \sum_{k=1}^{\delta - 1}\left(\sin(\theta(k + 1 / 2)) - \sin(\theta(k - 1 / 2))\right) \\
  %& = & \bigfrac{1}{2\sin(\theta / 2)}\left(\sin(\theta(\delta - 1/2)) - \sin(\theta / 2)\right) \\
  & = & \bigfrac{1}{2}\left(\bigfrac{\sin(\theta(\delta - 1/2))}{\sin(\theta / 2)} - 1\right).
\end{eqnarray*}

Accordingly, defining $l_\delta : \R \to \R$ by $l_\delta(\theta) := 1 + 2\sum_{k=1}^{\delta - 1} \cos(\theta k)$ we have that
\begin{eqnarray}
  \nu_{j+1} %& = & l_\delta(\theta_j) = 1 + 2 \sum_{k = 1}^{\delta - 1}\cos(\theta_j k) \notag \\
  & = & l_\delta(\theta_j) ~=~ \bigfrac{\sin(\theta_j(\delta - 1/2))}{\sin(\theta_j / 2)}. \label{equ:EigenFormula}
\end{eqnarray}
Thus, the eigenvalues of $\tilde{\X}_0$ are sampled from the $(\delta-1)^{\rm st}$ Dirichlet kernel.  
Of course, $\nu_1 = 2\delta - 1$ is the largest of these in magnitude, so the eigenvalue gap is at most equal to
\begin{eqnarray*}
  \nu_1 - \nu_2 & = & (2\delta -  1) - \bigfrac{\sin(\pi / d (2\delta - 1))}{\sin(\pi / d)} \\
  & \le & (2 \delta - 1) - \bigfrac{\pi / d(2 \delta - 1) - \frac{1}{6}(\pi / d (2 \delta - 1))^3}{\pi / d} \\
  & = & \frac{1}{6}\left(\frac{\pi}{d}\right)^2(2 \delta - 1)^3 \label{equ:gap12}.
\end{eqnarray*}
Thus, $\nu_1 - |\nu_2 |$ is $\mathcal{O}\left(\frac{\delta^3}{d^2} \right)$.  However, a lower bound on the spectral gap is more useful.  The following lemma establishes that the spectral gap is indeed $\sim \frac{\delta^3}{d^2}$ for most reasonable choices of $\delta < d$.

\begin{lem}
Let $\nu_1 = 2 \delta -1, \nu_2, \dots, \nu_d$ be the eigenvalues of $\tilde{\X}_0$.  Then, there exists a positive absolute constant $C \in \R^+$ such that
$$\min_{j \in \{2, 3, \dots, d \} } (\nu_1 - |\nu_j| ) \geq C \frac{\delta^3}{d^2}$$
whenever $d \geq 4(\delta - 1)$ and $\delta \geq 3$. % More generally, defining the function $l_{\delta, d} : \Z \to \R$ by 
%\[l_{\delta, d}(k) = (\delta - 1) - \sum_{i=1}^{\delta - 1} \cos\left(\frac{2 \pi i}{d} \right),\] we have that $\min_{k \in [d]} l_{\delta, d}(k) \simeq \delta^3 / d^2$.
\label{lem:EigGap}
\end{lem}

\begin{proof}

Let $\theta_j = \frac{2 \pi j}{d}$.  We find the lower bound by considering that $\theta_j \in [\pi / d, 2\pi - \pi / d]$ for every $j > 0$, so 
\begin{eqnarray*}
  \nu_1 - \max | \nu_j |& \ge & \nu_1 - \max_{\theta \in [\pi / d, 2\pi - \pi / d]} | l_\delta(\theta) | \\
  & = & (2 \delta - 1) - \max_{\theta \in [\pi / d, \pi]} | l_\delta(\theta) |
\end{eqnarray*}
where we have used our eigenvalue formula from \eqref{equ:EigenFormula}, and the symmetry of $l_\delta$ about $\theta = \pi$.

Consider now that 
\begin{align*}
  l_\delta'(\theta)  &=  \bigfrac{(\delta - 1/2)\cos((\delta - 1/2)\theta) \sin(\theta/2) - 1/2 \sin((\delta - 1/2)\theta) \cos(\theta / 2)}{\sin(\theta / 2) ^ 2} 
   \le  0\\ 
  & \iff  \\
  &(2\delta - 1) \sin(\theta/2)\cos((\delta - 1/2)\theta)  \le  \sin((\delta - 1/2)\theta) \cos(\theta /2 ). 
\end{align*}
Since $\tan$ is convex on $[0,\, \pi / 2)$, this last inequality will hold for $\theta \in [0,\, \frac{\pi}{2\delta - 1})$.  For $\theta \in [\frac{\pi}{2\delta-1}, \frac{2\pi}{2\delta -1})$, $\cos((\delta - 1/2)\theta) \leq 0$ while the remainder of the terms are non-negative, so the inequality also holds.  In addition, $l_\delta(\frac{2\pi}{2\delta - 1}) = 0$, so $l_\delta$ is decreasing on its way to its first zero, and does not become negative until $\theta > \frac{2\pi}{2\delta - 1}$.  Therefore,
\[\nu_1 - \max_{j > 1} |\nu_j| \ge (2 \delta - 1) - \max \left\{\nu_2, \max_{\theta \in [\frac{2\pi}{2\delta-1}, \pi]} |l_\delta(\theta)| \right\},\]
which permits us to look at $(2 \delta - 1) - \nu_2$ and $(2 \delta - 1) - \max\limits_{\theta \in [\frac{2\pi}{2\delta-1}, \pi]} |l_\delta(\theta)|$ separately.

For $\max\limits_{\theta \in [\frac{2\pi}{2\delta-1}, \pi]} |l_\delta(\theta)|$, we consider the Dirichlet kernel \eqref{equ:EigenFormula} %, \[l_\delta(\theta) = \bigfrac{\sin((\delta - 1/2)\theta)}{\sin(\theta / 2)},\] 
which allows us to devise an upper bound by considering it as a sine wave modulated by a decreasing envelope in the usual fashion; therefore, we take the first non-negative half-cycle of $\sin((\delta - 1/2)\theta)$ and bound $l_\delta$ there.  More specifically,
$$  \max_{\theta \in [\frac{2\pi}{2\delta-1}, \pi]} | l_\delta(\theta) |  \le \left| \sin \left(\frac{2\pi}{2\delta - 1} \right) \right|^{-1}  \le \bigfrac{2\delta - 1}{4},$$
where the last line uses that $\bigfrac{2 \pi}{2\delta - 1} \le \pi / 2$ (since $\delta \ge 3$).  If this is indeed the top eigenvalue, it gives an eigenvalue gap of $\bigfrac{3}{4}(2\delta - 1)$, which we shall see is much too large.

Now for $\nu_2$, we have $\theta_1 \cdot (\delta - 1) \le \pi / 2$ (since $4(\delta - 1) \le d$). Thus, $\cos$ will be concave on $[0,\, \theta_1(\delta-1)]$.  Considering \eqref{eqn:lambda}, this will give $\sum_{k=1}^{\delta - 1}\cos(k\theta_1) \le (\delta - 1) \cos \left(\theta_1 \frac{\delta}{2} \right)$, so

\begin{eqnarray*}
  \nu_1 - \nu_2 & \ge & 2(\delta - 1) \left(1 - \cos \left(\pi\frac{\delta}{d} \right) \right)\\
  & \ge & 2(\delta - 1) \left(\pi^2\frac{\delta^2}{2d^2} - \pi^4\frac{\delta^4}{24d^4} \right) \\
  & = & \pi^2\bigfrac{\delta^3}{d^2} - \pi^2\bigfrac{\delta^2}{d^2} -\pi^4\bigfrac{\delta^5}{12d^4} + \pi^4\bigfrac{\delta^4}{12d^4} \\
  & > & \frac{\pi^2}{2} \cdot \frac{\delta^3}{d^2} .
\end{eqnarray*}
The stated result follows.  %The lower bound on $l_{\delta, d}$ comes from considering that $l_{\delta, d}(k) = 1/2(\nu_1 - \nu_{k+1})$; the upper bound comes from the observation concerning $\nu_1 - \nu_2$ made in \eqref{equ:gap12}.
\end{proof}

We are now sufficiently well informed about $\tilde{\X}_0$ to consider perturbation results for its leading eigenvector.
