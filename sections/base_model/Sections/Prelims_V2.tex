%
%%
%
We now summarize a new %recently introduced \cite{IVW2015_FastPhase} 
fast (essentially linear-time) approach for recovering $\x_0$ from local
correlation measurements of the form (\ref{eq:loco_measurements}). 
More explicitly, we will consider the problem of recovering $\x_0 \in \mathbbm C^d$ from noisy
magnitude measurements $\y \in \mathbbm{R}^D$ of the form 
%
\begin{equation}
    \y = \left | M \x_0 \right |^2 + \n, 
    \label{equ:MeasModel}
\end{equation}
%
with $D \geq d$.  Here $M \in \mathbbm{C}^{D \times d}$ is our
measurement matrix, $| \cdot |^2: \mathbbm{C}^D \rightarrow
\mathbbm{R}^D$ computes the  componentwise squared magnitude of each
vector entry, and $\n \in \mathbbm{R}^D$ represents arbitrary
measurement noise.  Note that we will always consider the indices of vectors in
$\mathbbm{C}^d$ to be $\mod d$ and between $1$ and $d$; more
specifically, for any $\z \in \mathbbm{C}^d$, we always mean to indicate the entry 
$z_{(j-1) \mod d \ + 1}$ whenever we write $z_j$.


\subsection{A Banded Lifting Scheme}
\label{sec:LiftToBandedMatrix}

Solving for $\z$ via \eqref{equ:LinProb} in the noiseless setting yields a portion of the rank one matrix ${\bf x}_0 {\bf x}_0^*\in \mathbbm{C}^{d \times d}$.  More specifically, only the diagonal entries are recovered as part of the banded matrix
\begin{equation}
(\X_0)_{j,k} =  \left\{ \begin{array}{ll} ({\bf x}_0 {\bf x}_0^*)_{j,k} & \textrm{if}~| j - k ~{\rm mod}~d | < \delta \\ 0 & \textrm{otherwise} \end{array} \right..
\label{equ:PhaseMatrix}
\end{equation}
Normalizing each entry of $\X_0 \in \mathbbm{C}^{d \times d}$ one obtains the Hermitian matrix $\tilde{\X}_0 \in \mathbbm{C}^{d \times d}$ with entries
\begin{equation}
(\tilde{\X}_0)_{j,k} =  \left\{ \begin{array}{ll} \mathbbm{e}^{\mathbbm{i}(\phi_j - \phi_k)} & \textrm{if}~| j - k ~{\rm mod}~d | < \delta \\ 0 & \textrm{otherwise} \end{array} \right.,
\label{equ:MatrixofPhases}
\end{equation}
where $(x_0)_j = r_j \mathbbm{e}^{\mathbbm{i} \phi_j}$ for $j \in [d]$.  Our objective is now to use $\X_0 \in \mathbbm{C}^{d \times d}$ to recover $\phi_1, \dots, \phi_d \in [0, 2 \pi]$, the phases of the entries of ${\bf x}_0$.  That is, we want to recover the vector ${\bf \tilde{x}}_0 \in \mathbbm{C}^d$ with 
\begin{equation}
(\tilde{x}_0)_j := \mathbbm{e}^{\mathbbm{i}\phi_j} ~=~ \left\{\begin{array}{ll} (x_0)_j / |(x_0)_j| & \textrm{if}~|(x_0)_j| \neq 0\\ 1 & \textrm{else} \end{array} \right.
\label{Def:VecofPhases}
\end{equation} 
using $\tilde{\X}_0$.  The following theorem, proven in \cite{IV_SPIE}, demonstrates that this is possible in the noiseless setting by simply computing the top eigenvector of $\tilde{\X}_0$.  

\begin{thm}
The largest magnitude eigenvalue of $\tilde{\X}_0$ is $\nu_1 = 2 \delta -1$, and $\tilde{\x}_0$ is the only eigenvector of $\tilde{\X}_0$ with eigenvalue $\nu_1$.  Furthermore, $c \sqrt{\sum^d_{j=2} \nu^2_j} \leq \nu_1$ whenever $\delta \geq \frac{1+(d+1) c^2}{2(1+c^2)}$ for any chosen $c \in \mathbbm{R}^+$.
\label{thm:Noiseless_Spectrum}
\end{thm}

As demonstrated below, computing the leading eigenvector of $\tilde{\X}_0$ also turns out to be both highly robust to noise numerically, as well as provably robust theoretically.  In order to see this, explicit formulas are developed for the entire spectrum of any such $\tilde{\X}_0$ matrix below, in \S \ref{sec:Spectrum}.  Besides improving on Theorem~\ref{thm:Noiseless_Spectrum}, these formulas, when combined with matrix perturbation results, allow us to bound the effect of noise on the leading eigenvector of $\tilde{\X}_0$ in \S \ref{sec:Perturb}.  These bounds are then carefully combined with properties of the proposed measurements (e.g., Theorem~\ref{thm:WellCondMeas}) in order to develop theoretical robustness guarantees for the overall phase retrieval approach in \S \ref{sec:RecovGuarantee}.  The end result is a new phase retrieval method for measurements of the type discussed above (recall \S \ref{sec:MeasMatrix}) which is significantly more robust to noise than similarly fast methods for locally supported measurements.  We are now ready to begin by explicitly specifying our recovery algorithm in full detail.

