%--------------------------------------------------
% Introduction
%--------------------------------------------------
Throughout the paper, we denote vectors using boldface, and matrices with capital letters.
We will find it convenient to use the Kronecker product of two matrices $X\in \R^{m_1 \times n_1}$ and $Y\in \R^{m_2 \times n_2}$, defined as 

\[X \otimes Y := \left[\begin{array}{lll} X_{11}Y & \cdots &   X_{1,n_1}Y \\ \vdots & \ddots & \vdots \\ X_{m_1,1}& \cdots & X_{m_1,n_1}\end{array}\right  ],\]
where $X \otimes Y \in \R^{(m_1 m_2) \times (n_1 n_2)}$.
Further,  we define the ``vectorization" operation from $\R^{m \times n}$ to $\R^{mn}$ whereby $X$ maps to $\vec{X}$. Here, the entries of $\vec{X}$ are obtained by stacking the columns of $X$. In particular, for three matrices $A,X,B$ (of appropriate dimensions) we will find the following identity useful: 
\[\overrightarrow{AXB} = (B^T\otimes A) \vec{X}.\]

We denote the vector of all ones in $\R^d$ by $\mathbbm{1}_d$ and the normalized discrete Fourier transform matrix by $F$.


%We obtain noisy measurements $\y \in \mathbbm{R}^D$ of the form 
%\begin{equation}
%\y = \left | M \x_0 \right |^2 + \n, 
%\label{equ:MeasModel}
%\end{equation}
%with $D \geq d$.  Here $M \in \mathbbm{C}^{D \times d}$ is our measurement matrix, $| \cdot |^2: \mathbbm{C}^D \rightarrow \mathbbm{R}^D$ computes the  componentwise squared magnitude of each vector entry, and $\n \in \mathbbm{R}^D$ represents arbitrary measurement noise. \\
%
%GOAL:  to recover $\x_0$\\
%
%In our notation, we will always consider the indices of vectors in $\mathbbm{C}^d$ to be $\mod d$ and between $1$ and $d$; more specifically, for $z \in \mathbbm{C}^d$, by $z_i$ we always mean $z_{(i-1) \mod d \ + 1}$.
%
%%%%%%
%%% REMOVE THIS SECTION -- WANT TO TRY TO FLATTEN THE NOISE WITHOUT DESTROYING THE STRUCTURE OF \x_0 ???????????
%%%%%%%%%%%%%%%%%%%%%%%
%%\subsection{Flattening Vectors with High Probability}
%%
%%The following notion of flatness for a given vector will be valuable below.  More specifically, we will eventually shape our arbitrary noise vector $\n$ from \eqref{equ:MeasModel} with a random matrix so that it is ``flat'' as per the following definition with high probability.
%%
%%\begin{Def}
%%Let $m \in [D]$.  A vector ${\mathbf v} \in \mathbbm{C}^D$ will be called $m$-flat if its entries can be partitioned into at least $\left \lfloor \frac{D}{m} \right \rfloor$ contiguous blocks such that:
%%\begin{enumerate}
%%\item Every block contains either $m$ or $m+1$ entries, 
%%\item Every block contains at least one entry whose magnitude is $\geq \frac{\| {\mathbf v}  \|_2}{2\sqrt{D}}$, and
%%\item All entries of ${\mathbf v}$ have magnitude $\leq \sqrt{\frac{3 m + 3}{2 D }} \cdot \| {\mathbf v}  \|_2$.
%%\end{enumerate}
%%\end{Def}
%%
%%We may ``flatten'' a vector using a structured random matrix as follows.  Let $W \in \mathbbm{C}^{D \times D}$ be the random unitary matrix
%%\begin{equation}
%%W := P F B,
%%\label{def:Wdef}
%%\end{equation}
%%where $P \in \{ 0,1 \}^{D \times D}$ is a permutation matrix selected uniformly at random from the set of all $D \times D$ permutation matrices, $F$ is the unitary $D \times D$ discrete Fourier transform matrix, and $B \in \left\{ -1,0,1\right\}^{D \times D}$ is a random diagonal matrix with i.i.d. symmetric Bernoulli $\pm1$-entries on its diagonal.  The following theorem, proven in \cite{IVW2015_FastPhase}, specifies how flat $W \x$ is likely to be for an arbitrary vector $\x \in \mathbbm{C}^D$.
%%
%%\begin{thm}
%%Let $W \in \mathbbm{C}^{D \times D}$ be formed as per \eqref{def:Wdef} for $D \geq 8$.  Then, $W \x \in \mathbbm{C}^D$ will be $m$-flat with probability at least $1-\frac{1}{m}$ provided that $\sqrt{D} \geq m + 1 \geq C \cdot \ln^2(D) \cdot \ln^3 \left( \ln D \right)$.\footnote{Here $C \in \mathbbm{R}^+$  is a fixed absolute constant.}
%%\label{thm:flattenx}
%%\end{thm}
%%
%%We are now ready to define our measurement matrix $M$.
%%%%%%%%%%%%%%%%%%%%%%%
%
%\subsection{Our Measurement Matrices}
%\label{sec:MeasMatrix}
%
%Herein we will utilize the type of block-circulant measurements proposed in \cite{IVW2015_FastPhase}.  More specifically, for the sake of concrete analysis we will focus on windowed Fourier measurements with parameters $\delta \in \mathbbm{Z}^+$ and $a \in [4, \infty)$ corresponding to the masks
%\begin{equation}
%(m_l)_{i} = \left\{ \begin{array}{ll} \frac{\mathbbm{e}^{-i/a}}{\sqrt[4]{2\delta -1}} \cdot \mathbbm{e}^{\frac{2 \pi \mathbbm{i} \cdot (i - 1) \cdot (l - 1)}{2\delta - 1}} & \textrm{if}~i \leq \delta \\ 0 & \textrm{if}~i > \delta\end{array} \right.
%\label{equ:MeasDef}
%\end{equation}
%for $1 \leq l \leq 2\delta - 1$, and $1 \leq i \leq d$.  This yields $D = (2\delta - 1)d$ total overlapping windowed Fourier measurements of $\x_0 \in \mathbbm{C}^d$ which can be represented by the block-form measurement matrix
%\begin{equation}
%M :=  \left( \begin{array}{l}  M_1\\ M_2\\ \vdots \\ M_{2\delta - 1}\end{array} \right),
%\label{equ:MblockDecomp}
%\end{equation}
%where each block $M_l \in \mathbbm{C}^{d \times d}$ is a banded circulant submatrix with entries given by
%\begin{equation*}
%(M_l)_{i,j} := \left\{ \begin{array}{ll} \frac{\mathbbm{e}^{-((j-i)~{\rm mod}~d +1)/a}}{\sqrt[4]{2\delta -1}} \cdot \mathbbm{e}^{\frac{2 \pi \mathbbm{i} \cdot ((j-i)~{\rm mod}~d) \cdot (l - 1)}{2\delta - 1}} & \textrm{if}~(j-i)~{\rm mod}~d + 1 \leq \delta \\ 0 & \textrm{if}~(j-i)~{\rm mod}~d +1 > \delta\end{array} \right.
%\label{equ:SubMatDef}
%\end{equation*}
%for $1 \le l \le 2\delta - 1$.
%
%\begin{figure}[t]
%\resizebox{\columnwidth}{!}{ $\begingroup
%\begin{bmatrix}
%    |(x_0)_1|^2 & (x_0)_1\overline{(x_0)_2} &   
%    (x_0)_2\overline{(x_0)_1} & |(x_0)_2|^2 & (x_0)_2 \overline{(x_0)_3} & 
%    (x_0)_3 \overline{(x_0)_2} & |(x_0)_3|^2 & (x_0)_3 \overline{(x_0)_4} &
%    (x_0)_4\overline{(x_0)_3} & |x_4|^2 & (x_0)_4 \overline{(x_0)_1} & (x_0)_1\overline{(x_0)_4}
%\end{bmatrix}^T \endgroup$}
%\caption{The vector $\z \in \mathbbm{C}^{12}$  from \eqref{equ:Defz} and \eqref{equ:LinProb} with $\delta = 2$ for an $\x_0 \in \mathbbm{C}^4$.}
%%\resizebox{\columnwidth}{!}{ $\begingroup
%% {\bf b} = 
%%\begin{bmatrix}
%%    (b_1)_1 & (b_2)_1 & (b_3)_1 & 
%%    (b_1)_2 & (b_2)_2 & (b_3)_2 & 
%%    (b_1)_3 & (b_2)_3 & (b_3)_3 &
%%    (b_1)_4 & (b_2)_4 & (b_3)_4
%%\end{bmatrix}^T.  \endgroup$ }  % \]
%%\end{equation*} 
%\label{fig:Examplez}
%\end{figure}
%
%Let $\z \in \mathbbm{C}^{D}$ be defined by 
%\begin{equation}
%z_{i} :=  (x_0)_{\left \lceil \frac{i+\delta-1}{2\delta-1} \right \rceil} \overline{(x_0)}_{\left \lceil \frac{i+\delta-1}{2\delta-1} \right \rceil + ((i+\delta-2)~{\rm mod}~(2\delta-1)) - \delta+1}.
%\label{equ:Defz}
%\end{equation}
%Note that $\z$ contains all products of each entry of $\x_0$ with the complex conjugates of its ${2\delta -1}$ closest neighboring entries (see Figure~\ref{fig:Examplez} for an example).  As seen in \cite{IVW2015_FastPhase}, we may reorder the entries of $\left |  M \x_0 \right|^2$ via a simple permutation matrix $P \in \{ 0,1 \}^{D \times D}$ to obtain
%\begin{equation}
%P \left |  M \x_0 \right|^2 = M' \z := \left( \begin{array}{llllllll}  M'_1& M'_2 & \hdots & M'_{\delta} & 0 & 0 & \hdots & 0\\
%0 & M'_1& M'_2 & \hdots & M'_{\delta} & 0 & \hdots & 0\\ 
% & &  & \ddots &  &  &  & \\
% M'_2  & \hdots & M'_{\delta} & 0 & \hdots & 0 & \hdots & M'_1  
%\end{array} \right) \z
%\label{equ:LinProb}
%\end{equation}
%where the new blocks $M'_{l} \in \mathbbm{C}^{(2\delta - 1) \times (2\delta - 1)}$ have entries given by
%\begin{equation*}
%(M'_l)_{i,j} := \left\{ \begin{array}{ll} \frac{\mathbbm{e}^{-(2l+j-1)/ a}}{\sqrt{2\delta - 1}} \cdot \mathbbm{e}^{-\frac{2 \pi \mathbbm{i} \cdot (i - 1) \cdot (j-1)}{2\delta - 1}} & \textrm{if}~ 1 \leq j \leq \delta - l +1 \\
% 0 & \textrm{if}~ \delta - l + 2 \leq j \leq 2\delta - l -1 \\  \frac{\mathbbm{e}^{-(2l+j-2(\delta - 1))/a}}{\sqrt{2\delta - 1}} \cdot \mathbbm{e}^{-\frac{2 \pi \mathbbm{i} \cdot (i - 1) \cdot (j - 2 \delta)}{2\delta - 1}}  & \textrm{if}~2 \delta - l \leq j \leq 2\delta - 1, ~l < \delta \\
%  0 & \textrm{if}~ j > 1, ~\textrm{and}~ l = \delta  \end{array} \right..
%\end{equation*} 
%The permuted measurement matrix $M'$ in \eqref{equ:LinProb} is both well conditioned and rapidly invertible (see Figure~\ref{fig:MprimeExample} for an example $M'$ matrix).  In particular, the following theorem is proven \cite{IVW2015_FastPhase}.
%
%\begin{figure}[t]
%%\begin{center}
%%\begin{equation*}
%\resizebox{\columnwidth}{!}{
%$\begingroup\SmallColSep
%M^\prime \! = \!
%\begin{bmatrix}
%  |\overline{ (M'_1)_{1,1} }& \overline{ (M'_1)_{1,2} }& 
%  \overline{ (M'_1)_{1,2} }|&|\overline{ (M'_2)_{1,1} }& \overline{\quad 0 \quad} & 
%  \overline{ \quad 0 \quad }|& 0 & 0 & 0 & 0 & 0 & 0\\
%  |(M'_1)_{2,1} & (M'_1)_{2,2} & 
%  (M'_1)_{2,3} |&| (M'_2)_{2,1} & 0 & 
%  \quad 0 \quad |& 0 & 0 & 0 & 0 & 0 & 0\\
%  |\underline{(M'_1)_{3,1} }& \underline{ (M'_1)_{3,2} }& 
%  \underline{(M'_1)_{3,3} }|&|\underline{ (M'_2)_{3,1} }& \underline{ \quad 0 \quad }& 
%  \underline{ \quad 0 \quad }|& 0 & 0 & 0 & 0 & 0 & 0\\
%  
%  0 & 0 & 0 & |\overline{(M'_1)_{1,1}} & \overline{(M'_1)_{1,2}} &
%  \overline{(M'_1)_{1,3}}| & |\overline{(M'_2)_{1,1}} & \overline{\quad 0 \quad} & \overline{\quad 0 \quad }| & 0 & 0 & 0\\
%  0 & 0 & 0 & |(M'_1)_{2,1} & (M'_1)_{2,2} &
%  (M'_1)_{2,3}| & |(M'_2)_{2,1} & \quad 0 \quad & \quad 0 \quad |& 0 & 0 & 0\\
%  0 & 0 & 0 & |\underline{(M'_1)_{3,1}} & \underline{(M'_1)_{3,2}} &
%  \underline{(M'_1)_{3,3}}| & |\underline{(M'_2)_{3,1}} & \underline{\quad 0 \quad} & \underline{\quad 0 \quad}|& 0 & 0 & 0\\
%  
%  0 & 0 & 0 & 0 & 0 & 0 & |\overline{(M'_1)_{1,1}} & \overline{(M'_1)_{1,2}} &
%  \overline{(M'_1)_{1,3}}| & |\overline{(M'_2)_{1,1}} & \overline{\quad 0 \quad} & \overline{\quad 0 \quad }| \\
%  0 & 0 & 0 & 0 & 0 & 0 & |(M'_1)_{2,1} & (M'_1)_{2,2} &
%  (M'_1)_{2,3}| & |(M'_2)_{2,1} & \quad 0 \quad & \quad 0 \quad |\\
%  0 & 0 & 0 & 0 & 0 & 0 & |\underline{(M'_1)_{3,1}} & \underline{(M'_1)_{3,2}} &
%  \underline{(M'_1)_{3,3}}| & |\underline{(M'_2)_{3,1}} & \underline{\quad 0 \quad} & \underline{\quad 0 \quad}|\\
%
%    |\overline{(M'_2)_{1,1}} & \overline{\quad 0 \quad} & \overline{\quad 0 \quad }| & 0 & 0 & 
%  0 & 0 & 0 & 0 &  |\overline{(M'_1)_{1,1}} & \overline{(M'_1)_{1,2}} &
%  \overline{(M'_1)_{1,3}}|  \\
%  |(M'_2)_{2,1} & \quad 0 \quad & \quad 0 \quad | & 0 & 0 & 
%  0 & 0 & 0 & 0 &|(M'_1)_{2,1} & (M'_1)_{2,2} &
%  (M'_1)_{2,3}| \\
%  |\underline{(M'_2)_{3,1}} & \underline{\quad 0 \quad} & \underline{\quad 0 \quad}| & 0 & 0 & 
%  0 & 0 & 0 & 0 &|\underline{(M'_1)_{3,1}} & \underline{(M'_1)_{3,2}} &
%  \underline{(M'_1)_{3,3}}| \\
%\end{bmatrix},  \endgroup$ } 
%\caption{The associated matrix $M' \in \mathbbm{C}^{12 \times 12}$ from \eqref{equ:LinProb} with $d = 4$ and $\delta = 2$.}
%\label{fig:MprimeExample}
%\end{figure}
%
%\begin{thm}
%Define $M' \in \mathbbm{C}^{D \times D}$ as per \eqref{equ:LinProb} with $a:= \max \left\{ 4,~\frac{\delta - 1}{2} \right\}$.  Then, the condition number of $M'$ satisfies $$\kappa \left( M' \right) ~<~ \max \left\{ 144 \mathbbm{e}^2,~\frac{9 \mathbbm{e}^2}{4} \cdot (\delta - 1)^2 \right \},$$
%and the smallest singular value of $M'$ satisfies
%$$\sigma_{\rm min} \left( M' \right) > \frac{7}{20 a} \cdot \mathbbm{e}^{-(\delta + 1)/a} > \frac{C}{\delta}$$
%for an absolute constant $C \in \R^+$.  Furthermore, $M'$ can be inverted in $\mathcal{O} \left( \delta \cdot d \log d \right)$-time.
%\label{thm:WellCondMeas}
%\end{thm}
%
%This theorem indicates that one should be able to efficiently and stably solve for $\z$ using \eqref{equ:LinProb}.  
%
%\subsubsection{Reconstruction from a Special Class of Short-time Fourier Transform Measurements}
%
%TALK ABOUT THAT HERE...???
%
%\subsection{A Banded Lifting Scheme}
%\label{sec:LiftToBandedMatrix}
%
%Solving for $\z$ via \eqref{equ:LinProb} in the noiseless setting yields a portion of the rank one matrix ${\bf x}_0 {\bf x}_0^*\in \mathbbm{C}^{d \times d}$.  More specifically, only the diagonal entries are recovered as part of the banded matrix
%\begin{equation}
%(\X_0)_{j,k} =  \left\{ \begin{array}{ll} ({\bf x}_0 {\bf x}_0^*)_{j,k} & \textrm{if}~| j - k ~{\rm mod}~d | < \delta \\ 0 & \textrm{otherwise} \end{array} \right..
%\label{equ:PhaseMatrix}
%\end{equation}
%Normalizing each entry of $\X_0 \in \mathbbm{C}^{d \times d}$ one obtains the Hermitian matrix $\tilde{\X}_0 \in \mathbbm{C}^{d \times d}$ with entries
%\begin{equation}
%(\tilde{\X}_0)_{j,k} =  \left\{ \begin{array}{ll} \mathbbm{e}^{\mathbbm{i}(\phi_j - \phi_k)} & \textrm{if}~| j - k ~{\rm mod}~d | < \delta \\ 0 & \textrm{otherwise} \end{array} \right.,
%\label{equ:MatrixofPhases}
%\end{equation}
%where $(x_0)_j = r_j \mathbbm{e}^{\mathbbm{i} \phi_j}$ for $j \in [d]$.  Our objective is now to use $\X_0 \in \mathbbm{C}^{d \times d}$ to recover $\phi_1, \dots, \phi_d \in [0, 2 \pi]$, the phases of the entries of ${\bf x}_0$.  That is, we want to recover the vector ${\bf \tilde{x}}_0 \in \mathbbm{C}^d$ with 
%\begin{equation}
%(\tilde{x}_0)_j := \mathbbm{e}^{\mathbbm{i}\phi_j} ~=~ (x_0)_j / |(x_0)_j| 
%\end{equation} 
%using $\X_0$.  The following theorem, proven in \cite{IV_SPIE}, demonstrates that this is possible in the noiseless setting by simply computing the top eigenvector of $\X_0$.  
%
%\begin{thm}
%The largest magnitude eigenvalue of $\tilde{\X}_0$ is $\nu_1 = 2 \delta -1$, and $\tilde{\x}_0$ is the only eigenvector of $\tilde{\X}_0$ with eigenvalue $\nu_1$.  Furthermore, $c \sqrt{\sum^d_{j=2} \nu^2_j} \leq \nu_1$ whenever $\delta \geq \frac{1+(d+1) c^2}{2(1+c^2)}$ for any chosen $c \in \mathbbm{R}^+$.
%\label{thm:Noiseless_Spectrum}
%\end{thm}
%
%--Discuss theorem~\ref{thm:Noiseless_Spectrum} more -- We improve it below, calculate everything exactly -- life is wonderful!
%
%We are now ready to explicitly specify our recovery algorithm.
%
