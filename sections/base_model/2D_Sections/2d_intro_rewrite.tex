In \cref{ch:base_model}, we studied the phase retrieval problem \eqref{eq:phase_ret} in the case of a measurement model that we referred to as \emph{local measurement systems}.  This meant we had a small number of vector $m_j \in \Cd$, whose magnitude-squared inner products were taken with the objective vector $x \in \Cd$, for several different circular translations of $m_j$, written explicitly as \[ (\y_\ell)_j = |\langle \x_0, S^*_\ell \m_j \rangle|^2, \quad (j, \ell) \in [K] \times P\] in \eqref{eq:shift_model}.  Of course, considering this merely as a mathematical abstraction, it is entirely conceivable that $x \in \Cd$ represents a discretization and vectorization of a signal in two-dimensions, or for that matter, any arbitrary number of dimensions -- the mathematical formulation does not force or exclude any particular interpretation of the model.  Nonetheless, as is mentioned in \cref{sec:conn_pty}, one major strength of the model proposed and studied in this dissertation is that it somewhat more closely represents the reality of ptychography, the crucial application of phase retrieval that has motivated this work since the beginning.  Therefore, if we wish to extend this work to two dimensions, we ought to analyze an analogous model that adequately reflects the structure of two-dimensional ptychography.

Conveniently enough, it is not difficult to state the model for what we have in mind: we shall consider measurements of the form \begin{equation} \abs*{\inner*{Q, S^{\ell} A S^{-\ell'}}}^2 = \abs*{\sum_{j, k = 1}^d A_{j - \ell, k - \ell'} Q_{jk}}^2, \label{eq:2d_meas} \end{equation} where $Q \in \Cdxd$ is the two-dimensional objective signal that we want to recover (analogous to $x$ in \eqref{eq:pr_bare}), $A \in \Cdxd$ is a measurement matrix (analogous to one of the $m_j$), and $\ell, \ell' \in [d]_0$ are shifts (analogous to $\ell$).  For $A$ to represent a ``local measurement'' (with a support size of $\delta$), we require our measurement matrices to satisfy $\supp(A) \subset [\delta]^2$, and for simplicity of analysis, we are going to further require that these matrices are rank one; we will therefore index our measurement matrices by $A_{uv} = \a_u \b_v^*$, where $\supp(\a_u), \supp(\b_v) \subset [\delta]$.  For convenience, we set \begin{equation} \mathbf{Q} := \vec Q \vec Q ^*. \label{eq:vecQ} \end{equation}

We remark that this setup may be used to model two-dimensional ptychography, in a way similar to the work in \cref{sec:conn_pty}, by fixing $\a, \b \in \Cd$ (with $\supp(\a), \supp(\b) \subset [\delta]$) and setting \[\a_u = \sqrt{K} f_u^K \circ \a, \ \b_v = \sqrt{K} f_v^K \circ \b.\]
  
