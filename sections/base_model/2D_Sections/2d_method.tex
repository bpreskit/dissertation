Our recovery method, outlined in Algorithm \ref{Alg1}, aims to approximate an image $Q\in \C^{d\times d}$ from phaseless measurements of the form \eqref{eq:2d_meas}.
%
%
\begin{algorithm}[htbp]
\renewcommand{\algorithmicrequire}{\textbf{Input:}}
\renewcommand{\algorithmicensure}{\textbf{Output:}}
\caption{Two Dimensional Phase Retrieval from Local Measurements}
\label{Alg1}
\begin{algorithmic}[1]
    \REQUIRE Measurements $\y\in \mathbbm{R}^D$ as per \eqref{def:Measurements}
    \ENSURE $X \in \mathbbm{C}^{d \times d}$ with $X \approx \mathbbm{e}^{-\mathbbm{i} \theta} Q$ for some $\theta \in [0, 2 \pi]$ 
    \STATE Compute the Hermitian matrix $P = \Big( \left(\mathcal{M} \big|_{\mathcal{P}}\right)^{-1} {\bf y}\Big)/2 + \Big( \left(\mathcal{M} \big|_{\mathcal{P}}\right)^{-1} {\bf y}\Big)^*/2  \in \mathcal{P}\left(\mathbbm{C}^{d^2\times d^2}\right)$ as an estimate of $\mathcal{P} \left( \vec{Q} \vec{Q}^* \right)$.  $\mathcal{M}$ and $\mathcal{P}$ are as defined in \eqref{Def:Measure_Operator} and \S\ref{sec:linear}.
    \STATE Form the matrix of phases, $\widetilde{P} \in \mathcal{P}\left(\mathbbm{C}^{d^2\times d^2}\right)$, by normalizing the non-zero entries of $P$.
    \STATE Compute the principal eigenvector of $\widetilde{P}$ and use it to compute $U_{j,k} \approx \sgn\left( Q_{j,k}\right) ~\forall j,k \in [d]$ as per \S\ref{sec:Getphases}.
    \STATE Use the diagonal entries of $P$ to compute $M_{j,k} \approx \left| Q_{j,k} \right|^2$ for all $j,k \in [d]$ as per \S\ref{sec:Getmags}.
    \STATE Set $X_{j,k} = \sqrt{M_{j,k}} \cdot U_{j,k}$ for all $j,k \in [d]$ to form $X$
    %\STATE Set $\x = W^* \widetilde{\x}$ 
    \end{algorithmic}
\end{algorithm}
%
%
%
Specifically, we consider the collection of measurements given by 
\begin{equation}
y_{(\ell,\ell',u,v)} := \left| \left\langle Q, S_{\ell} \a_u \left(S_{\ell'} \b_v \right)^* \right\rangle \right|^2
\label{def:Measurements}
\end{equation}
for all $(\ell,\ell',u,v) \in [d]^2 \times\Omega^2$ where $\Omega \subset [d]$ has $|\Omega| = 2\delta-1$.  Thus, we collect a total of $D := (2\delta-1)^2 \cdot d^2 $ measurements where each measurement is due to a vertical and horizontal shift of a rank one illumination pattern $\a_u \b_v^* \in \mathbbm{C}^{d \times d}$.

Algorithm \ref{Alg1} consists of first rephrasing the system \eqref{def:Measurements} as a linear system on the space of $d^2 \times d^2$ matrices (following Candes, et al. \cite{candes2011phaselift}), and then estimating a projection $\mathcal{P}( \vec{Q}\vec{Q}^*)$ of the rank one matrix $\vec{Q}\vec{Q}^*$ from this system.  This process is described in \cref{sec:linear}.  In \cref{sec:Getphases,sec:Getmags} we show how the magnitudes of the entries of $Q$ are estimated directly from $\mathcal{P}(\vec{Q}\vec{Q}^*)$ and their phases are found from solving an eigenvector problem.  Together, the magnitude and phase estimates provide an approximation of $Q$.

%  \cite{candes2013phaselift} of the discrete 2D phase retrieval problem from local measurements of type \eqref{Prob_Disc}.

\subsection{The Linear Measurement Operator $\mathcal{M}$ and Its Inverse}
\label{sec:linear}
To produce the linear system of step 1, we observe that
\begin{align*}%
y_{(\ell,\ell',u,v)} &= \left| \left\langle Q, S_{\ell} \a_u \left(S_{\ell'} \b_v \right)^* \right\rangle \right|^2 = |\langle \vec{Q}, S_{\ell'} \bar{\b_u} \otimes S_{\ell} \a_v \rangle |^2 \\%
&= \left\langle  \vec{Q} \vec{Q}^*, ~S_{\ell'} \bar{\b_u} \otimes S_{\ell} \a_v \left(S_{\ell'} \bar{\b_u} \otimes S_{\ell} \a_v\right)^* \right\rangle,
\end{align*} %With identity \eqref{id:quadprod} %$|\langle \x, \y \rangle|^2 = (x^*y) (y^* x) = \Tr\left(x^* y y^* x \right) = \Tr(x x^* y y^*) = \langle x x^*, y y^* \rangle,$
%$|\langle \x, \y \rangle |^2 = \langle \x \x^*, \y \y^* \rangle$ for arbitrary $\x, \y \in \C^n$,
%we can further see that $$y_{(\ell,\ell',u,v)} = \left\langle  \vec{Q} \left( \vec{Q} \right)^*, ~S_{\ell'}^* \bar{\b_u} \otimes S_{\ell}^* \a_v \left(S_{\ell'}^* \bar{\b_u} \otimes S_{\ell}^* \a_v\right)^* \right\rangle$$
%% \begin{align*}
%% y_{(\ell,\ell',u,v)} &=\left\langle S^*_{\ell} \overline{\a_u} \otimes S^*_{\ell'} \b_v, \vec{Q^*} \right\rangle \overline{\left\langle S^*_{\ell} \overline{\a_u} \otimes S^*_{\ell'} \b_v, \vec{Q^*} \right\rangle } \\
%% &= \left( \vec{Q^*} \right)^* S^*_{\ell} \overline{\a_u} \otimes S^*_{\ell'} \b_v \left( S^*_{\ell} \overline{\a_u} \otimes S^*_{\ell'} \b_v \right)^* \vec{Q^*} \\
%% &= {\rm Trace}\left( \vec{Q^*} \left( \vec{Q^*} \right)^* S^*_{\ell} \overline{\a_u} \otimes S^*_{\ell'} \b_v \left( S^*_{\ell} \overline{\a_u} \otimes S^*_{\ell'} \b_v \right)^* \right)\\
%% &=\left\langle  \vec{Q^*} \left( \vec{Q^*} \right)^*, ~S^*_{\ell} \overline{\a_u} \otimes S^*_{\ell'} \b_v \left( S^*_{\ell} \overline{\a_u} \otimes S^*_{\ell'} \b_v \right)^*\right\rangle.
%% \end{align*}
which allows us to naturally define $\mathcal{M}: \mathbbm{C}^{d^2 \times d^2} \mapsto \mathbbm{R}^D$ as the linear measurement operator given by 
\begin{equation}
  \begin{aligned}
\left(\mathcal{M}(Z) \right)_{(\ell,\ell',u,v)} &:= \left\langle  Z, ~S_{\ell'} \bar{\b_u} \otimes S_{\ell} \a_v \left(S_{\ell'} \bar{\b_u} \otimes S_{\ell} \a_v\right)^*\right\rangle \\ &= \left\langle Z, ~S_{\ell'} \bar{\b_u}\bar{\b_u}^* S^*_{\ell'} \otimes S_\ell \a_v \a_v^* S^*_\ell \right\rangle,
  \end{aligned}
\label{Def:Measure_Operator}
\end{equation}
so that $\y = \mathcal{M}(\vec{Q} \vec{Q}^*)$.  This allows us to solve for $\Pc(\vec{Q} \vec{Q}^*)$, the projection of $\vec{Q} \vec{Q}^*$ onto the rowspace $\Pc(\C^{d^2 \times d^2})$ of $\Mc$.  For clarity, we will abbreviate $\Pc(\C^{d^2 \times d^2})$ as $\Pc$, identifying this subspace with its orthogonal projection operator.

We observe that the local supports of $\a_u$ and $\b_v$ ensure that $\mathcal{M}( \vec{E_{j,k}} \vec{E_{j',k'}}^*) = {\bf 0}$ whenever either $|j - j'| \geq \delta$ or $|k - k'| \geq \delta$ holds (this is clear from \eqref{Def:Measure_Operator} and \eqref{id:kronsimp}).  As a result we can see that $\Pc \subset \B$ where \begin{equation} \B := \Span\{ \vec{E_{j,k}} \vec{E_{j',k'}}^* ~{\bf \big |}~  |j - j'| < \delta, |k - k'| < \delta \}. \label{eq:defB} \end{equation}  %% \footnote{Note that projection onto $\Span(\B)$ can also be described as a restriction operator onto the indices associated with the elements of $\B$.  Our periodic boundary conditions also imply that, e.g., $|j - j'| < \delta \Leftrightarrow \exists h \in \mathbbm{Z}~{\rm with}~ |h| < \delta~{\rm s.t.}~j' + h \equiv j ~{\rm mod}~d$.}
In steps 2-4 of algorithm \ref{Alg1}, recovery of $Q$ from $\Pc(\vec{Q} \vec{Q}^*)$ relies on having $\Pc = \B$ exactly; we say in such a case that $\Mc|_\B$ is invertible.  Clearly, the invertibility of $\Mc$ over $\B$ will depend on our choice of $\a$ and $\b$.  We prove the following proposition, a corollary of which identifies pairs $\a, \b$ which produce an invertible linear system:
\begin{proposition}
  Let $T_\delta : \C^{d \times d} \to \C^{d \times d}$ be the operator given by \[T_\delta(X)_{ij} = \left\{\begin{array}{r@{,\quad}l}
  X_{ij} & |i - j| < \delta \mod d \\
  0 & \text{otherwise}\end{array}\right..\]
  If the space $T_{\delta}(\C^{d \times d})$ is spanned by the collection $\{a_j a_j^*\}_{j=1}^K$, then $\B$ is spanned by \[\{(a_j \otimes a_{j'}) (a_j \otimes a_{j'})^*\}_{(j, j') \in [K]^2} = \{(a_j a_j^*) \otimes (a_{j'} a_{j'}^*)\}_{(j, j') \in [K]^2}.\]
  \label{prop:kronspan}
\end{proposition}

\begin{proof}
  By \eqref{id:kronsimp} and \eqref{eq:defB}, it suffices to show that $$(e_k e_{k'}^*) \otimes (e_j e_{j'}^*) \in \Span\{(a_n a_n^*) \otimes (a_{n'} a_{n'}^*)\}_{(n, n') \in [K]^2}$$ for any $|j - j'|, |k - k'| < \delta$.  Indeed, we have that $\{E_{jj'} : |j - j'| < \delta \mod d\}$ forms a basis for $T_\delta(\C^{d \times d})$, so $E_{j j'}, E_{k k'} \in \Span\{a_n a_n^*\}_{n \in [K]}$ and $$(e_k e_{k'}^*) \otimes (e_j e_{j'}^*) \in \Span\{(a_n a_n^*) \otimes (a_{n'} a_{n'}^*)\}_{(n, n') \in [K]^2}.$$
\end{proof}

We remark that \cref{prop:kronspan} permits us to import all of the theory developed in \cref{ch:meas_sec}, as well as the mask constructions offered in \cref{sec:MeasMatrix}.  A trivial combination of \cref{prop:kronspan} with \cref{cor:gam_fam_span} gives us \cref{cor:2d_gam_span}, and with Example 2 of \cref{sec:MeasMatrix} gives us \cref{cor:2d_sparse}.

\begin{corollary}
  Fix $d, \delta \in \N$ such that $D := 2 \delta - 1 \le d$.  Suppose that $\gamma \in \Rd$ has $\supp(\gamma) \subset [\delta]$ and satisfies $f_j^d(\gamma \circ S^{-m} \gamma) \neq 0$ for all $j \in [d], m \in [\delta]_0$.  Then fixing $K \ge D$ and setting $\a_u = f^K_u \circ \gamma$, we have \[\Span\{S^\ell \a_u \a_u^* S^{-\ell} \otimes S^{\ell'} \a_v \a_v^* S^{-\ell'}\}_{(\ell, \ell, u, v) \in [d]^2 \times [D]^2} = \B.\]
  \label{cor:2d_gam_span}
\end{corollary}

\begin{corollary}
  Fix $d, \delta \in \N$ such that $D := 2 \delta - 1 \le d$.  Set $\a_1 = e_1, \a_{2j} = e_1 + e_{j+1}$, and $\a_{2j + 1} = e_1 + i e_{j+1}$ for $j \in [\delta - 1]$.  Then \[\Span\{S^\ell \a_u \a_u^* S^{-\ell} \otimes S^{\ell'} \a_v \a_v^* S^{-\ell'}\}_{(\ell, \ell, u, v) \in [d]^2 \times [D]^2} = \B.\]
  \label{cor:2d_sparse}
\end{corollary}

\subsection{Computing the Phases of the Entries of $Q$ after Inverting $\mathcal{M} \big|_{\B}$}
\label{sec:Getphases}

Assuming that $\Pc = \B$ so that we can recover $\Pc ( \vec{Q} \vec{Q}^* ) = \B ( \vec{Q} \vec{Q}^* )$ from our measurements $\y$, we are still left with the problem of how to recover $\vec{Q}$ from $\B ( \vec{Q} \vec{Q}^* )$.  Our first step in solving for $\vec{Q}$ will be to compute all the phases of the entries of $\vec{Q}$ from $\B(\vec{Q} \vec{Q}^*)$.  Thankfully, this can be solved as an angular synchronization problem as in \cref{sec:Perturb} by using the method described in \cref{thm:SpecGraphPertBound}.  Let $\one \in \mathbbm{C}^{d^2 \times d^2}$ be the matrix of all ones, and $\sgn: \mathbbm{C} \mapsto  \mathbbm{C}$ be 
$$\sgn(z) = \left\{\begin{array}{r@{,\qquad}l} \dfrac{z}{|z|} & z \neq 0 \\ 1 & \text{otherwise} \end{array}\right..$$
We now define $\widetilde{Q} \in \mathbbm{C}^{d^2 \times d^2}$ by $\widetilde{Q} = \B(\sgn(\vec{Q} \vec{Q}^*))$ (i.e. $\widetilde{Q}$ is $\B(\vec{Q}\vec{Q}^*)$ with its non-zero entries normalized). %
%% \begin{equation}
%% \widetilde{Q}_{j,k} := \left\{\begin{array}{r@{,\qquad}l} \sgn \left( \left[ \B \left( \vec{Q} \left( \vec{Q} \right)^* \right) \right]_{j,k} \right) & \left[ \B \left( \one \one^* \right) \right]_{j,k} \neq 0 \\ 0~~~~~~~~~~~~~~~~~~~~~~~~ & \text{otherwise} \end{array}\right..
%% \label{def:Qtilde}
%% \end{equation}
As we shall see, the principal eigenvector of $\widetilde{Q}$ will provide us with all of the phases of the entries of $\vec{Q}$.

Working toward that goal, we may note that
\begin{equation}
\widetilde{Q} = \diag \left( \sgn \left( \vec{Q} \right) \right) \B \left( \one \one^* \right) \diag \left( \overline{\sgn \left( \vec{Q} \right)} \right)
\label{equ:Qtilde_partial_factor}
\end{equation}
where $\sgn$ is applied component-wise to vectors, and where $\diag(\x)  \in \mathbbm{C}^{d^2 \times d^2}$ is diagonal with $\left(\diag(\x)\right)_{j,j} := x_j$ for all $\x \in \mathbbm{C}^{d^2}$ and $j \in [d^2]$.  After noting that $\diag(\sgn(\cdot))$ always produces a unitary diagonal matrix, we can further see that the spectral structure of $\widetilde{Q}$ is determined by $\B \left( \one \one^* \right)$.  In particular, the following theorem completely characterizes the eigenvalues and eigenvectors of $\B \left( \one \one^* \right)$.

\begin{thm} 
Let $F \in \mathbbm{C}^{d \times d}$ be the unitary discrete Fourier transform matrix with $F_{j,k} := \frac{1}{\sqrt{d}} \ee^{2 \pi \ii \frac{(j-1)(k-1)}{d}} ~\forall j,k \in [d]$, and let $D \in \mathbbm{C}^{d \times d}$ be the diagonal matrix with $D_{j,j} = 1 + 2 \sum^{\delta-1}_{k=1} \cos \left( \frac{2 \pi (j-1)k}{d} \right)~\forall j \in [d]$.  Then,
$$\B \left( \one \one^* \right) = \left( F \otimes F \right) \left( D \otimes D \right) \left( F \otimes F \right)^*.$$
In particular, the principal eigenvector of $\B \left( \one \one^* \right)$ is $\one$ and its associated eigenvector is $(2 \delta - 1)^2$. 
\label{thm:Factorized_P11}
\end{thm}

\begin{proof}
From the definition of $\B$ we have that 
\begin{align*}
  \B \left( \one \one^* \right) &= \sum^d_{j=1} ~\sum_{|j - j'| < \delta} ~\sum^d_{k=1} ~\sum_{|k - k'| < \delta}\vec{E_{j,k}} \left(\vec{E_{j',k'}} \right)^* \\
  &= \sum^d_{j=1} ~\sum_{|j - j'| < \delta} ~\sum^d_{k=1} ~\sum_{|k - k'| < \delta} E_{kk'} \otimes E_{jj'} \\
  &= \left(\sum^d_{k=1} ~\sum_{|k - k'| < \delta} E_{kk'}\right) \otimes \left( \sum^d_{j=1} ~\sum_{|j - j'| < \delta} E_{jj'} \right) \\
  &= T_\delta\left(\one \one^*\right) \otimes T_\delta \left(\one \one^*\right)
\end{align*}
Of course, the eigenvectors and eigenvalues of $T_\delta(\one \one^*)$ have already been studied in \cref{lem:spectrum}, so $T_\delta(\one \one^*) = F D F^*$ (with $D$ and $F$ as defined in the statement of this theorem) which then yields the desired result by Theorem 4.2.12 of Horn and Johnson \cite{horn1991topics}.
\end{proof}

Theorem~\ref{thm:Factorized_P11} in combination with \eqref{equ:Qtilde_partial_factor} makes it clear that $\sgn (\vec{Q})$ will be the principal eigenvector of $\widetilde{Q}$.  As a result, we can rapidly compute the phases of all the entries of $\vec{Q}$ by using, e.g., a shifted inverse power method \cite{trefethen1997numerical} in order to compute the eigenvector of $\widetilde{Q}$ corresponding to the eigenvalue $(2 \delta - 1)^2$. 

\subsection{Computing the Magnitudes of the Entries of $Q$ after Inverting $\mathcal{M} \big|_{\B}$}
\label{sec:Getmags}

Having found the phases of each entry of $\vec{Q}$ using $\B ( \vec{Q} \vec{Q}^* )$, it only remains to find each entry's magnitude as well.  This is comparably easy to achieve.  Note that $\B$ trivially contains $\vec{E_{j,k}}\vec{E_{j,k}}^* = e_ke_k^* \otimes e_je_j^*$ for all $j, k \in [d]$, so $\B ( \vec{Q} \vec{Q}^* )$ is guaranteed to always provide the diagonal entries of $\vec{Q} \vec{Q}^*$,which are exactly the squared magnitudes of the entries of $\vec{Q}$.  Combined with the phase information recovered in step 2 of \ref{Alg1}, we are finally able to reconstruct every entry of $\vec{Q}$ up to a global phase as in step 5.
