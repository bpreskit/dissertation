We will now demonstrate the efficiency and robustness of Algorithm~\ref{Alg1}. Figures
\ref{fig:recon} and \ref{fig:recon_msu} plot the reconstruction of two $256\times 256$ test 
images (shown in Figures \ref{fig:true} and \ref{fig:true_msu} respectively) from squared
magnitude local measurements of the type described in (\ref{def:Prob_Disc_Gen}). Here both $\a$ and $\b$ are
chosen to be the {\em deterministic} measurement vectors of Corollary 2, detailed in Example 2 of \cite[Section
2]{iwen2016phase}, with $\delta$ chosen to be $2$. The reconstructions were computed using an
implementation of Algorithm~\ref{Alg1} in Matlab$^\circledR$ running on a Laptop computer (Ubuntu
Linux 16.04 {\tt x86\_64}, Intel$^\circledR$ Core$^{\text{\tiny TM}}$M-5Y10c processor, 8GB RAM,
Matlab$^\circledR$ R2016b). More specifically, the linear measurement operator $\left. \mathcal M
\right \vert_{\mathcal P}$ was constructed by passing the standard basis elements $E_{i,j}, i,j \in
[d], |i-j|< \delta \mod d$ through (\ref{Def:Measure_Operator}). An LU decomposition of this sparse
and structured matrix was pre-computed and stored for different values of $d,\delta$ for use in the
numerical simulations below. We note that an FFT-based implementation of Step 1 of
Algorithm~\ref{Alg1} is likely to yield improved efficiency; we defer such an implementation to
future work. The relative errors, defined by the expression $\displaystyle \frac{\min_\theta\Vert
\mathbbm e^{\mathbbm i \theta} X-Q\Vert_F}{\Vert Q\Vert_F}$ (where $X$ denotes the
reconstruction (up to a global phase factor) of $Q$), were $4.288\times 10^{-16}$ and $2.857\times 10^{-16}$ for the
reconstructions in Figures \ref{fig:recon} and \ref{fig:recon_msu} respectively. The reconstructions
were computed in $16.318$ and $16.529$ seconds respectively. 

%
\begin{figure}[p]
    \centering
    \begin{subfigure}[b]{0.495\textwidth}
        \centering
        \includegraphics[scale=0.75]{true_ucsd}
        \caption{Test Image $1$ ($256\times 256$ pixels)}
        \label{fig:true}
    \end{subfigure}
    \hfill
    \begin{subfigure}[b]{0.495\textwidth}
        \centering
        \includegraphics[scale=0.75]{recon_ucsd}
        \caption{Reconstructed Image (Rel. error $4.288\times 10^{-16}$)}
        \label{fig:recon}
    \end{subfigure}
    \vspace{0.05in} \\
    \begin{subfigure}[b]{0.495\textwidth}
        \centering
        \includegraphics[scale=0.75]{true}
        \caption{Test Image $2$ ($256\times 256$ pixels)}
        \label{fig:true_msu}
    \end{subfigure}
    \hfill
    \begin{subfigure}[b]{0.495\textwidth}
        \centering
        \includegraphics[scale=0.75]{recon}
        \caption{Reconstructed Image (Rel. error $2.857\times 10^{-16}$)}
        \label{fig:recon_msu}
    \end{subfigure}
    \vspace{0.05in}
    \caption{Two Dimensional Image Reconstruction from Phaseless Local Measurements.}
    \label{fig:2d-recon}
\end{figure}
%


We next plot the execution time (in seconds, averaged over $50$ trials) required to implement
Algorithm~\ref{Alg1} for different values of $d$ in Fig. \ref{fig:etime}. In each case, $\delta$ was
chosen to be $\lceil \log_2(d) \rceil$, with the same choice of measurement vectors as for the
reconstruction in Fig.  \ref{fig:recon}. The plot confirms that the proposed method is extremely
efficient; indeed, the plot reveals an FFT-time empirical computational complexity of $\mathcal
O(d^2 \log_2(d^2))$.

Finally, Fig. \ref{fig:noise} illustrates the robustness of the proposed method to measurement
errors (an analysis of noise robustness is omitted from the paper and should follow from an appropriate generalization of the techniques used for the analysis of BlockPR\cite{iwen2016phase} in the one-dimensional case). The figure plots the reconstruction error (averaged over $50$ trials) in reconstructing a
$64\times 64$ random matrix with i.i.d zero-mean complex Gaussian entries from phaseless
measurements (with $\delta=6$, and with the same measurement construction as with Fig.
\ref{fig:recon}). An additive noise model with i.i.d. zero-mean Gaussian noise was used to corrupt
the measurements. The added noise as well as reconstruction error are reported in decibels, with
%
\[  \text{SNR (dB)} = 10 \log_{10}\left( \frac{\Vert \y \Vert_2^2}{D\sigma^2}\right), \qquad 
    \text{Error (dB)} = 10 \log_{10} \left( \frac{\min_\theta\Vert \mathbbm 
                e^{\mathbbm i \theta}X-Q\Vert^2_F}{\Vert Q\Vert^2_F}\right). \]
%
%For added robustness, an improved eigenvector-based magnitude estimation method detailed in 
%\cite[Section 6.1]{iwen2016phase} was utilized in place of Step 4 of Algorithm~\ref{Alg1}. 
%We observe that the proposed algorithm demonstrates robustness across a wide variety of SNRs. Moreover, 
%the test signals are reconstructed to (almost) the level of added noise. 
We observe that the proposed algorithm (indicated by the dashed line) demonstrates robustness 
across a wide variety of SNRs. Additionally, the results from utilizing an improved
(eigenvector-based) magnitude estimation method (detailed in \cite[Section 6.1]{iwen2016phase}) 
in place of Step 4 of Algorithm~\ref{Alg1} is plotted using the solid line. In both cases, we
observe that the test signals are reconstructed to (almost) the level of added noise. 
%
\begin{figure}[hbtp]
    \centering
    \begin{subfigure}[b]{0.495\textwidth}
        \centering
        \includegraphics[clip=true, trim = 1.10in 3.25in 0.75in 3.25in,
scale=0.55]{etime}
        \caption{Execution Time vs Problem Size}
        \label{fig:etime}
    \end{subfigure}
    \hfill
    \begin{subfigure}[b]{0.495\textwidth}
        \centering
        %\includegraphics[clip=true, trim = 1.10in 3.15in 0.75in 3in,scale=0.55]{noise}
        \includegraphics[clip=true, trim = 1.10in 3.25in 0.75in 3in,scale=0.575]{noise_alt}
        \caption{Noise Robustness of the Proposed Method}
        \label{fig:noise}
    \end{subfigure} 
    \vspace{0.05in}
    \caption{Evaluating the Efficiency and Robustness of the Proposed Two Dimensional Phase
    Retrieval Algorithm.}
    \label{fig:perf}
\end{figure}
%
