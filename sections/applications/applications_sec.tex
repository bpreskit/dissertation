One common technique used to generate redundancy in phase retrieval-type measurements is to design a system that illuminates only a small part of the sample at a time.  These ``partial snapshots'' are then positioned along an overlapping grid, which produces the redundancy.  The overlap is necessary since, even if you could solve phase retrieval perfectly on each patch, each of these patches would have their own phase ambiguities; these would need to be synchronized to achieve a single, coherent image of the original sample.  Usually, ptychography is performed by taking a single \emph{mask} or \emph{illumination function} with small supprt, say $\m \in \C^d$ with $\supp(m) \subset [\delta]$ where $\delta \ll d$ and shifting this mask to different positions relative to the sample. These measurements may then be modelled as \begin{equation} \y_{\ell, j} = |\mathcal{F}(S^\ell \m \circ \x)_j|^2+ \eta_{\ell, j} = |\langle f_j \circ \m, \x \rangle|^2 + \eta_{\ell, j}. \label{eq:ptych_meas} \end{equation}  This technique is the inspiration for our model, where we do not require our measurements to take this exact form.
