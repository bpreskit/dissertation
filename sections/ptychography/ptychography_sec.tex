In our model for the ptychographic setup of \eqref{eq:ptych_meas}, we assume that measurements are taken corresponding to all shifts $\ell \in [d]_0$.  Unfortunately, in practice, this is usually an impossibility, since in many cases an illumination of the sample can cause damage to the sample, and applying the illumination beam (which can be highly irradiative) repeatedly at a single point can destory it.  In usual ptychography, the beam is shifted by a far larger distance than the width of a single pixel -- instead of overlapping on $\delta - 1$ of $\delta$ pixels, adjacent illumination regions will typically overlap on a percentage of their support on the order of $50\%$ or even less.  Considering the risks to the sample and the costs of operating the measurement equipment, there are strong incentives to reduce the number of illuminations applied to any object, and therefore our theory ought to address a model that reflects this concern.

In particular, instead of taking all $d$ shifts in $[d]$, we hope to use only $\ell \in k [d / k]_0$, where $k$ is an integer divisor of $d$.
