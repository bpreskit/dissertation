In our model for the ptychographic setup of \eqref{eq:ptych_meas}, we assume that measurements are taken corresponding to all shifts $\ell \in [d]_0$.  Unfortunately, in practice, this is usually an impossibility, since in many cases an illumination of the sample can cause damage to the sample, and applying the illumination beam (which can be highly irradiative) repeatedly at a single point can destory it.  In usual ptychography, the beam is shifted by a far larger distance than the width of a single pixel -- instead of overlapping on $\delta - 1$ of $\delta$ pixels, adjacent illumination regions will typically overlap on a percentage of their support on the order of $50\%$ or even less.  Considering the risks to the sample and the costs of operating the measurement equipment, there are strong incentives to reduce the number of illuminations applied to any object, and therefore our theory ought to address a model that reflects this concern.

In particular, instead of taking all $d$ shifts in $[d]$, we hope to use only $\ell \in k [d / k]_0$, where $k$ is an integer divisor of $d$.

\section{Spanning masks and conditioning}
In the case of ptychography, instead of using all shifts in our lifted measurement system, we instead fix a shift size $s \in \N$ where $d = \dbar s$ with $\dbar \in \N$ and use $S^{s \ell} m_j m_j^* S^{-s \ell}$ for $\ell \in [\dbar]$.  Therefore, we introduce the following generalization of the lifted measurement system.

\begin{definition}
  Given a family of masks in $\{m_j\}_{j \in [D]} \subset \C^d$ and $s, \dbar \in \N$ with $\dbar = d / s$, the associated \emph{lifted measurement system of shift $s$} is the set $\mathcal{L}_{\{m_j\}}^s := \{S^{s \ell} m_j m_j^* S^{-s \ell}\}_{(\ell, j) \in [\dbar] \times [D]} \subset \C^{d \times d}.$
\end{definition}

Of course, with a shift size $s > 1$, it is impossible for $\mathcal{L}^s$ to span $T_\delta(\C^{d \times d})$, so we consider the analagous subspace.  We will define $\mathcal{J}_{\delta, s} = \bigcup_{\ell \in [\dbar]_0}\supp(S^{s \ell} \one \one^* S^{-s \ell})$ to be the set of indices ``reached'' by this system, and \[T_\delta^s(X) = \left\{\begin{array}{r@{,\quad}l} X_{ij} & (i, j) \in \mathcal{J}_{\delta, s} \\ 0 & \text{otherwise}\end{array}\right.\] to be the projection onto the associated subspace of $\C^{d \times d}$.  Namely, we observe that \[\left(S^{s \ell} m_k m_k^* S^{-s \ell}\right)_{ij} = (S^{s \ell} m_k)_i (\overline{S^{s \ell} m_k})_j = (m_k)_{i - s \ell} (\overline{m_k})_{j - s\ell},\] so $\left(S^{s \ell} m_k m_k^* S^{-s \ell}\right)_{ij} = 0$ when $(i - s \ell, j - s \ell) \notin [\delta]^2$, i.e.~when $(i, j) \notin [\delta]^2_{s \ell + 1}$.  Hence the indices onto which we are projecting are those in $\bigcup_{\ell \in [\dbar]_0} [\delta]_{s \ell + 1}^2$.  This set may be revisualized by calculating which $j$'s are admissible for each $i$; for a fixed $i$, we look at all shifts $\ell$ such that $i \in [\delta]_{s\ell + 1}$, and $j$ is allowed to be in their union.  

In the (pathological) case where $s \ge \delta$, obviously any given index can only appear in one of the $[\delta]_{s \ell + 1}$, namely $i \in [\delta]_{s \ell + 1}$ iff $i \mod s \le \delta$ and $\floor{i / s} = \ell$, so in this case we would have \[\mathcal{J}_{\delta, s} = \{(i, j) : \floor{i / s} = \floor{j / s} \ \text{and} \ i \mathbin{\mathrm{mod}} s, j \mod s \le \delta\}.\]  However, this case is not typical, since $T_{\delta, s}(\one \one^*)$ will be the adjacency matrix of an unconnected graph, and the phase synchronization of section \ref{sec:phase_synch} will fail, as the graph Laplacian \eqref{eq:graph_laplace} will be singular.  In the ordinary case, where $s < \delta$, it is clear that we need only consider the first and last shifts that cover $i$, namely $i \in [\delta]_{1 + s \ell}$ iff $\ceil{\frac{i - \delta}{s}} \le \ell \le \ceil{\frac{i - s}{s}}$, and therefore
\begin{gather*}
  \mathcal{J}_{\delta, s} = \left\{(i, j) : j \in \{\ceil*{\frac{i - \delta}{s}}s + 1, \cdots, \ceil*{\frac{i - \delta}{s}} + \delta\}\right. \\
  \left. \cup\, \{\ceil*{\frac{i - s}{s}}s + 1, \cdots, \ceil*{\frac{i - s}{s}}s + \delta\} \right\} \\
  = \left\{(i, j) : j = \ceil*{\frac{i - \delta}{s}}s + 1, \ldots, \ceil*{\frac{i - s}{s}}s + \delta \right\}
\end{gather*}
Unfortunately, this formulation is not particularly transparent, but we mention an important special case.  When $s$ is also a divisor of $\delta$, say $\delta = \deltabar s$, then this condition becomes
\begin{gather*}
  (i, j) \in \mathcal{J}_{\delta, s} \iff \left(\ceil*{\frac{i}{s}} - \deltabar\right) s + 1 \le j \le \left(\ceil*{\frac{i}{s}} - 1\right) s + \delta \\
  \iff \frac{1}{s} - \deltabar \le  \frac{j}{s} - \ceil*{\frac{i}{s}} \le \deltabar - 1 \\
%  \iff 1 - \deltabar \le \ceil*{\frac{j}{s}} - \ceil*{\frac{i}{s}} \le \deltabar - 1 \\
  \iff \left| \ceil*{\frac{j}{s}} - \ceil*{\frac{i}{s}} \right| < \deltabar.
\end{gather*}

Before addressing invertibility and conditioning of lifted measurement systems with shifts, for $N \in \N$, we introduce $\mathcal{T}_N : \bigcup_{\ell \in \N} \C^{\ell N \times m} \to \bigcup_{\ell \in \N} \C^{\ell m \times N},$ the blockwise transpose operator, defined by \[\mathcal{T}_N\left(\begin{bmatrix} V_1 \\ \vdots \\ V_\ell \end{bmatrix}\right) = \begin{bmatrix} V_1^* \\ \vdots \\ V_{\ell}^* \end{bmatrix}\] for $V_1, \ldots, V_\ell \in \C^{N \times m}$.  We also define, for $\{k_j\}_{j = 1}^n$ and $V \in \C^{m \times n}$, \[\mathcal{I}(V, (k_j)_{j = 1}^n) = \begin{bmatrix} v_1 \otimes I_{k_1} & \cdots & v_n \otimes I_{k_n} \end{bmatrix},\]  where $v_j = V e_j$ are the columns of $V$.  This prepares us to prove the following lemmas.

\begin{lemma}
  Given $k, N, m \in \N$ and $V \in \C^{kN \times M}$, we have \[\circop^N(V)^* = \circop^m\left( (R_k \otimes I_m) \mathcal{T}_N(V) \right).\] \label{lem:circ_transpose}
\end{lemma}

\begin{proof}
  Suppose $V_i$ are the $N \times m$ blocks of $V$, such that $V = \left[V_1^T \cdots V_k^T\right]^T$.  Indexing blockwise, we have $\circop^N(V)_{[ij]} = V_{i - j + 1}$, so that $\circop^N(V)^*_{[ij]} = V_{j - i + 1}^*$.  In other words,
  \[
  \circop^N(V)^* = \begin{bmatrix} V_1^* & V_2^* & \cdots & V_N^* \\ V_N^* & V_1^* & \cdots & V_{N - 1}^* \\ \vdots & & \ddots & \vdots \\ V_2^* & V_3^* & \cdots & V_1^* \end{bmatrix} = \circop^m((R_k \otimes I_m) \mathcal{T}_N(V) )
  %\begin{bmatrix} V_1 & V_N & \cdots & V_2 \\ V_2 & V_1 & \cdots & V_3 \\ \vdots & & \ddots & \vdots \\ V_N & V_{N - 1} & \cdots & V_1 \end{bmatrix}^* \\
  \]

  as claimed.
  
\end{proof}

\begin{lemma}
  Given $N_1, N_2, k, m \in \N$ and $V_i \in \C^{k N_1 \times m}$ for $i \in [N_2]$, we have \[\begin{bmatrix} \circop^{N_1}(V_1) & \cdots & \circop^{N_1}(V_{N_2}) \end{bmatrix} (P^{(k, N_2)} \otimes I_m)^* = \circop^{N_1}(\begin{bmatrix} V_1 & \cdots & V_{N_2}\end{bmatrix}).\] \label{lem:block_circ_right}
\end{lemma}

\begin{proof}
  We quote \eqref{eq:M_2} from lemma \ref{lem:interleave} and consider that $P^{(k, N_2)} \otimes I_m$ is a permutation that changes the blockwise indices of $m \times p$ blocks (or, acting from the right, $p \times m$ blocks) exactly the way that $P^{(k, N_2)}$ changes the indices of a vector.
\end{proof}

\begin{lemma}
  Given $k, n \in \N$ and $V_j \in \C^{m_j \times n_j}$, we have \[\diag(I_k \otimes V_j)_{j = 1}^n =  P_1 (I_k \otimes \diag(V_j)_{j = 1}^n) P_2^*\] where $P_1 = \mathcal{I}(P^{(n, k)}, (m_j)_{j = 1}^n)$ and $P_2 = \mathcal{I}(P^{(n, k)}, (n_j)_{j = 1}^n)$. \label{lem:diag_kron_perm}
\end{lemma}

\begin{proof}
  We immediately reduce to the case $m_j = n_j = 1$ (and we replace $V$ with $v \in \C^n$) for all $j$ by observing that $P_1$ and $P_2$ will act on blockwise indices precisely as $P^{(n, k)}$ acts on individual indices.  Hence, we need only remark that \[\left(\diag(I_k \otimes v_\ell)_{\ell = 1}^n\right)_{((i_1 - 1)k + i_2) ((j_1 - 1)k + j_2)} = \left\{\begin{array}{r@{,\quad}l} v_{i_1} & i_1 = j_1 \ \text{and} \ i_2 = j_2 \\ 0 & \text{otherwise} \end{array}\right.,\] while \begin{align*} (P^{(n, k)} (&I_k \otimes \diag(v)) P^{(n, k)*})_{((i_1 - 1)k + i_2) ((j_1 - 1)k + j_2)} \\ &= (I_k \otimes \diag(v))_{((i_2 - 1)n + i_1) ((j_2 - 1)n + j_1)} \\ &= \left\{\begin{array}{r@{,\quad}l} v_{i_1} & i_1 = j_1 \ \text{and} \ i_2 = j_2 \\ 0 & \text{otherwise} \end{array}\right..\end{align*}
\end{proof}

For the remainder of this section, we assume that $\delta > s$.  We now consider the question of when $\Span \mathcal{L}_{\{m_j\}}^s = T_{\delta, s}$ and what the condition number of $\mathcal{A}$ will be; naturally, this requires us to have redefined $\mathcal{A}$ by \[\mathcal{A}(X)_{(\ell, j)} = \langle S^{s \ell} m_j m_j^* S^{-s \ell}, X \rangle, \quad (\ell, j) \in [\dbar]_0 \times [D].\]  As in \eqref{eq:vectorized_meas}, we vectorize $X$ by its diagonals and write $A \in \C^{\dbar D \times (2 \delta - 1) d}$ such that \begin{equation*}\left(A \begin{bmatrix} \chi_{1 - \delta} \\ \vdots \\ \chi_{\delta - 1} \end{bmatrix}\right)_{(j-1) \dbar + \ell} = \mathcal{A}(X)_{(\ell, j)}, %\label{eq:vectorized_meas_ptych}
\end{equation*} which gives the $(j - 1) \dbar + \ell\th$ row of $A$ as \[\begin{bmatrix} S^{s (\ell - 1)} g_{1 - \delta}^j \\ \vdots \\ S^{s (\ell - 1)} g_{\delta - 1}^j \end{bmatrix}^*\] so that, by lemma \ref{lem:circ_transpose}, we have \begin{equation} A = \begin{bmatrix} \circop^s(g_{1 - \delta}^1) & \cdots & \circop^s(g_{1 - \delta}^D) \\ \vdots & \ddots & \vdots \\ \circop^s(g_{\delta - 1}^1) & \cdots & \circop^s(g_{\delta - 1}^D) \end{bmatrix}^* = \begin{bmatrix} \circop(R_{\dbar} \mathcal{T}_s g_{1 - \delta}^1) & \cdots & \circop(R_{\dbar} \mathcal{T}_s g_{\delta - 1}^1) \\ \vdots & \ddots & \vdots \\ \circop(R_{\dbar} \mathcal{T}_s g_{1 - \delta}^D) & \cdots & \circop(R_{\dbar} \mathcal{T}_s g_{\delta - 1}^D) \end{bmatrix}. \label{eq:A_block_ptych} \end{equation}  However, because $T_{\delta, s} \subsetneq T_\delta$, this operator can never be invertible.  Figure \ref{fig:T_delta_s} shows this visually.  Indeed, we consider that (restricting to $m \ge 0$), even if $\supp(m_k) = [\delta]$ for all $k$, $\supp(g^k_m) = [\delta - m],$ so when $\delta - m < s, \bigcup_{\ell = 1}^{\dbar} \supp(S^{s (\ell - 1)} g^k_m) \subsetneq [d]$.  In particular, $\circop^s(g^k_m)_{ij} = 0$ for all $j$ when $i \mod_1 s > \delta - m$.  By a similar argument, for $m < 0$ we have $\circop^s(g_m^k)_{ij} = 0$ when $i \mod_1 s < s - (\delta - |m|)$.  We remark that these inequalities can only be satisfied when $m > \delta - s$ or $m > \delta - s$, respectively.

\begin{figure}
  \centering
  \begin{subfigure}[b]{0.4\textwidth}    
    \begin{tikzpicture}[ampersand replacement=\&,baseline=-\the\dimexpr\fontdimen22\textfont2\relax]
    \matrix (m)[matrix of math nodes,left delimiter=(,right delimiter=)]
            {
              * \& * \& * \&   \&   \&   \& * \& * \\
              * \& * \& * \& * \&   \&   \&   \& * \\
              * \& * \& * \& * \& * \&   \&   \&   \\
                \& * \& * \& * \& * \& * \&   \&   \\
                \&   \& * \& * \& * \& * \& * \&   \\
                \&   \&   \& * \& * \& * \& * \& * \\
              * \&   \&   \&   \& * \& * \& * \& * \\
              * \& * \&   \&   \&   \& * \& * \& * \\
            };

            \begin{pgfonlayer}{myback}
              \fhighlight[blue!30]{m-1-1}{m-3-3}
              \fhighlight[blue!30]{m-2-2}{m-4-4}
              \fhighlight[blue!30]{m-3-3}{m-5-5}
              \fhighlight[blue!30]{m-4-4}{m-6-6}
              \fhighlight[blue!30]{m-5-5}{m-7-7}
              \fhighlight[blue!30]{m-6-6}{m-8-8}
              \fhighlight[blue!30]{m-7-1}{m-8-1}
              \fhighlight[blue!30]{m-1-7}{m-1-8}
              \fhighlight[blue!30]{m-8-8}{m-8-8}
              \fhighlight[blue!30]{m-8-1}{m-8-2}
              \fhighlight[blue!30]{m-1-8}{m-2-8}
              \fhighlight[blue!30]{m-1-1}{m-2-2}
            \end{pgfonlayer}
  \end{tikzpicture}
    \caption{$T_3(\C^{8 \times 8})$}
  \end{subfigure}
  \begin{subfigure}[b]{0.4\textwidth}
    \begin{tikzpicture}[ampersand replacement=\&,baseline=-\the\dimexpr\fontdimen22\textfont2\relax]
    \matrix (m)[matrix of math nodes,left delimiter=(,right delimiter=)]
            {
              * \& * \& * \&   \&   \&   \& * \& * \\
              * \& * \& * \& * \&   \&   \&   \& * \\
              * \& * \& * \& * \& * \&   \&   \&   \\
                \& * \& * \& * \& * \& * \&   \&   \\
                \&   \& * \& * \& * \& * \& * \&   \\
                \&   \&   \& * \& * \& * \& * \& * \\
              * \&   \&   \&   \& * \& * \& * \& * \\
              * \& * \&   \&   \&   \& * \& * \& * \\
            };

            \begin{pgfonlayer}{myback}
              \fhighlight[blue!30]{m-1-1}{m-3-3}
              %\fhighlight[blue!30]{m-2-2}{m-4-4}
              \fhighlight[blue!30]{m-3-3}{m-5-5}
              %\fhighlight[blue!30]{m-4-4}{m-6-6}
              \fhighlight[blue!30]{m-5-5}{m-7-7}
              %\fhighlight[blue!30]{m-6-6}{m-8-8}
              \fhighlight[blue!30]{m-7-1}{m-8-1}
              \fhighlight[blue!30]{m-1-7}{m-1-8}
              \fhighlight[blue!30]{m-7-7}{m-8-8}
              %\fhighlight[blue!30]{m-8-8}{m-8-8}
              %\fhighlight[blue!30]{m-8-1}{m-8-2}
              %\fhighlight[blue!30]{m-1-8}{m-2-8}
              %\fhighlight[blue!30]{m-1-1}{m-2-2}
            \end{pgfonlayer}
    \end{tikzpicture}
    \caption{$T_{3, 2}(\C^{8 \times 8})$}
  \end{subfigure}
  \caption{$T_\delta(\C^{d \times d})$ vs. $T_{\delta, s}(\C^{d \times d})$ for $d = 8, \delta = 3, s = 2$}
  \label{fig:T_delta_s}  
\end{figure}

By reference to \eqref{eq:A_block_ptych}, it is clear that each of these ``missing indices'' results in a column of all zeros in $A$; specifically, viewing $A e_{(m + \delta - 1)d + i}, (m, i) \in [2 \delta - 1]_{1 - \delta} \times [d]$ as the $i\th$ column of the $m + \delta\th$ block of $A$, we see \[A e_{(m + \delta - 1) d + i} = 0\quad \text{if} \quad \left\{\begin{array}{rcl@{\,,\quad}l} i \mod_1 s & > & \delta - m & m \ge 0 \\ i \mod_1 s & < & s - (\delta + m) & m \le 0 \end{array}\right..\]  Since $\delta > s$, we may reduce this condition to ``$i \mod s > \delta - m$ or $i \mod s < s - (\delta + m)$,'' or further to $i \mod s \notin [2 \delta - s + 1]_{s - \delta - m}$.  Therefore, the matrix representing $\mathcal{A}$ restricted to $T_{\delta, s}(\C^{d \times d})$ is found by right multiplying $A$ by \begin{equation}N = \diag(I_{\dbar} \otimes N_{j - \delta})_{j = 1}^{2 \delta - 1},\ \text{where} \ N_m = \left\{\begin{array}{c@{,\quad}l} \begin{bmatrix} 0_{\delta + m} \\ I_{s - (\delta + m)} \end{bmatrix} & m < s - \delta \vspace{2pt}\\ \begin{bmatrix} I_{s - (\delta - m)} \\ 0_{\delta - m} \end{bmatrix} & m > \delta - s \\ I_s & \text{otherwise} \end{array}\right.. \label{eq:N_and_N_m}\end{equation}

But does this restriction commute well with the permutations used in the condition number analysis of section \ref{sec:con_number}?  Thankfully it does; following the intuition of \eqref{eq:interleaved_meas} and making use of lemma \ref{lem:block_circ_right}, we can arrive at \begin{align*} A' := P^{(\dbar, D)} A \left(P^{(\dbar, 2 \delta - 1)} \otimes I_s\right)^* &= \circop^D\left(P^{(\dbar, D)} \begin{bmatrix} R_{\dbar} \mathcal{T}_s g_{1 - \delta}^1 & \cdots & R_{\dbar} \mathcal{T}_s g_{\delta - 1}^1 \\ \vdots & \ddots & \vdots \\ R_{\dbar} \mathcal{T}_s g_{1 - \delta}^D & \cdots & R_{\dbar} \mathcal{T}_s g_{\delta - 1}^D \end{bmatrix}\right). \\ &= \circop^D\left(P^{(\dbar, D)} (I_D \otimes R_{\dbar}) \begin{bmatrix}  \mathcal{T}_s g_{1 - \delta}^1 & \cdots &  \mathcal{T}_s g_{\delta - 1}^1 \\ \vdots & \ddots & \vdots \\  \mathcal{T}_s g_{1 - \delta}^D & \cdots &  \mathcal{T}_s g_{\delta - 1}^D \end{bmatrix}\right). \end{align*}  In the interest of finding the locations of the zero columns after this permutation, we remark that the inner matrix is of size $\dbar D \times s (2 \delta - 1)$, and that the $\circop^D$ operation will therefore repeat it $\dbar$ times.  It is then clear that \begin{gather*} A'e_{(\ell - 1)s (2 \delta - 1) + i} = 0 \iff \begin{bmatrix}  \mathcal{T}_s g_{1 - \delta}^1 & \cdots &  \mathcal{T}_s g_{\delta - 1}^1 \\ \vdots & \ddots & \vdots \\  \mathcal{T}_s g_{1 - \delta}^D & \cdots &  \mathcal{T}_s g_{\delta - 1}^D \end{bmatrix} e_i = 0,\ \text{and} \\ \begin{bmatrix}  \mathcal{T}_s g_{1 - \delta}^1 & \cdots &  \mathcal{T}_s g_{\delta - 1}^1 \\ \vdots & \ddots & \vdots \\  \mathcal{T}_s g_{1 - \delta}^D & \cdots &  \mathcal{T}_s g_{\delta - 1}^D \end{bmatrix} e_{(m + \delta - 1)s + i} = 0 \iff i \notin [2 \delta - s + 1]_{s - \delta - m},\end{gather*} so we may remove the zero columns from $A'$ by right multiplying the interior matrix by $N' = \diag(N_m)_{m = 1 - \delta}^{\delta - 1}$.  That is, \begin{align} A' (I_{\dbar} \otimes N') &= \circop^D\left(P^{(\dbar, D)} (I_D \otimes R_{\dbar}) \begin{bmatrix}  \mathcal{T}_s g_{1 - \delta}^1 & \cdots &  \mathcal{T}_s g_{\delta - 1}^1 \\ \vdots & \ddots & \vdots \\  \mathcal{T}_s g_{1 - \delta}^D & \cdots &  \mathcal{T}_s g_{\delta - 1}^D \end{bmatrix} \begin{bmatrix} N_{1 - \delta} & & \\ & \ddots & \\ & & N_{\delta - 1} \end{bmatrix}\right) \nonumber\\ & = P^{(\dbar, D)} A N P', \label{eq:ptych_permute_final}\end{align} where the second equality comes from lemma \ref{lem:diag_kron_perm} and \[P' = \mathcal{I}(P^{(\dbar, 2 \delta - 1)}, (\min\{s, \delta - |m|\})_{m = 1 - \delta}^{\delta - 1}).\]  This result, along with corollary \ref{cor:circ_diag_condition}, gives us the following proposition.

\begin{proposition}
  Taking $A$ as in \eqref{eq:A_block_ptych}, $N$ and $N_m$ as in \eqref{eq:N_and_N_m}, and setting \[H = P^{(\dbar, D)} (I_D \otimes R_{\dbar}) \begin{bmatrix}  \mathcal{T}_s g_{1 - \delta}^1 & \cdots &  \mathcal{T}_s g_{\delta - 1}^1 \\ \vdots & \ddots & \vdots \\  \mathcal{T}_s g_{1 - \delta}^D & \cdots &  \mathcal{T}_s g_{\delta - 1}^D \end{bmatrix} \diag(N_m)_{m = 1 - \delta}^{\delta - 1}\] and $M_j = \sqrt{\dbar}(f_j^{\dbar} \otimes I_D)^* H$ for $j \in [\dbar]$, the condition number of $AN$ is given by \[\dfrac{\max\limits_{i \in [\dbar]} \sigma_{\max} (M_i)}{\min\limits_{i \in [\dbar]} \sigma_{\min} (M_i)}.\]  In particular, $\left.\mathcal{A}\right|_{T_{\delta, s}(\C^{d \times d})}$ is invertible if and only if each of the $M_i$ are of full rank.
\end{proposition}

\label{sec:con_number_ptych}

\section{Recovery algorithm}
In this section, we discuss a number of algorithms by which we can recover an estimate of $x_0$ from $T_{\delta, s}(\mathcal{A}^{-1}(y))$.  We begin with \cref{sec:blocky_block}, which discusses some improvements that can be made to the magnitude estimation step of \cref{alg:phaseRetrieval1} (the $\sqrt{X_{j,j}}$ of line 4).  These improvements, which we call ``blockwise'' vector or magnitude estimation, were first implemented in the numerical study of \cref{sec:MagEstImpNumerical}, and while they empirically delivered better results, no proof was yet discovered to guarantee by how much they improve the estimate of magnitudes.  \Cref{sec:pty_std_rec} describes and proves the robustness bounds for an algorithm almost completely analogous to \cref{alg:phaseRetrieval1}, taking advantage of the results in \cref{sec:blocky_block} for the magnitude estimation.  Finally, in \cref{sec:pty_vec_sync}, we use blockwise vector estimation to devise a completely new recovery algorithm that is quite general with respect to the sets of shifts it permits.  Although its recovery guarantees are not attractive, we will find in \cref{sec:ptych_num} that these theoretical results belie strong empirical performance.

\subsection{Blockwise Magnitude and Vector Estimation}
\label{sec:blocky_block}
For the moment, we restrict our discussion to the special case of dense shifts in our measurements, where $s = 1$ as in the work of \cref{ch:base_model,ch:meas}.  In \cref{sec:MagEstImpNumerical}, it was noted that the technique used to calculate the magnitudes of the entries of $\ux$ from $X \approx T_\delta(\ux \ux^*)$ was functional, but rudimentary.  Recall from line 4 of \cref{alg:phaseRetrieval1} that we have $X = \Ac^{-1}(\Ac(x_0 x_0^*) + n)$, where $\ux \in \Cd$ is the ground truth objective vector, $\Ac : \Cdxd \to \R^{d D}$ is the linear measurement operator determined by the masks $\{m_j\}_{j \in [D]}$, as defined, for example, in \eqref{eq:meas_op} of \cref{sec:meas_intro}, and $n \in \R^{d D}$ is arbitrary noise.  Then $X = \uX + \Ac^{-1}(n)$, where $\uX = T_\delta(\ux \ux^*)$, and we estimate the magnitude of $\ux_i$ by simply taking $\abs{x_i} = \sqrt{X_{ii}} \approx \sqrt{\uX_{ii}} = \abs{\ux_i}$.  This technique works, as we are able to show in \cref{lem:diag_mag_diff}, but empirically, we found that a slightly more sophisticated technique does a much better job here.  While this comparison was not made explicit in \cref{sec:NumEval}, we briefly illustrate it in \cref{sec:ptcyh_num}.

We notice that taking $\abs{x_i} = X_{ii}$ is equivalent to taking $\abs{x_i}$ to be the rank-1 approximation of the $1 \times 1, i\th$ diagonal block matrix of $X$, namely $\begin{bmatrix} X_{ii} \end{bmatrix}$, since these diagonal blocks are equal to the diagonal blocks of the \emph{untruncated} $\ux \ux^*$ when there is no noise.  However, given the width of the diagonal band in $T_\delta(\Cdxd)$, we could just as easily take blocks of size up to $\delta \times \delta$ and calculate their top eigenvectors; this would give us $2 \delta - 1$ estimates for each entry's magnitude, so we can combine them by averaging them together.  %To denote these blocks, we will set, for $\{j_1, \ldots, j_k\} = J \subset [d], P_{J} =\rowmatfl{e_{j_1}}{e_{j_k}}$, and take $X^{(\ell)} = P_{[\delta]_\ell} X P_{[\delta]_\ell}^*$ so that $X^{(\ell)} \in \C^{\delta \times \delta}$ and $X^{(\ell)}_{ij} = X_{i + \ell - 1, j + \ell - 1}$.  We may then take $\tilde{u}^{(\ell)}$ to be the top eigenvector of $X^{(\ell)}$, normalized such that $\norm{\tilde{u}^{(\ell)}}_2 = \norm{X^{(\ell)}}_2$.  We then use $u^{(\ell)} = S_d^{\ell - 1} \Rc_d(\tilde{u}^{(\ell)}) \in \Cd$ as the ``spatially
To denote these blocks, we will set $\abs{X}^{(\ell)} = \diag(\one_{[\delta]_\ell}) \abs{X} \diag(\one_{[\delta]_\ell})$\footnote{Here, and in the remainder of this section, we emphasize that all indices of objects in $\Cd$ and $\Cdxd$ are taken modulo $d$.} and $u^{(\ell)}$ to be the top eigenvector of $\abs{X}^{(\ell)}$, normalized such that $\norm{u^{(\ell)}}_2 = \norm{\abs{X}^{(\ell)}}_2$.  We then produce our estimate of the magnitudes by taking $\abs{x} = \frac{1}{2 \delta - 1} \sum_{\ell = 1}^d u^{(\ell)}$.  

Before formally stating this algorithm, we observe a few ways in which it may be generalized.  Firstly, we notice that this method can easily handle arbitrary block sizes $m$ for the blockwise, eigenvector-based magnitude estimations by simply taking $\abs{X}^{(\ell, m)} = \diag(\one_{[m]_\ell}) \abs{X} \diag(\one_{[m]_\ell})$, as long as $m \le \delta$ -- although this would require us to change the denominator in the averaging step, using $\frac{1}{2 m - 1} \sum_{\ell = 1}^d u^{(\ell, m)}$.  Proceeding even further, we can generalize this technique to use any collection $\{J_i\}_{i = 1}^N, J_i \subset [d]$ satisfying \begin{equation} \begin{gathered} [d] \subset \bigcup_{i = 1}^N J_i \\ \one_{J_i} \one_{J_i}^* \in T_{\delta}(\Cdxd) \end{gathered} \label{eq:T_delta_cover}\end{equation}  We will call any collection satisfying \eqref{eq:T_delta_cover} a \emph{$(T_\delta, d)$-covering} (or just \emph{covering} when $T_\delta$ and $d$ are clear from context), and the process of estimating magnitudes of $\ux$ from $X$ with respect to a $(T_\delta, d)$-covering is described in \cref{alg:blocky_block}.\footnote{We remark that this definition and the recovery algorithm are very obviously extensible to the use of $T_{\delta, s}$ instead of $T_\delta$.  In fact, this is a \emph{restriction}, if we consider in \eqref{eq:T_delta_cover} that $T_{\delta, s} \subset T_\delta$.  The definition, therefore, of a $(T_{\delta, s}, d)$-covering, is made by analogy to \eqref{eq:T_delta_cover}.}  It is worth remarking that the ``averaging step,'' specified in line 3, is optimal in the least-squares sense.  We set $P_{J_i} = \diag(\one_{J_i})$ to be the orthogonal projections onto the coordinate subspace associated with $J_i$ and consider that the vectors $u^{(J_i)}$ (in line 2) represent estimates of the projections $P_{J_i} \abs{\ux}$.  The least squares solution to $P_{J_i} u = u^{(J_i)}$, or \[\colmatfl{P_{J_1}}{P_{J_N}} u = \colmatfl{u^{(J_1)}}{u^{(J_N)}}\] is obtained by taking the pseudoinverse.  In this case, we have \[\colmatfl{P_{J_1}}{P_{J_N}}^* \colmatfl{P_{J_1}}{P_{J_N}} = \sum_{i = 1}^N P_{J_i}^* P_{J_i} = \sum_{i = 1}^N P_{J_i} = \diag(\mu),\] with $\mu_j = \abs{\{i : j \in J_i\}}$ as in line 1 of \cref{alg:blocky_block}.  Considering that $P_{J_i} u^{(J_i)} = u^{(J_i)}$, we have \[u = \colmatfl{P_{J_1}}{P_{J_N}}^{\dag} \colmatfl{u^{(J_1)}}{u^{(J_N)}} =  \diag(\mu)^{-1} \colmatfl{P_{J_1}}{P_{J_N}}^* \colmatfl{u^{(J_1)}}{u^{(J_N)}} = D_\mu^{-1}\left(\sum_{i = 1}^N u^{(J_i)}\right).\]

\begin{algorithm}[htbp]
\renewcommand{\algorithmicrequire}{\textbf{Input:}}
\renewcommand{\algorithmicensure}{\textbf{Output:}}
\caption{Blockwise Magnitude Estimation}
\label{alg:blocky_block}
\begin{algorithmic}[1]
    \REQUIRE $X \in T_\delta(\H^d)$, typically assumed to be an approximation $X \approx T_\delta(\ux \ux^*)$.  A $(T_\delta, d)$-covering $\{J_i\}_{i \in [N]}$.
    \ENSURE An estimate $\abs{x}$ of $\abs{\ux}$.
    \STATE For $j \in [d]$, set $\mu_j = \abs{\{i : j \in J_i\}}$ to be the number of appearences the index $j$ makes in $\{J_i\}$.
    \STATE For $i \in [N]$, set $u^{(J_i)}$ to be the leading eigenvector of $X^{(J_i)} = \diag(\one_{J_i}) X \diag(\one_{J_i})$, normalized such that $\norm{u^{(J_i)}}_2 = \sqrt{\norm{X^{(J_i)}}_2}$.
    \STATE Return $\abs{x} = D_\mu^{-1}\left(\sum_{i = 1}^N u^{(J_i)}\right)$.
    \end{algorithmic}
\end{algorithm}

We will denote the output of \cref{alg:blocky_block} by $\abs{x} = \BlkMag(X, \{J_i\})$.  In an overloading of notation, when the covering consists of all intervals of length $m$, in the sense that $J_i = [m]_i$ for $i = 1, \ldots, d$, we will also write it as $\BlkMag(X, \{[m]_i\}_{i \in [d]}) = \BlkMag(X, m)$.  In this way, the magnitude estimation technique used in \cref{sec:NumEval} is simply $\abs{x} = \BlkMag(X, \delta)$, whereas the $\abs{x_i} = \sqrt{X_{ii}}$ technique stated in \cref{alg:phaseRetrieval1} is equivalent to $\abs{x} = \BlkMag(X, 1)$.

Using \cref{cor:rank1Bound}, we are able to quickly prove a bound on the error of the estimate produced by this method.

\begin{proposition}
  Let $\{J_i\}_{i \in [N]}$ be a $(T_\delta, d)$-covering, and suppose $\uX = T_\delta(\ux \ux^*)$ for some $\ux \in \Cd$.  Using the notation of \cref{alg:blocky_block} (in particular, $\mu_j$ is as in line 1, and $\uu^{(J_i)} = \abs{\diag(\one_{J_i}) \ux}$), given $X \in T_\delta(\H^d)$, we have that the output $\abs{x}$ satisfies
  \begin{equation}
  \begin{aligned}
    \BlkMag(\uX, \{J_i\}) &= \abs{\ux} \\
    \norm{\BlkMag(X, \{J_i\}) - \abs{\ux}}_2 &\le \dfrac{\max_j \mu_j}{\min_j \mu_j} \dfrac{1 + 2 \sqrt{2}}{\min_i \norm{\uu^{(J_i)}}_2} \norm{\uX - X}_F.
  \end{aligned}
  \label{eq:blocky_rec}
  \end{equation}
  As special cases, we have
  \begin{equation}
    \begin{aligned}
      \norm{\BlkMag(X, m) - \ux}_2 &\le \dfrac{1 + 2 \sqrt{2}}{\min_i \norm{\uu^{[m]_i}}_2} \norm{X - \uX}_F \\
      \norm{\BlkMag(X, \delta) - \ux}_2 &\le \dfrac{1 + 2 \sqrt{2}}{\min_i \norm{\uu^{[\delta]_i}}_2} \norm{X - \uX}_F \\
      \norm{\BlkMag(X, 1) - \ux}_2 &\le \dfrac{\norm{\diag(X - \uX)}_F}{\min_i \abs{\ux_i}} 
    \end{aligned}
    \label{eq:blocky_spec}
  \end{equation}
  \label{prop:blocky_block}
\end{proposition}

\begin{proof}[Proof of \cref{prop:blocky_block}]
  The first inequality of \eqref{eq:blocky_rec} is clear, since line 2 of \cref{alg:blocky_block} will always return $\uu^{(J_i)} = \one_{J_i} \circ \abs{\ux}$, so line 3 will give \[ \left(D_\mu^{-1}\left(\sum_{i = 1}^N \uu^{(J_i)}\right)\right)_j = \frac{1}{\mu_j} \sum_{i = 1}^N \abs{\ux}_j \one_{j \in J_i} = \abs{\ux}_j.\]  %The second comes quite easily by applying \cref{cor:rank1Bound}\footnote{Here, we use the substitution $\eta \norm{\x_0}_2 = \frac{\norm{X - X_0}_F}{\norm{\x_0}_2}$.} and writing
  The second comes by writing \begin{equation} \norm*{D_\mu^{-1}(\sum_{i = 1}^N u^{(J_i)} - \uu^{(J_i)})}_2^2 \le \left(\frac{1}{\min_j \mu_j}\right)^2 \norm*{\sum_{i = 1}^N (u^{(J_i)} - \uu^{(J_i)})}_2^2. \label{eq:bb1} \end{equation}  From there, we consider that the $j\th$ term in the summation \begin{equation}\norm*{\sum_{i = 1}^N (u^{(J_i)} - \uu^{(J_i)})}_2^2 = \sum_{j = 1}^d \left(\sum_{i = 1}^N u^{(J_i)} - \uu^{(J_i)}\right)_j^2\label{eq:bb2}\end{equation} has at most $\max_k \mu_k$ nonzero summands, so, by $(\sum_{i=1}^n a_i)^2 \le n \sum_{i=1}^n a_i^2$, we have \begin{equation}\norm*{\sum_{i = 1}^N (u^{(J_i)} - \uu^{(J_i)})}_2^2 \le \max_j \mu_j \sum_{i = 1}^N (u^{(J_i)} - \uu^{(J_i)})^2.\label{eq:bb3}\end{equation}  We then apply \cref{cor:rank1Bound}\footnote{Here, we use the substitution $\eta \norm{\x_0}_2 = \frac{\norm{X - X_0}_F}{\norm{\x_0}_2}$.} to get \begin{equation} \sum_{i = 1}^N (u^{(J_i)} - \uu^{(J_i)})^2 \le (1 + 2 \sqrt{2}) \dfrac{\norm{\uu^{(J_i)} \uu^{(J_i)*} - X^{(J_i)}}_F^2}{\norm{\uu^{(J_i)}}_2^2} \label{eq:bb4} \end{equation}  In this expression, we consider that, in the summation $\sum_{i = 1}^N \norm{\uu^{(J_i)} \uu^{(J_i)*} - X^{(J_i)}}_F^2,$ the term $(\uX_{ij} - X_{ij})^2$ appears at most $\max\{\mu_i, \mu_j\}$ times, such that $\sum_{i = 1}^N \norm{\uu^{(J_i)} \uu^{(J_i)*} - X^{(J_i)}}_F^2 \le \max_j \mu_j \norm{\uX - X}_F^2$.  Using this substitution, combining \eqref{eq:bb1}--\eqref{eq:bb4}, and taking the square root of both sides gives \[\norm*{D_\mu^{-1}(\sum_{i = 1}^N u^{(J_i)} - \uu^{(J_i)})}_2 \le \dfrac{\max_j \mu_j}{\min \mu_j} \dfrac{1 + 2 \sqrt{2}}{\min_i \norm{\uu^{(J_i)}}_2} \norm{\uX - X}_F\] as desired.

  The first two inequalities of \eqref{eq:blocky_spec} are immediate by observing that $\mu_j = 2m - 1$ when the covering is $\{[m]_i\}_{i \in [d]}$.  The third comes from setting $\epsilon_i = X_{ii} - \uX_{ii}$ and writing
  \begin{align*}
    \norm{\abs{x} - \abs{\ux}}_2^2 &= \sum_{i = 1}^d \left(\sqrt{X_{ii}} - \abs{\ux}_i\right)^2 = \sum_{i = 1}^d \left(\sqrt{\abs{\ux}_i^2 + \epsilon_i} - \sqrt{\abs{\ux}_i^2}\right)^2 \\
    &= \sum_{i = 1}^d \left(\frac{((\abs{\ux}_i^2 + \epsilon_i) - \abs{\ux}_i^2)^2}{\sqrt{\abs{\ux}_i^2 + \epsilon_i} + \abs{\ux}_i}\right)^2 \le \sum_{i = 1}^d \frac{\epsilon_i^2}{\min_i{\abs{\ux}_i^2}} \\ &= \frac{\norm{\diag(X - \uX)}_F^2}{\min_i \abs{\ux}_i^2}
  \end{align*}  
\end{proof}

We end with a few remarks on the results of \cref{prop:blocky_block}.  One immediate benefit from \cref{eq:blocky_spec} is that the estimation error from $\BlkMag(X, 1)$ no longer scales poorly with $d^{1/4}$ as in \cref{lem:diag_mag_diff}.  The $\min_i \abs{\ux_i}$ factor in the denominator is also not a problem, since in the recovery results of, say, \cref{thm:MainRes} or \cref{cor:main_improve}, the same term (to a higher power) already appears in the phase-error expression.

In the error bound for $\BlkMag(X, m)$ in \cref{eq:blocky_spec}, we notice that the bound is \emph{strictly decreasing} with $m$, since, for $m_1 > m_2$, we have \[\min_i \norm{\ux^{[m_1]_i}}_2^2 \ge (m_1 - m_2) \min_i \abs{\ux_i}^2 + \min_i \norm{\ux^{[m_2]_i}}_2^2 \ge m_1 \min_i \abs{\ux_i}^2.\]  Also, considering \eqref{eq:blocky_rec}, it is clear that $\BlkMag(X, \delta)$ gives the absolute best bound, since any $(T_\delta, d)$-covering $\{J_i\}_{i \in [N]}$ satisfies $J_i \subset [\delta]_\ell$ for some $\ell$.  This gives that $\min_i \norm{\uu^{(J_i)}}_2 \le \min_i \norm{\uu^{[\delta]_i}}_2$, and obviously $\frac{\max_j \mu_j}{\min_j \mu_j} \ge 1$, so the bound for $\BlkMag(X, \{J_i\})$ in \eqref{eq:blocky_rec} can never be better than that for $\BlkMag(X, \delta)$ in \eqref{eq:blocky_spec}.

We also remark that two easy ways to ensure $\frac{\max_j \mu_j}{\min_j \mu_j} = 1$ is minimized are to take some fixed $J_0 \subset [\delta]_0$ and let $J_\ell = J_0 + \ell$ be a ``cyclic'' covering, or to let $\{J_i\}$ be a partition of $[d]$.  These strategies will be relevant in \cref{sec:pty_std_rec}, but in the case of $s = 1$, $\BlkMag(X, \delta)$ always has the optimal bound for magnitude estimation error.

\subsection{Standard Recovery Algorithm for Ptychography}
\label{sec:pty_std_rec}
To recover $x$, an estimate of $\eit \ux$ from $X = \Ac^{-1}(y)$, we will not have to develop, nor even prove, any more technology.  The contents of \cref{ch:ang_sync,sec:con_number_ptych,sec:blocky_block} are sufficient to develop an algorithm, stated in \cref{alg:pty_pr} that is proven in \cref{thm:pty_rec} to stably produce an estimate of $\eit \ux$.  This algorithm differs very little from \cref{alg:pr_sdp}, except that in line 4, we use $\BlkMag(X, \{J_i\}) \circ \tbx$ instead of $\abs{\vec \diag(X)} \circ \tbx$.

\begin{algorithm}
\renewcommand{\algorithmicrequire}{\textbf{Input:}}
\renewcommand{\algorithmicensure}{\textbf{Output:}}
\caption{Phase Retrieval from Local Ptychographic Measurements}
\label{alg:pty_pr}
\begin{algorithmic}[1]
    \REQUIRE A family of masks $\{m_j\}_{j = 1}^D$ of support $\delta$; $s, d, \dbar \in \N$ satisfying $d = \dbar s \ge 2 \delta - 1$.  A $(T_{\delta, s}, d)$-covering $\{J_i\}_{i \in [N]}$.  Measurements $\y \approx \Ac(\ux \ux^*) \in \R^{\dbar D}$, as in \eqref{eq:pty_meas_op}.
    \ENSURE $x \in \Cd$ with $x \approx \eit \ux$ for some $\theta \in [0, 2 \pi]$
    \STATE Compute the Hermitian matrix $X = \Ac|_{T_{\delta, s}(\Cdxd)}^{-1} y \in T_{\delta, s}(\Hd)$ as an estimate of $T_{\delta, s}(\ux \ux^*)$
    \STATE Form the banded matrix of phases, $\tX \in T_{\delta, s}(\Cdxd)$, by normalizing the non-zero entries of $X$ %(replacing any zero entries in the band with $1$'s)
    \STATE Compute $\hZ$, the solution to \eqref{eq:ang_sync_sdp} with $L = (2 \delta - 1) I - \tX$, and take $\tbx = \sgn(u),$ where $u$ is the top eigenvector of $\hZ$.
    \STATE Return $x = \BlkMag(X, \{J_i\}) \circ \tbx$
\end{algorithmic}
\end{algorithm}

\Cref{thm:pty_rec} proves a bound on the accuracy of the output of \cref{alg:pty_pr}.  Here, we use the Laplacian matrix of the graph whose adjacency matrix is $T_{\delta, s}(\one \one^*) - I$, namely \begin{equation}D - W, \ \text{where} \ W = T_{\delta, s}(\one \one^*) - I \ \text{and} \ D = \diag(W \one).\label{eq:pty_lap}\end{equation}

\begin{theorem} \label{thm:pty_rec}
  Suppose we have a family of masks $\{m_j\}_{j \in [D]} \in \Cd$ of support $\delta$; $s, \dbar \in \N$ such that $s \dbar = d$; and a $(T_{\delta, s}, d)$-covering $\{J_i\}_{i \in [N]}$.  Further let $\ux \in \Cd, n \in \R^{\dbar D}$ be arbitrary and set $\uX = \ux \ux^*, X = \Ac^{-1}(\Ac(\uX) + n)$.  Define $\mu$ as in line 1 of \cref{alg:blocky_block}, $\tau_G = \lambda_2(D - W)$ for the matrices defined in \eqref{eq:pty_lap}, and $\SNR = \frac{\norm{\Ac(X)}_2}{\norm{n}_2}$%% , and $\alpha = \sum_{m = 1 - \delta}^{\delta - 1} \min\{s, \delta - \abs{m}\}$
  .  Then if line 3 of \cref{alg:pty_pr} solves \eqref{eq:ang_sync} exactly, then the output $x$ of \cref{alg:pty_pr} satisfies \begin{equation}\begin{aligned} \mintheta \norm{x - \eit \ux}_2 &\le \dfrac{4 \sqrt{2}}{\sqrt{\tau_G}} \dfrac{\sigma_{\min}^{-1} \norm{n}_2}{\abs{(\ux)_{\min}}^2} + \dfrac{\max_j \mu_j}{\min_j \mu_j} \dfrac{1 + 2 \sqrt{2}}{\min_i \norm{\uu^{(J_i)}}_2} \sigma_{\min}^{-1} \norm{n}_2 \\ \mintheta \norm{x - \eit \ux}_2 &\le \dfrac{4 \sqrt{2}}{\sqrt{\tau_G}} \dfrac{\kappa \norm{\uX}_F}{\abs{(\ux)_{\min}}^2 \SNR} + \dfrac{\max_j \mu_j}{\min_j \mu_j} \dfrac{1 + 2 \sqrt{2}}{\min_i \norm{\uu^{(J_i)}}_2} \kappa \frac{\norm{\uX}_F}{\SNR} \end{aligned} \label{eq:pty_rec}\end{equation}
\end{theorem}

\begin{proof}[Proof of \cref{thm:pty_rec}]
  This proof is a straightforward synthesis of the results in \cref{thm:improved_spec_pert,prop:blocky_block}.  In detail, we set $\tbx = \sgn(x), \tux = \sgn(\ux)$ and apply these two results by expanding
  \begin{align*}
    \mintheta \norm{x - \eit \ux}_2 &\le \mintheta \norm{\tbx - \eit \tux}_2 + \norm{\abs{x} - \abs{\ux}} \\
    &\le \dfrac{2 \sqrt{2} \norm{\tX - \widetilde{\uX}}_F}{\sqrt{\tau_G}} + \dfrac{\max_j \mu_j}{\min_j \mu_j} \dfrac{1 + 2 \sqrt{2}}{\min_i \norm{\uu^{(J_i)}}_2} \norm{X - \uX}_F.
  \end{align*}
  At this point, we quote \cref{lem:2d_srho} to get $\norm{\tX - \widetilde{\uX}}_F \le 2 \norm{X - \uX}_F / \abs{(\ux_{\min})}^2$ as usual, and finish by bounding $\norm{X - \uX}_F$ above by $\sigma_{\min}^{-1} \norm{n}_2$ and $\kappa \frac{\norm{\uX}_F}{\SNR}$.  
\end{proof}

We remark that a few elements of the bounds in \eqref{eq:pty_rec} are a bit unclear, most notably $\frac{\max_j \mu_j}{\min_j \mu_j}$ and $\tau_G$.  Of these, $\frac{\max_j \mu_j}{\min_j \mu_j}$ is a design feature that is chosen when we select our $(T_{\delta, s}, d)$-covering for the magnitude estimation step.  As is mentioned in \cref{sec:blocky_block}, one strategy for keeping this term under control would be making $\{J_i\}_{i \in [N]}$ a partition of $[d]$.  However, using the somewhat intuitive (if we follow the motivation behind \cref{sec:blocky_block}), maximal (it uses the largest possible $J_i$, which is an advantage for the $\min_i \norm{u^{(J_i)}}_2$ term) covering $J_i = \clopen{1 + (i - 1)s, \delta + 1 + (i - 1)s}$ for $i \in [\dbar]$, we notice that $\frac{\max_j \mu_j}{\min_j \mu_j} \in [1, 2]$.\footnote{The proof of this is brief: $j \in \clopen{1 + (\ell - 1) s, \delta + 1 + (\ell - 1)s}$ iff $\ell - 1 \in \clopen{\ceil{\frac{j- \delta}{s}}, \floor{\frac{j - 1}{s}} + 1}$, so that $\mu_j = \floor{\frac{j - 1}{s}} - \ceil{\frac{j - \delta}{s}} + 1$.  Clearly $\mu_j$ is periodic in $[s]$, so we take extrema over this interval, which clearly gives $\floor{\frac{\delta}{s}} \le \mu_j \le \floor{\frac{\delta - 1}{s}} + 1$.}

However, $\tau_G$ is still mysterious -- we have yet to obtain a convenient expression for $\tau_G$, so we relegate its study to the numerical experiments of \cref{sec:ptych_num}.

\subsection{Blockwise Vector Synchronization Method}
\label{sec:pty_vec_sync}
Blocky block plus vecky vec

\label{sec:ptych_recovery}
