With this setup in hand, we begin our analysis of the linear system $\Ac(X) = y$ with a number of lemmas that unravel the structure of this operator.  Our goal will be to proceed similarly to \cref{sec:con_number} by rewriting $\Ac$ as a product of block-circulant matrix with certain permutations, at which point we will be able to cite \cref{cor:circ_diag_condition}.

\subsection{Lemmas on Block Circulant Structure}
\label{sec:ptych_lem}
We begin with \cref{lem:circ_transpose}, which describes the transposes of block circulant matrices.
\begin{lemma}
\label{lem:circ_transpose} Given $k, N, m \in \N$ and $V \in \C^{kN \times M}$, we have \[\circop^N(V)^* = \circop^m\left( (R_k \otimes I_m) \Tc_N(V) \right).\] 
\end{lemma}
\begin{proof}[Proof of \cref{lem:circ_transpose}]
  Suppose $V_i$ are the $N \times m$ blocks of $V$, such that $V = \left[V_1^T \cdots V_k^T\right]^T$.  Indexing blockwise, we have $\circop^N(V)_{[ij]} = V_{i - j + 1}$, so that $\circop^N(V)^*_{[ij]} = V_{j - i + 1}^*$.  In other words,
  \[
  \circop^N(V)^* = \begin{bmatrix} V_1^* & V_2^* & \cdots & V_N^* \\ V_N^* & V_1^* & \cdots & V_{N - 1}^* \\ \vdots & & \ddots & \vdots \\ V_2^* & V_3^* & \cdots & V_1^* \end{bmatrix} = \circop^m((R_k \otimes I_m) \Tc_N(V) )
  \]
  as claimed.  
\end{proof}

\Cref{lem:block_circ_right,lem:diag_kron_perm} provide identities for a few block matrix structures that will be of interest.
\begin{lemma}\label{lem:block_circ_right}
  Given $N_1, N_2, k, m \in \N$ and $V_i \in \C^{k N_1 \times m}$ for $i \in [N_2]$, we have \[\begin{bmatrix} \circop^{N_1}(V_1) & \cdots & \circop^{N_1}(V_{N_2}) \end{bmatrix} (P^{(k, N_2)} \otimes I_m)^* = \circop^{N_1}(\begin{bmatrix} V_1 & \cdots & V_{N_2}\end{bmatrix}).\]
\end{lemma}

\begin{proof}[Proof of \cref{lem:block_circ_right}]
  We quote \eqref{eq:M_2} from \cref{lem:interleave} and consider that $P^{(k, N_2)} \otimes I_m$ is a permutation that changes the blockwise indices of $m \times p$ blocks (or, acting from the right, $p \times m$ blocks) exactly the way that $P^{(k, N_2)}$ changes the indices of a vector.
\end{proof}

\begin{lemma} \label{lem:diag_kron_perm}
  Given $k, n \in \N$ and $V_j \in \C^{m_j \times n_j}$ for $j \in [n]$ and setting $M = \sum_{j = 1}^n m_j, N = \sum_{j = 1}^N n_j,$ and $D = \diag(I_k \kron V_j)_{j = 1}^n \in \C^{k M \times k N}$, we have \[D = \diagcornmat{I_k \kron V_1}{I_k \kron V_n} = P_1 (I_k \otimes \diag(V_j)_{j = 1}^n) P_2^*\] where $P_1 = \Pc(P^{(n, k)}, (m_{j \mod_1 n})_{j = 1}^{kn})$ and $P_2 = \Pc(P^{(n, k)}, (n_{j \mod_1 n})_{j = 1}^{kn})$.
\end{lemma}

\begin{proof}[Proof of \cref{lem:diag_kron_perm}]
  We immediately reduce to the case $m_j = n_j = 1$ for all $j$ by observing that $P_1$ and $P_2$ will act on blockwise indices precisely as $P^{(n, k)}$ acts on individual indices.  Here, we replace $V_j$ with $v_j \in \C$, and note that $\diag(V_j)_{j = 1}^n = \diag(v)$.  Hence, we need only remark that \[\left(\diag(I_k \otimes v_\ell)_{\ell = 1}^n\right)_{((i_1 - 1)k + i_2) ((j_1 - 1)k + j_2)} = \left\{\begin{array}{r@{,\quad}l} v_{i_1} & i_1 = j_1 \ \text{and} \ i_2 = j_2 \\ 0 & \text{otherwise} \end{array}\right.,\] while \begin{align*} (P^{(n, k)} (&I_k \otimes \diag(v)) P^{(n, k)*})_{((i_1 - 1)k + i_2) ((j_1 - 1)k + j_2)} \\ &= (I_k \otimes \diag(v))_{((i_2 - 1)n + i_1) ((j_2 - 1)n + j_1)} \\ &= \left\{\begin{array}{r@{,\quad}l} v_{i_1} & i_1 = j_1 \ \text{and} \ i_2 = j_2 \\ 0 & \text{otherwise} \end{array}\right..\end{align*}
\end{proof}

\subsection{Zero Columns in Matrix Representation of $\Ac$}
\label{sec:pty_zero_col}
To begin the discussion of the matrix representation of $\Ac$, we refresh our notation: $d, \delta, \dbar, s \in N$ satisfy $d = \dbar s$ and $2 \delta - 1 \le d$.  We have $D \in \N$ (arbitrary for now) measurement vectors $\{m_j\}_{j \in [D]} \in \Cd$ satisfying $1 \in \supp(m_j) \subset [\delta]$, and we set $g_m^k = \diag(m_k m_k^*, m)$ for all $1 - \delta \le m \le \delta - 1$ and all $k \in [D]$.  The expressions $\Lc_{\{m_j\}}^s, T_{\delta, s},$ and $\Ac$ are defined in \cref{eq:lift_pty_sys,eq:T_delta_s,eq:pty_meas_op}.

We now consider the question of when $\Span \Lc_{\{m_j\}}^s = T_{\delta, s}$ and what the condition number of $\Ac$ will be.  As in \eqref{eq:diag_vec}, we vectorize $X$ by its diagonals\footnote{Notice that this will force $A$, the matrix representing $\Ac$, to be singular.  We expand on this later.} with $\Dc_\delta(X) \in \C^{d (2 \delta - 1)}$ and write $A \in \C^{\dbar D \times (2 \delta - 1) d}$ such that \begin{equation*} \left(A \Dc_\delta(X) \right)_{(j-1) \dbar + \ell} =  \left(A \colmatfl{\diag(X, 1 - \delta)}{\diag(X, \delta - 1)}\right)_{(j - 1) \dbar + \ell} = \Ac(X)_{(\ell - 1, j)}, %\label{eq:vectorized_meas_ptych}
\end{equation*}
%% \begin{equation*} \left(A \begin{bmatrix} \chi_{1 - \delta} \\ \vdots \\ \chi_{\delta - 1} \end{bmatrix}\right)_{(j-1) \dbar + \ell} = \Ac(X)_{(\ell, j)}, %\label{eq:vectorized_meas_ptych}
%% \end{equation*}
which gives the $(j - 1) \dbar + \ell\th$ row of $A$ as \[\begin{bmatrix} S^{s (\ell - 1)} g_{1 - \delta}^j \\ \vdots \\ S^{s (\ell - 1)} g_{\delta - 1}^j \end{bmatrix}^*\] so that, by \cref{lem:circ_transpose}, we have\footnote{For reference, we remark that $\circop^s(g_m^k) \in \C^{d \times \dbar}$ and $\circop(R_{\dbar} \Tc_s g_m^k) \in \C^{\dbar \times d}$.} \begin{equation} A = \begin{bmatrix} \circop^s(g_{1 - \delta}^1) & \cdots & \circop^s(g_{1 - \delta}^D) \\ \vdots & \ddots & \vdots \\ \circop^s(g_{\delta - 1}^1) & \cdots & \circop^s(g_{\delta - 1}^D) \end{bmatrix}^* = \begin{bmatrix} \circop(R_{\dbar} \Tc_s g_{1 - \delta}^1) & \cdots & \circop(R_{\dbar} \Tc_s g_{\delta - 1}^1) \\ \vdots & \ddots & \vdots \\ \circop(R_{\dbar} \Tc_s g_{1 - \delta}^D) & \cdots & \circop(R_{\dbar} \Tc_s g_{\delta - 1}^D) \end{bmatrix}. \label{eq:A_block_ptych} \end{equation}  However, because $T_{\delta, s} \subsetneq T_\delta$, this operator can never be invertible.  In fact, $A$ has several \emph{completely zero} columns, corresponding precisely to the coordinates of entries in $T_\delta / T_{\delta, s}$,\footnote{Here, by the quotient $V / W$ of nested subspaces $W \subset V$, we mean $V \cap W^\perp$.} in the sense that $\Dc_\delta(T_\delta / T_{\delta, s}) \subset \Nul(A)$.  The remainder of this section is dedicated to explicitly stating an orthonormal basis for $\Dc_\delta(T_{\delta, s})$ by enumerating the indices of these zero columns.  This is achieved in \cref{prop:zero_cols}, which provides this basis as the columns of a matrix $N$, such that $N^* \Dc_\delta(X)$ will be a convenient vectorization of $T_{\delta, s}(X)$ and $A N$ will represent $\left.\Ac\right|_{T_{\delta, s}(\Cdxd)}$.  The rest of this section is dedicated to constructing and verifying this matrix $N$.  For a first read, \cref{fig:zero_cols} may be a useful visual guide that stands to illuminate this otherwise opaquely stated result.

To describe this solution $N$, we must introduce one more indexed family of matrices.  Here, given $k_1, k_2, n \in \N$ with $k_1 \le k_2$, we enumerate the integer interval $[k_1, k_2) \cap \Z$ by $k_1 = a_1 < \cdots < a_{k_2 - k_1} = k_2 - 1$ and then set $b_i = a_i \mod_1 n \in [n]$ for $i \in [k_2 - k_1]$.  Then we set \begin{equation} J_{(k_1, k_2, n)} = \begin{piecewise} \rowmatfl{e_{b_1}^n}{e_{b_{k_2 - k_1}}^n} & k_2 - k_1 < n \\ I_n & \ow \end{piecewise} \label{eq:circ_mod_mat} \end{equation}  Then $J_{(k_1, k_2, n)} \in \C^{n \times \min\{n, k_2 - k_1\}}$ represents an orthogonal basis for $\Span\{e_i : i \mod n \in [k_1, k_2)\}$, where $i \mod n \in S$ here means that some element of $S$ is congruent to $i$ modulo $n$.  With this definition, we are prepared to state \cref{prop:zero_cols}.
\begin{proposition} \label{prop:zero_cols}
  Fix $d = \dbar s$ and $\delta$ satisfying $s, \dbar, \delta \in \N, 2 \delta - 1 \le d$.  We let
  \begin{equation}
    N = \diag(I_{\dbar} \otimes N_m)_{m = 1 - \delta}^{\delta - 1},\ \text{where} \ 
    %% N_m = \left\{
    %% \begin{array}{c@{,\quad}l}
    %%   \begin{bmatrix} 0_{\delta + m} \\ I_{s - (\delta + m)} \end{bmatrix} & m < s - \delta \vspace{2pt}\\
    %%   \begin{bmatrix} I_{s - (\delta - m)} \\ 0_{\delta - m} \end{bmatrix} & m > \delta - s \\
    %%   I_s & \ow \end{array}
    %% \right.
    N_m = \begin{piecewise}
      J_{(\abs{m} + 1, \delta + 1, s)} & m < 0 \\
      J_{(1, \delta - m + 1, s)} & \ow
    \end{piecewise}
    , \label{eq:N_and_N_m}
  \end{equation}
  and we note that $N_m \in \C^{s \times \min\{s, \delta - \abs{m}\}}$, so setting $\alpha = \sum_{m = 1 - \delta}^{\delta - 1} \min\{s, \delta - \abs{m}\}$, we have $N \in \C^{d (2 \delta - 1) \times \dbar \alpha}$.  Then $N$ is an orthogonal basis for $\Dc_\delta(T_{\delta, s}(\Cdxd))$ and therefore $A N \in \C^{\dbar D \times \dbar \alpha}$ is the matrix representation of $\left.\Ac\right|_{T_{\delta, s}(\Cdxd)}$ with respect to the basis $N^* \Dc_\delta(T_{\delta, s}) \in \C^{\dbar \alpha}$, in the sense that \[(A N N^* \Dc_\delta(X))_{(j - 1) \dbar + \ell} = (A \Dc_\delta(T_{\delta, s}(X)))_{(j - 1) \dbar + \ell} = \Ac(X)_{(\ell - 1, j)}.\]
\end{proposition}
\begin{figure}
  \centering
  \begin{tikzpicture}[ampersand replacement=\&,baseline=-\the\dimexpr\fontdimen22\textfont2\relax]
    \matrix (m)[matrix of math nodes,left delimiter=(,right delimiter=)]
            {
              * \& * \& * \& * \&   \&   \& * \& * \& * \\
              * \& * \& * \& * \& * \&   \&   \& * \& * \\
              * \& * \& * \& * \& * \& * \&   \&   \& * \\
              * \& * \& * \& * \& * \& * \& * \&   \&   \\
                \& * \& * \& * \& * \& * \& * \& * \&   \\
                \&   \& * \& * \& * \& * \& * \& * \& * \\
              * \&   \&   \& * \& * \& * \& * \& * \& * \\
              * \& * \&   \&   \& * \& * \& * \& * \& * \\
              * \& * \& * \&   \&   \& * \& * \& * \& * \\
            };
 
            \begin{pgfonlayer}{myback}
              \fhighlight[blue!30]{m-1-1}{m-4-4}
              %\fhighlight[blue!30]{m-2-2}{m-5-5}
              %\fhighlight[blue!30]{m-3-3}{m-6-6}
              \fhighlight[blue!30]{m-4-4}{m-7-7}
              %\fhighlight[blue!30]{m-5-5}{m-8-8}
              %\fhighlight[blue!30]{m-6-6}{m-9-9}
              \fhighlight[blue!30]{m-7-1}{m-9-1}
              \fhighlight[blue!30]{m-1-7}{m-1-9}
              \fhighlight[blue!30]{m-7-7}{m-9-9}
              \fhighlight[red!40!blue!40]{m-1-4}{m-1-4}
              \fhighlight[red!30]{m-2-5}{m-2-5}
              \fhighlight[red!30]{m-3-6}{m-3-6}
              \fhighlight[red!40!blue!40]{m-4-7}{m-4-7}
              \fhighlight[red!30]{m-5-8}{m-5-8}
              \fhighlight[red!30]{m-6-9}{m-6-9}
              \fhighlight[red!40!blue!40]{m-7-1}{m-7-1}
              \fhighlight[red!30]{m-8-2}{m-8-2}
              \fhighlight[red!30]{m-9-3}{m-9-3}
              
              \fhighlight[red!40!blue!40]{m-1-8}{m-1-8}
              \fhighlight[red!30]{m-2-9}{m-2-9}
              \fhighlight[red!40!blue!40]{m-3-1}{m-3-1}
              \fhighlight[red!40!blue!40]{m-4-2}{m-4-2}
              \fhighlight[red!30]{m-5-3}{m-5-3}
              \fhighlight[red!40!blue!40]{m-6-4}{m-6-4}
              \fhighlight[red!40!blue!40]{m-7-5}{m-7-5}
              \fhighlight[red!30]{m-8-6}{m-8-6}
              \fhighlight[red!40!blue!40]{m-9-7}{m-9-7}
              %\fhighlight[blue!30]{m-8-8}{m-8-8}
              %\fhighlight[blue!30]{m-8-1}{m-8-2}
              %\fhighlight[blue!30]{m-1-8}{m-2-8}
              %\fhighlight[blue!30]{m-1-1}{m-2-2}
            \end{pgfonlayer}
  \end{tikzpicture}
  \caption{$T_{4, 3}(\C^{9 \times 9})$}
  \label{fig:zero_cols}  
\end{figure}
\begin{proof}[Proof of \cref{prop:zero_cols}]
  We begin by fixing a diagonal $m \ge 0$ and considering which indices along this diagonal will be encountered by the shifting masks $S^\ell m_j m_j^* S^{-\ell}$.  We consider that $g_m^j$, the $m\th$ diagonal of $m_j m_j^*$, has support\footnote{It is worth emphasizing that equality here, namely the case that $\supp(g_m^j) = [1, \delta - m]$, is not only feasible, but common.} \[\supp(g_m^j) \subset [1, \delta - m] = \clopen{1, \delta - m + 1}.\]   Therefore, the set of all indices in $[d]$ hit by shifts of $g_m^j$ is given by \[\bigcup_{\ell \in [\dbar]_0} \clopen{1, \delta - m + 1} + \ell s = \{i \in [d] : i \mod s \in \clopen{1, \delta - m + 1}\}.\]  Recalling \eqref{eq:circ_mod_mat}, it is clear that $N_m = J_{(1, \delta - m + 1, s)} \in \C^{d \times \min\{s, \delta - m\}}$ spans exactly these indices, in the sense that \begin{equation} \label{eq:N_m_diag} (I_{\dbar} \kron N_m) (I_{\dbar} \kron N_m)^* \diag(\Cdxd, m) = \diag(T_{\delta, s}(\Cdxd), m). \end{equation}

  Similarly, for $m < 0,$ we have $\supp(g_m^j) \subset \clopen{\abs{m} + 1, \delta + 1},$ so we will simply use $N_m = I_{\dbar} \kron J_{(\abs{m} + 1, \delta + 1, s)}$ in this instance, and these $N_m$ satisfy \eqref{eq:N_m_diag} for negative $m$.  Synthesizing \eqref{eq:N_m_diag} for $m = 1 - \delta, \ldots, \delta - 1$, we have \[\Dc_\delta(T_{\delta, s}(X)) = \colmatfl{\diag(T_{\delta, s}(X), 1 - \delta)}{\diag(T_{\delta, s}(X), \delta - 1)} = \colmatfl{N_{1 - \delta} N_{1 - \delta}^* \diag(X, 1 - \delta)}{N_{\delta - 1} N_{\delta - 1}^* \diag(X, \delta - 1)} = N N^* \Dc_\delta(X),\] which completes the proof.
\end{proof}
%% \begin{proof}[Proof of \cref{prop:zero_cols}]
%%   To establish that the zero columns of $A$ correspond to coordinates of $T_\delta / T_{\delta, s}$, consider $g_m^k$, the $m\th$ diagonal of $m_k m_k^*$, for some $m \in [\delta]_0$.  If $\supp(m_k) = [\delta]$, we will have $\supp(g^k_m) = [\delta - m]$.  When $m$ is large enough that $\delta - m < s,$ the shifted images of this $m\th$ diagonal will not connect, and the $m\th$ diagonal of $T_{\delta, s}(\Hd)$ is incomplete in the sense that $\bigcup_{\ell = 1}^{\dbar} \supp(S^{s (\ell - 1)} g^k_m) \neq [d]$.  In particular, $\circop^s(g^k_m)_{ij} = 0$ for all $j$ when $i \mod_1 s > \delta - m$.\footnote{This may be clarified by taking an example from \cref{fig:zero_cols}, where $\delta = 4$ and $s = 3$: on the $m = 3^{\text{rd}}$ diagonal, entries are missing from $T_{\delta, s}(\Cdxd)$ when $i \mod_1 3 > 1$.}  By a similar argument, for $m < 0$ we have $\circop^s(g_m^k)_{ij} = 0$ when $i \mod_1 s < s - (\delta - |m|)$.  We remark that these inequalities can only be satisfied when $m > \delta - s$ or $m < s - \delta$, respectively.

%%   By reference to \eqref{eq:A_block_ptych}, it is clear that each of these ``missing indices'' results in a column of all zeros in $A$; specifically, viewing $A e_{(m + \delta - 1)d + i}, (m, i) \in [2 \delta - 1]_{1 - \delta} \times [d]$ as the $i\th$ column of the $m + \delta\th$ block of $A$, we see \[A e_{(m + \delta - 1) d + i} = 0\quad \text{if} \quad \left\{\begin{array}{rcl@{\,,\quad}l} i \mod_1 s & > & \delta - m & m \ge 0 \\ i \mod_1 s & < & s - (\delta + m) & m \le 0 \end{array}\right..\]  Hence, the condition to be a \emph{nonzero} column may be written as $s - (\delta + m) \le i \mod_1 s \le \delta - m$%% , or $i \mod_1 s \in [2 \delta - s + 1]_{s - \delta - m}$
%%   .  We view this in the following way: since $i \mod_1 s \in [s]$, these inequalities are only restrictive when $s - (\delta + m) > 1$ or when $\delta - m < s$, that is, when $m < s - \delta - 1$ or $m$  Therefore, the matrix representing $\Ac$ restricted to $T_{\delta, s}(\C^{d \times d})$ is found by right multiplying $A$ by \begin{equation}N = \diag(I_{\dbar} \otimes N_{j - \delta})_{j = 1}^{2 \delta - 1},\ \text{where} \ N_m = \left\{\begin{array}{c@{,\quad}l} \begin{bmatrix} 0_{\delta + m} \\ I_{s - (\delta + m)} \end{bmatrix} & s > \delta + m \vspace{2pt}\\ \begin{bmatrix} I_{s - (\delta - m)} \\ 0_{\delta - m} \end{bmatrix} & s > \delta - m \\ I_s & \ow \end{array}\right., \label{eq:N_and_N_m}\end{equation} and we note that $N_m \in \C^{s \times \min\{s, \delta - \abs{m}\}}$, so setting $\alpha = \sum_{m = 1 - \delta}^{\delta - 1} \min\{s, \delta - \abs{m}\}$, we have $N \in \C^{d \times \dbar \alpha}$.
%% \end{proof}

%% Here, we're using $(d, \delta, s) = (9, 4, 3)$, and the blue entries represent what is ``seen'' by $T_{4, 3}(\C^{9 \times 9})$; the red entries represent the missing entries of the third diagonal, and the purple entries are what is included of the third diagonal.
%% \begin{proposition} \label{prop:zero_cols}
%%   Fix $d = \dbar s$ and $\delta$ satisfying $s, \dbar, \delta \in \N, 2 \delta - 1 \le d$.  We let
%%   \begin{equation}
%%     N = \diag(I_{\dbar} \otimes N_m)_{m = 1 - \delta}^{\delta - 1},\ \text{where} \
%%     N_m = \left\{
%%     \begin{array}{c@{,\quad}l}
%%       \begin{bmatrix} 0_{\delta + m} \\ I_{s - (\delta + m)} \end{bmatrix} & m < s - \delta \vspace{2pt}\\
%%       \begin{bmatrix} I_{s - (\delta - m)} \\ 0_{\delta - m} \end{bmatrix} & m > \delta - s \\
%%       I_s & \ow \end{array}
%%     \right., \label{eq:N_and_N_m}
%%   \end{equation}
%%   and we note that $N_m \in \C^{s \times \min\{s, \delta - \abs{m}\}}$, so setting $\alpha = \sum_{m = 1 - \delta}^{\delta - 1} \min\{s, \delta - \abs{m}\}$, we have $N \in \C^{d (2 \delta - 1) \times \dbar \alpha}$.  Then $N$ is an orthogonal basis for $\Dc_\delta(T_{\delta, s}(\Cdxd))$ and therefore $A N \in \C^{\dbar D \times \dbar \alpha}$ is the matrix representation of $\left.\Ac\right|_{T_{\delta, s}(\Cdxd)}$ with respect to the basis $N^* \Dc_\delta(T_{\delta, s}) \in \C^{\dbar \alpha}$, in the sense that \[(A N N^* \Dc_\delta(X))_{(j - 1) \dbar + \ell} = \Ac(X)_{(\ell - 1, j)}.\]
%% \end{proposition}
%% \begin{figure}
%%   \centering
%%   \begin{tikzpicture}[ampersand replacement=\&,baseline=-\the\dimexpr\fontdimen22\textfont2\relax]
%%     \matrix (m)[matrix of math nodes,left delimiter=(,right delimiter=)]
%%             {
%%               * \& * \& * \& * \&   \&   \& * \& * \& * \\
%%               * \& * \& * \& * \& * \&   \&   \& * \& * \\
%%               * \& * \& * \& * \& * \& * \&   \&   \& * \\
%%               * \& * \& * \& * \& * \& * \& * \&   \&   \\
%%                 \& * \& * \& * \& * \& * \& * \& * \&   \\
%%                 \&   \& * \& * \& * \& * \& * \& * \& * \\
%%               * \&   \&   \& * \& * \& * \& * \& * \& * \\
%%               * \& * \&   \&   \& * \& * \& * \& * \& * \\
%%               * \& * \& * \&   \&   \& * \& * \& * \& * \\
%%             };
 
%%             \begin{pgfonlayer}{myback}
%%               \fhighlight[blue!30]{m-1-1}{m-4-4}
%%               %\fhighlight[blue!30]{m-2-2}{m-5-5}
%%               %\fhighlight[blue!30]{m-3-3}{m-6-6}
%%               \fhighlight[blue!30]{m-4-4}{m-7-7}
%%               %\fhighlight[blue!30]{m-5-5}{m-8-8}
%%               %\fhighlight[blue!30]{m-6-6}{m-9-9}
%%               \fhighlight[blue!30]{m-7-1}{m-9-1}
%%               \fhighlight[blue!30]{m-1-7}{m-1-9}
%%               \fhighlight[blue!30]{m-7-7}{m-9-9}
%%               \fhighlight[red!40!blue!40]{m-1-4}{m-1-4}
%%               \fhighlight[red!30]{m-2-5}{m-2-5}
%%               \fhighlight[red!30]{m-3-6}{m-3-6}
%%               \fhighlight[red!40!blue!40]{m-4-7}{m-4-7}
%%               \fhighlight[red!30]{m-5-8}{m-5-8}
%%               \fhighlight[red!30]{m-6-9}{m-6-9}
%%               \fhighlight[red!40!blue!40]{m-7-1}{m-7-1}
%%               \fhighlight[red!30]{m-8-2}{m-8-2}
%%               \fhighlight[red!30]{m-9-3}{m-9-3}
%%               %\fhighlight[blue!30]{m-8-8}{m-8-8}
%%               %\fhighlight[blue!30]{m-8-1}{m-8-2}
%%               %\fhighlight[blue!30]{m-1-8}{m-2-8}
%%               %\fhighlight[blue!30]{m-1-1}{m-2-2}
%%             \end{pgfonlayer}
%%   \end{tikzpicture}
%%   \caption{$T_{4, 3}(\C^{9 \times 9})$}
%%   \label{fig:zero_cols}  
%% \end{figure}
%% \begin{proof}[Proof of \cref{prop:zero_cols}]
%%   To establish that the zero columns of $A$ correspond to coordinates of $T_\delta / T_{\delta, s}$, consider $g_m^k$, the $m\th$ diagonal of $m_k m_k^*$, for some $m \in [\delta]_0$.  If $\supp(m_k) = [\delta]$, we will have $\supp(g^k_m) = [\delta - m]$.  When $m$ is large enough that $\delta - m < s,$ the shifted images of this $m\th$ diagonal will not connect, and the $m\th$ diagonal of $T_{\delta, s}(\Hd)$ is incomplete in the sense that $\bigcup_{\ell = 1}^{\dbar} \supp(S^{s (\ell - 1)} g^k_m) \neq [d]$.  In particular, $\circop^s(g^k_m)_{ij} = 0$ for all $j$ when $i \mod_1 s > \delta - m$.\footnote{This may be clarified by taking an example from \cref{fig:zero_cols}, where $\delta = 4$ and $s = 3$: on the $m = 3^{\text{rd}}$ diagonal, entries are missing from $T_{\delta, s}(\Cdxd)$ when $i \mod_1 3 > 1$.}  By a similar argument, for $m < 0$ we have $\circop^s(g_m^k)_{ij} = 0$ when $i \mod_1 s < s - (\delta - |m|)$.  We remark that these inequalities can only be satisfied when $m > \delta - s$ or $m < s - \delta$, respectively.

%%   By reference to \eqref{eq:A_block_ptych}, it is clear that each of these ``missing indices'' results in a column of all zeros in $A$; specifically, viewing $A e_{(m + \delta - 1)d + i}, (m, i) \in [2 \delta - 1]_{1 - \delta} \times [d]$ as the $i\th$ column of the $m + \delta\th$ block of $A$, we see \[A e_{(m + \delta - 1) d + i} = 0\quad \text{if} \quad \left\{\begin{array}{rcl@{\,,\quad}l} i \mod_1 s & > & \delta - m & m \ge 0 \\ i \mod_1 s & < & s - (\delta + m) & m \le 0 \end{array}\rfight..\]  Hence, the condition to be a \emph{nonzero} column may be written as $s - (\delta + m) \le i \mod_1 s \le \delta - m$%% , or $i \mod_1 s \in [2 \delta - s + 1]_{s - \delta - m}$
%%   .  We view this in the following way: since $i \mod_1 s \in [s]$, these inequalities are only restrictive when $s - (\delta + m) > 1$ or when $\delta - m < s$, that is, when $m < s - \delta - 1$ or $m$  Therefore, the matrix representing $\Ac$ restricted to $T_{\delta, s}(\C^{d \times d})$ is found by right multiplying $A$ by \begin{equation}N = \diag(I_{\dbar} \otimes N_{j - \delta})_{j = 1}^{2 \delta - 1},\ \text{where} \ N_m = \left\{\begin{array}{c@{,\quad}l} \begin{bmatrix} 0_{\delta + m} \\ I_{s - (\delta + m)} \end{bmatrix} & s > \delta + m \vspace{2pt}\\ \begin{bmatrix} I_{s - (\delta - m)} \\ 0_{\delta - m} \end{bmatrix} & s > \delta - m \\ I_s & \ow \end{array}\right., \label{eq:N_and_N_m}\end{equation} and we note that $N_m \in \C^{s \times \min\{s, \delta - \abs{m}\}}$, so setting $\alpha = \sum_{m = 1 - \delta}^{\delta - 1} \min\{s, \delta - \abs{m}\}$, we have $N \in \C^{d \times \dbar \alpha}$.
%% \end{proof}

\subsection{Main Result}
\label{sec:ptych_con_main}
To prove a condition number result analogous to that of \cref{thm:meas_cond} for $AN$, we will need to show that the restriction operator $N$ commutes well with the permutations used in the condition number analysis of \cref{sec:con_number}, preserving the block-circulant structures that made the analysis possible.  Thankfully it does; following the intuition of \eqref{eq:interleaved_meas}, referring to our expression of $A$ in \eqref{eq:A_block_ptych}, and making use of \cref{lem:interleave,lem:block_circ_right}, we can arrive at \begin{align*} A' := P^{(\dbar, D)} A \left(P^{(\dbar, 2 \delta - 1)} \otimes I_s\right)^* &= \circop^D\left(P^{(\dbar, D)} \begin{bmatrix} R_{\dbar} \Tc_s g_{1 - \delta}^1 & \cdots & R_{\dbar} \Tc_s g_{\delta - 1}^1 \\ \vdots & \ddots & \vdots \\ R_{\dbar} \Tc_s g_{1 - \delta}^D & \cdots & R_{\dbar} \Tc_s g_{\delta - 1}^D \end{bmatrix}\right). \\ &= \circop^D\left(P^{(\dbar, D)} (I_D \otimes R_{\dbar}) \begin{bmatrix}  \Tc_s g_{1 - \delta}^1 & \cdots &  \Tc_s g_{\delta - 1}^1 \\ \vdots & \ddots & \vdots \\  \Tc_s g_{1 - \delta}^D & \cdots &  \Tc_s g_{\delta - 1}^D \end{bmatrix}\right). \end{align*}  In the interest of finding the locations of the zero columns after this permutation, we remark that the inner matrix is of size $\dbar D \times s (2 \delta - 1)$, and that the $\circop^D$ operation will therefore repeat it $\dbar$ times, without moving the positions of the .  It is then clear that \begin{gather*} A'e_{(\ell - 1)s (2 \delta - 1) + i} = 0 \iff \begin{bmatrix}  \Tc_s g_{1 - \delta}^1 & \cdots &  \Tc_s g_{\delta - 1}^1 \\ \vdots & \ddots & \vdots \\  \Tc_s g_{1 - \delta}^D & \cdots &  \Tc_s g_{\delta - 1}^D \end{bmatrix} e_i = 0,\ \text{and} \\ \begin{bmatrix}  \Tc_s g_{1 - \delta}^1 & \cdots &  \Tc_s g_{\delta - 1}^1 \\ \vdots & \ddots & \vdots \\  \Tc_s g_{1 - \delta}^D & \cdots &  \Tc_s g_{\delta - 1}^D \end{bmatrix} e_{(m + \delta - 1)s + i} = 0 \iff i \notin [2 \delta - s + 1]_{s - \delta - m},\end{gather*} so we may remove the zero columns from $A'$ by right multiplying the interior matrix by $N' = \diag(N_m)_{m = 1 - \delta}^{\delta - 1}$.  That is,\begin{equation} A' (I_{\dbar} \otimes N') = \circop^D\left(P^{(\dbar, D)} (I_D \otimes R_{\dbar}) \begin{bmatrix}  \Tc_s g_{1 - \delta}^1 & \cdots &  \Tc_s g_{\delta - 1}^1 \\ \vdots & \ddots & \vdots \\  \Tc_s g_{1 - \delta}^D & \cdots &  \Tc_s g_{\delta - 1}^D \end{bmatrix} N'%% \begin{bmatrix} N_{1 - \delta} & & \\ & \ddots & \\ & & N_{\delta - 1} \end{bmatrix}
  \right). \label{eq:ptych_permute_final}\end{equation}
 %% \begin{align} A' (I_{\dbar} \otimes N') &= \circop^D\left(P^{(\dbar, D)} (I_D \otimes R_{\dbar}) \begin{bmatrix}  \Tc_s g_{1 - \delta}^1 & \cdots &  \Tc_s g_{\delta - 1}^1 \\ \vdots & \ddots & \vdots \\  \Tc_s g_{1 - \delta}^D & \cdots &  \Tc_s g_{\delta - 1}^D \end{bmatrix} N'%% \begin{bmatrix} N_{1 - \delta} & & \\ & \ddots & \\ & & N_{\delta - 1} \end{bmatrix}
  %% \right) \nonumber\\ & = P^{(\dbar, D)} A N P', \label{eq:ptych_permute_final}\end{align}
This may be reduced further by applying \cref{lem:diag_kron_perm}, which gives us that, setting \begin{align*} P_1 &= \Pc(P^{(2 \delta - 1, \dbar)}, (s)_{j = 1}^{\dbar (2 \delta - 1)}) = P^{(2 \delta - 1, \dbar)} \kron I_s \\ P_2 &= \Pc(P^{(2 \delta - 1, \dbar)}, (\min\{s, \delta - \abs{m}\})_{m = 1 - \delta}^{\delta - 1}), \end{align*} we will have $N = \diag(I_{\dbar} \kron N_m)_{m = 1 - \delta}^{\delta - 1} =  P_1 (I_{\dbar} \kron N') P_2^*.$  This gives \[A' (I_{\dbar} \kron N') = P^{(\dbar, D)} A P_1 (I_{\dbar} \kron N') = P^{(\dbar, D)} A N P_2,\] which, along with \cref{cor:circ_diag_condition}, gives us the following proposition.

\begin{proposition}
  Taking $A$ as in \eqref{eq:A_block_ptych}, $N$ and $N_m$ as in \eqref{eq:N_and_N_m}, and setting \[H = P^{(\dbar, D)} (I_D \otimes R_{\dbar}) \begin{bmatrix}  \Tc_s g_{1 - \delta}^1 & \cdots &  \Tc_s g_{\delta - 1}^1 \\ \vdots & \ddots & \vdots \\  \Tc_s g_{1 - \delta}^D & \cdots &  \Tc_s g_{\delta - 1}^D \end{bmatrix} \diag(N_m)_{m = 1 - \delta}^{\delta - 1}\] and $M_j = \sqrt{\dbar}(f_j^{\dbar} \otimes I_D)^* H$ for $j \in [\dbar]$, the condition number of $AN$ is given by \[\dfrac{\max\limits_{i \in [\dbar]} \sigma_{\max} (M_i)}{\min\limits_{i \in [\dbar]} \sigma_{\min} (M_i)}.\]  In particular, $\left.\Ac\right|_{T_{\delta, s}(\C^{d \times d})}$ is invertible if and only if each of the $M_i$ are of full rank.
\end{proposition}
