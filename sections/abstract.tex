In this dissertation, we study a new approach to the problem of phase retrieval, which is the task of reconstructing a complex-valued signal from magnitude-only measurements.  This problem occurs naturally in several specialized imaging applications such as electron microscopy and X-ray crystallography.  Although solutions were first proposed for this problem as early as the 1970s, these algorithms have lacked theoretical guarantees of success, and phase retrieval has suffered from a considerable gap between practice and theory for almost the entire history of its study.

A common technique in fields that use phase retrieval is that of \emph{ptychography}, where measurements are collected by only illuminating small sections of the sample at any time.  We refer to measurements designed in this way as \emph{local measurements}, and in this dissertation, we develop and expand the theory for solving phase retrieval in measurement regimes of this kind.  Our first contribution is a basic model for this setup in the case of a one-dimensional signal, along with an algorithm that robustly solves phase retrieval under this model.  This work is unique in many ways that represent substantial improvements over previously existing solutions: perhaps most significantly, many of the recovery guarantees in recent work rely on the measurements being generated by a random process, while we devise a class of measurements for which the conditioning of the system is known and quickly checkable.  These advantages constitute major progress towards producing theoretical results for phase retrieval that are directly usable in laboratory settings.

Chapter 1 conducts a survey of the history of phase retrieval and its applications, as well as the recent literature on the subject.  Chapter 2 presents co-authored results defining and establishing the setting and solution of the base model explored in this dissertation.  Chapter 3 expands the theory on what measurement schemes are admissible in our model, including an analysis of conditioning and runtime.  Chapter 4 introduces an alternate solution for angular synchronization that yields helpful theoretical results.  Chapter 5 brings our model nearer to the actual practice of ptychography.  Chapter 6 extends the base model to two dimensions.
