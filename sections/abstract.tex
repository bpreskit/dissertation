In this dissertation, we study a new approach to the problem of phase retrieval, which is the task of reconstructing a complex-valued signal from magnitude-only measurements.  This problem occurs naturally in several specialized imaging applications such as electron microscopy and X-ray crystallography.  Although solutions were first proposed for this problem as early as the 1970s, these algorithms have lacked theoretical guarantees of success, and phase retrieval has suffered from a considerable gap between practice and theory for almost the entire history of its study.

A common technique in fields that use phase retrieval is that of \emph{ptychography}, where measurements are collected by only illuminating small sections of the sample at any time.  We refer to measurements designed in this way as \emph{local measurements}, and in this dissertation, we develop and expand the theory for solving phase retrieval in measurement regimes of this kind.  Our first contribution is a basic model for this setup in the case of a one-dimensional signal, along with an algorithm that robustly solves phase retrieval under this model.  This work is unique in many ways that represent substantial improvements over previously existing solutions: perhaps most significantly, our model uses a deterministic measurement scheme, and our recovery algorithm is the first to have a solution that may be stated in exact arithmetic.  These advantages constitute major progress towards producing theoretical results for phase retrieval that are directly usable in laboratory settings.

Chapter 1 conducts a survey of the history of phase retrieval and its applications.  Chapter 2 reviews the mathematical literature on the subject, including the first solutions and the theoretical work of the last decade.  Chapter 3 presents co-authored results defining and establishing the setting and solution of the base model explored in this dissertation.  Chapter 4 expands the theory on what measurement schemes are admissible in our model, including an analysis of conditioning and runtime.  Chapter 5 explores results that bring our model nearer to the actual practice of ptychography.  Chapter 6 includes a few relevant results that may be used for future expansion on this topic.



%Phase retrieval is the problem of solving a system of equations of the form $\y = |A \x|^2 + \eta$, where $\x \in \C^d, A \in \C^{D \times d}, \y, \eta \in \R^D$, and $|\cdot|^2$ represents taking the elementwise magnitude-squared of a complex vector.  Here, $\x$ represents the objective vector to be found, $A$ is referred to as the ``measurement matrix,'' $\eta$ represents a perturbation or noise vector, and $\y$ represents the measurement data.  Given $\y$ and $A$, we attempt to solve for an approximation of $\x$.  We can imagine this as a set of noisy linear equations in complex variables where the \emph{phases} of the matrix-vector product $A\x$ have been erased; in solving for $\x$, it could therefore be said that we are reconstructing or ``retrieving'' its phase.

%This thesis develops and expands the theory for this problem with a particular family of measuremental setups: in particular, we study the special case where the rows of $A$ represent a collection of vectors with small support, which are each 
