In this section, we calculate the singular values, and therefore the condition number, of the measurement operator $\Ac$ for an arbitrary local measurement system $\{m_j\}_{j = 1}^d$ in \cref{thm:meas_cond}, the main result of this section.  We remark that this is an important element in using the error bounds proven in \cref{sec:recov_guar}, most notably \cref{thm:MainRes,Cor:RecovRes}, since they all unavoidably rely on the condition number $\kappa$ and singular values of the linear system solved in line 1 of \cref{alg:phaseRetrieval1}, and leaving this quantity unknown and unestimated renders these bounds far less useful.  We also remark that, in this section, as in \cref{sec:expl_span_fam,sec:span_fam_num}, we focus on calculations of the condition numbers $\kappa$ of the linear systems at hand, since $\kappa$ is scale invariant and reflects more completely on the strength of a local measurement system than does $\sigma_{\min}^{-1}$.  Indeed, if $\{m_j\}_{j \in [D]}$ is a local measurement system with $\sigma_{\min}^{-1} = s,$ then $\{t m_j\}_{j \in [D]}$ has $\sigma_{\min}^{-1} = \frac{s}{t^2},$ so it would appear that simply making $t$ large could arbitrarily improve the recovery guarantees of \cref{sec:recov_guar} which refer to $\sigma_{\min}^{-1}$; of course, this action would correspond to simply multiplying the observed measurements by the scalar $t^2$, and slips in its advantage by ignoring that $\norm{n}_2$ would also scale as $t^2$ in such a case.  This process clearly buys us nothing, so the $\sigma_{\min}^{-1}$ inequalities fail, in this way, to consider a sense of scale between $\norm{n}_2$ and $\norm{\Ac(x x^*)}_2$.  Therefore, by referencing $\kappa$ and $\SNR = \frac{\norm{\Ac(x x^*)}}{\norm{n}_2}$ in the alternative statements of \cref{thm:MainRes,Cor:RecovRes}, we have inequalities that more accurately -- if less tightly, by always aggravating the estimate with a factor of $\frac{\sigma_{\max} \norm{x x^*}_F}{\norm{\Ac(x x^*)}_2}$ -- describe the relationship between the design of the linear system and the accuracy of the estimate produced by \cref{alg:phaseRetrieval1}.

We emphasize further that, perhaps equally or even more significantly, this result gives a complete characterization of all local measurement systems that are usable for phase retrieval in \cref{alg:phaseRetrieval1}, since we may simply check whether $\{m_j\}_{j \in [D]}$ leads to a system with any singular values of $0$.  This is an important contribution to our understanding of the work presented in \cref{ch:base_model}, since previously we only possessed two examples of families of masks that produced invertible linear systems.



\begin{theorem} \label{thm:meas_cond} \label{prop:meas_cond}
  Given a family of masks $\{m_j\}_{j \in [D]}$ of support $\delta \le \frac{d + 1}{2}$, we define $g_m^j = \diag(m_j m_j^*, m),$ \[H = P^{(d, D)} \cornmatfun{R \conj{g}^@r_@c}{1}{D}{1 - \delta}{\delta - 1}%% \begin{bmatrix} R g_{1 - \delta}^1 & \cdots & R g_{\delta - 1}^1 \\ \vdots & \ddots & \vdots \\ R g_{1 - \delta}^D & \cdots & R g_{\delta - 1}^D \end{bmatrix}
  ,\] and $M_j = \sqrt{d}\left(f_j^d \otimes I_D\right)^* H$.  Then the singular values of $\Ac$ as defined in \eqref{eq:meas_op} are $\{\sigma_i(M_j)\}_{(i, j) \in [D] \times [d]}$ and its condition number is \[\kappa(\mathcal{A}) = \dfrac{\max\limits_{i \in [d]} \sigma_{\max} (M_i)}{\min\limits_{i \in [d]} \sigma_{\min} (M_i)}.\]
\end{theorem}

Although \cref{thm:meas_cond} is satisfyingly general, perhaps the most important and useful result proven in this section is the strictly narrower \cref{prop:span_fam_cond}.  This result generalizes Example 1 of \Cref{sec:MeasMatrix}, stated in \eqref{eq:MeasDef}, by abstracting out the term $\frac{\ee^{-n/a}}{\sqrt[4]{2\delta -1}}$ and considering what ``base vectors'' $\gamma \in \Rd$ will produce well-conditioned spanning families of masks when $m_j$ is taken to be the local Fourier measurement system with mask $\gamma$, and we determine a choice for the modulation index $K \in \N$ that guarantees the cleanest expression for the condition number.  

\begin{proposition}
\label{prop:span_fam_cond}
\label{prop:gam_fam_cond}
  Let $\{m_j\}_{j = 1}^D \subset \C^d$ be a local Fourier measurement system with support $\delta$, mask $\gamma$, and modulation index $K$, where $D = \min(d, 2 \delta - 1)$.  Let $\Ac$ be the associated measurement operator as in \eqref{eq:meas_op}, with canonical matrix representation $A$ as in \eqref{eq:meas_mat}.

If we additionally assume that $K = D,$ then the condition number of $\mathcal{A}$ is \begin{equation}\kappa(\mathcal{A}) = \dfrac{d^{-1/2} \norm{\gamma}_2^2}{\min\limits_{m \in [\delta]_0, j \in [d]} \lvert F_d^* (\gamma \circ S^{-m} \gamma)_j \rvert}.\label{eq:clean_cond}\end{equation}  Otherwise, if $K > D$, we may bound the condition number by \begin{equation}\kappa(\mathcal{A}) \le \dfrac{d^{-1/2} \norm{\gamma}_2^2}{\min\limits_{m \in [\delta]_0, j \in [d]} \lvert F_d^* (\gamma \circ S^{-m} \gamma)_j \rvert} \kappa(\tF_K), \label{eq:messy_cond}\end{equation} where $\tF_K \in \C^{D \times D}$ is the $D \times D$ principal submatrix of $F_K$.
\end{proposition}

We recall, as discussed in \cref{sec:MeasMatrix}, that the design of local Fourier measurement systems is motivated by the application of ptychography, described in \cref{sec:conn_pty} -- in this type of laboratory setup, $\gamma$ represents the ``illumination function,'' describing the intensity of radiation applied to each segment of the sample -- so it is appropriate to the end user that our simplest and most conveniently applied result pertains to a realistic, broad class of local measurement systems.  In particular, we are grateful to have determined precisely which illumination functions are admissible for our phase retrieval algorithm: following almost immediately from \cref{prop:span_fam_cond}, we have sufficient conditions for a local Fourier measurement system to be a spanning family, which we state in \cref{cor:gam_fam_span}.
\begin{corollary}
  Let $\{m_j\}_{j \in D}$ be a local Fourier measurement system of support $\delta$ with mask $\gamma \in \Rd$ and modulation index $K$, where $D = \min(d, 2 \delta - 1)$.  Then $\{m_j\}_{j \in [D]}$ is a spanning family if $F_d^*(\gamma \circ S^{-m} \gamma)_j \neq 0$ for all $m \in [\delta]_0, j \in [d]$ and $K \ge D$.
  \label{cor:gam_fam_span}
\end{corollary}

The proofs of \cref{thm:meas_cond} and \cref{prop:span_fam_cond} require some preliminary work in defining and studying a few new operators pertaining to the structure of \eqref{eq:meas_op}.  These definitions and a number of results concerning them are contained in \cref{sec:interlemma}.  \Cref{sec:span_fam_remarks} contains some remarks about the ``commonness'' of masks $\gamma$ that produce spanning Fourier families, as well as a technical result that supplements the simple, sufficient condition of \cref{cor:gam_fam_span} with a necessary and sufficient condition that covers the case $2 \delta - 1 > d$.

\subsection{Interleaving Operators and Circulant Structure}
\label{sec:interlemma}

To set the stage for the proof, we introduce a certain collection of permutation operators and study their interactions with circulant and block-circulant matrices.  The structure we identify here will be of much use to us in unraveling the linear systems we encounter in our model for phase retrieval with local correlation measurements.

To this end, we introduce the \emph{interleaving operators} $P^{(d, N)} : \C^{dN} \to \C^{dN}$ for any $d, N \in \N$, each of which is a permutation defined by \begin{equation} (P^{(d, N)} v)_{(i - 1)N + j} = v_{(j - 1)d + i}.\label{eq:interleave_def}\end{equation}  We can view this is beginning with $v \in C^{dN}$ written as $N$ blocks of $d$ entries, and interleaving them into $d$ blocks each of $N$ entries.  Additionally, for $\ell, N_1, N_2 \in \N, v \in \C^{\ell N_1}, k \in [\ell],$ and $H \in \C^{\ell N_1 \times N_2}$, we define the block circulant operator $\circop^{N_1}$ by
\begin{align*}
  \circop_k^{N_1}(v) &= \begin{bmatrix} v & S^{N_1} v & \cdots & S^{(k - 1)N_1} v \end{bmatrix} \\
  \circop_k^{N_1}(H) &= \begin{bmatrix} H & S^{N_1} H & \cdots & S^{(k - 1) N_1}H \end{bmatrix},
\end{align*}
where, as with $\circop(\cdot)$, when we omit the subscript we define $\circop^{N_1}(H) = \circop_\ell^{N_1}(H)$ and $\circop^{N_1}(v) = \circop_\ell^{N_1}(v)$.  We now proceed with the following lemmas; the first establishes the inverse of $P^{(d, N)}$.

\begin{lemma} \label{lem:interleave_inverse}
  For $d, N \in \N,$ we have \[(P^{(d, N)})^{-1} = P^{(d, N) *} = P^{(N, d)}.\]
\end{lemma}

\begin{proof}[Proof of \cref{lem:interleave_inverse}]
  To prove the statement, we simply take $v \in \C^{d N}$ and calculate, for $i \in [d], j \in [N]$,
  \begin{align*}
    (P^{(d, N)} P^{(N, d)} v)_{(i - 1) N + j} &= (P^{(d, N)} (P^{(N, d)} v))_{(i - 1) N + j} \\
    &= (P^{(N, d)} v)_{(j - 1) d + i} \\
    &= v_{(i - 1) N + j},
  \end{align*}
  with these equalities coming from the definition in \eqref{eq:interleave_def}.  
\end{proof}

We now observe some useful ways in which the interleaving operators commute with the construction of circulant matrices.

\begin{lemma}\label{lem:interleave}
  Suppose $V_i \in \C^{k \times n}, v_{ij} \in \C^k, w_j \in \C^{k N_1}$ for $i \in [N_1], j \in [N_2]$ and
  \begin{gather*}
    M_1 = \begin{bmatrix} \circop(V_1) \\ \vdots \\ \circop(V_{N_1}) \end{bmatrix},\quad
    M_2 = \begin{bmatrix} \circop^{N_1}(w_1) & \cdots & \circop^{N_1}(w_{N_2}) \end{bmatrix},\ \text{and} \\
    M_3 = \begin{bmatrix} \circop(v_{11}) & \cdots & \circop(v_{1 N_2}) \\ \vdots & \ddots & \vdots \\ \circop(v_{N_1 1}) & \cdots & \circop(v_{N_1 N_2}) \end{bmatrix}.\end{gather*}
  Then
  \begin{align}
    P^{(k, N_1)} M_1 &= \circop^{N_1}\left(P^{(k, N_1)} \begin{bmatrix} V_1 \\ \vdots \\ V_{N_1} \end{bmatrix}\right) \label{eq:M_1} \\
    M_2 P^{(k, N_2)*} &= \circop^{N_1}\left(\begin{bmatrix} w_1 & \cdots & w_{N_2} \end{bmatrix}\right) \label{eq:M_2} \\
    P^{(k, N_1)}M_3P^{(k, N_2)*} &= \circop^{N_1}\left(P^{(k, N_1)} \begin{bmatrix} v_{11} & \cdots & v_{1 N_2} \\ \vdots & \ddots & \vdots \\ v_{N_1 1} & \cdots & v_{N_1 N_2} \end{bmatrix}\right). \label{eq:M_3}
  \end{align}
\end{lemma}

\begin{proof}[Proof of lemma \ref{lem:interleave}]
  We index the matrices to check the equalities.  For \eqref{eq:M_1}, we take $(a, b, \ell, j) \in [d] \times [N_1] \times [k] \times [n]$ and have \begin{align*}
    (P^{(k, N_1)} M_1)_{(a-1)N_1 + b, (\ell - 1) n + j} &= (M_1)_{(b - 1) k + a, (\ell - 1)n + j} \\ &= \begin{bmatrix} S^{\ell - 1} V_1 \\ \vdots \\ S^{\ell - 1} V_{N_1} \end{bmatrix}_{(b - 1)k + a, j} \\ &= (S^{\ell - 1}V_b)_{a, j} = (V_b)_{a + \ell - 1, j}
  \end{align*}
  and
  \begin{align*}
    \circop^{N_1}\left(P^{(k, N_1)} \begin{bmatrix} V_1 \\ \vdots \\ V_{N_1} \end{bmatrix}\right)_{(a-1)N_1 + b, (\ell - 1) n + j} &= \left(P^{(k, N_1)} \begin{bmatrix} V_1 \\ \vdots \\ V_{N_1} \end{bmatrix}\right)_{(a - 1)N_1 + b + (\ell-1)N_1, j} \\
    &= %\begin{bmatrix} v_1 \\ \vdots \\ v_{N_1} \end{bmatrix}_{(b-1)k + a + j - 1} = 
    (V_b)_{a + \ell - 1, j}
  \end{align*}
  For \eqref{eq:M_2}, we take $(a, b, j) \in [k] \times [N_2] \times [k N_1]$ and have
%  \begin{align*}
 \[(P^{(k, N_2)} M_2^*)_{(a - 1)N_2 + b, j} = (M_2)_{j, (b - 1) k + a} = (w_b)_{j + (a - 1)N_1}\]
%  \end{align*}
  and
  \[\left(\circop^{N_1}\left(\begin{bmatrix} w_1 & \cdots & w_{N_2} \end{bmatrix}\right)\right)_{j, (a - 1)N_2 + b} = (S^{N_1(a - 1)} w_b)_j = (w_b)_{j + N_1(a - 1)}\]

  \eqref{eq:M_3} follows immediately by combining \eqref{eq:M_1} and \eqref{eq:M_2}.
\end{proof}

\Cref{lem:interkron} introduces a few useful identities relating the interleaving operators to kronecker products.

\begin{lemma}\label{lem:interkron}
  For $v \in \C^N, V = \rowmat{V}{1}{\ell} \in \C^{N \times \ell}, A = \rowmat{A}{1}{m} \in \C^{d \times m}$, and $B_i \in \C^{m \times k}, i \in [\ell]$, we have
  \begin{align}
    P^{(d, N)} (v \kron A) &= A \kron v
    \label{eq:interkron_vec} \\
    P^{(d, N)} (V \kron A) &= \rowmat{A \kron V}{1}{\ell}
    \label{eq:interkron_mat} \\
    P^{(d, N)} (V \kron A) P^{(\ell, m)} &= A \kron V
    \label{eq:interkron_swap} \\
    (V \kron A) \diagmat{B}{1}{\ell} &= \rowmatfun{V_@ \kron A B_@}{1}{\ell}
    \label{eq:kron_diag}
  \end{align}
\end{lemma}

\begin{proof}[Proof of \cref{lem:interkron}]
  For \eqref{eq:interkron_vec}, we see that, for $i, j, k \in [d] \times [N] \times [m]$, we have
  \begin{align*}
    (P^{(d, N)} v \otimes A)_{(i - 1) N + j, k} &= (v \otimes A)_{(j - 1) d + i, k} \\
    &= v_j A_{i k}, \ \text{while} \\
    (A \kron v)_{(i - 1) N + j, k} &= A_{i k} v_j,
  \end{align*}
  and \eqref{eq:interkron_mat} follows by considering that $V \otimes A = \rowmatfun{V_@ \kron A}{1}{\ell}.$  To get \eqref{eq:interkron_swap}, we trace the positions of columns, considering that $(V \kron A) e_{(i - 1) m + j} = V_j \kron A_i$.  From \eqref{eq:interkron_mat}, we observe that $P^{(d, N)} (V \kron A) e_{(i - 1) m + j} = A_j \kron V_i,$ so \begin{align*}
    P^{(d, N)} (V \kron A) P^{(m, \ell)} e_{(j - 1) \ell + i} &= P^{(d, N)} (V \kron A) e_{(i - 1) m + j} \\
    &= A_j \kron V_i = (A \kron V) e_{(j - 1) \ell + i}.
  \end{align*}
 As for \eqref{eq:kron_diag}, we remark that \begin{gather*} (V \kron A) \diagmat{B}{1}{\ell}%% \begin{bmatrix} B_1 & 0 & \cdots & 0 \\ 0 & \ddots &  & \vdots \\ \vdots & & \ddots & 0 \\ 0 & \cdots & 0 & B_\ell \end{bmatrix}
    = (V \kron A) \rowmatfun{e^\ell_@ \kron B_@}{1}{\ell}  \\ = \rowmatfun{(V \kron A) (e_@^\ell \kron B_@)}{1}{\ell} = \rowmatfun{V_@ \kron A B_@}{1}{\ell},\end{gather*} as desired.
\end{proof}

The following lemma is a standard result (e.g., Theorem 13.26 in \cite{laub2004matrix}) regarding the kronecker product.

\begin{lemma}\label{lem:kronvec}
  We have $\vec(A B C) = (C^T \kron A) \vec(B)$ for any $A \in \C^{m \times n}, B \in \C^{n \times p}, C \in \C^{p \times k}.$  In particular, for $a, b \in \Cd, \vec a b^* = \conj{b} \kron a$, and \begin{equation}\label{id:kronsimp} \vec E_{jk} (\vec E_{j' k'})^* = E_{k k'} \kron E_{j j'}. \end{equation}
\end{lemma}
The next lemma covers the standard result concerning the diagonalization of circulant matrices, as well as a generalization to block-circulant matrices.
\begin{lemma}
  For any $v \in \Cd$, we have \begin{equation} \circop(v) = F_d \diag(\sqrt{d} F_d^* v) F_d^* = \sqrt{d} \sum_{j = 1}^d (f_j^{d *} v) f_j^d f_j^{d *} \label{eq:circ_dft_diag} \end{equation}
  Suppose $V \in C^{k N \times m}$, then $\circop^N(V)$ is block diagonalizable by \begin{equation} \circop^N(V) = \left(F_k \otimes I_N\right) \left(\diag(M_1, \ldots, M_k)\right) \left(F_k \otimes I_m\right)^*,  \label{eq:circ_dft_blk} \end{equation} where \begin{equation} \sqrt{k}\left(F_k \otimes I_N\right)^* V = \begin{bmatrix} M_1 \\ \vdots \\ M_k \end{bmatrix}, \quad \text{or} \quad M_j = \sqrt{k} (f_j^k \otimes I_N)^* V \label{eq:M_ell}\end{equation} \label{lem:circ_diag}
\end{lemma}

\begin{proof}[Proof of lemma \ref{lem:circ_diag}]
  The diagonalization in \eqref{eq:circ_dft_diag} is a standard result: see, e.g., Theorem 7 of \cite{gray2006circulant}.

  To prove \eqref{eq:circ_dft_blk}, we set $V_i$ to be the $k \times m$ blocks of $V$ such that $V^* = \rowmat{V^*}{1}{k}$ and begin by observing that, for $u \in \C^k$ and $W \in \C^{m \times p}$, the $\ell\th$ $k \times p$ block of $\circop^N(V)(u \otimes W)$ is given by \[\left(\circop^N(V)(u \otimes W)\right)_{[\ell]} = \sum_{i = 1}^k u_i (S^{N (i - 1)}V)_\ell W = \sum_{i = 1}^k u_i V_{\ell - i + 1} W.\]  Taking $u = f_j^k$ and $W = I_m$, this gives \begin{align*} \left(\circop^N(V)(f_j^k \otimes I_m)\right)_{[\ell]} &= \frac{1}{\sqrt{k}}\sum_{i = 1}^k \omega_k^{(j - 1) (i - 1)} V_{\ell - i + 1} I_m \\ &= \frac{1}{\sqrt{k}} \omega_k^{(j - 1) (\ell - 1)} \sum_{i = 1}^k \omega_k^{-(j - 1)(i - 1)} V_i \\ &= (f_j^k)_\ell \left(\sqrt{k} (f_j^k \otimes I_N)^* V \right) = (f_j^k)_\ell M_j. \end{align*}  This relation is equivalent to having \[\circop^N(V) (f_j^k \otimes I_m) = (f_j^k \otimes M_j) = (f_j^k \otimes I_N) M_j,\] which is the statement of the lemma.
\end{proof}

Lemma \ref{lem:circ_diag} immediately gives the following corollary regarding the conditioning of $\circop^N(V)$, with which we return to considering spanning families of masks.

\begin{corollary} \label{cor:circ_diag_condition}
  With notation as in lemma \ref{lem:circ_diag}, the condition number of $\circop^N(V)$ is \[\dfrac{\max\limits_{i \in [k]} \sigma_{\max} (M_i)}{\min\limits_{i \in [k]} \sigma_{\min} (M_i)}.\]
\end{corollary}

\subsection{Proofs of Main Results}
\label{sec:gam_fam}
We begin with the proof of \cref{thm:meas_cond}, of which \cref{prop:span_fam_cond} is a special case.

\begin{proof}[Proof of \cref{thm:meas_cond}]
  We consider the rows of the matrix $A$ representing the measurement operator $\mathcal{A}$, defined in \eqref{eq:meas_mat} and \eqref{eq:meas_op}.  We vectorize $X \in T_\delta(\C^{d \times d})$ by its diagonals with $\Dc_\delta$, as in \eqref{eq:diag_vec} and set $\chi_m = \diag(X, m), m = 1 - \delta, \ldots, \delta - 1$.  Each measurement then looks like
  \begin{align*}
    \mathcal{A}(X)_{(\ell, j)} &= \langle S^{\ell} m_j m_j^* S^{-\ell}, X \rangle \\
    &= \sum_{m = 1 - \delta}^{\delta - 1} \langle S^{\ell} g_m^j, \chi_m \rangle,
  \end{align*}
  so that the definition of $A$ in \eqref{eq:meas_mat} immediately yields its $(j - 1) d + \ell\th$ row as $\rowmatfun{g_@^{j*} S^{1 - \ell}}{1 - \delta}{\delta - 1}$.  Transposing this expression, we see that the $(j - 1)d + 1\st$ through $(j - 1) d + d\th$ rows of $A$ compose $\rowmatfun{\circop(g^j_@)^*}{1 - \delta}{\delta - 1}$. %% \[\begin{bmatrix} S^{\ell - 1} g_{1 - \delta}^j \\ \vdots \\ S^{\ell - 1} g_{\delta - 1}^j \end{bmatrix}^*.\]  
  Together with $\circop(v)^* = \circop(R \conj{v})$, this determines $A$ to be the block matrix given by \[A = \cornmatfun{\circop(g_@c^@r)^*}{1}{D}{1 - \delta}{\delta - 1} = \cornmatfunexp{\circop(R \conj{g}_@c^@r)}{1}{D}{1 - \delta}{\delta - 1},\] which may be transformed, by \eqref{eq:M_3} of \cref{lem:interleave}, to
  \begin{equation}
    P^{(d, D)} A P^{(d, 2 \delta - 1)*} = \circop^D\left(P^{(d, D)} \cornmatfun{R \conj{g}_@c^@r}{1}{D}{1 - \delta}{\delta - 1} \right) = \circop^D(H).
    \label{eq:interleaved_meas}
  \end{equation}

%We label the $D \times 2\delta - 1$ blocks of $H$ by $H^* = \begin{bmatrix} H_1^* & \cdots & H_d^* \end{bmatrix}$, so that $(H_\ell)_{ij} = (R g_{j - \delta}^i)_\ell = (g_{j - \delta}^i)_{2 - \ell}$.
Quoting \cref{cor:circ_diag_condition} establishes the theorem.
\end{proof}

We are now able to prove \cref{prop:span_fam_cond}.

\begin{proof}[Proof of \cref{prop:span_fam_cond}]
  We begin by remarking that, for any $j, m \in [d]$, the Cauchy-Schwarz inequality gives us
  \begin{align*}
    f_j^{d*} (\gamma \circ S^{-m} \gamma) &= \left(D_{f_j^{d*}} \gamma\right) \cdot \left(S^{-m} \gamma\right) \\
    &\le \frac{1}{\sqrt{d}} \norm{\gamma}_2^2.
  \end{align*}
  Observing that $f_1^{d*}(\gamma \circ \gamma) = \frac{1}{\sqrt{d}} \norm{\gamma}_2^2$, we have that \begin{equation} \max_{(j, m) \in [d] \times [\delta]_0} F_d^* (\gamma \circ S^{-m} \gamma)_j = \frac{1}{\sqrt{d}} \norm{\gamma}_2^2. \label{eq:max_sv} \end{equation}

  For the moment, we assert that $D = 2 \delta - 1 \le d$ and set $\tF_K \in \C^{2 \delta - 1 \times 2 \delta - 1}, (\tF_K)_{ij} = \frac{1}{\sqrt{K}}\omega_K^{(i-1)(j-\delta)}$ to be the principal submatrix of $\diag(\sqrt{K} f^K_{2 - \delta}) F_K$.  For a local Fourier measurement system, we have \begin{equation} \begin{aligned} g_m^j &= \diag(m_j m_j^*, m) = \diag((\gamma \circ v_j) (\gamma \circ v_j)^*, m) \\ &= \diag(D_{v_j} \gamma \gamma^* D_{\overline{v_j}}, m) = \omega_K^{-m(j - 1)} \diag(\gamma \gamma^*, m), \end{aligned} \label{eq:gam_diag}\end{equation} so, setting $g_m = \diag(\gamma \gamma^*, m),$ we have $g_m^j = \diag(m_j m_j^*, m) = \omega_K^{-m(j - 1)} g_m$.  Therefore, we label the $2 \delta - 1 \times 2 \delta - 1$ blocks of $H$ by $H^* = \begin{bmatrix} H_1^* & \cdots & H_d^* \end{bmatrix}$, so that \[(H_\ell)_{ij} = (R \conj{g}_{j - \delta}^i)_\ell = \omega_K^{(i - 1)(j - \delta)}(R g_{j - \delta})_\ell\] and $M_\ell = \sqrt{d} (f_j^d \otimes I_D)^* H = \sum_{k = 1}^d \omega_d^{-(\ell - 1)(k - 1)} H_k,$ giving \begin{align*} (M_\ell)_{ij} &= \sum_{k = 1}^d \omega_d^{-(\ell - 1)(k - 1)} (H_k)_{ij} = \omega_K^{(i - 1)(j - \delta)} \sum_{k = 1}^d \omega_d^{-(\ell - 1)(k - 1)} (Rg_{j - \delta})_k \\ &= \sqrt{d} \omega_K^{(i - 1)(j - \delta)} (\conj{F}_d^* g_{j - \delta})_\ell. \end{align*}  In other words, \begin{equation} M_\ell = \sqrt{d K} \tF_K \diag(f_{2 - \ell}^{d*} g_{1 - \delta},\, \ldots\, , f_{2 - \ell}^{d*} g_{\delta - 1}). \label{eq:block_diag_components} \end{equation}  If $K = 2 \delta - 1$, then $\tF_K$ is unitary, and the singular values of $M_\ell$ are $\{\sqrt{d K} \abs{f_\ell^{d *} g_j}\}_{j = 1 - \delta}^{\delta - 1}$ (since $\abs{f_{2 - \ell}^{d*} g_j} = \abs{\conj{f}_{2 - \ell}^{d*} g_j} = \abs{f_\ell^{d*} g_j}$).  Recognizing that $S^j g_j = g_{-j}$, then \cref{prop:meas_cond} and \eqref{eq:max_sv} take us to \eqref{eq:clean_cond}.

  If $D = 2 \delta - 1 < K$, then the argument remains unchanged, except that the singular values of $M_\ell$, instead of being known explicitly, are bounded above and below by $\max\limits_{|j| < \delta} |f_\ell^{d *} g_j| \sigma_{\max}(\tF_K)$ and $\min\limits_{|j| < \delta} |f_\ell^{d *} g_j| \sigma_{\min}(\tF_K)$ respectively, which gives the more general result of \eqref{eq:messy_cond}.

  If $2 \delta - 1 > d = D$, then, by an argument similar to that used in \cref{prop:meas_cond}, we may obtain \[A = \cornmatfun{\circop(R \conj{g}_@c^@r)}{1}{d}{0}{d - 1},\] and a similar application of \eqref{eq:M_3} gives that \[P^{(d, d)} A P^{(d, d)*} = \circop^d\left( P^{(d, d)} \cornmatfun{R \conj{g}_@c^@r}{1}{d}{0}{d - 1} \right).\]  Setting \[H = P^{(d, d)} \cornmatfun{R \conj{g}_@c^@r}{1}{d}{0}{d - 1},\] and defining $H_\ell \in \C^{d \times d}$ by $H^* = \rowmat{H^*}{1}{d}$, %% then instead of using diagonals $1 - \delta, \ldots, \delta - 1$, we use diagonals $0, 1, \ldots, d - 1$.  This change propagates from \eqref{eq:vectorized_meas} to \eqref{eq:interleaved_meas}, so that
  we have \[(H_\ell)_{ij} = \omega_K^{(i - 1)(j - 1)} (R g_{j - 1})_\ell \quad \text{and} \quad (M_\ell)_{ij} = \sqrt{d} \omega_K^{(i - 1)(j - 1)} (\conj{F}_d^* g_{j - 1})_\ell,\] giving $M_\ell = \sqrt{d K} \mathcal{R}_{d \times d}(F_K) \diag(f_{2 - \ell}^{d*} g_0,\, \ldots\, , f_{2 - \ell}^{d*} g_{d - 1})$, which immediately gives us \eqref{eq:messy_cond}.  We remark that indexing only over the diagonals $m \in [\delta]_0$ in \eqref{eq:messy_cond} suffices, again because $S^j g_j = g_{-j}$, so having $2 \delta - 1 > d$ makes $1 - \delta, \ldots, -1$ redundant.  
\end{proof}

%%\subsection{Explicit Examples of Spanning Families}
\label{sec:reanalyze}

In this section, we analyze three explicit examples of masks $\gamma \in \Rd$ and their corresponding local Fourier measurement systems, and prove under what conditions these constitute spanning families.  The goal is to constructively provide examples of spanning families that are well-conditioned, and which are scalable in the sense that they may be used for any choice of $d$ and $\delta$.  Specifically, in \cref{sec:exp_mask}, we analyze Example 1 of \cref{sec:MeasMatrix} (also known as ``exponential masks,'' as we take $\gamma_i = C a^i$ for some $C, a \in \R$) with the new theory of this chapter, and find improvements on the bounds of its condition number, which scales roughly like $\kappa \approx \delta^2$.  In \cref{sec:nearflat_mask}, a new set of masks is studied.  These masks, referred to as ``near-flat masks,'' are constructed by taking $\gamma = a e_1 + \one_{[\delta]} \in \Rd$, and we provide a choice of $a$ that achieves a condition number that is asymptotically linear in $\delta$ -- a notable improvement over the conditioning of the exponential masks.  Finally, in \cref{sec:const_mask}, we note the somewhat curious case of a constant mask, $\gamma = \one_{[\delta]}$.  Here, $\gamma$ produces a spanning local Fourier measurement system -- with poor conditioning -- when $d$'s prime divisors are each greater than $\delta$.

\subsection{Exponential Masks}
\label{sec:exp_mask}
For our analysis of Example 1 of \cref{sec:MeasMatrix}, we rephrase the definition of \eqref{eq:MeasDef} somewhat slightly.  Here, we will take $d \in \N$ to be the ambient dimension and $\delta \le \frac{d + 1}{2}$.  Then, we let $\{m_j(a)\}_{j = 1}^{2 \delta - 1}$ be the local Fourier measurement system with mask $\gamma(a) \in \Rd$ defined by $\gamma(a)_i = a^{i - 1},$ for some $0 < a \in \R, a \neq 1$.  Denoting the condition number of the measurement operator associated to $\{m_j(a)\}_{j \in [2 \delta - 1]}$ by $kappa(a)$, we remark that it suffices to compute (or estimate) $\kappa(a)$ for all $a > 1$, as $\kappa(a) = \kappa(1 / a)$, which is proven in \cref{prop:kap_recip}.

\begin{proposition}
  Given $d \in \N, \delta \le \frac{d + 1}{2},$ and $a \in (0, 1) \cup (1, \infty),$ we define $\{m_j(a)\}_{j \in [2 \delta - 1]}$ to be the local Fourier measurement system of support $\delta$ with mask $\gamma(a) \in \Rd$ defined by \[\gamma(a)_i = \begin{piecewise} a^{i - 1} & i \in [\delta] \\ 0 & \ow \end{piecewise}.\]  Let $\kappa(a) := \kappa(\Ac)$ be the condition number of the measurement operator associated with $\{m_j(a)\}_{j \in [2 \delta - 1]}$.  Then $\kappa(a) = \kappa(1 / a)$.
  \label{prop:kap_recip}
\end{proposition}

\begin{proof}[Proof of \cref{prop:kap_recip}]
  Fix $a > 1$.  We begin by noting that $\gamma(a) = a^{\delta - 1} S^{\delta - 1} R \gamma(a^{-1})$, so that $\norm{\gamma(a)}_2^2 = a^{2 \delta - 2} \norm{\gamma(a^{-1})}_2^2$ and
  \begin{align*}
    \gamma(a) \circ S^{-m} \gamma(a) &= a^{2 \delta - 2} \left(S^{\delta - 1} R \gamma(a^{-1}) \circ S^{-m} S^{\delta - 1} R \gamma(a^{-1})\right) \\
    %% &= a^{2 \delta - 2} S^{\delta - 1} \left(R \gamma(a^{-1}) \circ R S^m \gamma(a^{-1})\right) \\
    %% &= a^{2 \delta - 2} S^{\delta - 1} R (\gamma(a^{-1}) \circ S^m \gamma(a^{-1})) \\
    &= a^{2 \delta - 2} S^{\delta - 1} R S^m (\gamma(a^{-1}) \circ S^{-m} \gamma(a^{-1})),
  \end{align*}
  so that
  \begin{align*}
    f_j^{d *} (\gamma(a) \circ S^{-m} \gamma(a)) &= a^{2 \delta - 2} \omega_d^{-(j - 1) (\delta - m - 1)} f_{2 - j}^{d *} (\gamma(a^{-1}) \circ S^{-m} \gamma(a^{-1})) \\
    &= a^{2 \delta - 2} \omega_d^{-(j - 1) (\delta - m - 1)} \conj{f_j^{d *} (\gamma(a^{-1}) \circ S^{-m} \gamma(a^{-1}))},
  \end{align*}
  which gives, by \cref{prop:gam_fam_cond}, that
  \begin{align*}
    \kappa(a) &= \dfrac{d^{-1/2} \norm{\gamma(a)}_2^2}{\min\limits_{m \in [\delta]_0, j \in [d]} \abs*{f_j^{d*} (\gamma(a) \circ S^{-m} \gamma(a))}} \\ %% \dfrac{\max\limits_{m \in [\delta]_0, j \in [d]} \abs*{a^{2 \delta - 2} \omega_d^{-(j - 1) (\delta - m - 1)} \conj{f_j^{d *} (\gamma(a^{-1})(a^{-1}) \circ S^{-m} \gamma(a^{-1})(a^{-1}))}}}{\min\limits_{m \in [\delta]_0, j \in [d]} \abs*{a^{2 \delta - 2} \omega_d^{-(j - 1) (\delta - m - 1)} \conj{f_j^{d *} (\gamma(a^{-1})(a^{-1}) \circ S^{-m} \gamma(a^{-1})(a^{-1}))}}} = 
    &= \dfrac{d^{-1/2} \norm{\gamma(a^{-1})}_2^2}{\min\limits_{m \in [\delta]_0, j \in [d]} \abs*{f_j^{d*} (\gamma(a^{-1}) \circ S^{-m} \gamma(a^{-1}))}} = \kappa(1 / a)
  \end{align*}  
\end{proof}

We now produce an estimate of $\kappa(a)$ in \cref{prop:exp_kappa}.

\begin{proposition}
  With $d, \delta, \{m_j(a)\}_{j \in [2 \delta - 1]}, \gamma(a), \kappa(a)$ defined as in \cref{prop:kap_recip}, when $a > 1$ we have \begin{equation}\kappa(a) \le \dfrac{a^{2 \delta} - 1}{a^{\delta + 1} - a^{\delta - 1}} \cdot \dfrac{a^2 + 1}{a^2 - 1},\label{eq:kap_est}\end{equation} and when $a \in (0, 1), \kappa(a) = \kappa(1 / a)$.  Taking \[a_\delta = \begin{piecewise} 1 + \frac{4}{\delta - 2} & \delta \ge 5 \\ \frac{1 + \sqrt{5}}{2} & \ow \end{piecewise},\] we may obtain \begin{equation} \kappa(a_\delta) \le \left(\dfrac{\ee (\delta + 2)}{4}\right)^2. \label{eq:kap_opt} \end{equation}
  \label{prop:exp_kappa}
\end{proposition}

\begin{proof}[Proof of \cref{prop:exp_kappa}]
  We fix $a > 1$, and we remark that \begin{equation}\norm{\gamma(a)}_2^2 = \sum_{i = 1}^\delta a^{2(i - 1)} = \dfrac{1 - a^{2 \delta}}{1 - a^2} = \dfrac{a^{2 \delta} - 1}{a^2 - 1}, \label{eq:norm_gam} \end{equation} so all that remains is to bound $\min_{j, m} \abs{f_j^{d*}(\gamma(a) \circ S^{-m} \gamma(a))}$ from below.  For convenience, we define $h(j, m) := \sqrt{d} f_j^{d*}(\gamma(a) \circ S^{-m} \gamma(a)).$  We begin by computing
  \begin{align}
    \abs{h(j, m)} &= \abs*{\sum_{i = 1}^{\delta - m} a^{m + 2 (i - 1)} \omega_d^{-(j - 1)(i - 1)}} \nonumber\\
    &= \abs*{\dfrac{a^m - a^{2 \delta - m} \omega_d^{-(j - 1)(\delta - m)}}{1 - a^2 \omega_d^{-(j - 1)}}} \nonumber\\
    &\ge \abs*{\dfrac{a^{2 \delta - m} - a^m}{a^2 + 1}}.\label{eq:ek_bound}
  \end{align}
  which is clearly decreasing with $m$, so it is minimized when $m = \delta - 1$.  This gives that $\min_{j, m} \abs{f_j^{d*}(\gamma(a) \circ S^{-m} \gamma(a))} \ge \frac{1}{\sqrt{d}} \frac{a^{\delta + 1} - a^{\delta - 1}}{a^2 + 1},$ which, with \eqref{eq:norm_gam} and \cref{prop:kap_recip,prop:gam_fam_cond}, yields \eqref{eq:kap_est}.

  To obtain \eqref{eq:kap_opt}, we consider that \eqref{eq:ek_bound} is necessarily suboptimal.  Consider that, trivially, $h(j, \delta - 1) = \abs{\gamma(a)_1 \gamma(a)_{\delta}} = a^{\delta - 1}$, whereas we have bounded it below by $\frac{a^{\delta + 1} - a^{\delta - 1}}{a^2 + 1} = a^{\delta - 1} \frac{a^2 - 1}{a^2 + 1}.$  Hence, a tighter lower bound may be to use $m = \delta - 2$ in the right hand side of \eqref{eq:kap_opt}.  Using \[\dfrac{a^{\delta + 2} - a^{\delta - 2}}{a^2 + 1} = a^{\delta - 2} \dfrac{a^4 - 1}{a^2 + 1} = a^{\delta - 2} (a^2 - 1),\] this yields \[h(j, m) \ge \min\left(a^{\delta - 1}, a^{\delta - 2} (a^2 - 1)\right).\]  We further observe that $a^{\delta - 2} (a^2 - 1) = a^{\delta - 1}$ exactly when $a^2 - a - 1 = 0$, yielding \[h(j, m) \ge \begin{piecewise} a^{\delta - 1} & a \ge \frac{1 + \sqrt{5}}{2} \\ a^{\delta - 2} (a^2 - 1) & 1 < a \le \frac{1 + \sqrt{5}}{2} \end{piecewise}.\]  Therefore, for $\delta \ge 5$, we have $(1 + \frac{4}{\delta - 2})^{1 / 2} \le \frac{1 + \sqrt{5}}{2}$ so that, choosing $a_\delta = 1 + \frac{4}{\delta - 2}$, we may guarantee
  \begin{equation} \begin{gathered}
    \kappa(a_\delta) \le \dfrac{a_\delta^{2 \delta} - 1}{a_\delta^{\delta - 2} (a_\delta^2 - 1)^2} \le \dfrac{a_\delta^{\delta + 2}}{(a_\delta^2 - 1)^2} \\
    = \dfrac{(1 + \frac{4}{\delta - 2})^{(\delta + 2) / 2}}{(4 / (\delta - 2))^2} \le \dfrac{\ee^2 (1 + \frac{4}{\delta - 2})^2}{(4 / (\delta - 2))^2} = \left(\dfrac{\ee (\delta + 2)}{4}\right)^2,
  \end{gathered} \label{eq:kapadel_approx} \end{equation}
  where the third inequality comes from $(1 + \frac{2}{x})^x \le \ee^2$ for $x > 0$.  For $\delta \in [2, 5)$, setting $\phi = \frac{1 + \sqrt{5}}{2}$, we may upper bound \[\kappa(\phi) \le \dfrac{\phi^{2 \delta} - 1}{\phi^{\delta - 1}},\] and numerically verify that this implies $\kappa(\phi) < (\ee(\delta + 2) / 4)^2$ for $\delta \in [2, 5)$.
  
  %% To find an optimal choice for $a$, we optimize both $\frac{(a^{2 \delta} - 1) / (a^2 - 1)}{a^{\delta - 1}}$ and $\frac{(a^{2 \delta} - 1) / (a^2 - 1)}{a^{\delta - 2}(a^2 - 1)}$.  For the moment, we require $\delta > 2$, and we find that 
  %% \begin{align*}
  %%   \dfrac{a^{2 \delta} - 1}{(a^2 - 1)a^{\delta - 1}} &\le \dfrac{a^{2 \delta}}{(a^2 - 1) a^{\delta - 1}} \\
  %%   &= \dfrac{a^{\delta + 1}}{a^2 - 1},
  %% \end{align*}
  %% which has a single critical point at \[a = \left(\frac{\delta + 1}{\delta - 1}\right)^{1 / 2} = \left(1 + \frac{2}{\delta - 1}\right)^{1 / 2} < (1 + \sqrt{5}) / 2,\] while \[\dfrac{a^{2 \delta} - 1}{a^{\delta - 2}(a^2 - 1)^2} \le \dfrac{a^{\delta + 2}}{(a^2 - 1)^2},\] which has a single minimum at \[a = \left(\frac{\delta + 2}{\delta - 2}\right)^{1 / 2} = \left(1 + \frac{4}{\delta - 2}\right)^{1 / 2},\] which is less than $(1 + \sqrt{5}) / 2$ for $\delta \ge 7$.
  %% \begin{align*}
  %%   \dfrac{a^{2 \delta} - 1}{a^{\delta + 1} - a^{\delta - 1}} \cdot \dfrac{a^2 + 1}{a^2 - 1} &\le \dfrac{a^{2 \delta}}{a^{\delta + 1} - a^{\delta - 1}} \cdot \dfrac{a^2 + 1}{a^2 - 1} \\
  %%   &= \dfrac{a^{\delta + 1}(a^2 + 1)}{(a^2 - 1)^2} \\
  %%   &= \dfrac{a^{\delta + 1}(a^2 + 1)}{(a - 1)^2 (a + 1)^2} \\
  %%   &\le \dfrac{a^{\delta + 1}}{(a - 1)^2}.
  %% \end{align*}
  %% We set $f(a) = a^{\delta + 1} / (a - 1)^2$ and find that its single critical value is a minimum at $a_{\delta} = 1 + \frac{2}{\delta - 1}$.  Using $(1 + x / n)^n \le e^x$ for $x \ge 0$, we have
  %% \begin{align}
  %%   \kappa(a_{\delta}) \le f(a_{\delta}) &= \dfrac{\left(1 + \frac{2}{\delta - 1}\right)^{\delta + 1}}{(2 / (\delta - 1))^2} \le \left(1 + \frac{2}{\delta - 1}\right)^2 \left(\dfrac{\ee (\delta - 1)}{2}\right)^2 \label{eq:a_delta_mid}\\
  %%   &= \left(\dfrac{\ee (\delta + 1)}{2}\right)^2, \nonumber
  %% \end{align}
  %% which completes the proof.
\end{proof}

We remark that the bound in \eqref{eq:kap_opt} is slightly stronger than a similar bound proven in Theorem 4 of \cite{IVW2015_FastPhase}, the work by \citeauthor*{IVW2015_FastPhase} where this family of masks was first studied.  They bounded the condition number for the optimal choice of $a$ by \[\kappa < \max\left\{144 \ee^2, \left(\frac{3 \ee (\delta - 1)}{2}\right)^2 \right\}.\]  For small delta, the term $144 \ee^2 > 1,000$ is much too loose, and even if this term was omitted, $\frac{\delta + 2}{4} < \frac{3(\delta - 1)}{2}$ for all $\delta > 1$.  For asymptotically large $\delta$, the bound proven above in \cref{prop:exp_kappa} is stronger by a constant factor of $36$.

\subsection{Near-flat masks}
\label{sec:nearflat_mask}
We now analyze masks of the form $\gamma = a e_1 + \one_{[\delta]}$ in \cref{prop:spike_one}.

\begin{proposition}
  Let $\{m_j\}_{j \in [D]} \subset \Cd$ be the local Fourier measurement system of support $\delta \le \frac{d + 1}{2}$ with mask $\gamma \in \Rd$ given by $\gamma = a e_1 + \one_{[\delta]}$ where $a > \delta - 1$.  Then this is a spanning family with condition number bounded above by \begin{equation} \kappa \le \dfrac{a^2 + 2 a + \delta}{a - \delta + 1}. \label{eq:spike_cond} \end{equation}  If we choose $a = 2 \delta - 1$, we have $\kappa \le 4 \delta + 1$.
  \label{prop:spike_one}
\end{proposition}

\begin{proof}[Proof of \cref{prop:spike_one}]
  We calculate the condition number directly.  We immediately have $\norm{\gamma}_2^2 = (a + 1)^2 + (\delta - 1) = a^2 + 2a + \delta$, which is the numerator of \eqref{eq:spike_cond}, so it remains only to provide a lower bound on $\sqrt{d} f_j^{d*}(\gamma \circ S^{-m} \gamma)$.  To achieve this, we remark that, for $m \ge 1$, \[\sqrt{d} f_j^{d*} (\gamma \circ S^{-m} \gamma) = a + \sum_{i = 1}^{\delta - m} \omega_d^{(j - 1)(i - 1)} = \begin{piecewise} a + \delta - m & j = 1 \\ a + \dfrac{1 - \omega_d^{(j - 1)(\delta - m)}}{1 - \omega_d^{j - 1}} & \ow \end{piecewise}.\]  Clearly, this expression has its maximum absolute value when $j = 1$, as $\abs{a + \sum_{i = 1}^{\delta - m} \omega_d^{(j - 1) (i - 1)}} \le a + \sum_{i = 1}^{\delta - m} \abs{\omega_d^{(j - 1)(i - 1)}} = a + \delta - m$, so we consider that, for $j \neq 1$, we have \[\abs*{\sqrt{d} f_j^{d*} (\gamma \circ S^{-m} \gamma)} \ge \abs*{\Re\left(a + \dfrac{1 - \omega_d^{(j - 1)(\delta - m)}}{1 - \omega_d^{j - 1}}\right)}.\]  We then reduce the term $\Re\left(\frac{1 - \omega_d^{(j - 1)(\delta - m)}}{1 - \omega_d^{j - 1}}\right)$ by setting $\eit := \omega_d^{j - 1}$ and $k := \delta - m$ and finding
  \begin{align*}
    \Re\left(\dfrac{1 - \ee^{\ii k \theta}}{1 - \eit}\right) &= \Re\left(\dfrac{(1 - \ee^{\ii k \theta})(1 - \ee^{-\theta \ii})}{2 - 2 \cos \theta}\right) \\
    &= \dfrac{(1 - \cos \theta) + \cos(k - 1) \theta - \cos k \theta}{2 - 2 \cos \theta} \\
    &= \frac{1}{2} + \dfrac{\sin(k - \frac{1}{2}) \theta \sin \frac{\theta}{2}}{2 \sin^2 \frac{\theta}{2}},
  \end{align*}
  where the third line comes from $\sin^2 x = (1 - \cos(2x)) / 2$ and $\cos a - \cos b = \frac{1}{2} \sin\left(\frac{a + b}{2}\right) \sin\left(\frac{b - a}{2}\right)$.  Using $\abs{\sin n\theta} \le n \abs{\sin \theta}$, this gives us that $\Re\left(\frac{1 - \ee^{\ii k \theta}}{1 - \eit}\right) \ge \frac{1}{2} - \frac{2k - 1}{2} = -k$, and hence \begin{equation} \abs*{\sqrt{d} f_j^{d*}(\gamma \circ S^{-m} \gamma)} \ge a - \delta + m \ge a - \delta + 1 \label{eq:nearflat_1} \end{equation} for all $1 \le m < \delta$.  For $m = 0$, a similar calculation gives that \begin{equation} \abs*{\sqrt{d} f_j^{d*}(\gamma \circ \gamma)} \ge a^2 + 2a - \delta, \label{eq:nearflat_2}\end{equation} and we notice that this bound is \emph{greater} than the bound for the case $1 \le m$, stated in \eqref{eq:nearflat_1} whenever $a \ge \frac{1 + \sqrt{5}}{2}$.  This means \eqref{eq:nearflat_1} is tighter than \eqref{eq:nearflat_2} whenever $a > \delta - 1$, which is necessary for these bounds to be positive and meaningful.  Therefore, we may restrict to choices of $a > \delta - 1$ and use the bound $\abs*{\sqrt{d} f_j^{d*}(\gamma \circ S^{-m} \gamma)} \ge a - \delta + 1$.  This completes the proof.
\end{proof}

We remark that this result is a very welcome contribution to our library of well-conditioned local measurement systems.  In Example 2 of \cref{sec:MeasMatrix}, we provided an example of a local measurement system that achieves a condition number that scales as $\bigO(\delta)$, but it was a somewhat cumbersome construction.  Each of its members $m_j$ was incredibly sparse, having either 1 or 2 nonzero entries, which is somewhat unrealizable since it is not even a local Fourier measurement system (which corresponds to the diffraction process that we expect to govern our measurement apparatus).  Furthermore, its sparsity at the level of discretization means a practitioner would have to have perfect control over the illumination of their sample at the scale of resolution they want for their image, which puts an unrealistic demand on the accuracy of the equipment being used (imagine taking a photograph by separately illuminating each pixel in a scene and recording the reflected light).  By contrast, the example proposed in \cref{prop:spike_one}, while not necessarily simple to achieve in the lab, at least has the merit of accomodating the Fresnel diffraction model which motivates the study of local Fourier measurement systems, and its conditioning asymptotically equals that of the sparse construction that was previously our most well-conditioned measurement system.

We further remark that a small improvement can be made to this bound.  By following the work of \citeauthor{mercerXXXXdirichlet} in \cite{mercerXXXXdirichlet}, we can asymptotically (as $d, \delta \to \infty$) bound the real part of $\sum_{i = 1}^m \omega_d^{(j - 1)(i - 1)}$ below by $C \delta$ where $C > - 1 / 4$.  At the very least, a choice of $a = 2 \delta$ would in this case yield a condition number that asymptotically converges to no more than $4 \delta^2 / (7 / 4 \delta) = \frac{16}{7} \delta$, but this yields an improvement of no more than a factor of $2$.  In fact, optimizing the choice of $a$ more carefully, with this improved constant, can get us down to $\kappa < \frac{7}{8} \delta$ as $\delta \to \infty$, but this proof is technical, so we relegate it to \cref{app:nearflat_more} and illustrate it numerically in \cref{sec:expl_num}.

%In the proof of \cref{prop:spike_one}, we restricted to choices of $a \ge \delta - 1$ in order to simplify the bounding process by excluding the lower bound for $\abs{f_j^{d*} (\gamma \circ \gamma)}$.  We remark that this restriction doesn't hurt the asymptotic behavior of the bound, since if we took $a$ small so that 

\subsection{Constant masks}
\label{sec:const_mask}
After these two examples, we also remark that the simplest type of mask -- a constant mask, where $\gamma = \one_{[\delta]}$ -- can actually produce a spanning family.  The conditions required of $\delta$ and $d$ to admit this are upsettingly number theoretical, so we present this result in \cref{prop:constant_gamma} primarily as an incidental curiosity.

\begin{proposition}
  Take $d \in \N$ and $\delta \le d$.  Then, with $D = \min(d, 2 \delta - 1)$, the local Fourier measurement system $\{m_j\}_{j = 1}^D$ of support $\delta$ and mask $\gamma = \one_{[\delta]}$ is a spanning family if and only if $d$ is strictly $\delta$-rough, in the sense that $k \mid d \implies k > \delta$.  In this event, and if we additionally take $d > 4$, the condition number of $\Ac$ is bounded by $\kappa \le \delta d^2 / 8$.
  \label{prop:constant_gamma}
\end{proposition}

\begin{proof}[Proof of \cref{prop:constant_gamma}]
  We begin by remarking that $\gamma \circ S^{-(\delta - k)} \gamma = \one_{[k]}$, such that \begin{equation} \sqrt{d} f_j^{d*} (\gamma \circ S^{-(\delta - k)} \gamma) = \sum_{i = 1}^k \omega_d^{(j - 1) (i - 1)} = \begin{piecewise} k & j = 1 \\ \dfrac{1 - \omega_d^{(j - 1) k}}{1 - \omega_d^{(j - 1)}} & \ow \end{piecewise}, \label{eq:const_fj}\end{equation} and hence $f_j^{d*}(\gamma \circ S^{- (\delta - k)} \gamma) = 0$ iff $(j - 1) k = n d$ for some positive integer $n$.  By \cref{cor:gam_fam_span}, this means that $\gamma$ produces a spanning family iff there does not exist a pair $(j, k) \in [d] \times [\delta]_0$ such that $j k = n d$ for some positive integer $n$.  This condition occurs iff there is no pair $(j, k) \in [d] \times [\delta]$ satisfying $j k = d$, which is to say that $\gamma$ produces a spanning family iff $d$ is strictly $\delta$-rough.

  To get the condition number, consider that $\norm{\gamma}_2^2 = \delta$.  It suffices, then, to get a lower bound on $\sqrt{d} \abs*{f_j^{d*} \one_{[k]}}$ for all $k \in [\delta]$.   Trivially following from \eqref{eq:const_fj}, we may write \[\sqrt{d} \abs*{f_j^{d*} \one_{[k]}} \ge \dfrac{\abs{1 - \omega_d}}{2} \ge \dfrac{\abs{\Re(1 - \omega_d)}}{2} = \dfrac{1 - \cos\left(\frac{2 \pi}{d}\right)}{2}.\]  For $d > 4$, we use $1 - \cos(x) \ge (2 x / \pi)^2$ to get that $(1 - \cos(\frac{2 \pi}{d})) / 2 \ge 8 / d^2$, which completes the proof.  (We also note that the only possible case when $d \le 4$ is $d = 3, \delta = 2$, and we can find by exhaustive calculation that $\kappa = \frac{2}{2 - \sqrt{3}}$).
\end{proof}

This result shows that, while constant masks can produce spanning families in some circumstances, the condition number of the resulting linear system is remarkably unstable as a function of the parameters of the discretization, $d$ and $\delta$.  At the very least, we have that if $(\delta, d)$ admits a constant spanning local Fourier measurement system, then $(\delta, d + 1)$ will not.  Since $d$ is intended to represent the number of pixels in the sensor array, this is a prohibitively specific requirement to be made of the discretization of the phase retrieval problem, so we emphasize that the result of \cref{prop:constant_gamma} is of strictly mathematical interest.


\subsection{Remarks on Spanning Family Results}
\label{sec:span_fam_remarks}

Having proven \cref{prop:span_fam_cond,cor:gam_fam_span}, we remark that the condition for a local Fourier measurement system to be a spanning family is generic, in the sense that it fails to hold only on a subset of $\R^d$ with Lebesgue measure zero, except possibly when $\delta > d / 2$.  We consider that, whenever $m \neq d / 2$, the set of $\gamma \in \R^d$ giving at least one zero in $F_d^*(\gamma \circ S^{-m} \gamma)$ is a finite union of zero sets of non-trivial quadratic polynomials and hence a set of zero measure.  %% Before we prove this, we also remark that this exception is quite pathological: since our intention is to have $\delta \ll d$, having $d / 2 = m < \delta$ will rarely be an impediment, since, in this case, we would have $T_\delta(\H^d) = \H^d$, against the intuition of having ``local'' measurements.  Nonetheless, in the case that we \emph{do} want to have $\Span \Lc_\gamma = \H^d$, then taking $\delta > d / 2 + 1$ gives some space for the condition $2|J_k| - \one_{0 \in J_k} \ge D$, and we again have that generic $\gamma$ will produce spanning families.
Indeed, when $m \neq d / 2$, we have that \[F_d^*((e_1 + e_{m + 1}) \circ S^m(e_1 + e_{m + 1}))_k = f_k^{d*} e_{m + 1} = \omega^{-m(k-1)},\] so $\gamma \mapsto F_d^*(\gamma \circ S^m \gamma)_k$ is a non-zero, homogeneous quadratic polynomial and therefore has a zero locus of measure zero.

However, when $d = 2 m$, then $\gamma \circ S^m \gamma$ is periodic with period $m$ and $F_d^*(\gamma \circ S^m \gamma)_{2i} = 0$ for $i \in [m]_0$.  In particular, if $\delta \ge d / 2$, then $D = d$ and, with reference to the notation of \cref{thm:meas_cond}, $M_{2i}$ will be singular for all $i$, making $A$ itself singular!  

Nonetheless, there is a way to have $\Lc_\gamma = \H^d$ when $d / 2$; the reason this does not contradict \cref{thm:meas_cond,prop:gam_fam_cond} is that, as was observed in the proof of \cref{cor:gam_fam_span}, these consider $A$ and $\Ac$ as operators on the complex vector space $\Cdxd$, and restricting to $\H^d$ as a vector space over $\R$ changes the behavior of the system in a way that must be accounted for.  

In this case, the behavior is changed in our favor, and while \cref{prop:gam_fam_cond} prohibits us from constructing a local Fourier measurement system that spans $\Cdxd$ when $2 \mid d$, \cref{prop:gam_span_herm} shows that we may have $\Span_\R \{S^{\ell} m_j m_j^* S^{-\ell}\} = \H^d$.  Necessary and sufficient conditions to procure a spanning family of $T_\delta(\H^d)$ in general, and $\H^d$ in particular when $\delta \ge \frac{d + 1}{2}$ are stated in \cref{prop:gam_span_herm}.  However, since the proof of this result is tedious, and since it represents an edge case that is largely uninteresting to the intended application of this dissertation, we relegate its proof to \cref{app:gam_span_herm}.

\begin{proposition}  \label{prop:span_fam} \label{prop:gam_span_herm}
  Suppose $\{m_j\}_{j \in [D]} \subset \Cd$ is a local Fourier measurement system of support $\delta$ with mask $\gamma \in \Rd$ and modulation index $K \ge D = \min (2 \delta - 1, d)$.  Then $\{m_j\}_{j \in [D]}$ is a spanning family, that is, \[\Span_\R\{S^{\ell} m_j m_j^* S^{-\ell}\}_{(\ell, j) \in [d]_0 \times [D]} = T_\delta(\H^d)\] if and only if each of the sets $J_k := \{m \in [\delta]_0 : (F_d^* (\gamma \circ S^{-m} \gamma))_k \neq 0\}$, for all $k \in [d]$, satisfy \begin{equation} %% \left\{\begin{array}{r@{,\quad}l} 2 |J_k| - 1 \ge D & 0 \in J_k \\ 2 |J_k| \ge D & \text{otherwise}\end{array}\right.
    2 \abs{J_k} - \one_{0 \in J_k} \ge D
    .\label{eq:J_k}\end{equation}  Equivalently, we may require \begin{equation} \#\mathrm{nnz}\left(f_k^{d*} D_\gamma \rowmatfun{S^@ \gamma}{1 - \delta}{\delta - 1}\right) \ge D, \ \text{for all} \ k \in [d].\label{eq:J_k_restate}\end{equation}
\end{proposition}

We remark that the maximum value of $2 \abs{J_k} - \one_{0 \in J_k}$ is $2 \delta - 1$, so when $D = 2 \delta - 1 \le d$, the condition of \eqref{eq:J_k} is equivalent to \cref{cor:gam_fam_span}: $F_d^* (\gamma \circ S^{-m} \gamma)$ must contain no zeros for all $m \in [\delta]_0$.  However, when $D = d \le 2 \delta - 1$, we notice that \eqref{eq:J_k} can be satisfied, even when some terms $F_d^* (\gamma \circ S^{-m}))_k$ are zero, as long as no particular Fourier component $f_k^d$ produces too many zeros as $m$ sweeps through $[\delta]_0$; this reforumlation of the condition is made somewhat more explicit in \eqref{eq:J_k_restate}.
