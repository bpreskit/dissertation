In this section, we use the results of \cref{sec:con_number} to explicitly state the inverse of the measurement operator $\Ac$, as well as the computational complexity of calculating its inverse.  Additionally, we calculate the variance in \emph{each entry} of $\Ac^{-1}(y)$ when $y = \Ac(T_\delta(x x^*)) + \eta$ is produced under an i.i.d.~Gaussian noise model, which will be useful to us in later analysis.

\subsection{Explicit inverse of $\Ac$}

We begin by fixing a local Fourier measurement system $\{m_j\}_{j = 1}^{2 \delta - 1}$ with support $\delta \le \frac{d + 1}{2}$, mask $\gamma$, and modulation index $K = D = 2 \delta - 1$.  We take $\Ac$ to be the associated measurement operator and $A$ its canonical matrix representation as in \cref{eq:meas_op,eq:meas_mat}.  We then remark from \eqref{eq:interleaved_meas} and \cref{lem:circ_diag} that \[A = P^{(D, d)} (F_d \otimes I_D) \diag(M_\ell)_{\ell = 1}^d (F_d \otimes I_D)^* P^{(d, D)}.\]  Recalling from \eqref{eq:block_diag_components} that \[M_\ell = \sqrt{D} \diag\left(f_\ell^{d*} \rowmat{g}{1 - \delta}{\delta - 1} \right)\tF_D,\] and defining $Z \in \C^{d \times D}$ by \begin{equation} Z_{\ell m} = \sqrt{D} f_\ell^{d*} g_{m - \delta} \label{eq:four_shift_mat}\end{equation} and setting $z_\ell = Z e_\ell$, we have \begin{equation} M_\ell = D_{z_\ell} \tF_D \ \text{and} \ \diag(M_\ell)_{\ell = 1}^d = \diag(\vec(Z)) (I_d \otimes \tF_D), \label{eq:M_rearr} \end{equation} which gives \[A = P^{(D, d)} (F_d \otimes I_D) \diag(\vec(Z)) (I_d \otimes \tF_D) (F_d \otimes I_D)^* P^{(d, D)}.\]  This immediately produces the inverse of $A$, which we state in \cref{prop:A_inv_gam}.

\begin{proposition}
  Let $A \in \C^{d \times 2 \delta - 1}$ be the canonical representation of the measurement operator $\Ac$ associated with a local Fourier measurement system $\{m_j\}_{j = 1}^d$ of support $\delta \le \frac{d + 1}{2}$ with mask $\gamma \in \R^d$.  Defining $Z$ as in \eqref{eq:four_shift_mat}, we have
  \begin{equation} A^{-1} = P^{(D, d)} (F_d \otimes I_D) (I_d \otimes \tF_D^*) (\diag(\vec(Z)))^{-1}  (F_d \otimes I_D)^* P^{(d, D)}. \label{eq:A_inv_gam} \end{equation}
  \label{prop:A_inv_gam}
\end{proposition}

This formulation makes it straightforward to deduce the computational complexity, as previously stated in \cref{sec:RuntimeAlg1}.  Namely, each permutation $P^{(D, d)}$ requires $\bigO(d D)$ operations to run on a vector.  Since, by \eqref{eq:interkron_swap} of \cref{lem:interkron}, we have that $F_d \kron I_D = P^{(d, D)}(I_D \kron F_d) P^{(D, d)}$, and considering also that $I_D \kron F_d$ comprises $D$ Fourier transforms of dimension $d$, multiplication by $F_d \kron I_D$ costs $\bigO(d D + D d \log d) = \bigO(d D \log d)$ operations.  Since $\tF_D = F_D S^{-\delta}$, multiplying by $I_d \kron \tF_D^*$ takes $\bigO(d D \log D)$ operations.  Finally, multiplying by $\diag(\vec(Z))^{-1}$ trivially has a cost of $\bigO(d D)$.  Putting all these considerations together, and recalling that $d \ge D = 2 \delta - 1$, the cost of inverting $A$ comes out to \[\bigO(d D) + \bigO(d D \log d) + \bigO(d D \log D) + \bigO(d D) + \bigO(d D \log d) + \bigO(d D) = \bigO(d D \log d),\] or $\bigO(\delta\, d \log d)$, as concluded in \cref{sec:RuntimeAlg1} and \cite{IVW2015_FastPhase}.

\subsection{Distribution of variance}
We remark that \eqref{eq:interkron_mat} of \cref{lem:interkron} gives that \[P^{(D, d)} (F_d \kron I_D) = \rowmat{I_D \kron f^d}{1}{d},\] so $A$ may be transformed by \eqref{eq:kron_diag} to give \[A = \rowmatfun{M_@ \kron f_@^d}{1}{d} \colmat{I_D \kron f^d}{1}{d} = \sum_{j = 1}^d M_j \kron f_j^d f_j^{d*}.\]  Setting $X = \rowmat{\chi}{1 - \delta}{\delta - 1} \in \C^{d \times 2 \delta - 1}$, \cref{lem:kronvec} gives us \[A \colmat{\chi}{1 - \delta}{\delta - 1} = \vec\left(\sum_{j = 1}^d f_j^d f_j^{d*} X M_j^T\right).\]  Recalling \eqref{eq:M_rearr} along with $\tF^{-T} = (\tF^T)^* = \overline{\tF}$, we have \[f_j^{d*} \mat_{(d, D)}\left(A \colmat{\chi}{1 - \delta}{\delta - 1}\right) \overline{\tF}_D = f_j^{d*} X D_{z_\ell}.\]


