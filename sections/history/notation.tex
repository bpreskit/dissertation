In this section, we gather some of the notation that is used throughout the dissertation.  \Cref{tab:notation} displays some of the most commonly used objects.  We remark that, in this table and throughout this work, indices of a vector $x \in \Cn$ or matrix $A \in \Cmxn$ are always taken modulo the appropriate dimension.  For example, $x_{n + 1} := x_1$ and $A_{00} = A_{mn}$.

We also introduce a few more operators that don't fit well into \cref{tab:notation}.  Given matrices $V_j \in \C^{m_j \times n_j}$ for $j \in [n]$, we write \[\diag(V_j)_{j = 1}^n = \begin{bmatrix} V_1 & & \\ & \ddots & \\ & & V_n \end{bmatrix} \in \C^{\sum m_j \times \sum n_j}.\]  To conveniently switch between matrices and vectors of different sizes, $\mathcal{R}_d : \bigcup_{k = 1}^\infty \C^k \to \C^d$ is a resize mapping, where for $v \in \C^k$ and $i \in [d],$ $$\mathcal{R}_d(v)_i = \left\{\begin{array}{r@{,\quad}l} v_i & i \le k \\ 0 & \text{otherwise} \end{array}\right. \text{for}\ i \in [d].$$  Similarly, $\mathcal{R}_{m \times n} : \bigcup_{k_1, k_2}^\infty \C^{k_1 \times k_2} \to \C^{m \times n}$ truncates or zero-pads matrices to size $m \times n$.  

We let $\vec : \bigcup_{m, n \in \N} \C^{m \times n} \to \bigcup_{k \in \N} \C^k$ be the columnwise vectorization operator, such that for $A \in \C^{m \times n}, \vec(A) \in \C^{m n}$ and  $\vec(A)_{(j - 1) m + i} = A_{i j}$ for $i, j \in [m] \times [n]$.  To invert $\vec$, we use $\mat_{(m, n)} : \C^{m n} \to \C^{m \times n},$ such that $\mat_{(m, n)}(v)_{ij} = v_{(j - 1) m + i}$.  By a slight overloading of notation, by $\vec(a_j)_{j = 1}^d$ or $(a_j)_{j=1}^d$, we intend the vector in $v \in \R^d$ satisfying $v_j = a_j$ -- this may come in handy to specify something such as $\begin{bmatrix} 1 & 2 & 4 & 8 \end{bmatrix}^T = \vec(2^{j - 1})_{j = 1}^4$.

{\pagebreak\newcommand{\env}[1]{\texttt{#1}}\renewcommand{\thefootnote}{\fnsymbol{footnote}}
  \centering \renewcommand{\arraystretch}{1.5}
\LTXtable{1.04\textwidth}{notation_table}}

{\renewcommand{\thefootnote}{\fnsymbol{footnote}}\footnotetext[3]{We omit the subscript (or superscript) when it is obvious from context.}}
