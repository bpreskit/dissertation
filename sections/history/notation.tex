\begin{itemize}
\item For $k \in \N, n \in \Z, [k]_n = \{n, n + 1, \ldots, n + k - 1\}$ and $[k] = [k]_1$.
\item Given $m, n, k \in \N, m \mod_k n$ is the unique element of $[n]_k$ such that $n \mid (m - k)$.  Without the subscript, we specify $m \mod n := m \mod_0 n \in \{0, \ldots, n - 1\}$ to be the usual modulo operator.
\item Indices of matrices in $\C^{d \times d}$ and vectors in $\C^d$ are always taken modulo $d$.
\item $S_d \in \R^{d \times d}$ is the $d \times d$ shift operator, such that $(S_d x)_i = x_{i - 1}$.  Typically we imply the subscript by context, writing $S$.
\item $R_d \in \R^{d \times d}$ is the operator that reverses a vector's entries, leaving the first entry fixed.  Namely, $(R_d x)_i = x_{2 - i}$.  Typically, we imply the dimension $d$ by context and write only $R$.
\item For $i, n \in \N, e_i^n \in \R^n$ is the $i^{\text{th}}$ column of the $n \times n$ identity matrix.  When context permits, $n$ is implied and we write $e_i$.  In particular, whenever $e_i$ is used in a matrix multiplication, $n$ is taken to be appropriate so that the multiplication is valid; for $A \in \C^{m \times n},$ the ``$e_i$'' in $A e_i$ is assumed to be $e_i^n$.
\item Given $x \in \C^d$ and $k \in [d], \circop_k(x) \in \C^{d \times k}$ denotes the first $k$ columns of the circulant matrix whose first column is $x$, defined by $\circop_k(x) e_i = S^{i - 1} x$ for $i \in [k]$.  Alternatively, \[\circop_k(x) = \begin{bmatrix} x & S x & \cdots & S^{k-1} x \end{bmatrix}.\]  When the subscript is omitted, $\circop(x) = \circop_d(x)$.
  \item Given any $A \subset B,$ we define the indicator function $\chi_A : B \to \{0, 1\}$ by \[\chi_A(x) = \begin{piecewise} 1 & x \in A \\ 0 & \text{otherwise} \end{piecewise}.\]
  \item $\one_d \in \C^d$ is the vector of all $1$'s.  When context makes the size clear, we write $\one$.  Given a set $A \subset [d], \one_A^d \in \C^d$ has $(\one_A)_i = \chi_A(i)$.
\item $\omega_d := \ee^{\frac{2 \pi \ii}{d}}$ is the $d^{\text{th}}$ root of unity.  When context permits, $d$ is implied and we use just $\omega$.
\item For $k \in \Z, F_k \in \C^{k \times k}$ is the $k \times k$ unitary Fourier matrix with $(F_k)_{ij} = \frac{1}{\sqrt{k}} \omega_k^{(i-1)(j-1)}$.
  \item For $m, n \in \N, f_n^m = F_m e_n$ is the $n^{\text{th}}$ column of the $m \times m$ unitary Fourier matrix, where $e_n \in \R^m$ has its index taken modulo $m$.
\item Given $x, y \in \C^d, x \circ y$ denotes the Hadamard/elementwise product of $x$ and $y$; specifically $(x \circ y)_i = x_i y_i$.
\item Given $A \in \C^{d \times d}, \diag(A, m) \in \C^d$ denotes the $m^{\text{th}}$ circulant off-diagonal of $A$.  That is, $\diag(A, m)_i = A_{i, i + m}$.
\item Given $x \in \C^d, \diag(x) \in \C^{d \times d}$ is the diagonal matrix whose diagonal entries are the entries of $x$.  Namely, $\diag(x) e_i = x_i e_i$.  We may also write this as $\diag(x_j)_{j = 1}^d$.  When the intention is clear from context, we may write $D_x := \diag(x)$.  Given matrices $V_j \in \C^{m_j \times n_j}$ for $j \in [n]$, we write \[\diag(V_j)_{j = 1}^n = \begin{bmatrix} V_1 & & \\ & \ddots & \\ & & V_n \end{bmatrix} \in \C^{\sum m_j \times \sum n_j}.\]
  \item $\H^d$ is the set of Hermitian matrices in $\C^{d \times d}$, to be viewed as a $d^2$-dimensional vector space over $\R$.  $\H^d_+ \subset \H^d$ is the complex Hermitian semi-definite cone, where $A \in \H^d_+$ if $A \succeq 0$.
  \item $\mathcal{R}_d : \bigcup_{k = 1}^\infty \C^k \to \C^d$ is a resize mapping, where for $v \in \C^k$ and $i \in [d],$ $$\mathcal{R}_d(v)_i = \left\{\begin{array}{r@{,\quad}l} v_i & i \le k \\ 0 & \text{otherwise} \end{array}\right. \text{for}\ i \in [d].$$  Similarly, $\mathcal{R}_{m \times n} : \bigcup_{k_1, k_2}^\infty \C^{k_1 \times k_2} \to \C^{m \times n}$ truncates or zero-pads matrices to size $m \times n.$
  \item Given $k, d \in \N$, we define the operator $T^d_k : \C^{d \times d} \to \C^{d \times d}$ by \[T_k^d(A)_{ij} = \left\{\begin{array}{r@{,\qquad}l} A_{ij} & |i - j| \mod d < k \\ 0 & \text{otherwise}. \end{array}\right.\]  Note that $T^d_k$ is simply the orthogonal projection operator onto its range $T_k^d(\C^{d \times d})$.  We use $T^d_k$ interchangeably to refer to both the operator and its range, and almost always exclude the dimension $d$ by context, writing $T_k$.
  \item $\mathcal{S}^n = \R^{n \times n} \cap \H^n$ is the set of real, symmetric $n \times n$ matrices.
    
    
\end{itemize}
