The history of modern algorithmic phase retrieval begins in the 1970's with \cite{gerchberg1972practical} by \citeauthor*{gerchberg1972practical}, where the measurement data corresponded to knowing the magnitude of both the image $\x_0$ and its Fourier transform.  This result was famously expanded upon by Fienup \cite{fienup1978reconstruction} later that decade, one significant improvement being that only the magnitude of the Fourier transform of $\x_0$ must be known in the case of a signal $\x_0$ belonging to some fixed convex set $\mathcal{C}$ (typically, $\mathcal{C}$ is the set of non-negative, real-valued signals restricted to a known domain).  Though these techniques work well in practice and have been popular for decades, they are notoriously difficult to analyze.  These are iterative methods that work by improving an initial guess until they stagnate.  In 2015, Marchesini et al.~proved that alternating projection schemes using generic measurements are guaranteed to converge to the correct solution {\em if provided with a sufficiently accurate initial guess} and algorithms for ptychography were explored in particular \cite{marchesini2015alternating}.  Waldspurger's recent paper proves a spectral initialization that reaches this basin of attraction with high probability using a simple Gaussian suite of measurements \cite{waldspurger2018gerchsax}.  The application of alternating minimizations to sparse phase retrieval has received considerable attention, as well \cite{jagatap2017fast,eldar2017fienup}, although results in \cite{IVW2017_easy} suggest that virtually any phase retrieval method may easily be composed with compressed sensing techniques to target sparse signals.  However, despite this impressive body of work, no global recovery guarantees currently exist for alternating projection techniques using local measurements (i.e., finding a sufficiently accurate initial guess is not generally easy).

Other authors have taken to proving probabilistic recovery guarantees when provided with globally supported Gaussian measurements.  Methods for which such results exist vary in their approach, and include convex relaxations \cite{candes2014solving,candes2012phaselift,hassibi2018phasemax,waldspurger2015phasecut}, gradient descent strategies \cite{candes2015wtf,gangwang2017quadratic}, graph-theoretic \cite{alexeev2014phase,salanevich2015polarization} and frame-based approaches \cite{balan2009painless, bodmann2013stable,bodmann2017frames}, and variants on conventional alternating minimization ideas \cite{netrapalli2013phase,waldspurger2018gerchsax}.  The approach of non-convex optimization by gradient descent, named \emph{Wirtinger Flow} in its first application to phase retrieval \cite{candes2015wtf}, has enjoyed recent success in a variety of phase retrieval applications \cite{soltanolkotabi2018multiplexed,soltanolkotabi2018accelwtf} as well as blind deconvolution \cite{strohmer2017wtf_deconv1,strohmer2017wtf_deconv2} and low-rank matrix recovery \cite{soltanolkotabi2016procrustes}.

Several recovery algorithms achieve theoretical recovery guarantees while using at most $D = \bigO(d \log^4 d)$ masked Fourier coded diffraction pattern measurements, including both {\em PhaseLift} \cite{Candes2014WF,gross2015improved}, and {\em Wirtinger Flow} \cite{candes2015wtf}.  However, until recently, there has been no construction of these measurements that were not randomized, and -- as of this dissertation -- the theory has not studied locally supported measurements of the type considered here.  Kueng, Gross, and others have tried to derandomize the constructions for \emph{PhaseLift} in particular by drawing the measurements from certain matrix groups \cite{kueng2015spherical,kueng2016clifford}, but the first completely deterministic construction of a measurement system with provable, global recovery via PhaseLift appeared this year with \cite{kech2018explicit}, suggesting exciting new directions of research.

Among the first treatments of local measurements are \cite{bendory2017stft,eldar2014sparse,jaganathan2016stft}, in which it is shown that STFT (short-time Fourier transform \cite{allen1977stft,portnoff1979stft}) measurements with specific properties can allow (sparse) phase retrieval in the noiseless setting, and several recovery methods have been proposed \cite{bendory2018stft,guo2018stft}.  Similarly, the phase retrieval approach from \cite{alexeev2014phase} was extended to STFT measurements in \cite{salanevich2015polarization} in order to produce recovery guarantees in the noiseless setting.  More recently, randomized robustness guarantees were developed for time-frequency measurements in \cite{salanevich2016polarization}.  However, no {\it deterministic} robust recovery guarantees have been proven in the noisy setting for any of these approaches.  Furthermore, none of the algorithms developed in these papers are demonstrated to be empirically competitive with standard alternating projection techniques for large signals when utilizing windowed Fourier and/or correlation-based measurements.  In \cite{IVW2015_FastPhase}, the authors first propose the measurement scheme developed in \cref{ch:base_model} and throughout this dissertation, and prove the first deterministic robustness results in the recent literature, although these results treat a ``greedy'' recovery algorithm, somewhat different from the one developed herein.  Indeed, some of the major contributions of the current work can be thought of as abstracting the components of the recovery strategy in \cite{IVW2015_FastPhase}, which then becomes equivalent to special cases of the algorithms studied here (see \cref{sec:ang_sync_tree,sec:blocky_block}).

In the midst of such an active and diverse field of research, the major contributions of the setup studied in this dissertation are that it takes into account the local measurements that match the models for key applications such as ptychography.  In this setting, we have uniquely produced a provably fast and stable recovery algorithm for a well-understood, deterministically stated class of measurement systems.
