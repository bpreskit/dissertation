\subsection{Historical Preliminaries}

The history of phase retrieval cannot be told without making mention of x-ray crystallography, the field that first brought scientific interest to this problem and by many metrics its most decorated and fruitful application.  In x-ray crystallography, the goal is to gain an image of the positions of atoms within a molecule by illuminating a crystallized sample with x-rays.  The molecular structure is deduced from the pattern of the radiation diffracted by the sample.  A rough diagram of this setup is shown in figure \ref{fig:xray_cryst}.

\begin{figure}
  \begin{center}
    %\includegraphics{figs/xray_cryst}
  \end{center}
  \caption{Experimental setup for x-ray crystallography}
  \label{fig:xray_cryst}
\end{figure}

This seemingly simple technique has been indispensable for the study of chemistry, biology, and physics, having been used to confirm or identify the arrangements of atoms in a wide variety of important compounds.  Over a dozen discoveries made through x-ray crystallography -- or made in developing the technique -- have been recognized by Nobel prizes in Physics, Chemistry, and Medicine or Physiology.  It has been used to produce accurate molecular models of a number of drugs (e.g., \cite{cell2001antibios, rasmussen2007adrenergic, schindler2000kinase}), including penicillin in Dorothy Crowfoot Hodgkin's Nobel prize-winning work in 1963 \cite{hodgkin1963penicillin}; human biological compounds, including innumerable proteins \cite{1993isomerase, , } and human DNA, for whose analysis in 1953 James Watson, Maurice Wilkins, and Francis Crick were awarded the Nobel prize in 1962 \cite{watson1962nobel_lecture}; and inorganic compounds, including lithium-ion batteries \cite{andrej2018battery, hausbrand2015battery} and carbon nanostructures such as fullerenes \cite{kroto1985fullerene, lamb1990carbon}, whose analysis earned the Nobel prize in 1996 \cite{galli2014nobel}.

The discovery of this technique, the development of its implementation, and many individual uses of it have each made their mark as substantial contributions to scientific knowledge, and have each been recognized with Nobel prizes.  Indeed, 

\subsection{Mathematical Preliminaries}
