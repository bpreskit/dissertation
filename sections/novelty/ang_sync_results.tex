\subsection{Introduction and existing results}
In this section, we consider the problem of angular synchronization, which appears as a subproblem in many approaches to phase retrieval, including ours.  Angular synchronization is the problem of recovering a vector of complex units $x_i \in \Sbb^1, i \in [n]$, or $x \in (\Sbb^1)^n$ from estimates of their relative phases \[\tX_{ij} = x_j^* x_i \eta_{ij}, (i, j) \in E\] where $\eta_{ij} \in \Sbb^1$ are the ``noise terms'' and $E \subset [n]^2$ is a list of the pairs of indices for which we have such estimates.  This immediately suggests associating a graph to this problem; namely, taking the vertex set to be $V = [n]$, we set $G := (V, E)$.  We also remark that this problem invites the same global phase ambiguity as phase retrieval: indeed $(\ee^{\ii \theta} x_i)^* (\ee^{\ii \theta} x_j) \eta_{ij} = x_i^* x_j \eta_{ij}$ for any $\theta \in [0, 2\pi)$.  Obviously, then, our goal is to get an estimate $\tilde{x}$ that is guaranteeably close to the ground truth $x$ in some chosen metric, say our usual \[d(x_1, x_2) = \min_{\theta \in \R} || x_1 - \ee^{\ii \theta} x_2 ||_2,\] and to acquire this estimate with the least computational cost.  Naturally, $x$ is not known, so we instead attempt to minimize some cost function corresponding to how well our estimate explains the measurement data, usually taking a form similar to the frustration function stated in \eqref{eq:frustration}.  Often, we more simply take only the numerator of this expression, \[\sum_{(i, j) \in E} w_{ij} |x_i - X_{ij} x_j|^2 = x^* (D - W \circ X) x, \label{eq:ang_sync_cost}\] where $D$ and $W$ are the degree and weight matrices specified in \eqref{eq:degandweight}.  The approach to angular synchronization that we consider in this dissertation, and the approach most studied in the literature, then is to attempt the non-convex optimization problem \begin{equation} \begin{array}{cc} \min\limits_{u \in \C^n} & u^* (D - W \circ X) u \\ \text{s.t.} & |u_i| = 1 \end{array} \label{eq:ang_sync }\end{equation}

The modern study of eigenvector-based methods for angular synchronization appears to have begun with a 2011 paper by Amit Singer \cite{singer2011ang_sync}, in which he proposed two ways of solving this problem which remain as the basis of the state of the art.  The first is almost identical to the eigenvector-based approach that we have used in the algorithm proposed in chapter \ref{ch:our_model}, and the second is a semidefinite relaxation of the same.  Namely, instead of numerically solving \[ \begin{array}{cc} \min\limits_{u \in \C^n} & u^* L_1 u \\ \text{s.t.} & ||u||_2 \le \sqrt{n} \end{array} \] to find the smallest eigenvector of $L_1$ and rounding to a vector of units by taking $\tx = \sgn(u)$, this method solves \begin{equation}\begin{array}{cccl} \min\limits_{Z \in \H^d} & \Tr(L_1 Z) \\ \text{s.t.} & Z & \succeq & 0 \\ & Z_{ii} & = & 1 \end{array}, \label{eq:angsyncSDPsinger}\end{equation} though he uses the unweighted adjacency matrix $X$ in lieu of $L_1$.  In this paper, he studies the problem under a noise model where the disturbances $\eta_{ij}$ are distributed according to \[\eta_{ij} = \left\{\begin{array}{r@{,\quad}l} 1 & \text{with probability}\ p \\ \mathrm{Unif}(\Sbb^1) & \text{with probability}\ 1 - p\end{array}\right.,\] such that the measurement $\tX_{ij}$ is exact with probability $p$ and is completely meaningless -- being drawn from the uniform distribution on $\Sbb^1$ -- with probability $1 - p$.  He proves the robustness of this method in the sense that there is a probability $p_c$, dependent on the spectral gap and size of the graph $G$, for which parameter values $p > p_c$ guarantee ``better than random'' approximations of $x$ to be recovered with high probability.  Moreover, he shows that, experimentally, both of these recovery algorithms work acceptably, if not extremely, well, with little to be gained by transferring from the eigenvector problem to the computationally more expensive semidefinite program.

The literature on angular synchronization since this paper has largely consisted of analyzing generalizations and variations of these methods.  One major generalization has been to apply these methods to larger classes of group synchronization problems such as synchronization over the orthogonal groups $O(d)$ or the special Euclidean groups $SE(d)$ \cite{Cheeger,briales2017cartan_sync,bandeira2016se_sync}.  Naturally, much of the interest in this subject has been the treatment of synchronization over $SO(3)$ and $SE(3)$, as these correspond to pose estimation problems fundamental to computer vision, as in \cite{govindu2006motion_avg, fischler1981ransac, olsson2017rot_avg, enqvist2011nonsequential}.  Significant results giving guarantees of robustness, as well as proofs that these relaxations are solved \emph{exactly} in certain cases may be found in \cite{alexeev2014phase, bandeira2016tightness, olsson2017rot_avg, bandeira2016se_sync}.

\subsection{Improvements to theorem \ref{thm:SpecGraphPertBound}}

We have already presented one result on this problem in theorem \ref{thm:SpecGraphPertBound} of \S\ref{sec:Perturb}, which drew largely on \cite{alexeev2014phase}.  % and which we restate here.
%% \begin{thm*}[Theorem \ref{thm:SpecGraphPertBound}]
%% Suppose that $G = (V=[d], E)$ is an undirected, connected, and unweighted graph (so that $W_{ij} = \chi_{E(i, j)}$) with spectral gap $\tau > 0$.  Let ${\bf u} \in \mathbbm{C}^d$ be an eigenvector of $L_1$ from \eqref{equ:ConnectLaplace} corresponding to its smallest eigenvalue, and let \[\widetilde{\bf x} = \sgn({\bf u}) \ \text{and} \ \tx_0 = \sgn(\x_0).\] Then for some universal constant $C \in \mathbbm{R}^+$, \[\min_{\theta \in [0, 2\pi]} ||\tx - \ee^{\ii \theta} \tx_0||_2 \le C \bigfrac{\|\tX - \widetilde{\X}_0\|_F}{\tau \cdot \sqrt{\min_{i \in V}(\deg(i))}},\] where $\tX$ and $\widetilde{\X}_0$ are defined as per \eqref{equ:DeftildeXviaGcomp} and \eqref{equ:DeftildeXviaGpure}, respectively. %%If, in addition, $G$ is $k$-regular, then we have \[\min_{\theta \in [0, 2\pi]} ||x - \ee^{\ii \theta}u||_2 \le C \bigfrac{||A_0 - A_1||_F}{k^{1/2}\tau}.\]
%% \end{thm*}
We remark that this theorem leaves something to be desired in that the graph $G$ is not permitted to be weighted, which restricts us from applying some knowledge that we may have about the problem.  

For example, suppose our relative phase measurements $\{X_{ij}\}_{(i,j) \in E}$ are disturbed by noise drawn from a fixed, phase-invariant probability distribution, say \[X_{ij} = \sgn(x_i^*x_j + \eta_{ij}), \eta_{ij} = a + bi \ \text{with} \ a, b \iid N(0, \sigma^2),\] then we will have more confidence in the relative phases represented by the larger magnitude entries of $X = \mathcal{A}^{-1}(\y)$.  It would be intuitive to use this knowledge to privilege some edges of the graph over others in the frustration function \eqref{eq:frustration} that we are trying to minimize, say by using $w_{ij} = |X_{ij}|$.  Unfortunately, theorem \ref{thm:SpecGraphPertBound} assumes an unweighted graph and its proof technique does not readily admit an adjustment towards weighted edges.  Therefore, we take a rather distinct approach, drawing upon work by Afonso Bandeira, et.~al in \cite{bandeira2016se_sync} and \cite{bandeira2016tightness}, which establishes solvability and robustness for an SDP relaxation similar to that in \eqref{eq:angsyncSDPsinger}.  Specifically, we rewrite 
  Of greatest interest to us is proposition 2 in \cite{bandeira2016se_sync}, which states

\begin{proposition}[Proposition 2 in \cite{bandeira2016se_sync}]
  
\end{proposition}

\cite{calafiore2016complex_pgo} uses similar methods to analyze $SE(2)$ synchronization, and in fact exchanges the rotations in $SO(2)$ for complex units, but does 

However, neither of these results may be directly applied, since \cite{bandeira2016tightness} is restricted to the case when $G = K_n$ is the complete graph and the weights $W = \one \one^* - I_n$ are constant.  The result stated in \cite{bandeira2016se_sync}, on the other hand, is far more general, applying to synchronization over $SE(n)$, but narrowing to angular synchronization (which is a special case of this problem) allows a considerable improvement over the error bound stated in the paper.  This sharpened result, as we shall see, leads to better guarantees than any we have yet proven.

To begin our analysis, we gather our notation: $G = (V = [n], E)$ is an arbitrary graph, with a weighted adjacency matrix $W = [w_{ij}] \in \sym^n$ satisfying $w_{ij} \ge 0$ and $w_{ij} \neq 0$ only if $(i, j) \in E$.  $D = \diag(W \one)$ is the degree matrix.  $\ux \in (\Sbb^1)^n$ is the ground truth vector, which we attempt to recover, and $X, \uX \in \H^n$ are our noisy and ground truth edge data matrices, satisfying \[X_{ij} = \begin{piecewise} \eta_{ij} x_i x_j^* & (i, j) \in E \\ 0 & \ow \end{piecewise} \quad \text{and} \quad \uX_{ij} = \begin{piecewise} x_i x_j^* & (i, j) \in E \\ 0 & \ow \end{piecewise},\] where $\eta_{ij} = \eta_{ji}^* \in \Sbb^1$ for each $(i, j) \in E$.  We then define $L = D - W \circ X$ and $\uL = D - W \circ \uX$.

Towards formulating the SDP relaxation, we consider the transformation $\mfr : \C \to \R^{2 \times 2}$ %and $\ulmfr : \C \to \R^2$ defined by \[\mfr(a + bi) = \begin{bmatrix} a & -b \\ b & a \end{bmatrix}, \ \ulmfr(a + bi) = \begin{bmatrix} a \\ b \end{bmatrix}.\]
defined by \[\mfr(a + bi) = \begin{bmatrix} a & -b \\ b & a \end{bmatrix}.\]
It is well-known that $\mfr$ is the canonical isomorphism from $\C$ into $\R^{2 \times 2}$, and indeed if we extend it to matrices by taking, for $A \in \C^{m \times n}$, $\mfr(A) \in \R^{2m \times 2n}$ to be a block matrix with $\mfr(A)_{ij} = \mfr(A_{ij})$, then it remains an isomorphism on $\C^{m \times n}$, and indeed it preserves the eigenvalues and eigenvectors of Hermitian matrices (see, e.g.~\cite[p.~101]{wedderburn1934matrices}).  In particular, a hermitian matrix $A \in \H^n$ is semi-definite if and only if $\mfr(A)$ is semi-definite; we notice that the multiplicities of its eigenvalues are all doubled, as $A v = \lambda v$ implies $\mfr(A) \mfr(v) = \lambda \mfr(v)$, giving that the two columns of $\mfr(v)$ are each eigenvectors of $\mfr(A)$ with eigenvalue $\lambda$.  This allows us to restate the SDP of \eqref{eq:angsyncSDPsinger} as a real-valued SDP in 
