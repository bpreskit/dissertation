\subsection{Introduction and existing results}
In this section, we consider the problem of angular synchronization, which appears as a subproblem in many approaches to phase retrieval, including ours.  Angular synchronization is the problem of recovering a vector of complex units $x_i \in \Sbb^1, i \in [n]$, or $x \in (\Sbb^1)^n$ from estimates of their relative phases \[\tX_{ij} = x_j^* x_i \eta_{ij}, (i, j) \in E\] where $\eta_{ij} \in \Sbb^1$ are the ``noise terms'' and $E \subset [n]^2$ is a list of the pairs of indices for which we have such estimates.  This immediately suggests associating a graph to this problem; namely, taking the vertex set to be $V = [n]$, we set $G := (V, E)$.  We also remark that this problem invites the same global phase ambiguity as phase retrieval: indeed $(\ee^{\ii \theta} x_i)^* (\ee^{\ii \theta} x_j) \eta_{ij} = x_i^* x_j \eta_{ij}$ for any $\theta \in [0, 2\pi)$.  Obviously, then, our goal is to get an estimate $\tilde{x}$ that is guaranteeably close to the ground truth $x$ in some chosen metric, say our usual \[d(x_1, x_2) = \min_{\theta \in \R} || x_1 - \ee^{\ii \theta} x_2 ||_2,\] and to acquire this estimate with the least computational cost.  Naturally, $x$ is not known, so we instead attempt to minimize some cost function corresponding to how well our estimate explains the measurement data, usually taking a form similar to the frustration function stated in \eqref{eq:frustration}.  Often, we more simply take only the numerator of this expression, \[\sum_{(i, j) \in E} w_{ij} |x_i - X_{ij} x_j|^2 = x^* (D - W \circ X) x, \label{eq:ang_sync_cost}\] where $D$ and $W$ are the degree and weight matrices specified in \eqref{eq:degandweight}.

The modern study of angular synchronization appears to have begun with a 2011 paper by Amit Singer \cite{singer2011ang_sync}, in which he proposed two ways of solving this problem which have mostly remained as the state of the art since.  The first is almost identical to the eigenvector-based approach that we have used in the algorithm proposed in chapter \ref{ch:our_model}, and the second is a semi-definite relaxation of the same.  Namely, instead of numerically solving \[\begin{array}{cc} \min\limits_{u \in \C^n} & u^* L_1 u \\ \text{s.t.} & ||u||_2 \le \sqrt{n} \end{array} \label{eq:angsyncSDPsinger}\] to find the smallest eigenvector of $L_1$ and rounding to a vector of units by taking $\tx = \sgn(u)$, this method solves \[\begin{array}{crcl} \min\limits_{Z \in \H^d} & \Tr(L_1 Z) \\ \text{s.t.} & Z & \succeq & 0 \\ & Z_{ii} & = & 1 \end{array},\] though he uses the unweighted adjacency matrix $X$ in lieu of $L_1$.  In this paper, he studies the problem under a noise model where the disturbances $\eta_{ij}$ are distributed according to \[\eta_{ij} = \left\{\begin{array}{r@{,\quad}l} 1 & \text{with probability}\ p \\ \mathop{Unif}(\Sbb^1) & \text{with probability}\ 1 - p\end{array}\right.,\] such that the measurement $\tX_{ij}$ is exact with probability $p$ and is completely meaningless -- being drawn from the uniform distribution on $\Sbb^1$ -- with probability $1 - p$.  He proves the robustness of this method in the sense that there is a probability $p_c$, dependent on the spectral gap and size of the graph $G$, 

The literature on angular synchronization since 2011 has largely consisted of analyses of these methods and of generalizations or variations of them.  One major generalization has been to apply these methods to larger classes of group synchronization problems such as synchronization over the orthogonal groups $O(d)$ or the special Euclidean groups $SE(d)$ \cite{Cheeger,briales2017cartan_sync,bandeira2016se_sync}.  Significant results giving guarantees of robustness, as well as proofs that these relaxations are solved \emph{exactly} in certain cases may be found in \cite{alexeev2014phase, bandeira2016tightness, olsson2017rot_avg, bandeira2016se_sync}.

\subsection{Improvements to theorem \ref{thm:SpecGraphPertBound}}

We have already presented one result on this problem in theorem \ref{thm:SpecGraphPertBound} of \S\ref{sec:Perturb}, which drew largely on \cite{alexeev2014phase}.  % and which we restate here.
%% \begin{thm*}[Theorem \ref{thm:SpecGraphPertBound}]
%% Suppose that $G = (V=[d], E)$ is an undirected, connected, and unweighted graph (so that $W_{ij} = \chi_{E(i, j)}$) with spectral gap $\tau > 0$.  Let ${\bf u} \in \mathbbm{C}^d$ be an eigenvector of $L_1$ from \eqref{equ:ConnectLaplace} corresponding to its smallest eigenvalue, and let \[\widetilde{\bf x} = \sgn({\bf u}) \ \text{and} \ \tx_0 = \sgn(\x_0).\] Then for some universal constant $C \in \mathbbm{R}^+$, \[\min_{\theta \in [0, 2\pi]} ||\tx - \ee^{\ii \theta} \tx_0||_2 \le C \bigfrac{\|\tX - \widetilde{\X}_0\|_F}{\tau \cdot \sqrt{\min_{i \in V}(\deg(i))}},\] where $\tX$ and $\widetilde{\X}_0$ are defined as per \eqref{equ:DeftildeXviaGcomp} and \eqref{equ:DeftildeXviaGpure}, respectively. %%If, in addition, $G$ is $k$-regular, then we have \[\min_{\theta \in [0, 2\pi]} ||x - \ee^{\ii \theta}u||_2 \le C \bigfrac{||A_0 - A_1||_F}{k^{1/2}\tau}.\]
%% \end{thm*}
We remark that this theorem leaves something to be desired in that the graph $G$ is not permitted to be weighted, which restricts us from applying some knowledge that we may have about the problem.  

For example, suppose our relative phase measurements $\{\rho_{ij}\}_{(i,j) \in E}$ are disturbed by noise drawn from a fixed, phase-invariant probability distribution, say \[\rho_{ij} = \sgn(x_i^*x_j + \eta_{ij}) = \sgn(X_{ij}), \eta_{ij} \iid N(0, \sigma^2),\] then we will have more confidence in the relative phases represented by the larger magnitude entries of $X = \mathcal{A}^{-1}(\y)$.  It would be intuitive to use this knowledge to privilege some edges of the graph over others in the frustration function \eqref{eq:frustration} that we are trying to minimize, say by using $w_{ij} = |X_{ij}|$.  Unfortunately, the proof technique used for theorem \ref{thm:SpecGraphPertBound} doesn't readily admit weighted edges, so we will begin by taking a rather distinct approach, drawing upon work by Afonso Bandeira, et. al in \cite{bandeira2016se_sync} and \cite{bandeira2016tightness}, which establishes solvability and robustness for an SDP relaxation similar to that in \eqref{eq:angsyncSDPsinger}.  However, neither of these results may be directly applied, since \cite{bandeira2016tightness} is restricted to the case when $G = K_n$ is the complete graph and the weights $W = \one \one^* - I_n$ are constant.  The result stated in \cite{bandeira2016se_sync}, on the other hand, is quite more general, applying to synchronization over $SE(n)$, but narrowing to angular synchronization (which is a special case of this problem) allows a considerable improvement over the error bound stated in the paper.  

Towards formulating the SDP relaxation, we prove a few lemmas about the isomorphisms $\ulmfr : \C \to \R^{2 \times 2}$ and $\mfr : \C \to \R^2$
