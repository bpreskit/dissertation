\label{sec:ang_sync}
\subsection{Introduction and existing results}
In this section, we consider the problem of angular synchronization, which appears as a subproblem in many approaches to phase retrieval, including ours.  Angular synchronization is the problem of recovering a vector of complex units $x_i \in \Sbb^1, i \in [n]$, or $x \in (\Sbb^1)^n$ from estimates of their relative phases \[\tX_{ij} = x_j^* x_i \eta_{ij}, (i, j) \in E\] where $\eta_{ij} \in \Sbb^1$ are the ``noise terms'' and $E \subset [n]^2$ is a list of the pairs of indices for which we have such estimates.  This immediately suggests associating a graph to this problem; namely, taking the vertex set to be $V = [n]$, we set $G = (V, E)$.  We also remark that this problem invites the same global phase ambiguity as phase retrieval: indeed $(\ee^{\ii \theta} x_i)^* (\ee^{\ii \theta} x_j) \eta_{ij} = x_i^* x_j \eta_{ij}$ for any $\theta \in [0, 2\pi)$.  Obviously, then, our goal is to get an estimate $\tilde{x}$ that is guaranteeably close to the ground truth $x$ in some chosen metric, say our usual $\min_{\theta \in \R} || x - \ee^{\ii \theta} y ||_2,$ and to acquire this estimate with the least computational cost.  Naturally, $x$ is not known, so we instead attempt to minimize some cost function corresponding to how well our estimate explains the measurement data, usually taking a form similar to the frustration function stated in \eqref{eq:frustration}.  Often, we more simply take only the numerator of this expression, \[\sum_{(i, j) \in E} w_{ij} |x_i - X_{ij} x_j|^2 = x^* (D - W \circ X) x, \label{eq:ang_sync_cost}\] where $D$ and $W$ are the degree and weight matrices specified in \eqref{eq:degandweight}.  For convenience, we define \begin{equation} L = D - W \circ X, \ \uL = D - W \circ x x^*, \ \text{and} \ L_G = D - W. \label{eq:L_defs} \end{equation}  The approach to angular synchronization that we consider in this dissertation, and the approach most studied in the literature, then is to attempt the non-convex optimization problem \begin{equation} \begin{array}{cl} \min\limits_{z \in \C^n} & z^* L z \\ \text{s.t.} & |z_i| = 1 \end{array} \label{eq:ang_sync}\end{equation}

The modern study of eigenvector-based methods for angular synchronization appears to have begun with a 2011 paper by Amit Singer \cite{singer2011ang_sync}, in which he proposed two ways of solving this problem which remain as the basis of the state of the art.  The first is almost identical to the eigenvector-based approach that we have used in the algorithm proposed in chapter \ref{ch:our_model}, and the second is a semidefinite relaxation of the same.  Namely, using the unweighted adjacency matrix $A_{ij} = X_{ij} \chi_E(i, j)$, his eigenvector method solves \[ \max_{||z||_2^2 = n} z^* A Z \] to find the largest eigenvector $\hz$ of $A$ and rounds to a vector of units by taking $\tx = \sgn(hz)$.  The SDP method solves \begin{equation}\begin{array}{cl} \max\limits_{Z \in \H^d} & \Tr(A Z) \\ \text{s.t.} & Z \succeq 0 \\ & Z_{ii} = 1 \end{array} \label{eq:angsyncSDPsinger}\end{equation}  In this paper, he studies the problem under a noise model where the disturbances $\eta_{ij}$ are distributed according to \[\eta_{ij} = \left\{\begin{array}{r@{,\quad}l} 1 & \text{with probability}\ p \\ \mathrm{Unif}(\Sbb^1) & \text{with probability}\ 1 - p\end{array}\right.,\] such that the measurement $\tX_{ij}$ is exact with probability $p$ and is completely meaningless -- being drawn from the uniform distribution on $\Sbb^1$ -- with probability $1 - p$.  He proves the robustness of this method in the sense that there is a probability $p_c$, dependent on the spectral gap and size of the graph $G$, for which parameter values $p > p_c$ guarantee ``better than random'' approximations of $x$ to be recovered with high probability.  Moreover, he shows that, experimentally, both of these recovery algorithms work acceptably, if not extremely, well, with little to be gained by transferring from the eigenvector problem to the computationally more expensive semidefinite program.

The literature on angular synchronization since this paper has largely consisted of analyzing generalizations and variations of these methods.  One major generalization has been to apply these methods to larger classes of group synchronization problems such as synchronization over the orthogonal groups $O(d)$ or the special Euclidean groups $SE(d)$ \cite{Cheeger,briales2017cartan_sync,bandeira2016se_sync}.  Naturally, much of the interest in this subject has been the treatment of synchronization over $SO(3)$ and $SE(3)$, as these correspond to pose estimation problems fundamental to computer vision, as in \cite{enqvist2011nonsequential, olsson2017rot_avg, fischler1981ransac, govindu2006motion_avg}.  Significant results giving guarantees of robustness, as well as proofs that these relaxations are solved \emph{exactly} in certain cases may be found in \cite{alexeev2014phase, bandeira2016tightness, olsson2017rot_avg, bandeira2016se_sync}.

\subsection{Tightness of SDP relaxation}

We have already presented one angular synchronization result in theorem \ref{thm:SpecGraphPertBound} of \S\ref{sec:Perturb}, which drew largely on \cite{alexeev2014phase}.  % and which we restate here.
%% \begin{thm*}[Theorem \ref{thm:SpecGraphPertBound}]
%% Suppose that $G = (V=[d], E)$ is an undirected, connected, and unweighted graph (so that $W_{ij} = \chi_{E(i, j)}$) with spectral gap $\tau > 0$.  Let ${\bf u} \in \mathbbm{C}^d$ be an eigenvector of $L_1$ from \eqref{equ:ConnectLaplace} corresponding to its smallest eigenvalue, and let \[\widetilde{\bf x} = \sgn({\bf u}) \ \text{and} \ \tx_0 = \sgn(\x_0).\] Then for some universal constant $C \in \mathbbm{R}^+$, \[\min_{\theta \in [0, 2\pi]} ||\tx - \ee^{\ii \theta} \tx_0||_2 \le C \bigfrac{\|\tX - \widetilde{\X}_0\|_F}{\tau \cdot \sqrt{\min_{i \in V}(\deg(i))}},\] where $\tX$ and $\widetilde{\X}_0$ are defined as per \eqref{equ:DeftildeXviaGcomp} and \eqref{equ:DeftildeXviaGpure}, respectively. %%If, in addition, $G$ is $k$-regular, then we have \[\min_{\theta \in [0, 2\pi]} ||x - \ee^{\ii \theta}u||_2 \le C \bigfrac{||A_0 - A_1||_F}{k^{1/2}\tau}.\]
%% \end{thm*}
We remark that this theorem leaves something to be desired in that the graph $G$ is not permitted to be weighted, which restricts us from applying some knowledge that we may have about the problem.  For example, suppose our relative phase measurements $\{X_{ij}\}_{(i,j) \in E}$ are disturbed by noise drawn from a fixed, phase-invariant probability distribution, say \[X_{ij} = \sgn(x_i^*x_j + \eta_{ij}), \eta_{ij} = a_{ij} + b_{ij}i \ \text{with} \ a_{ij}, b_{ij} \iid N(0, \sigma^2),\] then we will have more confidence in the relative phases represented by the larger magnitude entries of $X = \mathcal{A}^{-1}(\y)$.  It would be intuitive to use this knowledge to privilege some edges of the graph over others in the frustration function \eqref{eq:frustration} that we are trying to minimize, say by using $w_{ij} = |X_{ij}|$.  Unfortunately, theorem \ref{thm:SpecGraphPertBound} assumes an unweighted graph and its proof technique does not readily admit a satisfactory adjustment towards weighted edges, though we shall impose one later.  Therefore, we will take a distinct approach, drawing upon recent results in the literature that consider certain convex relaxations of \eqref{eq:ang_sync} \cite{bandeira2016tightness, calafiore2016complex_pgo, bandeira2016se_sync}.

To begin this discussion, we gather our notation: $G = (V = [n], E)$ is a connected graph with a weighted adjacency matrix $W = [w_{ij}] \in \sym^n$ satisfying $w_{ij} \ge 0$ and $w_{ij} \neq 0$ only if $(i, j) \in E$.  $D = \diag(W \one)$ is the degree matrix.  $\ux \in (\Sbb^1)^n$ is the ground truth vector, which we attempt to recover, and $X, \uX \in \H^n$ are our noisy and ground truth edge data matrices, satisfying \[X_{ij} = \begin{piecewise} \eta_{ij} x_i x_j^* & (i, j) \in E \\ 0 & \ow \end{piecewise} \quad \text{and} \quad \uX_{ij} = \begin{piecewise} x_i x_j^* & (i, j) \in E \\ 0 & \ow \end{piecewise},\] where $\eta_{ij} = \eta_{ji}^* \in \Sbb^1$ for each $(i, j) \in E$.  We then define $L, \uL$, and $L_G$ as in \eqref{eq:L_defs}.

Towards formulating the appropriate SDP relaxations, we recall the transformation $\mfr : \C \to \R^{2 \times 2}$ %and $\ulmfr : \C \to \R^2$ defined by \[\mfr(a + bi) = \begin{bmatrix} a & -b \\ b & a \end{bmatrix}, \ \ulmfr(a + bi) = \begin{bmatrix} a \\ b \end{bmatrix}.\]
defined by \[\mfr(a + bi) = \begin{bmatrix} a & -b \\ b & a \end{bmatrix}.\]
It is well-known that $\mfr$ is the canonical isomorphism from $\C$ into $\R^{2 \times 2}$, and indeed if we extend it to matrices by taking, for $A \in \C^{m \times n}$, $\mfr(A) \in \R^{2m \times 2n}$ to be a block matrix with $\mfr(A)_{ij} = \mfr(A_{ij})$, then it remains an isomorphism on $\C^{m \times n}$, and indeed it preserves the eigenvalues and eigenvectors of Hermitian matrices (see, e.g.~\cite[p.~101]{wedderburn1934matrices}).  In particular, a hermitian matrix $A \in \H^n$ is semi-definite if and only if $\mfr(A)$ is semi-definite; we notice that the multiplicities of its eigenvalues are all doubled, as $A v = \lambda v$ implies $\mfr(A) \mfr(v) = \lambda \mfr(v)$, giving that the two columns of $\mfr(v)$ are each eigenvectors of $\mfr(A)$ with eigenvalue $\lambda$.

With this, we consider SDP relaxations of \eqref{eq:ang_sync}.  Specifically, we observe that $z^* L z = \Tr(L z z^*)$, so an equivalent optimization problem will be
\begin{equation}
  \begin{array}{cl}
    \min\limits_{Z \in \H^n} & \Tr(L Z) \\
    \text{s.t.} & Z_{ii} = 1 \\
    & \rank(Z) = 1 \\
    & Z \succeq 0
  \end{array}, \label{eq:ang_sync_rankone}
\end{equation}
where the optimizer $\hat{z}$ of \eqref{eq:ang_sync} is recovered from the optimal matrix $\hZ$ by merely factoring $\hZ = \hat{z} \hat{z}^*$.  To get a convex relaxation, we simply omit the non-convex rank one constraint, yielding \begin{equation} \begin{array}{cl} \min\limits_{Z \in \H^n} & \Tr(L Z) \\ \text{s.t.} & Z_{ii} = 1 \\ & Z \succeq 0 \end{array}. \label{eq:ang_sync_sdp} \end{equation}  We remark that if an optimizer $\hZ$ of \eqref{eq:ang_sync_sdp} is rank one, then it is also an optimizer of \eqref{eq:ang_sync_rankone} since the feasible set of \eqref{eq:ang_sync_sdp} is strictly larger than that of \eqref{eq:ang_sync_rankone}; in this case, then, factoring $\hZ$ gives the global minimizer of \eqref{eq:ang_sync}.  Considering that many results in the optimization literature are written for real-valued SDPs, we further relax the feasible set by casting into the real domain with
\begin{equation}
  \begin{array}{cl}
    \min\limits_{Z \in \sym^{2n \times 2n}} & \Tr(\mfr(L) Z) \\
    \text{s.t.} & Z_{ii} = I_2 \\
    & Z \succeq 0
  \end{array}, \label{eq:ang_sync_real}
\end{equation}
where $Z_{ii} = [e_{2i - 1} \ e_{2i}]^* Z [e_{2i - 1} \ e_{2i}]$ in this case refers to the $i\th \ 2 \times 2$ diagonal block of $Z$.

%% However, neither of these results may be directly applied, since \cite{bandeira2016tightness} is restricted to  The result stated in \cite{bandeira2016se_sync}, on the other hand, is far more general, applying to synchronization over $SE(n)$, but narrowing to angular synchronization (which is a special case of this problem) allows a considerable improvement over the error bound stated in the paper.  This sharpened result, as we shall see, leads to better guarantees than any we have yet proven.

%% This allows us to restate the SDP of \eqref{eq:angsyncSDPsinger} as a real-valued SDP in 

At this point, we recognize the previous work existing on this problem.  Namely, in \cite{bandeira2016tightness}, Bandeira, Singer, and Boumal prove that the optimizer $\hZ$ of \eqref{eq:ang_sync_sdp} is rank one (and therefore yields a minimizer of \eqref{eq:ang_sync}) when $L$ is sufficiently close to $\uL$.  Unfortunately for our purposes, this paper only considers the case when $G = K_n$ is the complete graph and the weights $W = \one \one^* - I_n$ are constant.  A more general result appears in \cite{bandeira2016se_sync}, where Rosen, et.~al prove a similar result for synchronization over $SE(d)$, of which angular synchronization is a special case.  Moreover, these results allow for a weighted graph, and include a bound on $\min_{\theta \in [0, 2\pi)} ||\hat{z} - \ee^{\ii\theta} \ux||_2$ in terms of $||L - \uL||_2$ and the spectral gap of the graph.  Nonetheless, we find that narrowing to the case of $SO(2)$ (equivalent to angular synchronization) allows for a tighter error bound.  In \cite{calafiore2016complex_pgo}, Calafiore, Carlone, and Dellaert use methods similar to those in \cite{bandeira2016se_sync} to analyze $SE(2)$ synchronization.  Furthermore, and pertinent to the present work, the authors exchange the rotational components in $SO(2)$ for complex units, but they do not weight their cost function, nor do they supply explicit bounds on the error of their estimate or on what level of noise may be tolerated and still guarantee that their convex relaxation solves \eqref{eq:ang_sync} exactly.  Significantly, all three of these works supply an \emph{a posteriori}-certifiable condition that can verify whether the solution obtained is indeed optimal for \cref{eq:ang_sync,eq:ang_sync_sdp,eq:ang_sync_rankone,eq:ang_sync_real}.

Among the results just mentioned, of greatest interest to us is proposition 2 in \cite{bandeira2016se_sync}, which may be restated as

\begin{proposition}[Proposition 2 and Theorem 12 in \cite{bandeira2016se_sync}]
  There exists a constant $\beta > 0$, depending on $\uL$, such that, if $||L - \uL||_2 < \beta$, then \eqref{eq:ang_sync_real} has a unique solution $\hZ$ which may be factored as $\hZ = R R^*$, with $R = \mfr(\hat{z})$ where $\hat{z} \in (\Sbb^1)^n$ is a global optimizer of \eqref{eq:ang_sync}.  Furthermore, \[\min_{\theta \in [0, 2\pi)} ||\hz - \ee^{\ii \theta} \ux||_2 \le 2 \sqrt{\dfrac{n ||\uL - L||_2}{\lambda_2(L_G)}},\] where $\lambda_2(L_G)$ is the second smallest eigenvalue of $L_G$.
\end{proposition}

We strengthen this result by giving $\beta$ explicitly and by increasing the exponent of $|| \uL - L ||_2$ in the error bound, which improves the convergence rate as $L \to \uL$.

\begin{theorem}
  Given a connected, weighted graph $G = (V = [n], E)$ with spectral gap $\tau = \lambda_2(D - W)$ and rotational data $X_{ij} \in \Sbb^1$ for $(i, j) \in E$, suppose that $\hz$ is a minimizer of \eqref{eq:ang_sync}, where $L = D - W \circ X$.  By $\ux$ we denote the ground truth, and we take $\uL = D - W \circ \ux \ux^*$ and $\hat{L} = D - W \circ \hz \hz^*$.  Then if $||L - \hat{L}||_2 < \frac{\tau}{1 + \sqrt{n}}$, $\hZ = \hz \hz^*$ and $\mfr(\hZ)$ are the unique minimizers of \eqref{eq:ang_sync_sdp} and \eqref{eq:ang_sync_real} and we have \[\min_{\theta \in [0, 2\pi)} || \hz - \ee^{\ii \theta} \ux ||_2 \le \dfrac{2 \sqrt{n} ||\uL - L||_2}{\tau}.\]
  %Then $\hz$ is also the unique minimizer of \cref{eq:ang_sync_sdp,eq:ang_sync_rankone,eq:ang_sync_real} if it satisfies \begin{equation} L - \diag(\Re(L \hz \hz^*)) \succeq 0 \ \text{and} \ \Nul(L - \diag(\Re(L \hz \hz^*))) = \Span(\hz). \label{eq:ang_sync_unique_cond} \end{equation}  In particular, $\hz$ may be found by solving an SDP under these conditions.  Also, these conditions hold if $||W \circ (\hz \hz^* - X)||_2 < \tau$, and we have that \[\min_{\theta \in [0, 2\pi)} || \hz - \ee^{\ii \theta} \ux ||_2 \le \dfrac{2 \sqrt{n} ||L - \uL||_2}{\tau}.\]
  \label{thm:ang_sync_dual}
\end{theorem}

To prove theorem \ref{thm:ang_sync_dual}, we introduce dual problems for \eqref{eq:ang_sync} and \eqref{eq:ang_sync_real}.  Specifically, we give the Lagrangian function $\Lc : \Cn \times \Rn \to \R$ of \eqref{eq:ang_sync}, \[\Lc(z, \lambda) = z^* L z + \sum_{i = 1}^n \lambda_i (1 - z_i^* z_i) = z^*(L - \diag(\lambda)) z + \sum_{i = 1}^n \lambda_i,\] and the dual function $d : \Rn \to \R$ \[d(\lambda) = \inf_{z \in \Cn} \Lc(z, \lambda),\] which have the properties that for any $\lambda \in \Rn, z \in (\Sbb^1)^n,$ we have \[d(\lambda) \le \Lc(z, \lambda) = z^* L z.\]  In particular, $\sup\limits_{\lambda \in \Rn} d(\lambda) \le \min\limits_{z \in (\Sbb^1)^n} z^* L z$.  Additionally, for any $\lambda \in \Rn$ such that $L - \diag(\lambda) \nsucceq 0,$ we have $d(\lambda) = - \infty$.  Otherwise, \[d(\lambda) = \Lc(0, \lambda) = \sum_{i = 1}^n \lambda_i.\]  Therefore, we may state the dual problem of \eqref{eq:ang_sync} as
\begin{equation}
  \begin{array}{cl} \max\limits_{\Lambda \in \Rnxn} & \Tr(\Lambda) \\
    \text{s.t.} & L - \Lambda \succeq 0 \\
    & \Lambda = \diag(\lambda_1, \ldots, \lambda_n)
  \end{array}, \label{eq:ang_sync_dual}
\end{equation}
in the sense that, if $\Lambda^*$ optimizes \eqref{eq:ang_sync_dual} and $\hz$ optimizes \eqref{eq:ang_sync}, then $\Tr(\Lambda^*) \le \hz^* L \hz$.  By a similar argument, we may define $\Lc_{SDP} : \H_+^n \times \Rn \to \R$ and $d_{SDP} : \Rn \to \R$
\begin{align*}
  \Lc_{SDP}(Z, \lambda) &= \Tr(LZ) + \sum_{i = 1}^n \lambda_i(1 - Z_{ii}) = \Tr((L - \diag(\lambda)) Z) + \Tr(\diag(\lambda)), \ \text{and} \\
  d_{SDP}(\lambda) &= \inf_{Z \in \H^n_+} \Lc_{SDP}(Z, \lambda) = \begin{piecewise} \Tr(\diag(\lambda)) & L - \diag(\lambda) \succeq 0 \\ -\infty & \text{otherwise} \end{piecewise}
\end{align*}
to see that $\sup\limits_{\lambda \in \Rn} d_{SDP}(\lambda) \le \Lc_{SDP}(Z, \lambda) = \Tr(LZ)$ for any $Z$ which is feasible to \eqref{eq:ang_sync_sdp}; therefore, if $\Lambda^*$ and $\hZ$ optimize \eqref{eq:ang_sync_dual} and \eqref{eq:ang_sync_sdp}, we will have $\Tr(\Lambda^*) \le \Tr(L\hZ)$.

To prove the uniqueness of the solution to \eqref{eq:ang_sync_sdp}, we will need to quote a result from \cite{alizadeh1997nondegeneracy}.  They use the primal-dual format with the primal problem
\begin{equation}
  \begin{array}{rlc}
    \displaystyle \min_{X \in \sym^n} & \Tr(C X) \\
    \text{s.t.} & \Tr(A_k X) = b_k, & k \in [m] \\
    & X \succeq 0
  \end{array} \label{eq:alizadeh_primal}
\end{equation}
where $C, A_k \in \sym^n$ and $b \in \Rm$ are fixed.  The dual problem is
\begin{equation}
  \begin{array}{rl}
    \displaystyle\max_{y \in \Rm} & b^T y \\
    \text{s.t.} & C - \sum_{k = 1}^m y_k A_k \succeq 0
  \end{array} \label{eq:alizadeh_dual}
\end{equation}
where $b \in \Rm$ and $A_k$ are as in the primal.  Among other results, they prove the following:

\begin{proposition}[Theorems 9 and 10 in \cite{alizadeh1997nondegeneracy}]
  Suppose that $y \in \Rm$ is dual feasible and optimal, with $\rank(C - \sum_{k = 1}^m y_k A_k) = s$.  Let the columns of $Q \in \R^{n \times n - s}$ be an orthonormal basis for $\Nul(C - \sum_{k = 1}^m y_k A_k)$, such that  \[\Col(Q) = \Nul(C - \sum_{k = 1}^m y_k A_k) \quad \text{and} \quad Q^T Q = I.\] then if \[\Span \{Q^T A_k Q\}_{k \in [m]} = \sym^{n - s},\] there is a unqiue optimal primal solution $X$. \label{prop:primal_unique}
\end{proposition}

In order to use this result, we will need the dual of \eqref{eq:ang_sync_real}.  Following \cref{eq:alizadeh_primal,eq:alizadeh_dual}, this gives
\begin{equation}
  \begin{array}{cl}
    \max\limits_{\Lambda \in \sym^{2n}} & \Tr(\Lambda) \\
    \text{s.t.} & \mfr(L) - \Lambda \succeq 0 \\
    & \Lambda = \diag(\Lambda_1, \ldots, \Lambda_n) \\
    & \Lambda_i \in \sym^2
  \end{array} \label{eq:ang_sync_real_dual}
\end{equation}
Applying \cref{prop:primal_unique} in this case yields the following corollary:

\begin{corollary}
  Suppose that $Z$ and $\Lambda$ are feasible for \eqref{eq:ang_sync_real} and \eqref{eq:ang_sync_real_dual}, respectively, and that $\Tr(\Lambda) = \Tr(\mfr(L) Z)$.  Then
\end{corollary}

With this, we state a few lemmas whose proofs will constitute a proof of \cref{thm:ang_sync_dual}.  Each of these assumes the notation and hypotheses of \cref{thm:ang_sync_dual}.

\begin{lemma}[Sufficient conditions for strong duality]
  If $\hz \in (\Sbb^1)^n$ satisfies \begin{gather} L - \diag\Re(\hz \hz^* L) \succeq 0, \ \text{and} \label{eq:ang_sync_certificate} \\ \Nul(L - \diag\Re(\hz \hz^* L)) = \Span(\hz), \label{eq:ang_sync_nullspace} \end{gather} then $\hZ = \hz \hz^*$ is the unique optimizer of \eqref{eq:ang_sync_sdp} and $\mfr(\hZ)$ is the unique optimizer of \eqref{eq:ang_sync_real}.
  \label{lem:necsuff_ang_sync_dual}
\end{lemma}

\begin{lemma}
  If $||L - \hat{L}||_2 < \frac{\tau}{1 + \sqrt{n}}$, then $\hz$ meets the conditions of \cref{lem:necsuff_ang_sync_dual} and $\hZ = \hz \hz^*$ is the unique optimizer of \eqref{eq:ang_sync_sdp} and $\mfr(\hZ)$ is the unique optimizer of \eqref{eq:ang_sync_real}.
  \label{lem:unique_ang_sync}
\end{lemma}

\begin{lemma}
  Suppose that $\hz$ minimizes \eqref{eq:ang_sync}.  Then \[\min_{\theta \in [0, 2\pi)} || \hz - \ee^{\ii \theta} \ux ||_2 \le \dfrac{2 \sqrt{n} ||\uL - L||_2}{\tau}.\]
    \label{lem:error_ang_sync}
\end{lemma}

Before proving these lemmas, we note the origin of the $L - \diag \Re (\hz \hz^* L)$ term.  We observe that $(\Sbb^1)^n$ is an $n$-dimensional manifold with charts given by \[\phi_U : U \to (\Sbb^1)^n,\quad \phi_U(\theta_1, \ldots, \theta_n)_i = \ee^{\ii \theta_i},\] where $U \subset \Rn$ is any open set satisfying $||x - y||_\infty < 2 \pi$ for all $x, y \in U$ (or, more generally, $x - y \notin 2 \pi (\Z^n)$ for all $x, y \in U$).  Assume $\hz$ is a minimizer of \eqref{eq:ang_sync}.  Because $z^* L z$ is smooth on $(\Sbb^1)^n$, for any chart $\phi_U$ with $\hz \in \phi_U(U)$, we must have $\left.\nabla (L \circ \phi_U)\right|_{\phi_U^{-1}(\hz)} = 0$.  In particular, if $\theta \in \Rn$ is such that $\hz_i = \ee^{\ii \theta_i}$, then \[\left.\nabla (L \circ \phi_U)\right|_{\phi_U^{-1}(\hz)} = \nabla_\theta \hz^* L \hz.\]  We remark that \begin{align*} \hz^* L \hz &= \sum_{ij} \ee^{\ii (\theta_j - \theta_i)} L_{ij} = \sum_i \ee^{\ii \theta_i} \hz^* L_i \end{align*} where $L_i = L e_i$ and $L_{ij} = (L_i)_j$.  This gives \begin{align*} \frac{\partial}{\partial \theta_j} \hz^* L \hz &= \ii \ee^{\ii \theta_j} \hz^* L_j + \sum_{i = 1}^n \ee^{\ii \theta_i} \frac{\partial}{\partial \theta_j} \hz^* L_i \\ &= \ii \ee^{\ii \theta_j} \hz^* L_j - \ii \ee^{\ii \theta_j} \sum_{i = 1}^n \ee^{-\ii \theta_i} L_{ji} \\ &= \ii \ee^{\ii \theta_j} \hz^* L_j - \ii\ee^{-\ii \theta_j} L_j^* \hz \\ &= -2 \Im(\ee^{\ii \theta_j} \hz^* L_j) = -2 \Im (\hz \hz^* L)_{jj}\end{align*}  Therefore, $\left.\nabla (L \circ \phi_U)\right|_{\phi_U^{-1}(\hz)} = 0$ if and only if $\diag \Im(\hz \hz^* L) = 0$.  On the other hand, we observe that \begin{align*} ((L - \diag \Re(\hz \hz^* L)) \hz)_i &= L_i^* \hz - \Re(\hz_i \hz^* L_i) \hz_i \\ &= L_i^* \hz - \Re(\overline{\hz}_i L_i^* \hz) \hz_i,\end{align*} such that $\hz \in \Nul(L - \diag \Re(\hz \hz^* L))$ iff
\[
\begin{array}{crcll}
  & L_i^* \hz & = & \Re(\overline{\hz}_i L_i^* \hz) \hz_i, & \text{for all}\ i \\
  \iff & \overline{\hz}_i L_i^* \hz & = & \Re(\overline{\hz}_i L_i^* \hz), & \text{for all}\ i \\
  \iff & \overline{\hz}_i L_i^* \hz & \in & \R & \text{for all}\ i \\
  \iff & \diag \Im(\hz \hz^* L) & = & 0
\end{array}
\]
This may be summarized as \begin{equation} z \in \Nul(L - \diag \Re(\hz \hz^* L)) \iff \overline{\hz}_i L_i^* \hz \in \R \ \text{for all}\ i \label{eq:null_iff_real} \end{equation} so that $\hz$ is a minimizer of \eqref{eq:ang_sync} only if $\hz \in \Nul(L - \diag\Re(\hz\hz^* L))$, and this further suggests that the matrix $L - \diag\Re(\hz\hz^* L)$ can play a role in certifying minima of these optimization problems.

  Now for the proof.

\begin{proof}[Proof of \cref{lem:necsuff_ang_sync_dual}]
  Suppose that $\hz \in (\Sbb^1)^n$ satisfies $L - \diag \Re(\hz \hz^* L) \succeq 0$ and $(L - \diag \Re(\hz \hz^* L))z = 0$, and set $\hZ = \hz \hz^*$.  We begin by showing that \[\hz \in \Nul(L -  \diag \Re(\hz \hz^* L)) \ \text{and} \ L - \diag \Re(\hz \hz^* L) \succeq 0 \] suffices to show that $\hZ = \hz \hz^*$ and $\mfr(\hZ)$ are optimizers of \eqref{eq:ang_sync_sdp} and \eqref{eq:ang_sync_real}, respectively.  We then show that, if $\Nul(L -  \diag \Re(\hz \hz^* L)) = \Span(\hz)$ holds in addition to these, then they are the \emph{unique} optimizers of their problems.

  Setting $\hat{\Lambda} = \diag \Re(\hz \hz^* L), \hat{\Lambda}$ is feasible for \eqref{eq:ang_sync_dual} since we have assumed $L - \diag \Re(\hz \hz^* L) \succeq 0$.  Then we have
  \[
  \begin{array}{rclcll}
    \Tr(\hat{\Lambda}) & = & \Tr(\diag\Re(\hz \hz^* L))& = & \Tr(\Re(\hz \hz^* L)) \\
    & = & \sum_{i = 1}^n \Re(\hz_i \hz^* L_i) & = & \sum_{i = 1}^n \hz_i \hz^* L_i & \text{(from \eqref{eq:null_iff_real})} \\
    & = & \hz^* L \hz
  \end{array}
  \]
  Therefore $(\hz, \hat{\Lambda})$ is a primal-dual pair for \eqref{eq:ang_sync} and \eqref{eq:ang_sync_dual} with equal objective function values, showing that they are both optimizers for their respective problems.  Since \eqref{eq:ang_sync_dual} is also the dual problem for \eqref{eq:ang_sync_sdp}, and since $\Tr(L \hZ) = \Tr(L \hz \hz^*) = \hz^* L \hz$, $\hat{\Lambda}$ also certifies optimality of $\hZ$ for \emph{its} problem, and similarly also for $\mfr{\hZ}$ by observing that $\mfr(\hat{\Lambda})$ and $\mfr(\hZ) = \mfr(\hz) \mfr(\hz)^*$ are feasible for their problems, and $\Tr(\mfr(A)) = 2 \Tr(A)$ for general $A \in \H^n$.

  To show uniqueness, we quote theorems 9 and 10 of \cite{alizadeh1997nondegeneracy}, which state -- in our case -- that if $\Lambda \in \S^{2n}$ is feasible for \eqref{eq:ang_sync_real_dual}

\end{proof}
